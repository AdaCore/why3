\chapter{Language Reference}
\label{chap:syntaxref}

This chapter gives the grammar and semantics for \why and \whyml input files.

\section{Lexical conventions}

Lexical conventions are common to \why and \whyml.

% TODO: blanks

\paragraph{Comments.}
Comments are enclosed by \texttt{(*} and \texttt{*)} and can be nested.

\paragraph{Strings.}
Strings are enclosed in double quotes (\verb!"!). Double quotes can be
inserted in strings using the backslash character (\verb!\!).
In the following, strings are referred to with the non-terminal
\nonterm{string}{}.

% TODO: escape sequences for strings

\paragraph{Identifiers.} The syntax distinguishes lowercase and
uppercase identifiers and, similarly, lowercase and uppercase
qualified identifiers.

\begin{center}\framebox{\input{./qualid_bnf.tex}}\end{center}

\paragraph{Constants.}
The syntax for constants is given in Figure~\ref{fig:bnf:constant}.
Integer and real constants have arbitrary precision.
Integer constants may be given in base 16, 10, 8 or 2.
Real constants may be given in base 16 or 10.

\begin{figure}
\begin{center}\framebox{\input{./constant_bnf.tex}}\end{center}
  \caption{Syntax for constants.}
\label{fig:bnf:constant}
\end{figure}

\paragraph{Operators.}
Prefix and infix operators are built from characters organized in four
categories (\textsl{op-char-1} to \textsl{op-char-4}).
\begin{center}\framebox{\input{./operator_bnf.tex}}\end{center}
Infix operators are classified into 4 categories, according to the
characters they are built from:
\begin{itemize}
\item level 4: operators containing only characters from
\textit{op-char-4};
\item level 3: those containing
 characters from \textit{op-char-3} or \textit{op-char-4};
\item level 2: those containing
 characters from \textit{op-char-2}, \textit{op-char-3} or
 \textit{op-char-4};
\item level 1: all other operators (non-terminal \textit{infix-op-1}).
\end{itemize}

\paragraph{Labels.} Identifiers, terms, formulas, program expressions
can all be labeled, either with a string, or with a location tag.
\begin{center}\framebox{\input{./label_bnf.tex}}\end{center}
A location tag consists of a file name, a line number, and starting
and ending characters.

%%%%%%%%%%%%%%%%%%%%%%%%%%%%%%%%%%%%%%%%%%%%%%%%%%%%%%%%%%%%%%%%%%%%%%%%%%%%%%

\section{Why3 Language}

\paragraph{Terms.}
The syntax for terms is given in Figure~\ref{fig:bnf:term}.
The various constructs have the following priorities and
associativities, from lowest to greatest priority:
\begin{center}
  \begin{tabular}{|l|l|}
    \hline
    construct & associativity \\
    \hline\hline
    \texttt{if then else} / \texttt{let in} & -- \\
    label & -- \\
    cast  & -- \\
    infix-op level 1 & left \\
    infix-op level 2 & left \\
    infix-op level 3 & left \\
    infix-op level 4 & left \\
    prefix-op     & --   \\
    function application & left \\
    brackets / ternary brackets & -- \\
    bang-op       & --   \\
    \hline
  \end{tabular}
\end{center}

Note the curryfied syntax for function application, though partial
application is not allowed (rejected at typing).

\begin{figure}
  \begin{center}\framebox{\input{./term_bnf.tex}}\end{center}
  \caption{Syntax for terms.}
\label{fig:bnf:term}
\end{figure}

\paragraph{Type Expressions.} The syntax for type expressions is the following:
\begin{center}\framebox{\input{./type_bnf.tex}}\end{center}
Built-in types are \texttt{int}, \texttt{real}, and tuple types.
Note that the syntax for type
expressions notably differs from the usual ML syntax (\eg the
type of polymorphic lists is written \texttt{list 'a}, not \texttt{'a list}).

\paragraph{Formulas.}
The syntax for formulas is given Figure~\ref{fig:bnf:formula}.
The various constructs have the following priorities and
associativities, from lowest to greatest priority:
\begin{center}
  \begin{tabular}{|l|l|}
    \hline
    construct & associativity \\
    \hline\hline
    \texttt{if then else} / \texttt{let in} & -- \\
    label & -- \\
    \texttt{->} / \texttt{<->} & right \\
    \verb!\/! / \verb!||! & right \\
    \verb|/\| / \verb!&&! & right \\
    \texttt{not}  & -- \\
    infix level 1 & left \\
    infix level 2 & left \\
    infix level 3 & left \\
    infix level 4 & left \\
    prefix        & --   \\
    \hline
  \end{tabular}
\end{center}
Note that infix symbols of level 1 include equality (\texttt{=}) and
disequality (\texttt{<>}).

\begin{figure}
  \begin{center}\framebox{\input{./formula_bnf.tex}}\end{center}
  \caption{Syntax for formulas.}
\label{fig:bnf:formula}
\end{figure}

Notice that there are two symbols for the conjunction: \verb|/\|
and \verb|&&|, and similarly for disjunction. They are logically
equivalent, but may be treated slightly differently by some
transformations. For instance, \texttt{split} transforms the goal
\verb|A /\ B| into subgoals \verb|A| and \verb|B|, whereas it transforms
\verb|A && B| into subgoals \verb|A| and \verb|A -> B|. Similarly, it
transforms \verb!not (A || B)! into subgoals \verb|not A| and
\verb|not ((not A) /\ B)|.

\paragraph{Theories.}
The syntax for theories is given on Figure~\ref{fig:bnf:theorya} and~\ref{fig:bnf:theoryb}.

\begin{figure}
  \begin{center}\framebox{\input{./theory_bnf.tex}}\end{center}
  \caption{Syntax for theories (part 1).}
\label{fig:bnf:theorya}
\end{figure}

\begin{figure}
  \begin{center}\framebox{\input{./theory2_bnf.tex}}\end{center}
  \caption{Syntax for theories (part 2).}
\label{fig:bnf:theoryb}
\end{figure}

\paragraph{Files.}
A \why input file is a (possibly empty) list of theories.
\begin{center}\framebox{\input{./why_file_bnf.tex}}\end{center}


%%%%%%%%%%%%%%%%%%%%%%%%%%%%%%%%%%%%%%%%%%%%%%%%%%%%%%%%%%%%%%%%%%%%%%%%%%%%%%
\clearpage
\section{WhyML Language}\label{sec:syntax:whyml}

\subsection{Specification}

The syntax for specification clauses in programs 
is given in Figure~\ref{fig:bnf:spec}.
\begin{figure}
  \begin{center}\framebox{\input{./spec_bnf.tex}}\end{center}
  \caption{Specification clauses in programs.}
\label{fig:bnf:spec}
\end{figure}
Within specifications, terms are extended with new constructs
\verb|old| and \verb|at|:
\begin{center}\framebox{\input{./term_old_at_bnf.tex}}\end{center}
Within a postcondition, $\verb|old|~t$ refers to the value of term $t$
in the prestate. Within the scope of a code mark $L$,
the term $\verb|at|~t~\verb|'|L$ refers to the value of term $t$ at the program
point corresponding to $L$.

\subsection{Expressions}

The syntax for program expressions is given in
Figure~\ref{fig:bnf:expra} and~Figure~\ref{fig:bnf:exprb}.
\begin{figure}
  \begin{center}\framebox{\input{./expr_bnf.tex}}\end{center}
  \caption{Syntax for program expressions (part 1).}
\label{fig:bnf:expra}
\end{figure}

\begin{figure}
  \begin{center}\framebox{\input{./expr2_bnf.tex}}\end{center}
  \caption{Syntax for program expressions (part 2).}
\label{fig:bnf:exprb}
\end{figure}

In applications, arguments are evaluated from right to left.
This includes applications of infix operators, with the only exception of
lazy operators \verb|&&| and \verb+||+ that evaluate from left to
right, lazily.


% In the following we describe the informal semantics of each
% constructs, and provide the corresponding rule for computing the
% weakest precondition.


% \subsubsection{Constant Expressions, Unary and Binary Operators}


% Integer and real constants are as in the logic language, as weel as the unary and binary operators.


% \subsubsection{Array accesses and updates, fields access and updates}

% \todo{}

% \subsubsection{Let binding, sequences}

% \todo{}

% \subsubsection{Function definition}

% \todo{fun, let rec}

% \subsubsection{Function call}

% \todo{}

% \subsubsection{Exception throwing and catching}

% \todo{raise, try with end}

% \subsubsection{Conditional expression, pattern matching}

% \todo{if then else. Discuss standard WP versus fast WP}

% \subsubsection{Iteration Expressions}

% There are three kind of expressions for iterating: bounded, unbounded and infinite.

% \begin{itemize}
% \item A bounded iteration has the form
% \begin{flushleft}\ttfamily
%   for $i$=$a$ to $b$ do invariant \{ $I$ \} $e$ done
% \end{flushleft}
% Expressions $a$ and $b$ are evaluated first and only once, then expression $e$ si evaluated successively for $i$ from $a$ to $b$ included. Nothing is executed if $a > b$. The invariant $I$ must hold at each iteration including before entering the loop and when leaving it. The rule for computing WP is as follows:
% \begin{eqnarray*}
%   WP(\texttt{for} \ldots, Q) &=& I(a) \land \\
% && \forall \vec{w} (\forall i, a \leq i \leq b \land I(i) \rightarrow WP(e,I(i+1))) \land (I(b+1) \rightarrow Q)
% \end{eqnarray*}
% where $\vec{w}$ is the set of references modified by $e$.

% A downward bounded iteration is also available, under the form
% \begin{flushleft}\ttfamily
%   for $i$=$a$ downto $b$ do invariant \{ $I$ \} $e$ done
% \end{flushleft}

% \item An unbounded iteration has the form
% \begin{flushleft}\ttfamily
%   while $c$ do invariant \{ $I$ \} $e$ done
% \end{flushleft}
% it iterates the loop body $e$ until the condition $c$ becomes false. 
% \begin{eqnarray*}
%   WP(\texttt{while} \ldots, Q) &=& I \land \\
% && \forall \vec{w} (c \land I \rightarrow WP(e,I)) \land (\neg c \land I \rightarrow Q)
% \end{eqnarray*}
% where $\vec{w}$ is the set of references modified by $e$.

% Additionally, such a loop can be given a variant $v$, a quantity that must decreases ar each iteration, so as to prove its termination.


% \item An infinite iteration has the form
% \begin{flushleft}\ttfamily
%   loop invariant \{ $I$ \} $e$ end
% \end{flushleft}
% it iterates the loop forever, hence the only way to exit such a loop is to raise an exception.
% \begin{eqnarray*}
%   WP(\texttt{loop} \ldots, Q \mid Exc \Rightarrow R) &=& I \land \\
% && \forall \vec{w} (I \rightarrow WP(e,I)) \land (I \rightarrow WP(e,Exc \Rightarrow R))
% \end{eqnarray*}
% \end{itemize}

% \subsubsection{Assertions, Code Contracts}

% There are several form of expressions for inserting annotations in the code.
% The first form of those are the \emph{assertions} which have the form
% \begin{flushleft}\ttfamily
%   \textsl{keyword} \{ \textsl{proposition} \}
% \end{flushleft}
% where \textsl{keyword} is either \texttt{assert}, \texttt{assume} or
% \texttt{check}. They all state that the given proposition holds at the given program point. The differences are:
% \begin{itemize}
% \item \texttt{assert} requires to prove that the proposition holds, and then make it available in the context of the remaining of the code
% \item \texttt{check} requires to prove that the proposition holds, but
%   does not make it visible in the remaining
% \item \texttt{assume} assumes that the proposition holds and make it
%   visible in the context of the remaining code, without requiring to
%   prove it. It acts like an axiom, but within a program.
% \end{itemize}
% The corresponding rules for computing WP are as follows:
% \begin{eqnarray*}
%   WP(\texttt{assert} \{ P \}, Q) &=& P \mathop{\&\&} Q = P \land (P \rightarrow Q)\\
%   WP(\texttt{check} \{ P \}, Q) &=& P \land Q \\
%   WP(\texttt{assume} \{ P \}, Q) &=& P \rightarrow Q
% \end{eqnarray*}

% The other forms of code contracts allow to abstract a piece of code by specifications.
% \begin{itemize}
% \item $\texttt{any}~\{ P \}~\tau~\epsilon~\{ Q \}$ is a
%   non-deterministic expression that requires the precondition $P$ to
%   hold, then makes some side effects $\epsilon$, and returns any value
%   of type $\tau$ such that $Q$ holds. This construct acts as an axiom
%   in the sense that it does not check whether there exists any program
%   that can effectively establish the post-condition (similarly as the
%   introduction of a \texttt{val} at the global level).
% \item $\texttt{abstract}~e~\{ Q \}$ makes sure that the evaluation of
%   expression $e$ establishes the post-condition $Q$, and then abstract
%   away the program state by the post-condition $Q$ (similarly to the
%   \texttt{any} construct).
% \end{itemize}
% The corresponding rules for computing WP are as follows:
% \[
% \begin{array}{l}
%   WP(\texttt{any}~\{ P \}~\tau~\epsilon~\{ Q \mid exn_i \Rightarrow R_i \} ,
%   Q'  exn_i \Rightarrow R'_i) = \\
%   \qquad\qquad P \mathop{\&\&} \forall result, \epsilon.
%   (Q \rightarrow Q') \land (R_i \rightarrow R'_i) \\
%   WP(\texttt{abstract}~e~\{ Q \mid exn_i \Rightarrow R_i \} ,
%   Q' \mid exn_i \Rightarrow R'_i) = \\
%   \qquad\qquad WP(e,Q \mid exn_i \Rightarrow R_i) \land
%   \forall result, \epsilon, Q \rightarrow Q' \land R_i \rightarrow R'_i
% \end{array}
% \]

% \subsubsection{Labels, Operators \texttt{old} and \texttt{at}}

% \todo{Labels, old, at}

\subsection{Modules}

The syntax for modules is given in Figure~\ref{fig:bnf:module}.
\begin{figure}
  \begin{center}\framebox{\input{./module_bnf.tex}}\end{center}
  \caption{Syntax for modules.}
\label{fig:bnf:module}
\end{figure}
Any declaration which is accepted in a theory is also accepted in a
module. Additionally, modules can introduce record types with mutable
fields and declarations which are specific to programs (global
variables, functions, exceptions).

\subsection{Files}

A \whyml input file is a (possibly empty) list of theories and modules.
\begin{center}\framebox{\input{./whyml_file_bnf.tex}}\end{center}
A theory defined in a \whyml\ file can only be used within that
file. If a theory is supposed to be reused from other files, be they
\why\ or \whyml\ files, it should be defined in a \why\ file.


%%% Local Variables:
%%% mode: latex
%%% TeX-PDF-mode: t
%%% TeX-master: "manual"
%%% End:
