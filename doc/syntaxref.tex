\chapter{Syntax Reference}
\label{chap:syntaxref}

This chapter gives the grammar for \why\ input files.

% TODO: blanks

\paragraph{Comments.}
Comments are enclosed by \texttt{(*} and \texttt{*)} and can be nested.

\paragraph{Strings.}
Strings are enclosed by double quotes (\verb!"!). Double quotes can be
inserted in strings using the backslash character (\verb!\!).

% TODO: escape sequences for strings

\paragraph{Identifiers.} The syntax distinguishes lowercase and
uppercase identifiers and, similarly, lowercase and uppercase
qualified identifiers.
 
\begin{center}\framebox{\input{./qualid_bnf.tex}}\end{center}

\paragraph{Constants.}
The syntax for constants is given Figure~\ref{fig:bnf:constant}.
Integer and real constants have arbitrary precision.
Integer constants may be given in base 16, 10, 8 or 2.
Real constants may be given in base 16 or 10.
 
\begin{figure}[p]
\begin{center}\framebox{\input{./constant_bnf.tex}}\end{center}
  \caption{Syntax for constants.}
\label{fig:bnf:constant}
\end{figure}

\paragraph{Operators.} 
Prefix and infix operators are built from characters organized in four
categories (\textsl{op-char-1} to \textsl{op-char-4}).
\begin{center}\framebox{\input{./operator_bnf.tex}}\end{center}
Infix operators are classified into 4 categories, according to the
characters they are built from:
\begin{itemize}
\item level 4: operators containing only characters from
\textit{op-char-4};
\item level 3: those containing
 characters from \textit{op-char-3} or \textit{op-char-4};
\item level 2: those containing
 characters from \textit{op-char-2}, \textit{op-char-3} or
 \textit{op-char-4}; 
\item level 1: all other operators.
\end{itemize}

\paragraph{Terms.}
The syntax for terms is given Figure~\ref{fig:bnf:term}.
The various constructs have the following priorities and
associativities, from lowest to greatest priority:
\begin{center}
  \begin{tabular}{|l|l|}
    \hline
    construct & associativity \\
    \hline\hline
    \texttt{if then else} / \texttt{let in} & -- \\
    label & -- \\
    cast  & -- \\
    infix level 1 & left \\
    infix level 2 & left \\
    infix level 3 & left \\
    infix level 4 & left \\
    prefix        & --   \\
    \hline
  \end{tabular}
\end{center}

Note the curryfied syntax for function application, though partial
application is not allowed (rejected at typing).

\begin{figure}[p]
  \begin{center}\framebox{\input{./term_bnf.tex}}\end{center}
  \caption{Syntax for terms.}
\label{fig:bnf:term}
\end{figure}

\paragraph{Type Expressions.} The syntax for type
expressions notably differs from the usual ML syntax.
\begin{center}\framebox{\input{./type_bnf.tex}}\end{center}

\paragraph{Formulas.}
The syntax for formulas is given Figure~\ref{fig:bnf:formula}.
The various constructs have the following priorities and
associativities, from lowest to greatest priority:
\begin{center}
  \begin{tabular}{|l|l|}
    \hline
    construct & associativity \\
    \hline\hline
    \texttt{if then else} / \texttt{let in} & -- \\
    label & -- \\
    \texttt{->} / \texttt{<->} & right \\
    \texttt{or} / \verb!\/! & right \\
    \texttt{and} / \verb!/\! & right \\
    \texttt{not}  & -- \\
    infix level 1 & left \\
    infix level 2 & left \\
    infix level 3 & left \\
    infix level 4 & left \\
    prefix        & --   \\
    \hline
  \end{tabular}
\end{center}
Note that infix symbols of level 1 include equality (\texttt{=}) and
disequality (\texttt{<>}).

\begin{figure}[p]
  \begin{center}\framebox{\input{./formula_bnf.tex}}\end{center}
  \caption{Syntax for formulas.}
\label{fig:bnf:formula}
\end{figure}

\paragraph{Theories.}
The syntax for theories is given Figure~\ref{fig:bnf:theory}.

\begin{figure}[p]
  \begin{center}\framebox{\input{./theory_bnf.tex}}\end{center}
  \caption{Syntax for theories.}
\label{fig:bnf:theory}
\end{figure}

%%% Local Variables:
%%% mode: latex
%%% TeX-PDF-mode: t
%%% TeX-master: "manual"
%%% End:
