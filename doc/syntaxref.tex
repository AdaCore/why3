\chapter{Syntax Reference}
\label{chap:syntaxref}

This chapter gives the grammar for the input files.

TODO: constants

\paragraph{Identifiers.} The syntax distinguishes lowercase and
uppercase identifiers and, consequently, lowercase and uppercase
qualified identifiers.
 
\begin{center}\framebox{\input{./qualid_bnf.tex}}\end{center}

\paragraph{Type Expressions.} The syntax for type
expressions notably differs from the usual ML syntax.
\begin{center}\framebox{\input{./type_bnf.tex}}\end{center}

\paragraph{Terms and Types.}
The syntax for terms is given Figure~\ref{fig:bnf:term}.
Note the curryfied syntax for function application, though partial
application is not allowed (rejected at typing).

TODO: prefix and infix operators

\begin{figure}
  \begin{center}\framebox{\input{./term_bnf.tex}}\end{center}
  \caption{Syntax for terms.}
\label{fig:bnf:term}
\end{figure}

\paragraph{Formulas.}
The syntax for formulas is given Figure~\ref{fig:bnf:formula}.

\begin{figure}
  \begin{center}\framebox{\input{./formula_bnf.tex}}\end{center}
  \caption{Syntax for formulas.}
\label{fig:bnf:formula}
\end{figure}

\paragraph{Theories.}

\begin{figure}
  \begin{center}\framebox{\input{./theory_bnf.tex}}\end{center}
  \caption{Syntax for theories.}
\label{fig:bnf:theory}
\end{figure}

%%% Local Variables:
%%% mode: latex
%%% TeX-PDF-mode: t
%%% TeX-master: "manual"
%%% End:
