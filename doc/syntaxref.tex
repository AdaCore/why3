\chapter{Syntax Reference}
\label{chap:syntaxref}

This chapter gives the grammar for \why\ input files.

% TODO: blanks

\paragraph{Comments.}
Comments are enclosed by \texttt{(*} and \texttt{*)} and can be nested.

\paragraph{Strings.}
Strings are enclosed by double quotes (\verb!"!).

% TODO: escape sequences for strings

\paragraph{Identifiers.} The syntax distinguishes lowercase and
uppercase identifiers and, similarly, lowercase and uppercase
qualified identifiers.
 
\begin{center}\framebox{\input{./qualid_bnf.tex}}\end{center}

\paragraph{Constants.}
The syntax for constants is given Figure~\ref{fig:bnf:constant}.

\begin{figure}[p]
\begin{center}\framebox{\input{./constant_bnf.tex}}\end{center}
  \caption{Syntax for constants.}
\label{fig:bnf:constant}
\end{figure}

\paragraph{Operators.} Prefix operators have greater priority than
infix operators. Infix operators are left associative and have
priorities set as follows, from greatest to lowest priorities:
\begin{itemize}
\item operators containing only characters from
\textit{op-char-4};
\item those containing
 characters from \textit{op-char-3} or \textit{op-char-4};
\item those containing
 characters from \textit{op-char-2}, \textit{op-char-3} or
 \textit{op-char-4}; 
\item all other operators.
\end{itemize}

\begin{center}\framebox{\input{./operator_bnf.tex}}\end{center}

\paragraph{Terms.}
The syntax for terms is given Figure~\ref{fig:bnf:term}.
Note the curryfied syntax for function application, though partial
application is not allowed (rejected at typing).

\begin{figure}[p]
  \begin{center}\framebox{\input{./term_bnf.tex}}\end{center}
  \caption{Syntax for terms.}
\label{fig:bnf:term}
\end{figure}

\paragraph{Type Expressions.} The syntax for type
expressions notably differs from the usual ML syntax.
\begin{center}\framebox{\input{./type_bnf.tex}}\end{center}

\paragraph{Formulas.}
The syntax for formulas is given Figure~\ref{fig:bnf:formula}.

\begin{figure}[p]
  \begin{center}\framebox{\input{./formula_bnf.tex}}\end{center}
  \caption{Syntax for formulas.}
\label{fig:bnf:formula}
\end{figure}

\paragraph{Theories.}
The syntax for theories is given Figure~\ref{fig:bnf:theory}.

\begin{figure}[p]
  \begin{center}\framebox{\input{./theory_bnf.tex}}\end{center}
  \caption{Syntax for theories.}
\label{fig:bnf:theory}
\end{figure}

%%% Local Variables:
%%% mode: latex
%%% TeX-PDF-mode: t
%%% TeX-master: "manual"
%%% End:
