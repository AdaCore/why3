\chapter{The \why\ API}
\label{chap:api}

This chapter is a tutorial for the users who want to link their own
OCaml code with the \why\ library. We progressively introduce the way
one can use the library to build terms, formulas, theories, proof
tasks, call external provers on tasks, and apply transformations on
tasks. The complete documentation for API calls is given
[TODO in Chapter~ref{chap:apidoc}.]

We assume the reader has a fair knowledge of the OCaml
language. Notice that the \why\ library must be installed,
see Section~\ref{sec:installlib}.


\section{Building Propositional Formulas}

The first step is to know how to build propositional formulas. The
module \texttt{Term} gives a few functions for building these. Here is
a piece of OCaml code for building the formula $true \lor false$.
\begin{verbatim}
(* opening the Why3 library *)
open Why

(* a ground propositional goal: true or false *)
let fmla_true : Term.fmla = Term.f_true
let fmla_false : Term.fmla = Term.f_false
let fmla1 : Term.fmla = Term.f_or fmla_true fmla_false
\end{verbatim}
As one can guess, the type \texttt{fmla} is the type of formulas in
the library.

Such a formula can be printed using the module \texttt{Pretty}
providing pretty-printers.
\begin{verbatim}
(* printing the formula *)
open Format
let () = printf "@[formula 1 is:@ %a@]@." Pretty.print_fmla fmla1
\end{verbatim}

Assuming the lines above are written in a file \texttt{f.ml}, it can
be compiled using
\begin{verbatim}
ocamlc str.cma unix.cma nums.cma dynlink.cma \
        -I +ocamlgraph -I +why3 graph.cma why.cma f.ml -o f
\end{verbatim}
Running the generated executable \texttt{f} results in the following output.
\begin{verbatim}
formula 1 is: true or false
\end{verbatim}

Let's now build a formula with propositional variables: $A \land B
\rightarrow A$. Propositional variables must be declared first before
using them in formulas. This is done as follows.
\begin{verbatim}
let prop_var_A : Term.lsymbol =
  Term.create_psymbol (Ident.id_fresh "A") []
let prop_var_B : Term.lsymbol =
  Term.create_psymbol (Ident.id_fresh "B") []
\end{verbatim}
The type \texttt{lsymbol} is the type of logic symbols. Then the atoms $A$ and $B$
must be built by the general function for applying a predicate symbol to a list of terms. Here we just need the empty list of arguments.
\begin{verbatim}
let atom_A : Term.fmla = Term.f_app prop_var_A []
let atom_B : Term.fmla = Term.f_app prop_var_B []
let fmla2 : Term.fmla =
  Term.f_implies (Term.f_and atom_A atom_B) atom_A
let () = printf "@[formula 2 is:@ %a@]@." Pretty.print_fmla fmla2
\end{verbatim}

As expected, the output is as follows.
\begin{verbatim}
formula 2 is: A and B -> A
\end{verbatim}
Notice that the concrete syntax of \why\ forbids predicate identifiers
to start with a capital letter. This constraint does not exist when
building those directly using library calls.

\section{Building Tasks}

Let's see how we can call a prover to prove a formula. As said in
previous chapters, a prover must be given a task, so we need to build
tasks from our formulas. Task can be build incrementally from an empty
task by adding declaration to it, using the functions
\texttt{add\_*\_decl} of module \texttt{Task}. For the formula $true \lor
false$ above, this is done as follows.
\begin{verbatim}
let task1 : Task.task = None (* empty task *)
let goal_id1 : Decl.prsymbol =
  Decl.create_prsymbol (Ident.id_fresh "goal1")
let task1 : Task.task =
  Task.add_prop_decl task1 Decl.Pgoal goal_id1 fmla1
\end{verbatim}
To make the formula a goal, we must give a name to it, here "goal1". A
goal name has type \texttt{prsymbol}, for identifiers denoting
propositions in a theory or a task. Notice again that the concrete
syntax of \why\ requires these symbols to be capitalized, but it is not
mandatory when using the library. The second argument of
\texttt{add\_prop\_decl} is the kind of the proposition:
\texttt{Paxiom}, \texttt{Plemma} or \texttt{Pgoal}
(notice, however, that lemmas are not allowed in tasks
and can only be used in theories).


Once a task is built, it can be printed.
\begin{verbatim}
(* printing the task *)
let () = printf "@[task 1 is:@\n%a@]@." Pretty.print_task task1
\end{verbatim}

The task for our second formula is a bit more complex to build, because
the variables A and B must be added as logic declarations in the task.
\begin{verbatim}
(* task for formula 2 *)
let task2 = None
let task2 = Task.add_logic_decl task2 [prop_var_A, None]
let task2 = Task.add_logic_decl task2 [prop_var_B, None]
let goal_id2 = Decl.create_prsymbol (Ident.id_fresh "goal2")
let task2 = Task.add_prop_decl task2 Decl.Pgoal goal_id2 fmla2
let () = printf "@[task 2 is:@\n%a@]@." Pretty.print_task task2
\end{verbatim}
The argument \texttt{None} is the declarations of logic symbols means
that they do not have any definition.

Execution of our OCaml program now outputs:
\begin{verbatim}
task 1 is:
theory Task
  goal Goal1 : true or false
end

task 2 is:
theory Task
  logic A

  logic B

  goal Goal2 : A and B -> A
end
\end{verbatim}

\section{Calling External Provers}

To call an external prover, we need to access the Why configuration
file \texttt{why.conf}, as it was build using the \texttt{why3config}
command line tool or the \textsf{Detect Provers} menu of the graphical
IDE. The following API calls allow to access the content of this
configuration file.
\begin{verbatim}
(* reads the config file *)
let config : Whyconf.config = Whyconf.read_config None
(* the [main] section of the config file *)
let main : Whyconf.main = Whyconf.get_main config
(* all the provers detected, from the config file *)
let provers : Whyconf.config_prover Util.Mstr.t =
  Whyconf.get_provers config
\end{verbatim}
The type \texttt{'a Util.Mstr.t} is a map indexed by strings. This map
can provide the set of existing provers. In the following, we directly
attempt to access the prover Alt-Ergo, which is known to be identified
with id \texttt{"alt-ergo"}.
\begin{verbatim}
(* the [prover alt-ergo] section of the config file *)
let alt_ergo : Whyconf.config_prover =
  try
    Util.Mstr.find "alt-ergo" provers
  with Not_found ->
    eprintf "Prover alt-ergo not installed or not configured@.";
    exit 0
\end{verbatim}

The next step is to obtain the driver associated to this prover. A
driver typically depends on the standard theories so these should be
loaded first.
\begin{verbatim}
(* builds the environment from the [loadpath] *)
let env : Env.env =
  Env.create_env (Lexer.retrieve (Whyconf.loadpath main))
(* loading the Alt-Ergo driver *)
let alt_ergo_driver : Driver.driver =
  Driver.load_driver env alt_ergo.Whyconf.driver
\end{verbatim}

We are now ready to call the prover on the tasks. This is done by a
function call that launches the external executable and waits for its
termination. Here is a simple way to proceed:
\begin{verbatim}
(* calls Alt-Ergo *)
let result1 : Call_provers.prover_result =
  Driver.prove_task ~command:alt_ergo.Whyconf.command
    alt_ergo_driver task1 () ()
(* prints Alt-Ergo answer *)
let () = printf "@[On task 1, alt-ergo answers %a@]@."
  Call_provers.print_prover_result result1
\end{verbatim}
This way to call a prover is in general too naive, since it may never
return if the prover runs without time limit. The function
\texttt{prove\_task} has two optional parameters: \texttt{timelimit}
is the maximum allowed running time in seconds, and \texttt{memlimit}
is the maximum allowed memory in megabytes.  The type
\texttt{prover\_result} is a record with three fields:
\begin{itemize}
\item \texttt{pr\_answer}: the prover answer, explained below;
\item \texttt{pr\_output}: the output of the prover, i.e. both
  standard output and the standard error of the process
  (a redirection in \texttt{why.conf} is required);
\item \texttt{pr\_time} : the time taken by the prover, in seconds.
\end{itemize}
A \texttt{pr\_answer} is a sum of several kind of answers:
\begin{itemize}
\item \texttt{Valid}: the task is valid according to the prover.
\item \texttt{Invalid}: the task is invalid.
\item \texttt{Timeout}: the prover exceeds the time or memory limit.
\item \texttt{Unknown} $msg$: the prover can't determine if the task
  is valid; the string parameter $msg$ indicates some extra
  information.
\item \texttt{Failure} $msg$: the prover reports a failure, i.e.~it
  was unable to read correctly its input task.
\item \texttt{HighFailure}: an error occurred while trying to call the
  prover, or the prover answer was not understood (i.e.~none of the
  given regular expressions in the driver file matches the output
  of the prover).
\end{itemize}
Here is thus another way of calling the Alt-Ergo prover, on our second
task.
\begin{verbatim}
let result2 : Call_provers.prover_result =
  Driver.prove_task ~command:alt_ergo.Whyconf.command
    ~timelimit:10
    alt_ergo_driver task2 () ()

let () =
  printf "@[On task 2, alt-ergo answers %a in %5.2f seconds@."
    Call_provers.print_prover_answer
    result1.Call_provers.pr_answer
    result1.Call_provers.pr_time
\end{verbatim}
The output of our program is now as follows.
\begin{verbatim}
On task 1, alt-ergo answers Valid (0.01s)
On task 2, alt-ergo answers Valid in  0.01 seconds
\end{verbatim}

\section{Building Terms}

An important feature of the functions for building terms and formulas
is that they statically guarantee that only well-typed terms can be
constructed.

Here is the way we build the formula $2+2=4$. The main difficulty is to
access the internal identifier for addition: it must be retrieved from
the standard theory \texttt{Int} of the file \texttt{int.why} (see
Chap~\ref{chap:library}).
\begin{verbatim}
let two : Term.term = Term.t_const (Term.ConstInt "2")
let four : Term.term = Term.t_const (Term.ConstInt "4")
let int_theory : Theory.theory =
  Env.find_theory env ["int"] "Int"
let plus_symbol : Term.lsymbol =
  Theory.ns_find_ls int_theory.Theory.th_export ["infix +"]
let two_plus_two : Term.term =
  Term.t_app_infer plus_symbol [two;two]
let fmla3 : Term.fmla = Term.f_equ two_plus_two four
\end{verbatim}
An important point to notice as that when building the application of
$+$ to the arguments, it is checked that the types are correct. Indeed
the constructor \texttt{t\_app\_infer} infers the type of the resulting
term. One could also provide the expected type as follows.
\begin{verbatim}
let two_plus_two : Term.term =
  Term.t_app plus_symbol [two;two] Ty.ty_int
\end{verbatim}

When building a task with this formula, we need to declare that we use
theory \texttt{Int}:
\begin{verbatim}
let task3 = None
let task3 = Task.use_export task3 int_theory
let goal_id3 = Decl.create_prsymbol (Ident.id_fresh "goal3")
let task3 = Task.add_prop_decl task3 Decl.Pgoal goal_id3 fmla3
\end{verbatim}

\section{Building Quantified Formulas}

To illustrate how to build quantified formulas, let us consider
the formula $\forall x:int. x*x \geq 0$. The first step is to
obtain the symbols from \texttt{Int}.
\begin{verbatim}
let zero : Term.term = Term.t_const (Term.ConstInt "0")
let mult_symbol : Term.lsymbol =
  Theory.ns_find_ls int_theory.Theory.th_export ["infix *"]
let ge_symbol : Term.lsymbol =
  Theory.ns_find_ls int_theory.Theory.th_export ["infix >="]
\end{verbatim}
The next step is to introduce the variable $x$ with the type int.
\begin{verbatim}
let var_x : Term.vsymbol =
  Term.create_vsymbol (Ident.id_fresh "x") Ty.ty_int
\end{verbatim}
The formula $x*x \geq 0$ is obtained as in the previous example.
\begin{verbatim}
let x : Term.term = Term.t_var var_x
let x_times_x : Term.term = Term.t_app_infer mult_symbol [x;x]
let fmla4_aux : Term.fmla = Term.f_app ge_symbol [x_times_x;zero]
\end{verbatim}
To quantify on $x$, one can first build an intermediate
value of type \texttt{fmla\_quant}, representing a closure
under a quantifier:
\begin{verbatim}
let fmla4_quant : Term.fmla_quant = Term.f_close_quant [var_x] [] fmla4_aux
let fmla4 : Term.fmla = Term.f_forall fmla4_quant
\end{verbatim}
The second argument of \texttt{f\_close\_quant} is a list of triggers.

A simpler method would be to use an appropriate function:
\begin{verbatim}
let fmla4bis : Term.fmla = Term.f_forall_close [var_x] [] fmla4_aux
\end{verbatim}

\section{Building Theories}

[TO BE COMPLETED]

\section{Applying transformations}

[TO BE COMPLETED]

\section{Writing new functions on term}

[TO BE COMPLETED]
% pattern-matching on terms, opening a quantifier




%%% Local Variables:
%%% mode: latex
%%% TeX-PDF-mode: t
%%% TeX-master: "manual"
%%% End:
