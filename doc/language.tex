\documentclass[a4paper,11pt,twoside,openright]{memoir}

% rubber: module index

%BEGIN LATEX
\usepackage{comment,todonotes}
\newcommand{\ahref}[2]{{#2}}
\excludecomment{htmlonly}
\newenvironment{latexonly}{}{}
%END LATEX

%HEVEA\@addimagenopt{-pdf}

%BEGIN LATEX
% tells memoir style to number subsections
\setsecnumdepth{subsection}
%END LATEX

\usepackage[T1]{fontenc}
\usepackage{lmodern}
%\usepackage{url}
\usepackage[pdftex,colorlinks=true,urlcolor=blue,pdfstartview=FitH]{hyperref}

%BEGIN LATEX
\usepackage{graphicx}
%END LATEX
%HEVEA \newcommand{\includegraphics}[2][2]{\imgsrc{#2}}

\usepackage{listings}
\usepackage{xspace}

%BEGIN LATEX
\setulmarginsandblock{30mm}{30mm}{*}
\setlrmarginsandblock{30mm}{30mm}{*}
\setheadfoot{15pt}{38pt}
\checkandfixthelayout

% placement of figures
\renewcommand{\textfraction}{0.01}
\renewcommand{\topfraction}{0.99}
\renewcommand{\bottomfraction}{0.99}
%END LATEX
\setcounter{topnumber}{4}
\setcounter{bottomnumber}{4}
\setcounter{totalnumber}{8}

%HEVEA \newstyle{table.lstframe}{width:100\%;border-width:1px;}

% \usepackage[toc,nonumberlist]{glossaries}
% \makeglossaries

% \usepackage{glossary}
% \makeglossary
% \glossary{name={entry name}, description={entry description}}

% for ocamldoc generated pages
%\usepackage{ocamldoc}
%\let\tt\ttfamily
%\let\bf\bfseries

\usepackage{ifthen}

%\newcommand{\todo}[1]{{\Huge\bfseries TODO: #1}}

\newcommand{\why}{\textsf{Why3}\xspace}
\newcommand{\whyml}{\textsf{WhyML}\xspace}
\newcommand{\eg}{\emph{e.g.}\xspace}
\newcommand{\ie}{\emph{i.e.}\xspace}
\newcommand{\opam}{\textsf{OPAM}\xspace}

\newcommand{\emptyitem}{%
%BEGIN LATEX
~
%END LATEX
}

% BNF grammar
\newcommand{\keyword}[1]{\texttt{#1}}
\newcommand{\indextt}[1]{\index{#1@\protect\keyword{#1}}}
\newcommand{\indexttbs}[1]{\index{#1@\protect\keywordbs{#1}}}
\newif\ifspace
\newif\ifnewentry
\newcommand{\addspace}{\ifspace \: \spacefalse \fi}
\newcommand{\term}[2]{\addspace\mbox{\lstinline|#1|%
\ifthenelse{\equal{#2}{}}{}{\indexttbase{#2}{#1}}}\spacetrue}
\newcommand{\nonterm}[2]{%
  \ifthenelse{\equal{#2}{}}%
             {\addspace\mbox{\textsl{#1}\ifnewentry\index{#1@\textsl{#1}}\fi}\spacetrue}%
             {\addspace\mbox{\textsl{#1}\footnote{#2}\ifnewentry\index{grammar entries!\textsl{#1}}\fi}\spacetrue}}
\newcommand{\repetstar}{$^*$\spacetrue}
\newcommand{\repetplus}{$^+$\spacetrue}
\newcommand{\repetone}{$^?$\spacetrue}
\newcommand{\lparen}{\addspace(}
\newcommand{\rparen}{)}
\newcommand{\orelse}{\addspace$\mid$\spacetrue}
\newcommand{\sep}{ \\[2mm] \spacefalse\newentrytrue}
\newcommand{\newl}{ \\ & & \spacefalse}
\newcommand{\alt}{ \\ & $\mid$ & \spacefalse}
\newcommand{\is}{ & $::=$ & \spacefalse\newentryfalse}
\newenvironment{syntax}{\begin{tabular}{@{}rrll@{}}\spacefalse\newentrytrue}{\end{tabular}}
\newcommand{\synt}[1]{$\spacefalse#1$}
\newcommand{\emptystring}{$\epsilon$}
\newcommand{\below}{See\; below}

%%% listings for Why3 %%%%%%%%%%%%%%%%%%%%%%%%%%%%%%%%%%%%%%%%%%%%%%%%


% cannot use \usepackage here because it breaks hevea (no colored keywords anymore :-()

\RequirePackage{listings}
\RequirePackage{amssymb}

\lstdefinelanguage{why3}
{
basicstyle=\ttfamily,%
morekeywords=[1]{abstract,absurd,any,assert,assume,axiom,by,%
check,clone,coinductive,constant,diverges,do,done,downto,%
else,end,ensures,exception,exists,export,for,forall,fun,%
function,ghost,goal,if,import,in,inductive,invariant,lemma,%
let,loop,match,meta,model,module,mutable,namespace,not,old,%
predicate,private,raise,raises,reads,rec,requires,result,%
returns,so,then,theory,to,try,type,use,val,variant,while,%
with,writes},%
string=[b]",%
%keywordstyle=[1]{\color{red}},%
morekeywords=[2]{true,false},%
%keywordstyle=[2]{\color{blue}},%
otherkeywords={},%
commentstyle=\itshape,%
columns=[l]fullflexible,%
sensitive=true,%
morecomment=[s]{(*}{*)},%
escapeinside={*?}{?*},%
keepspaces=true,%
literate=%
% {'a}{$\alpha$}{1}%
% {'b}{$\beta$}{1}%
% {<}{$<$}{1}%
% {>}{$>$}{1}%
% {<=}{$\le$}{1}%
% {>=}{$\ge$}{1}%
% {<>}{$\ne$}{1}%
% {/\\}{$\land$}{1}%
% {\\/}{ $\lor$ }{3}%
% {\ or(}{ $\lor$(}{3}%
% {not\ }{$\lnot$ }{1}%
% {not(}{$\lnot$(}{1}%
% {+->}{\texttt{+->}}{2}%
% {+->}{$\mapsto$}{2}%
% {-->}{\texttt{-\relax->}}{2}%
% {-->}{$\longrightarrow$}{2}%
% {->}{$\rightarrow$}{2}%
% {<-}{$\leftarrow$}{2}%
% {<->}{$\leftrightarrow$}{2}%
%
%
}

\lstnewenvironment{why3}{\lstset{language=why3}}{}

\newcommand{\whyf}[1]{\lstinline[language=why3]{#1}}


% \RequirePackage{listings}
% \RequirePackage{amssymb}

\lstset{
  basicstyle={\ttfamily},
  framesep=2pt,
  frame=single,
  keywordstyle={\color{blue}},
  stringstyle=\itshape,
  commentstyle=\itshape,
  columns=[l]fullflexible,
  showstringspaces=false,
}

\lstnewenvironment{whycode}{\lstset{language=why3}}{}
\lstnewenvironment{ocamlcode}{\lstset{language={[Objective]Caml}}}{}

%%% Local Variables:
%%% mode: latex
%%% TeX-master: "manual"
%%% TeX-PDF-mode: t
%%% End:

\input{./version.tex}
\let\of\texttt

\makeindex

%HEVEA\title{}

\begin{document}
\sloppy
%BEGIN LATEX
\hbadness=5000
%END LATEX

\thispagestyle{empty}

\begin{center}

%BEGIN LATEX
\rule\textwidth{0.8mm}
%END LATEX

\vfill

{
%BEGIN LATEX
\fontsize{40}{40pt}\selectfont
%END LATEX
%HEVEA \Huge
\bfseries\sffamily WhyML \\[1em] Syntax and Semantics}

\vfill

%BEGIN LATEX
\rule\textwidth{0.8mm}
%END LATEX

\vfill

% \todo{NE PAS DISTRIBUER TANT QU'IL RESTE DES TODOS}

%BEGIN LATEX
\begin{LARGE}
%END LATEX
  Version \whyversion{}
%BEGIN LATEX
\end{LARGE}
%END LATEX

\vfill

%BEGIN LATEX
\begin{Large}
%END LATEX
  \begin{tabular}{c}
  Fran\c{c}ois Bobot$^{1,2}$ \\
  Jean-Christophe Filli\^atre$^{1,2}$  \\
  Claude March\'e$^{2,1}$ \\
  Guillaume Melquiond$^{2,1}$\\
  Andrei Paskevich$^{1,2}$
\end{tabular}
%BEGIN LATEX
\end{Large}
%END LATEX
\vfill

\begin{flushleft}

\begin{tabular}{l}
$^1$ LRI, CNRS \& University Paris-Sud, Orsay, F-91405 \\
$^2$ Inria Saclay -- \^Ile-de-France, Palaiseau, F-91120
\end{tabular}

%BEGIN LATEX
\bigskip
%END LATEX

  \textcopyright 2010-2014 University Paris-Sud, CNRS, Inria

\urldef{\urlutcat}{\url}{http://frama-c.com/u3cat/}
\urldef{\urlhilite}{\url}{http://www.open-do.org/projects/hi-lite/}
\urldef{\urlbware}{\url}{http://bware.lri.fr/}

This work has been partly supported by the `\ahref{\urlutcat}{U3CAT}'
national ANR project (ANR-08-SEGI-021-08\begin{latexonly},
  \urlutcat\end{latexonly}), the `\ahref{\urlhilite}{Hi-Lite}'
\begin{latexonly}(\urlhilite)\end{latexonly} FUI project of the
System@tic competitivity cluster, and the `\ahref{\urlbware}{BWare}'
ANR project (ANR-12-INSE-0010\begin{latexonly},
  \urlbware\end{latexonly}).

\end{flushleft}
\end{center}

\newpage
\tableofcontents

\chapter{Lexical conventions}

\whyml\ assumes a uni-byte encoding of input files, such as ASCII or
Latin-1.

\paragraph{Comments.}
Comments are enclosed by \texttt{(*} and \texttt{*)} and can be nested.
The sequence of characters \of{(*)} is not considered as the beginning
of a comment.

\paragraph{Strings.}
Strings are enclosed in double quotes (\verb!"!). Double quotes can be
inserted in strings using the backslash character (\verb!\!).
In the following, strings are referred to with the non-terminal
\nonterm{string}{}.

\todo{escape sequences for strings}

\paragraph{Identifiers.} The syntax distinguishes lowercase and
uppercase identifiers and, similarly, lowercase and uppercase
qualified identifiers.

\begin{center}\framebox{\input{./generated/qualid_bnf.tex}}\end{center}

\paragraph{Constants.}
The syntax for constants is given in Figure~\ref{fig:bnf:constant}.
Integer and real constants have arbitrary precision.
Integer constants may be given in base 16, 10, 8 or 2.
Real constants may be given in base 16 or 10.

\begin{figure}
\begin{center}\framebox{\input{./generated/constant_bnf.tex}}\end{center}
  \caption{Syntax for constants.}
\label{fig:bnf:constant}
\end{figure}

\paragraph{Operators.}
Prefix and infix operators are built from characters organized in four
categories (\textsl{op-char-1} to \textsl{op-char-4}).
\begin{center}\framebox{\input{./generated/operator_bnf.tex}}\end{center}
Infix operators are classified into 4 categories, according to the
characters they are built from:
\begin{itemize}
\item level 4: operators containing only characters from
\textit{op-char-4};
\item level 3: those containing
 characters from \textit{op-char-3} or \textit{op-char-4};
\item level 2: those containing
 characters from \textit{op-char-2}, \textit{op-char-3} or
 \textit{op-char-4};
\item level 1: all other operators (non-terminal \textit{infix-op-1}).
\end{itemize}

\paragraph{Labels.} Identifiers, terms, formulas, program expressions
can all be labeled, either with a string, or with a location tag.
\begin{center}\framebox{\input{./generated/label_bnf.tex}}\end{center}
A location tag consists of a file name, a line number, and starting
and ending characters.

\chapter{}

\bibliographystyle{abbrv}
\bibliography{manual}
%\input{biblio-demons}

\cleardoublepage
\listoffigures
\cleardoublepage
\printindex

\end{document}

%%% Local Variables:
%%% mode: latex
%%% TeX-PDF-mode: t
%%% TeX-master: t
%%% compile-command: "rubber -d language"
%%% ispell-local-dictionary: "american"
%%% End:
