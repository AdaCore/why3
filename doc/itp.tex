
\chapter{Interactive Proof Assistants}


% ... We then provide specific information about some ITPs.

\section{Using an Interactive Proof Assistant to Discharge Goals}

Instead of calling an automated theorem prover to discharge a goal,
\why offers the possibility to call an interactive theorem prover
instead. In that case, the interaction is decomposed into two distinct
phases:
\begin{itemize}
\item Edition of a proof script for the goal, typically inside a proof editor
  provided by the external interactive theorem prover;
\item Replay of an existing proof script.
\end{itemize}
An example of such an interaction is given in the tutorial
section~\ref{sec:gui}.

Some proof assistants offer more than one possible editor, e.g. a
choice between the use of a dedicated editor and the use of the Emacs
editor and the ProofGeneral mode. Selection of the preferred mode can
be made in \texttt{why3ide} preferences, under the ``Editors'' tab.

\section{Theory Realizations}
\label{sec:realizations}

Given a \why theory, one can use a proof assistant to make a
\emph{realization} of this theory, that is to provide definitions for
some of its uninterpreted symbols and proofs for some of its
axioms. This way, one can show the consistency of an axiomatized
theory and/or make a connection to an existing library (of the proof
assistant) to ease some proofs.
%Currently, realizations are supported for the proof assistants Coq and PVS.

\subsection{Generating a realization}

Generating the skeleton for a theory is done by passing to the
\texttt{realize} command a driver suitable for realizations, the names of
the theories to realize, and a target directory.
\index{realize@\texttt{realize}}

\begin{verbatim}
why3 realize -D path/to/drivers/prover-realize.drv
             -T env_path.theory_name -o path/to/target/dir/
\end{verbatim}
\index{driver@\verb+--driver+}
\index{theory@\verb+--theory+}

The theory is looked into the files from the environment, \eg the standard
library. If the theory is stored in a different location, option \texttt{-L}
should be used.

The name of the generated file is inferred from the theory name. If the
target directory already contains a file with the same name, \why
extracts all the parts that it assumes to be user-edited and merges them in
the generated file.

Note that \why does not track dependencies between realizations and
theories, so a realization will become outdated if the corresponding
theory is modified.
It is up to the user to handle such dependencies, for instance using a
\texttt{Makefile}.

\subsection{Using realizations inside proofs}

If a theory has been realized, the \why printer for the corresponding prover
will no longer output declarations for that theory but instead simply put
a directive to load the realization. In order to tell the printer
that a given theory is realized, one has to add a meta declaration in the
corresponding theory section of the driver.
\index{driver file}

\begin{verbatim}
theory env_path.theory_name
  meta "realized_theory" "env_path.theory_name", "optional_naming"
end
\end{verbatim}

The first parameter is the theory name for \why. The second
parameter, if not empty, provides a name to be used inside generated
scripts to point to the realization, in case the default name is not
suitable for the interactive prover.
\index{realized_theory@\verb+realized_theory+}

\subsection{Shipping libraries of realizations}

While modifying an existing driver file might be sufficient for local
use, it does not scale well when the realizations are to be shipped to
other users. Instead, one should create two additional files: a
configuration file that indicates how to modify paths, provers, and
editors, and a driver file that contains only the needed
\verb+meta "realized_theory"+ declarations. The configuration file should be as
follows.
\index{configuration file}

\begin{verbatim}
[main]
loadpath="path/to/theories"

[prover_modifiers]
name="Coq"
option="-R path/to/vo/files Logical_directory"
driver="path/to/file/with/meta.drv"

[editor_modifiers coqide]
option="-R path/to/vo/files Logical_directory"

[editor_modifiers proofgeneral-coq]
option="--eval \"(setq coq-load-path (cons '(\\\"path/to/vo/files\\\" \
  \\\"Logical_directory\\\") coq-load-path))\""
\end{verbatim}

This configuration file can be passed to \why thanks to the
\verb+--extra-config+ option.
\index{extra-config@\verb+--extra-config+}
\index{prover_modifiers@\verb+prover_modifiers+}
\index{editor_modifiers@\verb+editor_modifiers+}
\index{option@\verb+option+}
\index{driver@\verb+driver+}



\section{Coq}
\label{sec:coq}
\index{Coq proof assistant}

This section describes the content of the Coq files generated by \why for
both proof obligations and theory realizations. When reading a Coq
script, \why is guided by the presence of empty lines to split the
script, so the user should refrain from removing empty lines around
generated parts or adding empty lines inside them.

\begin{enumerate}
\item	The header of the file contains all the library inclusions
	required by the driver file. Any user-made changes to this part
	will be lost when the file is regenerated by \why. This part ends
	at the first empty line.
\item	Abstract logic symbols are assumed with the vernacular directive
	\verb+Parameter+. Axioms are assumed with the \verb+Axiom+
	directive. When regenerating a script, \why assumes that all such
	symbols have been generated by a previous run. As a consequence,
	the user should not introduce new symbols with these two
	directives, as they would be lost.
\item	Definitions of functions and inductive types in theories are
	printed in a block that starts with \verb+(* Why3 assumption *)+.
	This comment should not be removed; otherwise \why will assume
	that the definition is user-made.
\item	Finally, proof obligations and symbols to be realized are
	introduced by \verb+(* Why3 goal *)+. The user is supposed to
	fill the script after the statement. \why assumes that the
	user-made part extends up to \verb+Qed+, \verb+Admitted+,
	\verb+Save+, or \verb+Defined+, whichever comes first. In the
	case of definitions, the original statement can be replaced by
	a \verb+Notation+ directive, in order to ease the usage of
	already defined symbols. \why also recognizes \verb+Variable+
	and \verb+Hypothesis+ and preserves them; they should be used in
	conjunction with Coq's \verb+Section+ mechanism to realize
	theories that still need some abstract symbols and axioms.
\end{enumerate}

Currently, the parser for Coq scripts is rather naive and does not know
much about comments. For instance, \why can easily be confused by
some terminating directive like \verb+Qed+ that would be present in a
comment.


%%% Local Variables:
%%% mode: latex
%%% TeX-PDF-mode: t
%%% TeX-master: "manual"
%%% End:


\chapter{Coq Tactic}
\label{chap:tactic}
\index{Coq proof assistant}

\why\ provides a Coq tactic to call external theorem provers as oracles.

\section{Installation}

You need Coq version 8.3 or greater. If this is the case, \why's
configuration detects it, then compiles and installs the Coq tactic.
The Coq tactic is installed in
\begin{center}
  \textit{why3-lib-dir}\texttt{/coq-tactic/}
\end{center}
where \textit{why3-lib-dir} is \why's library directory, as reported
by \verb+why3 --print-libdir+. This directory
is automatically added to Coq's load path if you are
calling Coq via \why (from \texttt{why3ide}, \texttt{why3replayer},
etc.). If you are calling Coq by yourself, you need to add
this directory to Coq's load path, either using Coq's command line
option \texttt{-I} or by adding
\begin{center}
  \verb+Add LoadPath "+\textit{why3-lib-dir}\verb+/coq-tactic/".+
\end{center}
to your \texttt{\~{}/.coqrc} resource file.

\section{Usage}

The Coq tactic is called \texttt{why3} and is used as follows:
\begin{center}
  \texttt{why3} \verb+"+\textit{prover-name}\verb+"+
  $[$\texttt{timelimit} \textit{n}$]$.
\end{center}
The string \textit{prover-name} identifies one of the automated theorem provers
supported by \why, as reported by \verb+why3 --list-provers+
(interactive provers excluded).
The current goal is then translated to \why's logic and the prover is
called. If it reports the goal to be valid, then Coq's \texttt{admit}
tactic is used to assume the goal. The prover is called with a time
limit in seconds as given by \why's configuration file
(see page~\pageref{sec:whyconffile}). A different value may be given
using the \texttt{timelimit} keyword.

\paragraph{Error messages.} The following errors may be reported by
the Coq tactic.
\begin{description}
\item[\texttt{Not a first order goal}]\emptyitem
  The Coq goal could not be translated to \why's logic.
\item[\texttt{Timeout}]\emptyitem
  There was no answer from the prover within the given time limit.
\item[\texttt{Don't know}]\emptyitem
  The prover stopped without validating the goal.
\item[\texttt{Invalid}]\emptyitem
  The prover stopped, reporting the goal to be invalid.
\item[\texttt{Failure}]\emptyitem
  The prover failed. Depending on the message that follows, you may
  want to file a bug report, either to the \why\ developers or to the
  prover developers.
\end{description}

%%% Local Variables:
%%% mode: latex
%%% compile-command: "make -C .. doc"
%%% TeX-PDF-mode: t
%%% TeX-master: "manual"
%%% End:


\section{Isabelle/HOL}
\label{sec:isabelle}

\index{Isabelle proof assistant}

When using Isabelle from \why, files generated from \why theories and
goals are stored in a dedicated XML format. Those files should not be
edited. Instead, the proofs must be completed in a file with the same
name and extension \texttt{.thy}. This is the file that is opened when
using ``Edit'' action in \texttt{why3ide}.

\subsection{Installation}

You need version Isabelle2015 or Isabelle2016. Former versions are not
supported. We assume below that your version is 2016, please replace
2016 by 2015 otherwise.

Isabelle must be installed before compiling \why. After compilation
and installation of \why, you must manually add the path
\begin{verbatim}
<Why3 lib dir>/isabelle
\end{verbatim}
into either the user file
\begin{verbatim}
.isabelle/Isabelle2016/etc/components
\end{verbatim}
or the system-wide file
\begin{verbatim}
<Isabelle install dir>/etc/components
\end{verbatim}

\subsection{Usage}

The most convenient way to call Isabelle for discharging a \why goal
is to start the Isabelle/jedit interface in server mode. In this mode,
one must start the server once, before launching \texttt{why3ide},
using
\begin{verbatim}
isabelle why3_jedit
\end{verbatim}
Then, inside a \texttt{why3ide} session, any use of ``Edit'' will
transfer the file to the already opened instance of jEdit. When the
proof is completed, the user must send back the edited proof to
\texttt{why3ide} by closing the opened buffer, typically by hitting
\texttt{Ctrl-w}.

\subsection{Realizations}

Realizations must be designed in some \texttt{.thy} as follows.
The realization file corresponding to some \why file \texttt{f.why}
should have the following form.
\begin{verbatim}
theory Why3_f
imports Why3_Setup
begin

section {* realization of theory T *}

why3_open "f/T.xml"

why3_vc <some lemma>
<proof>

why3_vc <some other lemma> by proof

[...]

why3_end
\end{verbatim}

See directory \texttt{lib/isabelle} for examples.


%%% Local Variables:
%%% mode: latex
%%% TeX-PDF-mode: t
%%% TeX-master: "manual"
%%% End:


\subsection{PVS}

When a PVS file is regenerated, the old version is split into chunks,
according to blank lines. Chunks corresponding to \why declarations
are identified with a comment starting with \verb+% Why3+, \eg
\begin{verbatim}
  % Why3 f
  f(x: int) : int
\end{verbatim}
Other chunks are considered to be user PVS declarations.
Thus a comment such as \verb+% Why3 f+ must not be removed;
otherwise, there will be two
declarations for \texttt{f} in the next version of the file (one being
regenerated and another one considered to be a user-edited chunk).

The user is allowed to perform the following actions on a PVS
realization:
\begin{itemize}
\item give a definition to an uninterpreted symbol (type, function, or
  predicate symbol), by adding an equal sign (\texttt{=}) and a
  right-hand side to the definition. When the declaration is
  regenerated, the left-hand side is updated and the right-hand side
  is reprinted as is. In particular, the names of a function or
  predicate arguments should not be modified. In addition, the
  \texttt{MACRO} keyword may be inserted and it will be kept in
  further generations.

\item turn an axiom into a lemma, that is to replace the PVS keyword
  \texttt{AXIOM} with either \texttt{LEMMA} or \texttt{THEOREM}.

\item insert anything between generated declarations, such as a lemma,
  an extra definition for the purpose of a proof, an extra
  \texttt{IMPORTING} command, etc. Do not forget to surround these
  extra declarations with blank lines.
\end{itemize}
\why makes some effort to merge new declarations with old ones
and with user chunks. If it happens that some chunks could not be
merged, they are appended at the end of the file, in comments.



%%% Local Variables:
%%% mode: latex
%%% TeX-PDF-mode: t
%%% TeX-master: "manual"
%%% End:
