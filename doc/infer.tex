
\chapter{Inference of Loop Invariants}
\label{chp:infer-loop}

This chapter shows how to install and use \texttt{loop-infer}, an
utility based in \emph{abstract interpretation} to infer loop
invariants. This is still work in progress and many features are still
not covered.

\section{Availability}

The \texttt{loop-infer} utility has the following OCaml dependencies.
%
\begin{itemize}
\item \texttt{apron}: can be installed using \opam.
\item \texttt{camllib}: required by \texttt{fixpoint} and can also be
  installed using \opam.
\item \texttt{fixpoint}: follow instructions below.
\end{itemize}

The \texttt{fixpoint} library is not available in \opam. To install it
one needs to download the sources and install it. The following
instructions can be performed in any directory.
%
\begin{verbatim}
> svn co svn://scm.gforge.inria.fr/svnroot/bjeannet/pkg/fixpoint
> cd fixpoint/trunk/
> cp Makefile.config.model Makefile.config
> # if required make modifications to Makefile.config
> make all     # compiles
> make install # uses ocamlfind to install the library
\end{verbatim}

Once the dependencies are installed, the configuration script of \why
should enable the compilation of the \texttt{loop-infer} utility when
the flag \texttt{--enable-infer} is passed.

\begin{verbatim}
> ./configure --enable-infer
# ...
Components
    Invariant inference     : yes
# ...
\end{verbatim}

\section{Running it}

\todo{check whether we generate a tool or only support inference of
  loop invariants through the API}

Why3 compilation should generate a new why3 subcommand 'infer'

run it with
\begin{verbatim}
why3 infer <file>.mlw
\end{verbatim}

it produces a file \verb|<file>.mlw_inferred.mlw| where the invariants found are inserted


For information about how to use the API to infer loop invariants
refer to Section~\ref{sec:infer-loop-api}.

%%% Local Variables:
%%% mode: latex
%%% TeX-PDF-mode: t
%%% TeX-master: "manual"
%%% End: