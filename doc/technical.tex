\chapter{Technical Informations}


\section{Structure of Session Files}

The proof session state is stored in an XML file named
\texttt{\textsl{<dir>}/why3session.xml}, where \texttt{\textsl{<dir>}}
is the directory of the project.
The XML file follows the DTD given in \texttt{share/why3session.dtd} and reproduced below.
\lstinputlisting{../share/why3session.dtd}


\section{Prover Detection}
\label{sec:proverdetecttiondata}

All the necessary data configuration for the automatic detection of
installed provers is stored in the file
\texttt{provers-detection-data.conf} typically located in directory
\verb|/usr/local/share/why3| after installation. The contents of this
file is reproduced below.
%BEGIN LATEX
{\footnotesize
%END LATEX
\lstinputlisting{../share/provers-detection-data.conf}
%BEGIN LATEX
}
%END LATEX

\section{The \texttt{why3.conf} Configuration File}
\label{sec:whyconffile}
\index{why3.conf@\texttt{why3.conf}}\index{configuration file}

One can use a custom configuration file. The \why
tools look for it in the following order:
\begin{enumerate}
\item the file specified by the \texttt{-C} or \texttt{-{}-config} options,
\item the file specified by the environment variable
  \texttt{WHY3CONFIG} if set,
\item the file \texttt{\$HOME/.why3.conf}
  (\texttt{\$USERPROFILE/.why3.conf} under Windows) or, in the case of
  local installation, \texttt{why3.conf} in the top directory of \why sources.
\end{enumerate}
If none of these files exist, a built-in default configuration is used.

The configuration file is a human-readable text file, which consists
of association pairs arranged in sections.
%BEGIN LATEX
Figure~\ref{fig:why3conf} shows an example of configuration file.
%END LATEX
%HEVEA Below is an example of configuration file.

%BEGIN LATEX
\begin{figure}[p]
{\footnotesize
%END LATEX
\begin{verbatim}
[main]
loadpath = "/usr/local/share/why3/theories"
loadpath = "/usr/local/share/why3/modules"
magic = 14
memlimit = 0
plugin = "/usr/local/lib/why3/plugins/tptp"
plugin = "/usr/local/lib/why3/plugins/genequlin"
plugin = "/usr/local/lib/why3/plugins/hypothesis_selection"
running_provers_max = 4
timelimit = 2

[ide]
default_editor = "editor %f"
error_color = "orange"
goal_color = "gold"
iconset = "fatcow"
intro_premises = true
premise_color = "chartreuse"
print_labels = false
print_locs = false
print_time_limit = false
saving_policy = 2
task_height = 404
tree_width = 512
verbose = 0
window_height = 1173
window_width = 1024

[prover]
command = "'why3-cpulimit' 0 %m -s coqtop -batch -I %l/coq-tactic -R %l/coq Why3 -l %f"
driver = "/usr/local/share/why3/drivers/coq.drv"
editor = "coqide"
in_place = false
interactive = true
name = "Coq"
shortcut = "coq"
version = "8.3pl4"

[prover]
command = "'why3-cpulimit' %t %m -s alt-ergo %f"
driver = "/usr/local/share/why3/drivers/alt_ergo_0.93.drv"
editor = ""
in_place = false
interactive = false
name = "Alt-Ergo"
shortcut = "altergo"
shortcut = "alt-ergo"
version = "0.93.1"

[editor coqide]
command = "coqide -I %l/coq-tactic -R %l/coq Why3 %f"
name = "CoqIDE"
\end{verbatim}
%BEGIN LATEX
}
\caption{Sample \texttt{why3.conf} file}
\label{fig:why3conf}
\end{figure}
%END LATEX

A section begins with a header inside square brackets and ends at the
beginning of the next section. The header of a
section can be only one identifier, \texttt{main} and \texttt{ide} in
the example, or it can be composed by a family name and one family
argument, \texttt{prover} is one family name, \texttt{coq} and
\texttt{alt-ergo} are the family argument.

Sections contain associations \texttt{key=value}. A value is either
an integer (\eg \texttt{-555}), a boolean (\texttt{true}, \texttt{false}),
or a string (\eg \texttt{"emacs"}). Some specific keys can be attributed
multiple values and are
thus allowed to occur several times inside a given section. In that
case, the relative order of these associations matter.

\section{Drivers for External Provers}
\label{sec:drivers}

Drivers for external provers are readable files from directory
\texttt{drivers}. Experimented users can modify them to change the way
the external provers are called, in particular which transformations
are applied to goals.

[TO BE COMPLETED LATER]

\section{Transformations}
\label{sec:transformations}

This section documents the available transformations. We first
describe the most important ones, and then we provide a quick
documentation of the others, first the non-splitting ones, \eg those
which produce exactly one goal as result, and the others which produce any
number of goals.

Notice that the set of available transformations in your own
installation is given by
\begin{verbatim}
why3 --list-transforms
\end{verbatim}
\index{list-transforms@\verb+--list-transforms+}

\subsection{Inlining definitions}

Those transformations generally amount to replace some applications of
function or predicate symbols with its definition.

\begin{description}

\item[inline\_trivial]
  expands and removes definitions of the form
\begin{whycode}
function  f x_1 ... x_n = (g e_1 ... e_k)
predicate p x_1 ... x_n = (q e_1 ... e_k)
\end{whycode}
when each $e_i$ is either a ground term or one of the $x_j$, and
each $x_1 \dots x_n$ occurs at most once in all the $e_i$.
\index{inline-trivial@\verb+inline_trivial+}

\item[inline\_goal] expands all outermost symbols of the goal that
  have a non-recursive definition.
\index{inline-goal@\verb+inline_goal+}

\item[inline\_all]
  expands all non-recursive definitions.
\index{inline-all@\verb+inline_all+}

\end{description}


\subsection{Induction Transformations}

\begin{description}
\item[induction\_ty\_lex]:
  \index{induction-ty-lex@\verb+induction_ty_lex+}
  This transformation performs structural, lexicographic induction on
  goals involving universally quantified variables of algebraic data
  types, such as lists, trees, etc. For instance, it transforms the
  following goal
\begin{whycode}
goal G: forall l: list 'a. length l >= 0
\end{whycode}
  into this one:
\begin{whycode}
goal G :
  forall l:list 'a.
     match l with
     | Nil -> length l >= 0
     | Cons a l1 -> length l1 >= 0 -> length l >= 0
     end
\end{whycode}
  When induction can be applied to several variables, the transformation
  picks one heuristically. A label \verb|"induction"| can be used to
  force induction over one particular variable, \eg with
\begin{whycode}
goal G: forall l1 "induction" l2 l3: list 'a.
        l1 ++ (l2 ++ l3) = (l1 ++ l2) ++ l3
\end{whycode}
induction will be applied on \verb|l1|. If this label is attached on
several variables, a lexicographic induction is performed on these
variables, from left to right.

%\item[] Induction on inductive predicates.

%[TO BE COMPLETED]

% TODO: implement also induction on int !

\end{description}

\subsection{Simplification by Computation}

These transformations simplify the goal by applying several kinds of
simplification, described below. The transformations differ only by
the kind of rules they apply:
\begin{description}
\item[compute\_in\_goal] aggressively applies all known
  computation/simplification rules
  \index{compute-in-goal@\verb+compute_in_goal+}

\item[compute\_specified] performs rewriting using only built-in
  operators and user-provided rules
  \index{compute-specified@\verb+compute_specified+}
\end{description}

The kinds of simplification are as follows.
\begin{itemize}
\item Computations with built-in symbols, \eg operations on integers,
  when applied to explicit constants, are evaluated. This includes
  evaluation of equality when a decision can be made (on integer
  constants, on constructors of algebraic data types), Boolean
  evaluation, simplification of pattern-matching/conditional expression,
  extraction of record fields, and beta-reduction.
  At best, these computations reduce the goal to
  \verb|true| and the transformations thus does not produce any sub-goal.
  For example, a goal
  like \verb|6*7=42| is solved by those transformations.
\item Unfolding of definitions, as done by \verb|inline_goal|.
  \verb|compute_in_goal| unfold all definitions, including recursive ones.
  For \verb|compute_specified|, the user can enable unfolding of a specific
  logic symbol by attaching the meta \verb|rewrite_def| to the symbol.
\begin{whycode}
function sqr (x:int) : int = x * x
meta "rewrite_def" function sqr
\end{whycode}
\item Rewriting using axioms or lemmas declared as rewrite rules. When
  an axiom (or a lemma) has one of the forms
\begin{whycode}
axiom a: forall ... t1 = t2
\end{whycode}
  or
\begin{whycode}
axiom a: forall ... f1 <-> f2
\end{whycode}
  then the user can declare
\begin{whycode}
meta "rewrite" prop a
\end{whycode}
  to turn this axiom into a rewrite rule. Rewriting is always done
  from left to right. Beware that there is no check for termination
  nor for confluence of the set of rewrite rules declared.
\end{itemize}

\paragraph{Bound on the number of reductions}
The computations performed by these transformations can take an
arbitrarily large number of steps, or even not terminate. For this
reason, the number of steps is bounded by a maximal value, which is
set by default to 1000. This value can be increased by another meta,
\eg
\begin{whycode}
meta "compute_max_steps" 1_000_000
\end{whycode}
When this upper limit is reached, a warning is issued, and the
partly-reduced goal is returned as the result of the transformation.


\subsection{Other Non-Splitting Transformations}

\begin{description}

\item[eliminate\_algebraic] replaces algebraic data types by first-order
definitions~\cite{paskevich09rr}.
\index{eliminate-algebraic@\verb+eliminate_algebraic+}

\item[eliminate\_builtin] removes definitions of symbols that are
  declared as builtin in the driver, \ie with a ``syntax'' rule.
\index{eliminate-builtin@\verb+eliminate_builtin+}

\item[eliminate\_definition\_func]
  replaces all function definitions with axioms.
\index{eliminate-definition-func@\verb+eliminate_definition_func+}

\item[eliminate\_definition\_pred]
  replaces all predicate definitions with axioms.
\index{eliminate-definition-pred@\verb+eliminate_definition_pred+}

\item[eliminate\_definition]
  applies both transformations above.
\index{eliminate-definition@\verb+eliminate_definition+}

\item[eliminate\_mutual\_recursion]
  replaces mutually recursive definitions with axioms.
\index{eliminate-mutual-recursion@\verb+eliminate_mutual_recursion+}

\item[eliminate\_recursion]
  replaces all recursive definitions with axioms.
\index{eliminate-recursion@\verb+eliminate_recursion+}

\item[eliminate\_if\_term] replaces terms of the form \texttt{if
    formula then t2 else t3} by lifting them at the level of formulas.
  This may introduce \texttt{if then else } in formulas.
\index{eliminate-if-term@\verb+eliminate_if_term+}

\item[eliminate\_if\_fmla] replaces formulas of the form \texttt{if f1 then f2
  else f3} by an equivalent formula using implications and other
  connectives.
\index{eliminate-if-fmla@\verb+eliminate_if_fmla+}

\item[eliminate\_if]
  applies both transformations above.
\index{eliminate-if@\verb+eliminate_if+}

\item[eliminate\_inductive] replaces inductive predicates by
  (incomplete) axiomatic definitions, \ie construction axioms and
  an inversion axiom.
\index{eliminate-inductive@\verb+eliminate_inductive+}

\item[eliminate\_let\_fmla]
  eliminates \texttt{let} by substitution, at the predicate level.
\index{eliminate-let-fmla@\verb+eliminate_let_fmla+}

\item[eliminate\_let\_term]
  eliminates \texttt{let} by substitution, at the term level.
\index{eliminate-let-term@\verb+eliminate_let_term+}

\item[eliminate\_let]
  applies both transformations above.
\index{eliminate-let@\verb+eliminate_let+}

% \item[encoding\_decorate\_mono]

% \item[encoding\_enumeration]

\item[encoding\_smt]
  encodes polymorphic types into monomorphic type~\cite{conchon08smt}.
\index{encoding-smt@\verb+encoding_smt+}

\item[encoding\_tptp]
  encodes theories into unsorted logic. %~\cite{cruanes10}.
\index{encoding-tptp@\verb+encoding_tptp+}

% \item[filter\_trigger] *)

% \item[filter\_trigger\_builtin] *)

% \item[filter\_trigger\_no\_predicate] *)

% \item[hypothesis\_selection] *)
%   Filter hypothesis of goals~\cite{couchot07ftp,cruanes10}. *)

\item[introduce\_premises] moves antecedents of implications and
  universal quantifications of the goal into the premises of the task.
\index{introduce-premises@\verb+introduce_premises+}

% \item[remove\_triggers] *)
%   removes the triggers in all quantifications. *)

\item[simplify\_array] automatically rewrites the task using the lemma
  \verb|Select_eq| of theory \verb|map.Map|.
\index{simplify-array@\verb+simplify_array+}

\item[simplify\_formula] reduces trivial equalities $t=t$ to true and
  then simplifies propositional structure: removes true, false, simplifies
  $f \land f$ to $f$, etc.
\index{simplify-formula@\verb+simplify_formula+}

\item[simplify\_recursive\_definition] reduces mutually recursive
  definitions if they are not really mutually recursive, \eg
\begin{whycode}
function f : ... = .... g ...
with g : ... = e
\end{whycode}
becomes
\begin{whycode}
function g : ... = e
function f : ... = ... g ...
\end{whycode}
if $f$ does not occur in $e$.
\index{simplify-recursive-definition@\verb+simplify_recursive_definition+}

\item[simplify\_trivial\_quantification]
  simplifies quantifications of the form
\begin{whycode}
forall x, x=t -> P(x)
\end{whycode}
or
\begin{whycode}
forall x, t=x -> P(x)
\end{whycode}
when $x$ does not occur in $t$ into
\begin{whycode}
P(t)
\end{whycode}
  More generally, it applies this simplification whenever $x=t$ appears
  in a negative position.
\index{simplify-trivial-quantification@\verb+simplify_trivial_quantification+}

\item[simplify\_trivial\_quantification\_in\_goal]
  is the same as above but it applies only in the goal.
\index{simplify-trivial-quantification-in-goal@\verb+simplify_trivial_quantification_in_goal+}

\item[split\_premise] replaces axioms in conjunctive form
  by an equivalent collection of axioms.
  In absence of case analysis labels (see \texttt{split\_goal} for details),
  the number of axiom generated per initial axiom is
  linear in the size of that initial axiom.
\index{split-premise@\verb+split_premise+}

\item[split\_premise\_full]: This transformation is similar to \texttt{split\_premise}, but also converts the axioms to conjunctive normal form. The number of axioms generated per initial axiom may be exponential in the size of the initial axiom.
\index{split-premise-full@\verb+split_premise_full+}

\end{description}

\subsection{Other Splitting Transformations}~\label{tech:trans:split}

\begin{description}

\item[simplify\_formula\_and\_task] is the same as \texttt{simplify\_formula}
  but it also removes the goal if it is equivalent to true.
\index{simplify-formula-and-task@\verb+simplify_formula_and_task+}

\item[split\_goal]: if the goal is a conjunction of goals, returns the
  corresponding set of subgoals. In absence of case analysis labels,
  the number of subgoals generated is linear in the size of the initial goal.

  \paragraph{Behavior on asymetric connectives and \texttt{by}/\texttt{so}}

  The transformation treat specially asymetric and \texttt{by}/\texttt{so}
  connectives. Asymetric conjunction \verb|A && B| in goal position
  is handled as syntactic sugar for \verb|A /\ (A -> B)|.
  The conclusion of the first subgoal can then be used to prove the second one.

  Asymetric disjunction \verb+A || B+ in hypothesis position is handled as
  syntactic sugar for \verb|A \/ ((not A) /\ B)|.
  In particular, a case analysis on such hypothesis would give the negation of
  the first hypothesis in the second case.

  The \texttt{by} connective is treated as a proof indication. In hypothesis position, \verb|A by B| is treated as if it were syntactic sugar for its regular interpretation \verb|A|. In goal position, it is treated as if \verb|B| was an intermediate step for proving \verb|A|. \verb|A by B| is then replaced by |B| and the transformation also generates a side-condition subgoal \verb|B -> A| representing the logical cut.

  Although splitting stops at disjunctive points like
  symetric disjunction and left-hand sides of implications, the occurences of the \texttt{by} connective are not restricted. For instance:
  \begin{itemize}
  \item Splitting
\begin{whycode}
goal G : (A by B) && C
\end{whycode}
generates the subgoals
\begin{whycode}
goal G1 : B
goal G2 : A -> C
goal G3 : B -> A (* side-condition *)
\end{whycode}
\item Splitting
\begin{whycode}
goal G : (A by B) \/ (C by D)
\end{whycode}
generates
\begin{whycode}
goal G1 : B \/ D
goal G2 : B -> A (* side-condition *)
goal G3 : D -> C (* side-condition *)
\end{whycode}
\item Splitting
\begin{whycode}
goal G : (A by B) || (C by D)
\end{whycode}
generates
\begin{whycode}
goal G1 : B || D
goal G2 : B -> A        (* side-condition *)
goal G3 : B || (D -> C) (* side-condition *)
\end{whycode}
Note that due to the asymetric disjunction, the disjunction is kept in the
second side-condition subgoal.
\item Splitting
\begin{whycode}
goal G : exists x. P x by x = 42
\end{whycode}
generates
\begin{whycode}
goal G1 : exists x. x = 42
goal G2 : forall x. x = 42 -> P x (* side-condition *)
\end{whycode}
Note that in the side-condition subgoal, the context is universally closed.
\end{itemize}

The \texttt{so} connective plays a similar role in hypothesis position, as it serves as a consequence indication. In goal position, \verb|A so B| is treated as if it were syntatic sugar for its regular interpretation \verb|A|. In hypothesis position, it is treated as if both \verb|A| and \verb|B| were true because \verb|B| is a consequence of \verb|A|. \verb|A so B| is replaced by \verb|A /\ B| and the transformation also generates a side-condition subgoal \verb|A -> B| corresponding to the consequence relation between formula.

  As for the \texttt{by} connective, occurences of \texttt{so} are unrestricted.
  For instance:
  \begin{itemize}
  \item Splitting
\begin{whycode}
goal G : (((A so B) \/ C) -> D) && E
\end{whycode}
generates
\begin{whycode}
goal G1 : ((A /\ B) \/ C) -> D
goal G2 : (A \/ C -> D) -> E
goal G3 : A -> B               (* side-condition *)
\end{whycode}
\item Splitting
\begin{whycode}
goal G : A by exists x. P x so Q x so R x by T x
(* reads: A by (exists x. P x so (Q x so (R x by T x))) *)
\end{whycode}
\begin{whycode}
goal G1 : exists x. P x
goal G2 : forall x. P x -> Q x               (* side-condition *)
goal G3 : forall x. P x -> Q x -> T x        (* side-condition *)
goal G4 : forall x. P x -> Q x -> T x -> R x (* side-condition *)
goal G5 : (exists x. P x /\ Q x /\ R x) -> A (* side-condition *)
\end{whycode}
In natural language, this corresponds to the following proof schema for \verb|A|:
There exists a \verb|x| for which \verb|P| holds. Then, for that witness \verb|Q| and \verb|R| also holds. The last one holds because \verb|T| holds as well. And from those three conditions on \verb|x|, we can deduce \verb|A|.
\end{itemize}

\paragraph{Labels controlling the transformation}

The transformations in the split family can be finely controlled by using
labels on formula.

The label \verb|"stop_split"| can be used to block the splitting of a formula.
The label is removed after blocking, so applying the transformation a second
time will split the formula. This is can be used to decompose the splitting process in blocks. Also, if a formula with this label is found in non-goal position,
its \texttt{by}/\texttt{so} proof indication will be erased by the
transformation. In a sense, formula tagged by \verb|"stop_split"| are handled
as if they were local lemmas.

The label \verb|"case_split"| can be used to force case analysis on hypotheses.
For instance, applying \texttt{split\_goal} on
\begin{whycode}
goal G : ("case_split" A \/ B) -> C
\end{whycode}
generates the subgoals
\begin{whycode}
goal G1 : A -> C
goal G2 : B -> C
\end{whycode}
Without the label, the transformation do nothing because undesired case analysis
may easily lead to an exponential blow-up.

Note that the precise behavior of splitting transformation in presence of
the \verb|"case_split"| is not yet specified
and is likely to change in future versions.

\index{split-goal@\verb+split_goal+}

\item[split\_all]
  performs both \texttt{split\_premise} and \texttt{split\_goal}.
\index{split-all@\verb+split_all+}

\item[split\_intro]: Composition of \texttt{split\_goal} and \texttt{introduce\_premises}.
\index{split-intro@\verb+split_intro+}

\item[split\_goal\_full]: This transformation has a behavior similar
  to \texttt{split\_goal}, but also converts the goal to conjunctive normal form.
  The number of subgoals generated may be exponential in the size of the initial goal.
\index{split-goal-full@\verb+split_goal_full+}

\item[split\_all\_full]
  performs both \texttt{split\_premise} and \texttt{split\_goal\_full}.
\index{split-all-full@\verb+split_all_full+}


\end{description}


%%% Local Variables:
%%% mode: latex
%%% TeX-PDF-mode: t
%%% TeX-master: "manual"
%%% End:

