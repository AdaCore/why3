\chapter{Technical Informations}


\section{Structure of Session Files}

The proof session state is stored in an XML file named
\texttt{\textsl{<dir>}/why3session.xml}, where \texttt{\textsl{<dir>}}
is the directory of the project.
The XML file follows the DTD given in \texttt{share/why3session.dtd} and reproduced below.
\lstinputlisting{../share/why3session.dtd}


\section{Provers detection data}
\label{sec:proverdetecttiondata}

All the necessary data configuration for the automatic detection of
installed provers is stored in the file
\texttt{provers-detection-data.conf} typically located in directory
\verb|/usr/local/share/why3| after installation. The contents of this
file is reproduced below.
%BEGIN LATEX
{\footnotesize
%END LATEX
\lstinputlisting{../share/provers-detection-data.conf}
%BEGIN LATEX
}
%END LATEX

\section{The \texttt{why3.conf} configuration file}
\label{sec:whyconffile}
\index{why3.conf@\texttt{why3.conf}}\index{configuration file}

\begin{figure}[p]
%BEGIN LATEX
{\footnotesize
%END LATEX
\begin{verbatim}
[main]
loadpath = "/usr/local/share/why3/theories"
loadpath = "/usr/local/share/why3/modules"
magic = 14
memlimit = 0
plugin = "/usr/local/lib/why3/plugins/tptp"
plugin = "/usr/local/lib/why3/plugins/genequlin"
plugin = "/usr/local/lib/why3/plugins/hypothesis_selection"
running_provers_max = 4
timelimit = 2

[ide]
default_editor = "editor %f"
error_color = "orange"
goal_color = "gold"
iconset = "boomy"
intro_premises = true
premise_color = "chartreuse"
print_labels = false
print_locs = false
print_time_limit = false
saving_policy = 2
task_height = 404
tree_width = 512
verbose = 0
window_height = 1173
window_width = 1024

[prover]
command = "'why3-cpulimit' 0 %m -s coqtop -batch -I %l/coq-tactic -R %l/coq Why3 -l %f"
driver = "/usr/local/share/why3/drivers/coq.drv"
editor = "coqide"
in_place = false
interactive = true
name = "Coq"
shortcut = "coq"
version = "8.3pl4"

[prover]
command = "'why3-cpulimit' %t %m -s alt-ergo %f"
driver = "/usr/local/share/why3/drivers/alt_ergo_0.93.drv"
editor = ""
in_place = false
interactive = false
name = "Alt-Ergo"
shortcut = "altergo"
shortcut = "alt-ergo"
version = "0.93.1"

[editor coqide]
command = "coqide -I %l/coq-tactic -R %l/coq Why3 %f"
name = "CoqIDE"
\end{verbatim}
%BEGIN LATEX
}
%END LATEX
  \caption{Sample why3.conf file}
\label{fig:why3conf}
\end{figure}



One can use a custom configuration file. \texttt{why3config}
and other \texttt{why3} tools use priorities for which
user's configuration file to consider:
\begin{itemize}
\item the file specified by the \texttt{-C} or \texttt{-{}-config} options,
\item the file specified by the environment variable
  \texttt{WHY3CONFIG} if set.
\item the file \texttt{\$HOME/.why3.conf}
  (\texttt{\$USERPROFILE/.why3.conf} under Windows) or, in the case of
  local installation, \texttt{why3.conf} in Why3 sources top directory.
\end{itemize}
If none of these files exists, a built-in default configuration is used.

The configuration file is a human-readable text file, which consists
of association pairs arranged in sections. Figure~\ref{fig:why3conf}
shows an example of configuration file.

A section begins with a header inside square brackets and ends at the
beginning of the next section. The header of a
section can be only one identifier, \texttt{main} and \texttt{ide} in
the example, or it can be composed by a family name and one family
argument, \texttt{prover} is one family name, \texttt{coq} and
\texttt{alt-ergo} are the family argument.

Inside a section, one key can be associated with an integer (\eg{} -555),
a boolean (true, false) or a string (\eg{} "emacs"). One key can appear
only once except if its a multi-value key. The order of apparition of
the keys inside a section matter only for the multi-value key.

\section{Drivers of External Provers}
\label{sec:drivers}

The drivers of external provers are readable files, in directory
\texttt{drivers}. Experimented users can modify them to change the way
the external provers are called, in particular which transformations
are applied to goals.

[TO BE COMPLETED LATER]

\section{Transformations}
\label{sec:transformations}

Here is a quick documentation of provided transformations. We give
first the non-splitting ones, \eg{} those which produce one goal as
result, and others which produces any number of goals.

Notice that the set of available transformations in your own
installation is given by
\begin{verbatim}
why3 --list-transforms
\end{verbatim}

\subsection{Non-splitting transformations}

\begin{description}

\item[eliminate\_algebraic] replaces algebraic data types by first-order
definitions~\cite{paskevich09rr}.

\item[eliminate\_builtin] removes definitions of symbols that are
  declared as builtin in the driver, \ie with a ``syntax'' rule.
\item[eliminate\_definition\_func]
  replaces all function definitions with axioms.
\item[eliminate\_definition\_pred]
  replaces all predicate definitions with axioms.
\item[eliminate\_definition]
  applies both transformations above.
\item[eliminate\_mutual\_recursion]
  replaces mutually recursive definitions with axioms.
\item[eliminate\_recursion]
  replaces all recursive definitions with axioms.

\item[eliminate\_if\_term] replaces terms of the form \texttt{if
    formula then t2 else t3} by lifting them at the level of formulas.
  This may introduce \texttt{if then else } in formulas.

\item[eliminate\_if\_fmla] replaces formulas of the form \texttt{if f1 then f2
  else f3} by an equivalent formula using implications and other
  connectives.

\item[eliminate\_if]
  applies both transformations above.

\item[eliminate\_inductive] replaces inductive predicates by
  (incomplete) axiomatic definitions, \ie construction axioms and
  an inversion axiom.

\item[eliminate\_let\_fmla]
  eliminates \texttt{let} by substitution, at the predicate level.

\item[eliminate\_let\_term]
  eliminates \texttt{let} by substitution, at the term level.

\item[eliminate\_let]
  applies both transformations above.

% \item[encoding\_decorate\_mono]

% \item[encoding\_enumeration]

\item[encoding\_smt]
  encodes polymorphic types into monomorphic type~\cite{conchon08smt}.

\item[encoding\_tptp]
  encodes theories into unsorted logic. %~\cite{cruanes10}.

% \item[filter\_trigger] *)

% \item[filter\_trigger\_builtin] *)

% \item[filter\_trigger\_no\_predicate] *)

% \item[hypothesis\_selection] *)
%   Filter hypothesis of goals~\cite{couchot07ftp,cruanes10}. *)

\item[inline\_all]
  expands all non-recursive definitions.

\item[inline\_goal] expands all outermost symbols of the goal that
  have a non-recursive definition.

\item[inline\_trivial]
  removes definitions of the form

\begin{whycode}
function  f x_1 ... x_n = (g e_1 ... e_k)
predicate p x_1 ... x_n = (q e_1 ... e_k)
\end{whycode}
when each $e_i$ is either a ground term or one of the $x_j$, and
each $x_1 \dots x_n$ occurs at most once in all the $e_i$.

\item[introduce\_premises] moves antecedents of implications and
  universal quantifications of the goal into the premises of the task.

% \item[remove\_triggers] *)
%   removes the triggers in all quantifications. *)

\item[simplify\_array] automatically rewrites the task using the lemma
  \verb|Select_eq| of theory \verb|array.Array|.

\item[simplify\_formula] reduces trivial equalities $t=t$ to true and
  then simplifies propositional structure: removes true, false, simplifies
  $f \land f$ to $f$, etc.

\item[simplify\_recursive\_definition] reduces mutually recursive
  definitions if they are not really mutually recursive, \eg
\begin{whycode}
function f : ... = .... g ...
with g : ... = e
\end{whycode}
becomes
\begin{whycode}
function g : ... = e
function f : ... = ... g ...
\end{whycode}
if $f$ does not occur in $e$.

\item[simplify\_trivial\_quantification]
  simplifies quantifications of the form
\begin{verbatim}
forall x, x=t -> P(x)
\end{verbatim}
or
\begin{verbatim}
forall x, t=x -> P(x)
\end{verbatim}
when $x$ does not occur in $t$ into
\begin{verbatim}
P(t)
\end{verbatim}
  More generally, it applies this simplification whenever $x=t$ appears
  in a negative position.

\item[simplify\_trivial\_quantification\_in\_goal]
  is the same as above but it applies only in the goal.

\item[split\_premise]
  splits conjunctive premises.

\end{description}

\subsection{Splitting transformations}

\begin{description}

\item[full\_split\_all]
  performs both \texttt{split\_premise} and \texttt{full\_split\_goal}.

\item[full\_split\_goal] puts the goal in a conjunctive form,
  returns the corresponding set of subgoals. The number of subgoals
  generated may be exponential in the size of the initial goal.

\item[simplify\_formula\_and\_task] is the same as \texttt{simplify\_formula}
  but it also removes the goal if it is equivalent to true.

\item[split\_all]
  performs both \texttt{split\_premise} and \texttt{split\_goal}.

\item[split\_goal] if the goal is a conjunction of goals, returns the
  corresponding set of subgoals. The number of subgoals generated is linear in
  the size of the initial goal.

\item[split\_intro]
  moves the antecedents into the premises when a goal is an implication.

\end{description}


%%% Local Variables:
%%% mode: latex
%%% TeX-PDF-mode: t
%%% TeX-master: "manual"
%%% End:

