\newcommand{\why}{\textsc{Why}}

\newglossaryentry{ident}{
  name={identifier},
  description={
    (type \texttt{Ident.ident}) --- a common type to represent
    a name of an object. Every \gls{tysymbol}, \gls{lsymbol},
    \gls{prsymbol}, \gls{tvsymbol}, \gls{vsymbol}, and
    \gls{theory} has a unique identifier that can be used
    to distinguish it from any other object of the same type.
\glspar
    Every identifier has an associated, possibly non-unique,
    string. To avoid collisions in \gls{prettyprinting} and
    to make the output conform to a given format, various
    \glspl{printer} may \glslink{sanitization}{sanitize}
    identifiers. Identifiers may also bear a \gls{location}
    of their origin and a list of \glspl{label}.
\glspar
    An identifier is \gls{unigen} from a \gls{preid} and
    is suitable for \gls{weakmemo}.
\nopostdesc}
}

\newglossaryentry{label}{
  name={label},
  description={
\nopostdesc}
}

\newglossaryentry{location}{
  name={location},
  description={
\nopostdesc}
}

\newglossaryentry{decl}{
  name={declaration},
  description={(type \texttt{Decl.decl})
\nopostdesc}
}

\newglossaryentry{algtype}{
  name={algebraic type},
  description={
\nopostdesc}
}

\newglossaryentry{constructor}{
  name={constructor},
  description={--- a \gls{lsymbol} introduced in
  an \gls{algtype} \gls{decl}. Can be used in \glspl{pattern}
  and as a usual function symbol.
\nopostdesc}
}

\newglossaryentry{lsymbol}{
  name={logical symbol},
  description={(type \texttt{Term.lsymbol}) --- a type representing
  function and predicate symbols. A logical symbol bears a unique
  \gls{ident} and a type signature, describing what types a symbol
  admits in its arguments.
\glspar
  A logical symbol is \gls{unigen} from a \gls{preid} and
  is suitable for \gls{weakmemo}.
\nopostdesc}
}

\newglossaryentry{preid}{
  name={pre-identifier},
  description={(type \texttt{Ident.preid}) --- a preliminary
  non-unique object used to produce unique \glspl{ident}.
\nopostdesc}
}

\newglossaryentry{prettyprinting}{
  name={pretty-printing},
  description={
\nopostdesc}
}

\newglossaryentry{printer}{
  name={printer},
  description={
\nopostdesc}
}

\newglossaryentry{proposition}{
  name={printer},
  description={
\nopostdesc}
}

\newglossaryentry{prsymbol}{
  name={proposition symbol},
  description={(type \texttt{Decl.prsymbol}) --- a type representing
  \gls{proposition} names. A proposition symbol bears a unique \gls{ident},
  is \gls{unigen} from a \gls{preid}, and is suitable for \gls{weakmemo}.
\nopostdesc}
}

\newglossaryentry{sanitization}{
  name={sanitization},
  description={
\nopostdesc}
}

\newglossaryentry{theory}{
  name={theory},
  description={
\nopostdesc}
}

\newglossaryentry{tvsymbol}{
  name={type variable},
  description={
    (type \texttt{Ty.tvsymbol}) --- a type representing symbols of
    type variables. A type variable bears a unique \gls{ident}
    and is \gls{unigen} from a \gls{preid}.
\nopostdesc}
}

\newglossaryentry{hashcons}{
  name={hash-consed},
  description={
    Objects of a given type are hash-consed whenever (a) every
    two semantically equal objects are also physically equal;
    and (b) contrary to \gls{unigen} objects, one can construct
    an hash-consed object equal to an existing one.
\glspar
    Hash-consed objects provide efficient comparison and hash
    operations and are suitable for \gls{weakmemo}.
\nopostdesc}
}

\newglossaryentry{pattern}{
  name={pattern},
  description={(type \texttt{Term.pattern}) --- objects used
  in pattern-matching expressions. A pattern is built of
  \glspl{constructor} and \glspl{vsymbol} with the help of
  smart constructors guaranteeing that every pattern is well-typed.
\glspar
  Patterns are \gls{hashcons}.
\nopostdesc}
}

\newglossaryentry{type}{
  name={type},
  description={(type \texttt{Ty.ty}) --- a type representing, well,
  types in {\why}. Types are built with the help of smart constructors
  guaranteeing that every type is well-formed.
\glspar
  Types are \gls{hashcons} and suitable for \gls{weakmemo}.
\nopostdesc}
}

\newglossaryentry{term}{
  name={term},
  description={(type \texttt{Term.term}) --- a type of logical terms.
  Every term has a \gls{type}. Terms are built with the help of
  smart constructors guaranteeing that every term is well-typed.
  A term can bear a list of \glspl{label}.
\glspar
  In addition to usual first-order terms, {\why} terms can contain
  if-then-else expressions, let-expressions, and \gls{pattern} matching.
\glspar
  Terms are \gls{hashcons}.
\nopostdesc}
}

\newglossaryentry{formula}{
  name={formula},
  description={(type \texttt{Term.fmla}) --- a type of logical formulas.
  Formulas are built with the help of smart constructors guaranteeing
  that every formula is well-typed. A formula can bear a list of
  \glspl{label}.
\glspar
  In addition to usual first-order formulas, {\why} formulas can contain
  if-then-else expressions, let-expressions, and \gls{pattern} matching.
\glspar
  Formulas are \gls{hashcons}.
\nopostdesc}
}

\newglossaryentry{tysymbol}{
  name={type symbol},
  description={(type \texttt{Ty.tysymbol}) --- a type representing
  \gls{type} constructors. A type symbol bears a unique \gls{ident}
  and an arity, is \gls{unigen} from a \gls{preid}, and
  is suitable for \gls{weakmemo}.
\nopostdesc}
}

\newglossaryentry{unigen}{
  name={uniquely generated},
  description={
    Objects of a given type are uniquely generated whenever every
    newly constructed object is physically and semantically distinct
    from any other value of the same type. For example, \glspl{ident}
    are uniquely generated, while \glspl{preid} or \glspl{term} are not.
\glspar
    Uniquely generated objects usually provide efficient comparison
    and hash operations. Thus, they can be used in \gls{hashcons}
    objects and are suitable for \gls{weakmemo}.
\nopostdesc}
}

\newglossaryentry{vsymbol}{
  name={variable},
  description={(type \texttt{Ty.vsymbol}) --- a type representing
  variable symbols. A variable symbol bears a unique \gls{ident}
  and has an associated \gls{type}.
\glspar
  A variable is \gls{unigen} from a \gls{preid}.
\nopostdesc}
}

\newglossaryentry{weakmemo}{
  name={forgetful memoization},
  description={(module \texttt{Hashweak}) --- memoization technique
  that does not create an additional reference to a key, allowing it
  (and its associated value) to be garbage-collected even while
  the memoization table is still accessible. Forgetful memoization
  is used, in particular, to implement \gls{task} \glspl{trans}:
  the transformation results are memoized, but as soon as the
  original task is garbage-collected, these results can be
  dropped, too.
\glspar
  \Glspl{tysymbol}, \glspl{lsymbol}, \glspl{prsymbol},
  \glspl{type}, \glspl{decl}, \glspl{task}, and \glspl{env}
  can be used as keys in forgetful memoization.
\nopostdesc}
}

\newglossaryentry{trans}{
  name={transformation},
  description={(type \texttt{Trans.trans})
\nopostdesc}
}

\newglossaryentry{task}{
  name={task},
  description={(type \texttt{Task.task})
\nopostdesc}
}

\newglossaryentry{env}{
  name={environment},
  description={(type \texttt{Env.env})
\nopostdesc}
}


\printglossary

\gls{type} \gls{term} \gls{formula} \gls{ident} \gls{weakmemo}
\gls{vsymbol} \gls{unigen} \gls{tysymbol} \gls{pattern}
