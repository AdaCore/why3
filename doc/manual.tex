\documentclass[a4paper,11pt,twoside,openright]{memoir}

% rubber: module index

%BEGIN LATEX
\usepackage{comment}
\newcommand{\ahref}[2]{{#2}}
\excludecomment{htmlonly}
\newenvironment{latexonly}{}{}
%END LATEX

%HEVEA\@addimagenopt{-pdf}

%BEGIN LATEX
% tells memoir style to number subsections
\setsecnumdepth{subsection}
%END LATEX

\usepackage[T1]{fontenc}
\usepackage{lmodern}
%\usepackage{url}
\usepackage[pdftex,colorlinks=true,urlcolor=blue,pdfstartview=FitH]{hyperref}

%BEGIN LATEX
\usepackage{upquote}
%END LATEX

%BEGIN LATEX
\usepackage{graphicx}
%END LATEX
%HEVEA \newcommand{\includegraphics}[2][2]{\imgsrc{#2}}

\usepackage{listings}
\usepackage{xspace}

%BEGIN LATEX
\setulmarginsandblock{30mm}{30mm}{*}
\setlrmarginsandblock{30mm}{30mm}{*}
\setheadfoot{15pt}{38pt}
\checkandfixthelayout

% placement of figures
\renewcommand{\textfraction}{0.01}
\renewcommand{\topfraction}{0.99}
\renewcommand{\bottomfraction}{0.99}
%END LATEX
\setcounter{topnumber}{4}
\setcounter{bottomnumber}{4}
\setcounter{totalnumber}{8}

%HEVEA \newstyle{table.lstframe}{width:100\%;border-width:1px;}

% \usepackage[toc,nonumberlist]{glossaries}
% \makeglossaries

% \usepackage{glossary}
% \makeglossary
% \glossary{name={entry name}, description={entry description}}

% for ocamldoc generated pages
%\usepackage{ocamldoc}
%\let\tt\ttfamily
%\let\bf\bfseries

\usepackage{ifthen}


\newcommand{\why}{\textsf{Why3}}
\newcommand{\eg}{\emph{e.g.}}

% BNF grammar
\newcommand{\indextt}[1]{\index{#1@\protect\keyword{#1}}}
\newcommand{\indexttbs}[1]{\index{#1@\protect\keywordbs{#1}}}
\newif\ifspace
\newif\ifnewentry
\newcommand{\addspace}{\ifspace ~ \spacefalse \fi}
\newcommand{\term}[2]{\addspace\hbox{\lstinline|#1|%
\ifthenelse{\equal{#2}{}}{}{\indexttbase{#2}{#1}}}\spacetrue}
\newcommand{\nonterm}[2]{%
  \ifthenelse{\equal{#2}{}}%
             {\addspace\hbox{\textsl{#1}\ifnewentry\index{grammar entries!\textsl{#1}}\fi}\spacetrue}%
             {\addspace\hbox{\textsl{#1}\footnote{#2}\ifnewentry\index{grammar entries!\textsl{#1}}\fi}\spacetrue}}
\newcommand{\repetstar}{$^*$\spacetrue}
\newcommand{\repetplus}{$^+$\spacetrue}
\newcommand{\repetone}{$^?$\spacetrue}
\newcommand{\lparen}{\addspace(}
\newcommand{\rparen}{)}
\newcommand{\orelse}{\addspace$\mid$\spacetrue}
\newcommand{\sep}{ \\[2mm] \spacefalse\newentrytrue}
\newcommand{\newl}{ \\ & & \spacefalse}
\newcommand{\alt}{ \\ & $\mid$ & \spacefalse}
\newcommand{\is}{ & $::=$ & \newentryfalse}
\newenvironment{syntax}{\begin{tabular}{@{}rrll@{}}\spacefalse\newentrytrue}{\end{tabular}}
\newcommand{\synt}[1]{$\spacefalse#1$}
\newcommand{\emptystring}{$\epsilon$}
\newcommand{\below}{See\; below}

%%% Local Variables: 
%%% mode: latex
%%% TeX-master: "manual"
%%% End: 

\usepackage{xcolor}
\usepackage{colortbl}
\usepackage{rotating}

\newcommand{\provername}[1]{\cellcolor{yellow!25}
\begin{sideways}\textbf{#1}~~\end{sideways}}
\newcommand{\explanation}[1]{\cellcolor{yellow!13}\textsl{#1}}

\newcommand{\valid}[1]{\cellcolor{green!13}#1}
\newcommand{\unknown}[1]{\cellcolor{red!20}#1}
\newcommand{\invalid}[1]{\cellcolor{red!50}#1}
\newcommand{\timeout}[1]{\cellcolor{red!20}(#1)}
\newcommand{\outofmemory}[1]{\cellcolor{red!20}(#1)}
\newcommand{\noresult}{\multicolumn{1}{>{\columncolor[gray]{0.8}}c|}{~}}
\newcommand{\failure}{\cellcolor{red!20}failure}
\newcommand{\highfailure}{\cellcolor{red!50}FAILURE}




\input{./version.tex}

\makeindex

%HEVEA\title{The Why3 platform}

\begin{document}
\sloppy
%BEGIN LATEX
\hbadness=5000
%END LATEX

\thispagestyle{empty}

\begin{center}

%BEGIN LATEX
\rule\textwidth{0.8mm}
%END LATEX

\vfill

{
%BEGIN LATEX
\fontsize{40}{40pt}\selectfont
%END LATEX
%HEVEA \Huge
\bfseries\sffamily The Why3 platform}

\vfill

%BEGIN LATEX
\rule\textwidth{0.8mm}
%END LATEX

\vfill

% \todo{NE PAS DISTRIBUER TANT QU'IL RESTE DES TODOS}

%BEGIN LATEX
\begin{LARGE}
%END LATEX
  Version \whyversion{}, January 2018
%BEGIN LATEX
\end{LARGE}
%END LATEX

\vfill

%BEGIN LATEX
\begin{Large}
%END LATEX
  \begin{tabular}{c}
  Fran\c{c}ois Bobot$^{1,2}$ \\
  Jean-Christophe Filli\^atre$^{1,2}$  \\
  Claude March\'e$^{2,1}$ \\
  Guillaume Melquiond$^{2,1}$\\
  Andrei Paskevich$^{1,2}$
\end{tabular}
%BEGIN LATEX
\end{Large}
%END LATEX
\vfill

\begin{flushleft}

\begin{tabular}{l}
$^1$ LRI, CNRS \& University Paris-Sud, Orsay, F-91405 \\
$^2$ Inria Saclay -- \^Ile-de-France, Palaiseau, F-91120
\end{tabular}

%BEGIN LATEX
\bigskip
%END LATEX

\textcopyright 2010--2016 University Paris-Sud, CNRS, Inria

\urldef{\urlutcat}{\url}{http://frama-c.com/u3cat/}
\urldef{\urlhilite}{\url}{http://www.open-do.org/projects/hi-lite/}
\urldef{\urlbware}{\url}{http://bware.lri.fr/}
\urldef{\urlproofinuse}{\url}{http://www.spark-2014.org/proofinuse}

This work has been partly supported by the `\ahref{\urlutcat}{U3CAT}'
national ANR project (ANR-08-SEGI-021-08\begin{latexonly},
  \urlutcat\end{latexonly}) ; the `\ahref{\urlhilite}{Hi-Lite}'
\begin{latexonly}(\urlhilite)\end{latexonly} FUI project of the
System@tic competitivity cluster ; the `\ahref{\urlbware}{BWare}'
ANR project (ANR-12-INSE-0010\begin{latexonly},
  \urlbware\end{latexonly}) ; and the \ahref{\urlproofinuse}{Joint Laboratory ProofInUse} (ANR-13-LAB3-0007\begin{latexonly}, \urlproofinuse\end{latexonly})
\end{flushleft}
\end{center}

\chapter*{Foreword}
%HEVEA\cutname{foreword.html}

%This is the manual for the Why platform version 3, or \why for
%short.
\why is a platform for deductive program verification. It provides
a rich language for specification and programming, called \whyml, and
relies on external theorem provers, both automated and interactive,
to discharge verification conditions. \why comes with a standard
library of logical theories (integer and real arithmetic, Boolean
operations, sets and maps, etc.) and basic programming data structures
(arrays, queues, hash tables, etc.). A user can write \whyml programs
directly and get correct-by-construction OCaml programs through an
automated extraction mechanism. \whyml is also used as an intermediate
language for the verification of C, Java, or Ada programs.

\why is a complete reimplementation %~\cite{boogie11why3}
of the former Why platform~\cite{filliatre07cav}.
%for program verification.
Among the new features are: numerous
extensions to the input language, a new architecture for calling
external provers, and a well-designed API, allowing to use \why as a
software library.  An important emphasis is put on modularity and
genericity, giving the end user a possibility to easily reuse \why
formalizations or to add support for a new external prover if wanted.

\subsection*{Availability}

\why project page is \url{http://why3.lri.fr/}.  The last distribution
is available there, in source format, together with this documentation
and several examples.

\why is distributed as open source and freely available under the
terms of the GNU LGPL 2.1. See the file \texttt{LICENSE}.

See the file \texttt{INSTALL} for quick installation instructions, and
Section~\ref{sec:install} of this document for more detailed
instructions.

\subsection*{Contact}

There is a public mailing list for users' discussions:
\url{http://lists.gforge.inria.fr/mailman/listinfo/why3-club}.

Report any bug to the \why Bug Tracking System:
\url{https://gforge.inria.fr/tracker/?atid=10293&group_id=2990&func=browse}.


\subsection*{Acknowledgements}

We gratefully thank the people who contributed to \why, directly or
indirectly: Romain Bardou, Stefan Berghofer, Sylvie Boldo, Martin
Clochard, Simon Cruanes, L\'eon Gondelman, Johannes Kanig, St\'ephane
Lescuyer, David Mentr\'e, Sim\~ao Melo de Sousa, Benjamin Monate,
Thi-Minh-Tuyen Nguyen, M\'ario Pereira, Asma Tafat, Piotr Trojanek.


\cleardoublepage

%BEGIN LATEX
\tableofcontents
%END LATEX

%
\chapter{Introduction}


\section{Organization of this document}

This document is organized as follows. The first three chapters are
user manuals, to learn how to use Why3. Other chapters are reference
manuals, giving detailed technical informations.

Chapter~\ref{chap:syntax} presents the input syntax for file defining
Why theories. The semantics is given informally with examples.
Chapter~\ref{chap:ide} explains how to use the Why IDE for visualizing
theories and goals, call external provers for trying to solve them,
apply transformations to simplify them. The two first chapters are
thus to read for the beginners.

Chapter~\ref{chap:api} presents how to use Why3 programmatically,
using the API.  It is for the more advanced users, who wants to link
Why3 library with their own code. 

Chapter~\ref{chap:manpages} are the technical manual pages for the tools of
the platform. All tool options, and all the configuration files are described in details there.

Chapter~\ref{chap:apidoc} is the technical documentation of the API.



%%% Local Variables:
%%% mode: latex
%%% TeX-PDF-mode: t
%%% TeX-master: "manual"
%%% End:


\part{Tutorial}

\chapter{Getting Started}
\label{chap:starting}

\section{Hello Proofs}

The first and basic step in using Why3 is to write a suitable input
file. When one wants to learn a programming language, you start by
writing a basic program. Here we start by writing a file containing a
basic set of goals. 

Here is our first Why3 file, which is the file
\texttt{examples/hello\_proof.why} of the distribution.
\verbatiminput{../examples/hello_proof.why} Any declaration must occur
inside a theory, which is in that example called TheoryProof and
labelled with a comment inside double quotes. It contains three goals
named $G_1,G_2,G_3$. The first two are basic propositional goals,
whereas the third involves some integer arithmetic, and thus it
requires to import the theory of integer arithmetic from the Why3
standard library, which is done by the \texttt{use} declaration above.

We don't give more details here about the syntax and refer to
Chapter~\ref{chap:syntax} for detailed explanations. In the following,
we show how this file is handled in the Why3 GUI
(Section~\ref{sec:gui}) then in batch mode using the \texttt{why3}
executable (Section~\ref{sec:batch}). 


\section{Getting Started with the GUI}
\label{sec:gui}

The graphical interface allows to browse into a file or a set of
files, and check the validity of goals with external provers, in a
friendly way. This section presents the basic use of this GUI. Please
refer to Section~\ref{sec:ideref} for a more complete description.

\begin{figure}[tbp]
  \includegraphics[width=\textwidth]{gui1.png}
  \caption{The GUI when started the very first time}
  \label{fig:gui1}
\end{figure}

The GUI is launched on the file above as follows.
\begin{verbatim}
why3ide hello_proof.why
\end{verbatim}
When the GUI is started for the first time, you should get a window
which looks like the screenshot of Figure~\ref{fig:gui1}. First of
all, the left row is a tool bar which provide different actions to
apply on goals. In this case, the section ``Provers'' is empty, which
means that you did not perform prover detection yet. You should do it
now using the menu \textsf{File/Detect provers}. Second, the middle
part is a tree view that allows to browse inside the
theories. Initially, the item of this tree are closed. You should now
expand this view using the menu \textsf{View/Expand all} or its
shortcut \textsf{Ctrl-E}. This should result is something like the
screenshot of Figure~\ref{fig:gui2}.

\begin{figure}[tbp]
  \includegraphics[width=\textwidth]{gui2.png}
  \caption{The GUI with provers detected and tree view expanded}
  \label{fig:gui2}
\end{figure}

In the tree view, we have now a strctured view of the file: this file
contains one theory, itself containg three goals. In
Figure~\ref{fig:gui2}, we also clicked on the row corresponding to
goal $G_1$. The \emph{task} associated with this goal is then
displayed on the top right, and the corresponding part of the input
file is shown on the bottom right part.

Notice also that three provers where detected, and are now shown as
button in the ``provers'' section of the left toolbar. In that case,
detected provers are Alt-Ergo~\cite{ergo}, Coq~\cite{CoqArt} and
Simplify~\cite{simplify05}. 

\subsection{Calling provers on goals}

You are now ready to call these provers on the goals. Whenever you
click on a prover button, this prover is called on the goal selected
in the tree view. You can even select several goals at a time, either
by using multi-selection (typically by clicking while pressing the
\textsf{Shift} or \textsf{Ctrl} key) or by selecting the parent theory
or the parent file. Let us now select the theory ``HelloProof'' and
click on the \textsf{Simplify} button. After a short time, you should
get the display of Figure~\ref{fig:gui3}.

\begin{figure}[tbp]
  \includegraphics[width=\textwidth]{gui3.png}
  \caption{The GUI after Simplify prover is run on each goal}
  \label{fig:gui3}
\end{figure}

The row corresponding to goal $G_1$ is now closed, and marked with
green ``checked'' icon in the status column. This means that the goal
is proved by the Simplify prover. On the contrary, the two other goals
are not proved, they are marked with an orange question mark.

You can immediately attempt to prove the remaining goals using another
prover, {\eg} Alt-Ergo, by clicking on the corresponding button. The
goal $G_3$ should be proved now, but not $G_2$. 

\subsection{Applying transformations}

Instead of calling a prover on a goal, you can apply a transformation
to it.  Since $G_2$ is a conjunction, a possibility is to split it
into subgoals. You can do that by clicking on the \textsf{Split}
button of section ``Transformations'' of the left toolbar. Now you
have two subgoals, and you can try again a prover on them, for example
Simplify. Assuming we expand everything again, you should see now what
is displayed on Figure~\ref{fig:gui4}.

\begin{figure}[tbp]
  \includegraphics[width=\textwidth]{gui4.png}
  \caption{The GUI after splitting goal $G_2$}
  \label{fig:gui4}
\end{figure}

The first part of goal $G_2$ is still unproved. As a last resort, we
can try to call the Coq proof assistant. The first step is to click on
the \textsf{Coq} button. A new sub-row appear for Coq, and
unsurprisingly the goal is not proved by Coq either. What can be done
now is editing the proof: select that row and then click on the
\textsf{Edit} button in section ``Tools'' of the toolbar. This should
launch the Coq proof editor, which is \texttt{coqide} by default (see
Section~\ref{sec:ideref} for details on how to configure this). You get
now a regular Coq file fo fill in, as shown on Figure~\ref{fig:coqide}.
Please take care of the comments of this file. Only the part between
the two last comments can be modified. Moreover, these comments
themselves should not be modified at all, they are use to mark the
part you modify, in order to regenerate the file if the goal is
changed. 

\begin{figure}[tbp]
  \includegraphics[width=\textwidth]{coqide.png}
  \caption{CoqIDE on subgoal 1 of $G_2$}
  \label{fig:coqide}
\end{figure}

Of course, in that particular case, the goal cannot be proved since it
is not valid. The only thing to do is to fix the input file, as
explained below.

\subsection{Modifying the input}

Currently, the GUI does not allow to modify the input file. You must
exit the GUI and modify the file by some editor of your choice. Let's assume we change the goal $G_2$ by replacing the first occurrence of true by false, \eg
\begin{verbatim}
  goal G2 : (false -> false) and (true or false)
\end{verbatim}
Starting the IDE on the modified file and expanding everything with
\textsf{Ctrl-E}, we get the tree view shown on Figure~\ref{fig:gui5}.

\begin{figure}[tbp]
  \includegraphics[width=\textwidth]{gui5.png}
  \caption{The GUI restarted after modifying goal $G_2$}
  \label{fig:gui5}
\end{figure}

The important feature to notice first is that all the previous proof
attempts and transformations where saved in some database. Then, for
all the goals that remain unchanged, the previous proofs are shown
again. For the parts that changed, the previous proofs attempts are
shown but marked with "(obsolete)" so that you know the results are
not accurate. You can now retry to prove all what remains unproved
using any of the provers.

\section{Getting Started with the Why3 Command}
\label{sec:batch}

The why3 command allows to check the validity of goals with external
provers, in batch mode. This section presents the basic use of this
tool. Refer to Section~\ref{sec:why3ref} for a more complete
description of this tool and all its command-line options.

The very first time you want to use Why, you should proceed with
autodetection of external provers. This is done as follows, where ``>'' is the prompt.
\begin{verbatim}
> why3config --autodetect-provers
\end{verbatim}
This prints some information messages on what detections are attempted. To know which
provers have been successfully detected, you can run do as follows.
\begin{verbatim}
> why3 --list-provers
Known provers:
  alt-ergo (Alt-Ergo)
  coq (Coq)
  simplify (Simplify)
\end{verbatim}
The first word of each line is a unique identifier for the associated prover. We thus
have now the three provers Alt-Ergo~\cite{ergo}, Coq~\cite{CoqArt} and
Simplify~\cite{simplify05}. 

Let's assume now we want to run Simplify on the HelloProof
example. The command to type and its output are as follows, where the
\verb|-P| option is follow by the unique identifier as shown as above.
\begin{verbatim}
> why3 -P simplify hello_proof.why
hello_proof.why HelloProof G1 : Valid (0.10s)
hello_proof.why HelloProof G2 : Unknown: Unknown (0.01s)
hello_proof.why HelloProof G3 : Unknown: Unknown (0.00s)
\end{verbatim}

We can also specify which goal(s) to prove. This is done by giving
first a theory identifier, then goal identifier(s). Here is the way to
call Alt-Ergo on goals $G_2$ and $G_3$.
\begin{verbatim}
> why3 -P alt-ergo hello_proof.why -T HelloProof -G G2 -G G3
hello_proof.why HelloProof G2 : Unknown: Unknown (0.01s)
hello_proof.why HelloProof G3 : Valid (0.01s)
\end{verbatim}

Finally, a transformation to apply to goals before proving them can be
specified. To know the unique identifier associated to
transformations, do as follows.
\begin{verbatim}
> why3 --list-transforms
Known non-splitting transformations:
  [...]

Known splitting transformations:
  [...]
  split_goal
  split_intro
\end{verbatim}
Here is how you can split the goal $G_2$ before calling
Simplify on resulting subgoals.
\begin{verbatim}
> why3 -P simplify hello_proof.why -a split_goal -T HelloProof -G G2
hello_proof.why HelloProof G2 : Unknown: Unknown (0.00s)
hello_proof.why HelloProof G2 : Valid (0.00s)
\end{verbatim}
Section~\ref{sec:transformations} gives the description of the various
transformations available.



%%% Local Variables:
%%% mode: latex
%%% TeX-PDF-mode: t
%%% TeX-master: "manual"
%%% End:


\chapter{The \why Language}
\label{chap:syntax}
%HEVEA\cutname{syntax.html}

This chapter describes the input syntax, and informally gives its semantics,
illustrated by examples.

A \why text contains a list of \emph{theories}.
A theory is a list of \emph{declarations}. Declarations introduce new
types, functions and predicates, state axioms, lemmas and goals.
These declarations can be directly written in the theory or taken from
existing theories. The base logic of \why is first-order
logic with polymorphic types.

\section{Example 1: Lists}

%BEGIN LATEX
Figure~\ref{fig:tutorial1} contains an example of \why input
text, containing three theories.
%END LATEX
%HEVEA The code below is an example of \why input text, containing three theories.

%BEGIN LATEX
\begin{figure}
\centering
%END LATEX
\begin{whycode}
theory List
  type list 'a = Nil | Cons 'a (list 'a)
end

theory Length
  use import List
  use import int.Int

  function length (l : list 'a) : int =
    match l with
    | Nil      -> 0
    | Cons _ r -> 1 + length r
    end

  lemma Length_nonnegative : forall l:list 'a. length l >= 0
end

theory Sorted
  use import List
  use import int.Int

  inductive sorted (list int) =
    | Sorted_Nil :
        sorted Nil
    | Sorted_One :
        forall x:int. sorted (Cons x Nil)
    | Sorted_Two :
        forall x y : int, l : list int.
        x <= y -> sorted (Cons y l) -> sorted (Cons x (Cons y l))
end
\end{whycode}
%BEGIN LATEX
\vspace*{-1em}%\hrulefill
\caption{Example of \why text}
\label{fig:tutorial1}
\end{figure}
%END LATEX

The first theory, \texttt{List},
declares a new algebraic type for polymorphic lists, \texttt{list 'a}.
As in ML, \texttt{'a} stands for a type variable.
The type \texttt{list 'a} has two constructors, \texttt{Nil} and
\texttt{Cons}. Both constructors can be used as usual function
symbols, respectively of type \texttt{list 'a} and \texttt{'a
  $\times$ list 'a $\rightarrow$ list 'a}.
We deliberately make this theory that short, for reasons which will be
discussed later.

The next theory, \texttt{Length}, introduces the notion of list
length. The \texttt{use import List} command indicates that this new
theory may refer to symbols from theory \texttt{List}. These symbols
are accessible in a qualified form, such as \texttt{List.list} or
\texttt{List.Cons}. The \texttt{import} qualifier additionally allows
us to use them without qualification.

Similarly, the next command \texttt{use import int.Int} adds to our
context the theory \texttt{int.Int} from the standard library. The
prefix \texttt{int} indicates the file in the standard library
containing theory \texttt{Int}. Theories referred to without prefix
either appear earlier in the current file, \eg\ \texttt{List}, or are
predefined.

The next declaration defines a recursive function, \texttt{length},
which computes the length of a list. The \texttt{function} and
\texttt{predicate} keywords are used to introduce function and
predicate symbols, respectively.
\why checks every recursive, or mutually recursive, definition for
termination. Basically, we require a lexicographic and structural
descent for every recursive call for some reordering of arguments.
Notice that matching must be exhaustive and that every \texttt{match}
expression must be terminated by the \texttt{end} keyword.

Despite using higher-order ``curried'' syntax, \why does not permit
partial application: function and predicate arities must be respected.

The last declaration in theory \texttt{Length} is a lemma stating that
the length of a list is non-negative.

The third theory, \texttt{Sorted}, demonstrates the definition of
an inductive predicate. Every such definition is a list of clauses:
universally closed implications where the consequent is an instance
of the defined predicate. Moreover, the defined predicate may only
occur in positive positions in the antecedent. For example, a clause:
\begin{whycode}
  | Sorted_Bad :
      forall x y : int, l : list int.
      (sorted (Cons y l) -> y > x) -> sorted (Cons x (Cons y l))
\end{whycode}
would not be allowed. This positivity condition assures the logical
soundness of an inductive definition.

Note that the type signature of \lstinline{sorted} predicate does not
include the name of a parameter (see \texttt{l} in the definition
of \texttt{length}): it is unused and therefore optional.

\section{Example 1 (continued): Lists and Abstract Orderings}

In the previous section we have seen how a theory can reuse the
declarations of another theory, coming either from the same input
text or from the library. Another way to referring to a theory is
by ``cloning''. A \texttt{clone} declaration constructs a local
copy of the cloned theory, possibly instantiating some of its
abstract (\ie declared but not defined) symbols.

%BEGIN LATEX
Consider the continued example in Figure~\ref{fig:tutorial2}.
%END LATEX
%HEVEA Consider the continued example below.
We write an abstract theory of partial orders, declaring an
abstract type \texttt{t} and an abstract binary predicate
\texttt{<=}. Notice that an infix operation must be enclosed
in parentheses when used outside a term. We also specify
three axioms of a partial order.

%BEGIN LATEX
\begin{figure}
\centering
%END LATEX
\begin{whycode}
theory Order
  type t
  predicate (<=) t t

  axiom Le_refl : forall x : t. x <= x
  axiom Le_asym : forall x y : t. x <= y -> y <= x -> x = y
  axiom Le_trans: forall x y z : t. x <= y -> y <= z -> x <= z
end

theory SortedGen
  use import List
  clone import Order as O

  inductive sorted (l : list t) =
    | Sorted_Nil :
        sorted Nil
    | Sorted_One :
        forall x:t. sorted (Cons x Nil)
    | Sorted_Two :
        forall x y : t, l : list t.
        x <= y -> sorted (Cons y l) -> sorted (Cons x (Cons y l))
end

theory SortedIntList
  use import int.Int
  clone SortedGen with type O.t = int, predicate O.(<=) = (<=)
end
\end{whycode}
%BEGIN LATEX
\vspace*{-1em}%\hrulefill
\caption{Example of \why text (continued)}
\label{fig:tutorial2}
\end{figure}
%END LATEX

There is little value in \texttt{use}'ing such a theory: this
would constrain us to stay with the type \texttt{t}. However,
we can construct an instance of theory \texttt{Order} for
any suitable type and predicate. Moreover, we can build some
further abstract theories using order, and then instantiate
those theories.

Consider theory \texttt{SortedGen}. In the beginning, we
\texttt{use} the earlier theory \texttt{List}. Then we
make a simple \texttt{clone} theory \texttt{Order}.
This is pretty much equivalent to copy-pasting every
declaration from \texttt{Order} to \texttt{SortedGen};
the only difference is that \why traces the history
of cloning and transformations and drivers often make
use of it (see Section~\ref{sec:drivers}).

Notice an important difference between \texttt{use}
and \texttt{clone}. If we \texttt{use} a theory, say
\texttt{List}, twice (directly or indirectly: \eg by
making \texttt{use} of both \texttt{Length} and
\texttt{Sorted}), there is no duplication: there is
still only one type of lists and a unique pair
of constructors. On the contrary, when we \texttt{clone}
a theory, we create a local copy of every cloned
declaration, and the newly created symbols, despite
having the same names, are different from their originals.

Returning to the example, we finish theory \texttt{SortedGen}
with a familiar definition of predicate \texttt{sorted};
this time we use the abstract order on the values of type
\texttt{t}.

Now, we can instantiate theory \texttt{SortedGen} to any
ordered type, without having to retype the definition of
\texttt{sorted}. For example, theory \texttt{SortedIntList}
makes \texttt{clone} of \texttt{SortedGen} (\ie copies its
declarations) substituting type \texttt{int} for type
\texttt{O.t} of \texttt{SortedGen} and the default order
on integers for predicate \texttt{O.(<=)}. \why will
control that the result of cloning is well-typed.

Several remarks ought to be made here. First of all, why should
we clone theory \texttt{Order} in \texttt{SortedGen} if we make
no instantiation? Couldn't we write \texttt{use import Order as O}
instead? The answer is no, we could not. When cloning a theory,
we only can instantiate the symbols declared locally in this theory,
not the symbols imported with \texttt{use}. Therefore, we create
a local copy of \texttt{Order} in \texttt{SortedGen} to be able
to instantiate \texttt{t} and \texttt{(<=)} later.

Secondly, when we instantiate an abstract symbol, its declaration
is not copied from the theory being cloned. Thus, we will not create
a second declaration of type \texttt{int} in \texttt{SortedIntList}.

The mechanism of cloning bears some resemblance to modules and functors
of ML-like languages. Unlike those languages, \why makes no distinction
between modules and module signatures, modules and functors. Any \why
theory can be \texttt{use}'d directly or instantiated in any of its
abstract symbols.

The command-line tool \texttt{why3} (described in
Section~\ref{sec:batch}), allows us to see the effect of cloning.
If the input file containing our example is called \texttt{lists.why},
we can launch the following command:
\begin{verbatim}
> why3 lists.why -T SortedIntList
\end{verbatim}
to see the resulting theory \texttt{SortedIntList}:
\begin{whycode}
theory SortedIntList
  (* use BuiltIn *)
  (* use Int *)
  (* use List *)

  axiom Le_refl : forall x:int. x <= x
  axiom Le_asym : forall x:int, y:int. x <= y -> y <= x -> x = y
  axiom Le_trans : forall x:int, y:int, z:int. x <= y -> y <= z
    -> x <= z

  (* clone Order with type t = int, predicate (<=) = (<=),
    prop Le_trans1 = Le_trans, prop Le_asym1 = Le_asym,
    prop Le_refl1 = Le_refl *)

  inductive sorted (list int) =
    | Sorted_Nil : sorted (Nil:list int)
    | Sorted_One : forall x:int. sorted (Cons x (Nil:list int))
    | Sorted_Two : forall x:int, y:int, l:list int. x <= y ->
        sorted (Cons y l) -> sorted (Cons x (Cons y l))

  (* clone SortedGen with type t1 = int, predicate sorted1 = sorted,
    predicate (<=) = (<=), prop Sorted_Two1 = Sorted_Two,
    prop Sorted_One1 = Sorted_One, prop Sorted_Nil1 = Sorted_Nil,
    prop Le_trans2 = Le_trans, prop Le_asym2 = Le_asym,
    prop Le_refl2 = Le_refl *)
end
\end{whycode}

In conclusion, let us briefly explain the concept of namespaces
in \why. Both \texttt{use} and \texttt{clone} instructions can
be used in three forms (the examples below are given for \texttt{use},
the semantics for \texttt{clone} is the same):
\begin{itemize}
\item \texttt{use List as L} --- every symbol $s$ of theory \texttt{List}
is accessible under the name \texttt{L.$s$}. The \texttt{as L} part is
optional, if it is omitted, the name of the symbol is \texttt{List.$s$}.

\item \texttt{use import List as L} --- every symbol $s$ from
\texttt{List} is accessible under the name \texttt{L.$s$}. It is also
accessible simply as \texttt{$s$}, but only up to the end of the current
namespace, \eg the current theory. If the current theory, that is the
one making \texttt{use}, is later used under the name \texttt{T},
the name of the symbol would be \texttt{T.L.$s$}. (This is why we
could refer directly to the symbols of \texttt{Order} in theory
\texttt{SortedGen}, but had to qualify them with \texttt{O.}
in \texttt{SortedIntList}.)
As in the previous case, \texttt{as L} part is optional.

\item \texttt{use export List} --- every symbol $s$ from \texttt{List}
is accessible simply as \texttt{$s$}. If the current theory
is later used under the name \texttt{T}, the name of the symbol
would be \texttt{T.$s$}.
\end{itemize}

\why allows to open new namespaces explicitly in the text. In particular,
the instruction ``\texttt{clone import Order as O}'' can be equivalently
written as:
\begin{whycode}
namespace import O
  clone export Order
end
\end{whycode}
However, since \why favors short theories over long and complex ones,
this feature is rarely used.

\section{Example 2: Einstein's Problem}
\index{Einstein's logic problem}

We now consider another, slightly more complex example: how to use \why
to solve a little puzzle known as ``Einstein's logic
problem''.%
%BEGIN LATEX
\footnote{This \why example was contributed by St\'ephane Lescuyer.}
%END LATEX
%HEVEA {} (This \why example was contributed by St\'ephane Lescuyer.)
The code given below is available in the source distribution in
directory \verb|examples/logic/einstein.why|.

The problem is stated as follows. Five persons, of five
different nationalities, live in five houses in a row, all
painted with different colors.
These five persons own different pets, drink different beverages and
smoke different brands of cigars.
We are given the following information:
\begin{itemize}
\item The Englishman lives in a red house;

\item The Swede has dogs;

\item The Dane drinks tea;

\item The green house is on the left of the white one;

\item The green house's owner drinks coffee;

\item The person who smokes Pall Mall has birds;

\item The yellow house's owner smokes Dunhill;

\item In the house in the center lives someone who drinks milk;

\item The Norwegian lives in the first house;

\item The man who smokes Blends lives next to the one who has cats;

\item The man who owns a horse lives next to the one who smokes Dunhills;

\item The man who smokes Blue Masters drinks beer;

\item The German smokes Prince;

\item The Norwegian lives next to the blue house;

\item The man who smokes Blends has a neighbour who drinks water.
\end{itemize}
The question is: what is the nationality of the fish's owner?

We start by introducing a general-purpose theory defining the notion
of \emph{bijection}, as two abstract types together with two functions from
one to the other and two axioms stating that these functions are
inverse of each other.
\begin{whycode}
theory Bijection
  type t
  type u

  function of t : u
  function to_ u : t

  axiom To_of : forall x : t. to_ (of x) = x
  axiom Of_to : forall y : u. of (to_ y) = y
end
\end{whycode}

We now start a new theory, \texttt{Einstein}, which will contain all
the individuals of the problem.
\begin{whycode}
theory Einstein "Einstein's problem"
\end{whycode}
First we introduce enumeration types for houses, colors, persons,
drinks, cigars and pets.
\begin{whycode}
  type house  = H1 | H2 | H3 | H4 | H5
  type color  = Blue | Green | Red | White | Yellow
  type person = Dane | Englishman | German | Norwegian | Swede
  type drink  = Beer | Coffee | Milk | Tea | Water
  type cigar  = Blend | BlueMaster | Dunhill | PallMall | Prince
  type pet    = Birds | Cats | Dogs | Fish | Horse
\end{whycode}
We now express that each house is associated bijectively to a color,
by cloning the \texttt{Bijection} theory appropriately.
\begin{whycode}
  clone Bijection as Color with type t = house, type u = color
\end{whycode}
It introduces two functions, namely \texttt{Color.of} and
\texttt{Color.to\_}, from houses to colors and colors to houses,
respectively, and the two axioms relating them.
Similarly, we express that each house is associated bijectively to a
person
\begin{whycode}
  clone Bijection as Owner with type t = house, type u = person
\end{whycode}
and that drinks, cigars and pets are all associated bijectively to persons:
\begin{whycode}
  clone Bijection as Drink with type t = person, type u = drink
  clone Bijection as Cigar with type t = person, type u = cigar
  clone Bijection as Pet   with type t = person, type u = pet
\end{whycode}
Next we need a way to state that a person lives next to another. We
first define a predicate \texttt{leftof} over two houses.
\begin{whycode}
  predicate leftof (h1 h2 : house) =
    match h1, h2 with
    | H1, H2
    | H2, H3
    | H3, H4
    | H4, H5 -> true
    | _      -> false
    end
\end{whycode}
Note how we advantageously used pattern matching, with an or-pattern
for the four positive cases and a universal pattern for the remaining
21 cases. It is then immediate to define a \texttt{neighbour}
predicate over two houses, which completes theory \texttt{Einstein}.
\begin{whycode}
  predicate rightof (h1 h2 : house) =
    leftof h2 h1
  predicate neighbour (h1 h2 : house) =
    leftof h1 h2 \/ rightof h1 h2
end
\end{whycode}

The next theory contains the 15 hypotheses. It starts by importing
theory \texttt{Einstein}.
\begin{whycode}
theory EinsteinHints "Hints"
  use import Einstein
\end{whycode}
Then each hypothesis is stated in terms of \texttt{to\_} and \texttt{of}
functions. For instance, the hypothesis ``The Englishman lives in a
red house'' is declared as the following axiom.
\begin{whycode}
  axiom Hint1: Color.of (Owner.to_ Englishman) = Red
\end{whycode}
And so on for all other hypotheses, up to
``The man who smokes Blends has a neighbour who drinks water'', which completes
this theory.
\begin{whycode}
  ...
  axiom Hint15:
    neighbour (Owner.to_ (Cigar.to_ Blend)) (Owner.to_ (Drink.to_ Water))
end
\end{whycode}
Finally, we declare the goal in the fourth theory:
\begin{whycode}
theory Problem "Goal of Einstein's problem"
  use import Einstein
  use import EinsteinHints

  goal G: Pet.to_ Fish = German
end
\end{whycode}
and we are ready to use \why to discharge this goal with any prover
of our choice.

%%% Local Variables:
%%% mode: latex
%%% TeX-PDF-mode: t
%%% TeX-master: "manual"
%%% End:


% \chapter{Tools}
\label{chap:tools}

\section{Why3 command line tool}

\section{IDE}

\section{Other tools}

\begin{itemize}
\item why-config
\item why-bench
\end{itemize}



%%% Local Variables:
%%% mode: latex
%%% TeX-PDF-mode: t
%%% TeX-master: "manual"
%%% End:


\chapter{The Why3ML Programming Language}
\label{chap:whyml}

This chapter describes the \whyml\ programming language.

% simple message = all Why3 language + mutable record fields + model

% files .mlw
% command line
% tutorial with an example (same fringe)

%%% Local Variables:
%%% mode: latex
%%% TeX-PDF-mode: t
%%% TeX-master: "manual"
%%% End:


\chapter{The \why API}
\label{chap:api}

This chapter is a tutorial for the users who want to link their own
OCaml code with the \why library. We progressively introduce the way
one can use the library to build terms, formulas, theories, proof
tasks, call external provers on tasks, and apply transformations on
tasks. The complete documentation for API calls is given
at URL~\url{http://why3.lri.fr/api/}.

We assume the reader has a fair knowledge of the OCaml
language. Notice that the \why library must be installed, see
Section~\ref{sec:installlib}. The OCaml code given below is available in
the source distribution as \url{examples/use_api.ml}.


\section{Building Propositional Formulas}

The first step is to know how to build propositional formulas. The
module \texttt{Term} gives a few functions for building these. Here is
a piece of OCaml code for building the formula $true \lor false$.
\begin{verbatim}
(* opening the Why3 library *)
open Why3

(* a ground propositional goal: true or false *)
let fmla_true : Term.term = Term.t_true
let fmla_false : Term.term = Term.t_false
let fmla1 : Term.term = Term.t_or fmla_true fmla_false
\end{verbatim}
The library uses the common type \texttt{term} both for terms
(i.e.~expressions that produce a value of some particular type)
and formulas (i.e.~boolean-valued expressions).
% To distinguish terms from formulas, one can look at the
% \texttt{t_ty} field of the \texttt{term} record: in formulas,
% this field has the value \texttt{None}, and in terms,
% \texttt{Some t}, where \texttt{t} is of type \texttt{Ty.ty}.

Such a formula can be printed using the module \texttt{Pretty}
providing pretty-printers.
\begin{verbatim}
(* printing it *)
open Format
let () = printf "@[formula 1 is:@ %a@]@." Pretty.print_term fmla1
\end{verbatim}

Assuming the lines above are written in a file \texttt{f.ml}, it can
be compiled using
\begin{verbatim}
ocamlc str.cma unix.cma nums.cma dynlink.cma \
        -I +ocamlgraph -I +why3 graph.cma why.cma f.ml -o f
\end{verbatim}
Running the generated executable \texttt{f} results in the following output.
\begin{verbatim}
formula 1 is: true \/ false
\end{verbatim}

Let's now build a formula with propositional variables: $A \land B
\rightarrow A$. Propositional variables must be declared first before
using them in formulas. This is done as follows.
\begin{verbatim}
let prop_var_A : Term.lsymbol =
  Term.create_psymbol (Ident.id_fresh "A") []
let prop_var_B : Term.lsymbol =
  Term.create_psymbol (Ident.id_fresh "B") []
\end{verbatim}
The type \texttt{lsymbol} is the type of function and predicate symbols (which
we call logic symbols for brevity). Then the atoms $A$ and $B$ must be built
by the general function for applying a predicate symbol to a list of terms.
Here we just need the empty list of arguments.
\begin{verbatim}
let atom_A : Term.term = Term.ps_app prop_var_A []
let atom_B : Term.term = Term.ps_app prop_var_B []
let fmla2 : Term.term =
  Term.t_implies (Term.t_and atom_A atom_B) atom_A
let () = printf "@[formula 2 is:@ %a@]@." Pretty.print_term fmla2
\end{verbatim}

As expected, the output is as follows.
\begin{verbatim}
formula 2 is: A /\ B -> A
\end{verbatim}
Notice that the concrete syntax of \why forbids function and predicate
names to start with a capital letter (except for the algebraic type
constructors which must start with one). This constraint is not enforced
when building those directly using library calls.

\section{Building Tasks}

Let's see how we can call a prover to prove a formula. As said in
previous chapters, a prover must be given a task, so we need to build
tasks from our formulas. Task can be build incrementally from an empty
task by adding declaration to it, using the functions
\texttt{add\_*\_decl} of module \texttt{Task}. For the formula $true \lor
false$ above, this is done as follows.
\begin{verbatim}
let task1 : Task.task = None (* empty task *)
let goal_id1 : Decl.prsymbol =
  Decl.create_prsymbol (Ident.id_fresh "goal1")
let task1 : Task.task =
  Task.add_prop_decl task1 Decl.Pgoal goal_id1 fmla1
\end{verbatim}
To make the formula a goal, we must give a name to it, here "goal1". A
goal name has type \texttt{prsymbol}, for identifiers denoting
propositions in a theory or a task. Notice again that the concrete
syntax of \why requires these symbols to be capitalized, but it is not
mandatory when using the library. The second argument of
\texttt{add\_prop\_decl} is the kind of the proposition:
\texttt{Paxiom}, \texttt{Plemma} or \texttt{Pgoal}
(notice, however, that lemmas are not allowed in tasks
and can only be used in theories).


Once a task is built, it can be printed.
\begin{verbatim}
(* printing the task *)
let () = printf "@[task 1 is:@\n%a@]@." Pretty.print_task task1
\end{verbatim}

The task for our second formula is a bit more complex to build, because
the variables A and B must be added as logic declarations in the task.
\begin{verbatim}
(* task for formula 2 *)
let task2 = None
let task2 = Task.add_logic_decl task2 [prop_var_A, None]
let task2 = Task.add_logic_decl task2 [prop_var_B, None]
let goal_id2 = Decl.create_prsymbol (Ident.id_fresh "goal2")
let task2 = Task.add_prop_decl task2 Decl.Pgoal goal_id2 fmla2
let () = printf "@[task 2 is:@\n%a@]@." Pretty.print_task task2
\end{verbatim}
The argument \texttt{None} is the declarations of logic symbols means
that they do not have any definition.

Execution of our OCaml program now outputs:
\begin{verbatim}
task 1 is:
theory Task
  goal Goal1 : true \/ false
end

task 2 is:
theory Task
  predicate A

  predicate B

  goal Goal2 : A /\ B -> A
end
\end{verbatim}

\section{Calling External Provers}

To call an external prover, we need to access the Why configuration
file \texttt{why3.conf}, as it was built using the \texttt{why3config}
command line tool or the \textsf{Detect Provers} menu of the graphical
IDE. The following API calls allow to access the content of this
configuration file.
\begin{verbatim}
(* reads the config file *)
let config : Whyconf.config = Whyconf.read_config None
(* the [main] section of the config file *)
let main : Whyconf.main = Whyconf.get_main config
(* all the provers detected, from the config file *)
let provers : Whyconf.config_prover Util.Mstr.t =
  Whyconf.get_provers config
\end{verbatim}
The type \texttt{'a Util.Mstr.t} is a map indexed by strings. This map
can provide the set of existing provers. In the following, we directly
attempt to access the prover Alt-Ergo, which is known to be identified
with id \texttt{"alt-ergo"}.
\begin{verbatim}
(* the [prover alt-ergo] section of the config file *)
let alt_ergo : Whyconf.config_prover =
  try
    Util.Mstr.find "alt-ergo" provers
  with Not_found ->
    eprintf "Prover alt-ergo not installed or not configured@.";
    exit 0
\end{verbatim}

The next step is to obtain the driver associated to this prover. A
driver typically depends on the standard theories so these should be
loaded first.
\begin{verbatim}
(* builds the environment from the [loadpath] *)
let env : Env.env =
  Env.create_env (Whyconf.loadpath main)
(* loading the Alt-Ergo driver *)
let alt_ergo_driver : Driver.driver =
  try
    Driver.load_driver env alt_ergo.Whyconf.driver
  with e ->
    eprintf "Failed to load driver for alt-ergo: %a@."
      Exn_printer.exn_printer e;
    exit 1
\end{verbatim}

We are now ready to call the prover on the tasks. This is done by a
function call that launches the external executable and waits for its
termination. Here is a simple way to proceed:
\begin{verbatim}
(* calls Alt-Ergo *)
let result1 : Call_provers.prover_result =
  Call_provers.wait_on_call
    (Driver.prove_task ~command:alt_ergo.Whyconf.command
    alt_ergo_driver task1 ()) ()
(* prints Alt-Ergo answer *)
let () = printf "@[On task 1, alt-ergo answers %a@]@."
  Call_provers.print_prover_result result1
\end{verbatim}
This way to call a prover is in general too naive, since it may never
return if the prover runs without time limit. The function
\texttt{prove\_task} has two optional parameters: \texttt{timelimit}
is the maximum allowed running time in seconds, and \texttt{memlimit}
is the maximum allowed memory in megabytes.  The type
\texttt{prover\_result} is a record with three fields:
\begin{itemize}
\item \texttt{pr\_answer}: the prover answer, explained below;
\item \texttt{pr\_output}: the output of the prover, i.e. both
  standard output and the standard error of the process
  (a redirection in \texttt{why3.conf} is required);
\item \texttt{pr\_time} : the time taken by the prover, in seconds.
\end{itemize}
A \texttt{pr\_answer} is a sum of several kind of answers:
\begin{itemize}
\item \texttt{Valid}: the task is valid according to the prover.
\item \texttt{Invalid}: the task is invalid.
\item \texttt{Timeout}: the prover exceeds the time or memory limit.
\item \texttt{Unknown} $msg$: the prover can't determine if the task
  is valid; the string parameter $msg$ indicates some extra
  information.
\item \texttt{Failure} $msg$: the prover reports a failure, i.e.~it
  was unable to read correctly its input task.
\item \texttt{HighFailure}: an error occurred while trying to call the
  prover, or the prover answer was not understood (i.e.~none of the
  given regular expressions in the driver file matches the output
  of the prover).
\end{itemize}
Here is thus another way of calling the Alt-Ergo prover, on our second
task.
\begin{verbatim}
let result2 : Call_provers.prover_result =
   Call_provers.wait_on_call
    (Driver.prove_task ~command:alt_ergo.Whyconf.command
    ~timelimit:10
    alt_ergo_driver task2 ()) ()

let () =
  printf "@[On task 2, alt-ergo answers %a in %5.2f seconds@."
    Call_provers.print_prover_answer
    result1.Call_provers.pr_answer
    result1.Call_provers.pr_time
\end{verbatim}
The output of our program is now as follows.
\begin{verbatim}
On task 1, alt-ergo answers Valid (0.01s)
On task 2, alt-ergo answers Valid in  0.01 seconds
\end{verbatim}

\section{Building Terms}

An important feature of the functions for building terms and formulas
is that they statically guarantee that only well-typed terms can be
constructed.

Here is the way we build the formula $2+2=4$. The main difficulty is to
access the internal identifier for addition: it must be retrieved from
the standard theory \texttt{Int} of the file \texttt{int.why} (see
Chap~\ref{chap:library}).
\begin{verbatim}
let two : Term.term = Term.t_const (Term.ConstInt "2")
let four : Term.term = Term.t_const (Term.ConstInt "4")
let int_theory : Theory.theory =
  Env.find_theory env ["int"] "Int"
let plus_symbol : Term.lsymbol =
  Theory.ns_find_ls int_theory.Theory.th_export ["infix +"]
let two_plus_two : Term.term =
  Term.t_app_infer plus_symbol [two;two]
let fmla3 : Term.term = Term.t_equ two_plus_two four
\end{verbatim}
An important point to notice as that when building the application of
$+$ to the arguments, it is checked that the types are correct. Indeed
the constructor \texttt{t\_app\_infer} infers the type of the resulting
term. One could also provide the expected type as follows.
\begin{verbatim}
let two_plus_two : Term.term =
  Term.fs_app plus_symbol [two;two] Ty.ty_int
\end{verbatim}

When building a task with this formula, we need to declare that we use
theory \texttt{Int}:
\begin{verbatim}
let task3 = None
let task3 = Task.use_export task3 int_theory
let goal_id3 = Decl.create_prsymbol (Ident.id_fresh "goal3")
let task3 = Task.add_prop_decl task3 Decl.Pgoal goal_id3 fmla3
\end{verbatim}

\section{Building Quantified Formulas}

To illustrate how to build quantified formulas, let us consider
the formula $\forall x:int. x*x \geq 0$. The first step is to
obtain the symbols from \texttt{Int}.
\begin{verbatim}
let zero : Term.term = Term.t_const (Term.ConstInt "0")
let mult_symbol : Term.lsymbol =
  Theory.ns_find_ls int_theory.Theory.th_export ["infix *"]
let ge_symbol : Term.lsymbol =
  Theory.ns_find_ls int_theory.Theory.th_export ["infix >="]
\end{verbatim}
The next step is to introduce the variable $x$ with the type int.
\begin{verbatim}
let var_x : Term.vsymbol =
  Term.create_vsymbol (Ident.id_fresh "x") Ty.ty_int
\end{verbatim}
The formula $x*x \geq 0$ is obtained as in the previous example.
\begin{verbatim}
let x : Term.term = Term.t_var var_x
let x_times_x : Term.term = Term.t_app_infer mult_symbol [x;x]
let fmla4_aux : Term.term = Term.ps_app ge_symbol [x_times_x;zero]
\end{verbatim}
To quantify on $x$, we use the appropriate smart constructor as follows.
\begin{verbatim}
let fmla4 : Term.term = Term.t_forall_close [var_x] [] fmla4_aux
\end{verbatim}

\section{Building Theories}

[TO BE COMPLETED]

\section{Applying transformations}

[TO BE COMPLETED]

\section{Writing new functions on term}

[TO BE COMPLETED]
% pattern-matching on terms, opening a quantifier




%%% Local Variables:
%%% mode: latex
%%% TeX-PDF-mode: t
%%% TeX-master: "manual"
%%% End:


\part{Reference Manual}


\chapter{Compilation, Installation}
\label{sec:install}
%HEVEA\cutname{install.html}

In short, installation proceeds as follows.
\begin{flushleft}\ttfamily
  ./configure\\
  make\\
  make install \mbox{\rmfamily (as super-user)}
\end{flushleft}

\section{Installation Instructions from Source Distribution}

After unpacking the distribution, go to the newly created directory
\texttt{why3-\whyversion}. Compilation must start with a
configuration phase which is run as 
\begin{verbatim}
./configure
\end{verbatim}
This analyzes your current configuration and checks if requirements hold.
Compilation requires:
\begin{itemize}
\item The Objective Caml compiler, version 3.11.2 or higher. It is
  available as a binary package for most Unix distributions. For
  Debian-based Linux distributions, you can install the packages
\begin{verbatim}
ocaml ocaml-native-compilers
\end{verbatim}
It is also installable from sources, downloadable from the site
\url{http://caml.inria.fr/ocaml/}
\end{itemize}

\noindent
For some of the \why tools, additional OCaml libraries are needed:
\begin{itemize}
\item For the graphical interface, the Lablgtk2 library is needed.
  It provides OCaml
  bindings of the gtk2 graphical library. For Debian-based Linux
  distributions, you can install the packages
\begin{verbatim}
liblablgtk2-ocaml-dev liblablgtksourceview2-ocaml-dev
\end{verbatim}
It is also installable from sources, available from the site
\url{http://wwwfun.kurims.kyoto-u.ac.jp/soft/olabl/lablgtk.html}

\item For \texttt{why3 bench}, the OCaml bindings of the sqlite3 library
are needed.
For Debian-based Linux distributions, you can install the package
\begin{verbatim}
libsqlite3-ocaml-dev
\end{verbatim}
It is also installable from sources, available from the site
\url{http://ocaml.info/home/ocaml_sources.html#ocaml-sqlite3}
\end{itemize}


If you want to use the specific Coq features, \ie the Coq tactic
(Section~\ref{sec:coqtactic}) and Coq realizations
(Section~\ref{sec:realizations}), then Coq has to be installed before
\why. Look at the summary printed at the end of the configuration
script to check if Coq has been detected properly. Similarly, for
using PVS (Section~\ref{sec:pvs}) or Isabelle (Section~\ref{sec:isabelle}) to discharge proofs, PVS and Isabelle must be
installed before \why. You should check that those proof assistants
are correctly detected by the configure script.

When configuration is finished, you can compile \why.
\begin{verbatim}
make
\end{verbatim}
Installation is performed (as super-user if needed) using
\begin{verbatim}
make install
\end{verbatim}
Installation can be tested as follows:
\begin{enumerate}
\item install some external provers (see~Section~\ref{provers} below)
\item run \verb|why3 config --detect|
\item run some examples from the distribution, \eg you should
obtain the following:
\begin{verbatim}
$ cd examples
$ why3 replay logic/scottish-private-club
Opening session... done
Progress: 4/4
 1/1
Everything OK.
$ why3 replay programs/same_fringe
Opening session... done
Progress: 12/12
 3/3
Everything OK.
\end{verbatim}
\end{enumerate}

\section{Local Use, Without Installation}

It is not mandatory to install \why into system directories.
\why can be configured and compiled for local use as follows:
\begin{verbatim}
./configure --enable-local
make
\end{verbatim}
The \why executables are then available in the subdirectory
\texttt{bin/}. This directory can be added in your \texttt{PATH}.

\section{Installation of the \why API}
\label{sec:installlib}\index{API}

By default, the \why API is not installed. It can be installed using
\begin{flushleft}\ttfamily
make byte opt \\
make install-lib \mbox{\rmfamily (as super-user)}
\end{flushleft}

\section{Installation of External Provers}
\label{provers}

\why can use a wide range of external theorem provers. These need to
be installed separately, and then \why needs to be configured to use
them. There is no need to install automatic provers, \eg SMT solvers,
before compiling and installing \why.

For installation of external provers, please refer to the specific
section about provers on the Web page \url{http://why3.lri.fr/}.

For configuring \why to use the provers, follow instructions given in
Section~\ref{sec:why3config}.

\section{Multiple Versions of the Same Prover}

\why is able to use several versions of the same
prover, \eg it can use both CVC3 2.2 and CVC3 2.4.1 at the same time.
The automatic detection of provers looks for typical names for their
executable command, \eg \texttt{cvc3} for CVC3. However, if you
install several version of the same prover it is likely that you would
use specialized executable names, such as \texttt{cvc3-2.2} or
\texttt{cvc3-2.4.1}. To allow the \why detection process to recognize
these, you can use the option \verb|--add-prover| with the
\texttt{config} command, \eg
\index{add-prover@\verb+--add-prover+}
\begin{verbatim}
why3 config --detect --add-prover cvc3-2.4 /usr/local/bin/cvc3-2.4.1
\end{verbatim}
the first argument (here \verb|cvc3-2.4|) must be one of the class of
provers known in the file \verb|provers-detection-data.conf| typically
located in \verb|/usr/local/share/why3| after installation. See
Appendix~\ref{sec:proverdetecttiondata} for details.


\section{Session Update after Prover Upgrade}
\label{sec:uninstalledprovers}

If you happen to upgrade a prover, \eg installing CVC3 2.4.1 in place
of CVC3 2.2, then the proof sessions formerly recorded will still
refer to the old version of the prover. If you open one such a session
with the GUI, and replay the proofs, you will be asked to choose
between 3 options:
\begin{itemize}
\item Keep the former proofs as they are. They will be marked as
  ``archived''.
\item Upgrade the former proofs to an installed prover (typically a
  upgraded version). The corresponding proof attempts will become
  attached to this new prover, and marked as obsolete,
  to make their replay mandatory.
\item Copy the former proofs to an installed prover. This is a
  combination of the actions above: each proof attempt is duplicated,
  one with the former prover marked as archived, and one for the new
  prover marked as obsolete.
\end{itemize}

Notice that if the prover under consideration is an interactive one, then
the copy option will duplicate also the edited proof scripts, whereas
the upgrade-without-archive option will just reuse the former proof scripts.

Your choice between the three options above will be recorded, one for
each prover, in the \why configuration file. Within the GUI, you can
discard these choices via the \textsf{Preferences} dialog.

Outside the GUI, the prover upgrades are handled as follows. The
\texttt{replay} command will just ignore proof attempts marked as
archived\index{archived}.
Conversely, a non-archived proof attempt with a non-existent
prover will be reported as a replay failure. The
\texttt{session} command performs move or copy operations on
proof attempts in a fine-grained way, using filters, as detailed in
Section~\ref{sec:why3session}.


% pour l'instant on ne documente pas
% {que devient l'option -to-known-prover de why3session ?
%   (d'ailleurs documenté en tant que --convert-unknown ??) Pourrait-on
%   permettre à why3session d'appliquer les choix d'association
%   vieux-prover/nouveau-prouveur stockés par l'IDE ?}


%%% Local Variables:
%%% mode: latex
%%% TeX-PDF-mode: t
%%% TeX-master: "manual"
%%% End:


\chapter{Reference manuals for the \why tools}
\label{chap:manpages}

\section{Compilation, Installation}
\label{sec:install}

Compilation of \why must start with a configuration phase which is run as
\begin{verbatim}
./configure
\end{verbatim}
This analyzes your current configuration and checks if requirements hold.
Compilation requires:
\begin{itemize}
\item The Objective Caml compiler, version 3.10 or higher. It is
  available as a binary package for most Unix distributions. For
  Debian-based Linux distributions, you can install the packages
\begin{verbatim}
ocaml ocaml-native-compilers
\end{verbatim}
It is also installable from sources, downloadable from the site
\url{http://caml.inria.fr/ocaml/}
\end{itemize}

\noindent
For some tools, additional OCaml libraries are needed:
\begin{itemize}
\item For the IDE: the Lablgtk2 library for OCaml bindings of the gtk2
  graphical library. For Debian-based Linux distributions, you can
  install the packages
\begin{verbatim}
liblablgtk2-ocaml-dev liblablgtksourceview2-ocaml-dev
\end{verbatim}
It is also installable from sources, available from the site
\url{http://wwwfun.kurims.kyoto-u.ac.jp/soft/olabl/lablgtk.html}

\item For \texttt{why3bench}: The OCaml bindings of the sqlite3 library.
For Debian-based Linux distributions, you can install the package
\begin{verbatim}
libsqlite3-ocaml-dev
\end{verbatim}
It is also installable from sources, available from the site
\url{http://ocaml.info/home/ocaml_sources.html#ocaml-sqlite3}
\end{itemize}

When configuration is finished, you can compile \why.
\begin{verbatim}
make
\end{verbatim}
Installation is performed (as super-user if needed) using
\begin{verbatim}
make install
\end{verbatim}
Installation can be tested as follows: 
\begin{enumerate}
\item install some external provers (see~Section\ref{provers} below)
\item run \verb|why3config --detect|
\item run some examples from the distribution, \emph{e.g.} you should
obtain the following:
\begin{verbatim}
$ cd examples
$ why3replayer scottish-private-club
Info: found directory 'scottish-private-club' for the project
Opening session...[Xml warning] prolog ignored
[Reload] file '../scottish-private-club.why'
[Reload] theory 'ScottishClubProblem'
 done
Progress: 4/4
 1/1
Everything OK.
$ why3replayer programs/same_fringe
Info: found directory 'programs/same_fringe' for the project
Opening session...[Xml warning] prolog ignored
[Reload] file '../same_fringe.mlw'
[Reload] theory 'WP SameFringe'
[Reload] transformation split_goal for goal WP_parameter enum 
[Reload] transformation split_goal for goal WP_parameter eq_enum 
 done
Progress: 12/12
 3/3
Everything OK.
\end{verbatim}
\end{enumerate}

\subsection{Local use, without installation}

It is not mandatory to install \why into system directories.
\why can be configured and compiled for local use as follows:
\begin{verbatim}
./configure --enable-local
make
\end{verbatim}
The \why executables are then available in the subdirectory \texttt{bin/}.

\subsection{Installation of the \why library}
\label{sec:installlib}

By default, the \why library is not installed. It can be installed using
\begin{verbatim}
make byte opt
make install_lib
\end{verbatim}

\section{Installation of external provers}\label{provers}

\why can use a wide range of external theorem provers. These need to
be installed separately, and then \why needs to be configured to use
them. There is no need to install these provers before compiling and
installing Why.

For installation of external provers, please look at the Why provers
tips page \url{http://why.lri.fr/provers.en.html}.

For configuring \why to use the provers, follow instructions given in
Section~\ref{sec:why3config}.

\section{The \texttt{why3config} command-line tool}
\label{sec:why3config}

\why must be configured to access external provers. Typically, this is done
by running either the command line tool
\begin{verbatim}
why3config
\end{verbatim}
or using the menu
\begin{verbatim}
File/Detect provers
\end{verbatim}
of the IDE. This must be redone each time a new prover is installed.

The provers which \why attempts to detect are described in
the readable configuration file \texttt{provers-detection-data.conf}
of the \why data directory (\eg{}
\texttt{/usr/local/share/why3}). Advanced users may try to modify this
file to add support for detection of other provers. (In that case,
please consider submitting a new prover configuration on the bug
tracking system).

The result of provers detection is stored in the user's
configuration file (\verb+~/.why3.conf+ or, in the case of local
installation, \verb+why3.conf+ in Why3 sources top directory). This file
is also human-readable, and advanced users may modify it in order to
experiment with different ways of calling provers, \eg{} different
versions of the same prover, or with different options.

The provers which are typically looked for are
\begin{itemize}
\item Alt-Ergo~\cite{conchon08smt,ergo}: \url{http://alt-ergo.lri.fr}
\item CVC3~\cite{BarTin-CAV-07}: \url{http://cs.nyu.edu/acsys/cvc3/}
\item Coq~\cite{CoqArt}: \url{http://coq.inria.fr}
\item Eprover~\cite{schulz04ijcar}: \url{http://www4.informatik.tu-muenchen.de/~schulz/WORK/eprover.html}
\item Gappa~\cite{melquiond08rnc}: \url{http://gappa.gforge.inria.fr/}
\item Simplify~\cite{simplify05}: \url{http://secure.ucd.ie/products/opensource/Simplify/}
\item Spass: \url{http://www.spass-prover.org/}
\item Vampire: \url{http://www.voronkov.com/vampire.cgi}
\item VeriT: \url{http://www.verit-solver.org/}
\item Yices~\cite{DM06}: \url{http://yices.csl.sri.com/}, only versions 1.xx since versions 2.xx do not support quantifiers
\item Z3~\cite{z3}: \url{http://research.microsoft.com/en-us/um/redmond/projects/z3/}
\end{itemize}

\texttt{why3config} also detects the plugins installed in the \why
plugins directory (\eg{} \texttt{/usr/local/lib/why3/plugins}). A
plugin must register itself as a parser, a transformation or a
printer, as explained in the corresponding section.

If the user's configuration file is already present,
\texttt{why3config} will only reset unset variables to default value,
but will not try to detect provers.
The option \verb|--detect-provers| should be used to force
\why to detect again the available
provers and to replace them in the configuration file. The option
\verb|--detect-plugins| will do the same for plugins.

\section{The \texttt{why3} command-line tool}
\label{sec:why3ref}

\why is primarily used to call provers on goals contained in an
input file. By default, such a file must be written in \why language
and have the extension \texttt{.why}. However, a dynamically loaded
plugin can register a parser for some other format of logical problems,
\eg{} TPTP or SMTlib.

The \texttt{why3} tool executes the following steps:
\begin{enumerate}
\item Parse the command line and report errors if needed.
\item Read the configuration file using the priority defined in
  Section~\ref{sec:whyconffile}.
\item Load the plugins mentioned in the configuration. It will not
  stop if some plugin fails to load.
\item Parse and typecheck the given files using the correct parser in order
  to obtain a set of \why theories for each file. It uses
  the filename extension or the \verb|--format| option to choose
  among the available parsers. The \verb|--list-format| option gives
  the list of registered parsers.
\item Extract the selected goals inside each of the selected theories
  into tasks. The goals and theories are selected using the options
  \verb|-G/--goal| and \verb|-T/--theory|. The option
  \verb|-T/--theory| applies to the last file appearing on the
  command line, the option \verb|-G/--goal| applies to the last theory
  appearing on the command line. If no theories are selected in a file,
  then every theory is considered as selected. If no goals are selected
  in a theory, then every goal is considered as selected.
\item Apply the transformation requested
  with \verb|-a/--apply-transform| in their order of appearance on the
  command line. \verb|--list-transforms| list the known
  transformations, plugins can add more of them.
\item Apply the driver selected with the \verb|-D/--driver| option,
  or the driver of the prover selected with \verb|-P/--prover|
  option. \verb|--list-provers| lists the known provers, i.e.~the ones
  which appear in the configuration file.
\item If the option \verb|-P/--prover| is given, call the selected prover
  on each generated task and print the results. If the option
  \verb|-D/--driver| is given, print each generated task using
  the format specified in the selected driver.
\end{enumerate}

%\texttt{why3} calls the provers sequentially, use \texttt{why3bench} if *)
%you want to call the provers concurrently.  *)

\noindent
The provers can give the following output:
\begin{description}
\item[Valid] the goal is proved in the given context,
\item[Unknown] the prover has stopped its search,
\item[Timeout] the prover has reached the time limit,
\item[Failure] an error has occurred,
\item[Invalid] the prover knows the goal cannot be proved.
\end{description}
% \why can also be *)
% used to provide other informations : *)
% \begin{itemize} *)
% \item \texttt{print-namespace} print the namespace of the selected *)
%   theories *)
% \item TO BE COMPLETED *)
% \end{itemize} *)

The option \verb|--bisect| changes the behavior of why3. With this
option, \verb|-P/--prover| and \verb|-o/--output| must be given
and a valid goal must be selected. The last step executed by why3 is
replaced by computing a minimal set (in the great majority of the
case) of declarations that still prove the goal. Currently it does not
use any information provided by the prover, it call the prover
multiple times with reduced context. The minimal set of declarations is
then written in the prover syntax into a file located in the directory
given to the \verb|-o/--output| option.

\section{The \texttt{why3ide} GUI}
\label{sec:ideref}

The basic usage of the GUI is described by the tutorial of
Section~\ref{sec:gui}. We describe here the command-line options and
the actions of the various menus and buttons of the interface.

\subsection{Command-line options}

\begin{description}
\item[-I] $d$: adds $d$ in the load path, to search for theories.
\end{description}

\subsection{Left toolbar actions}

\begin{description}
\item[Context] The context in which the other tools below will
  apply. If ``only unproved goals'' is selected, no action will ever
  be applied to an already proved goal.  If ``all goals'', then
  actions are performed even if the goal is already proved. The second
  choice allows to compare provers on the same goal.

\item[Provers] To each detected prover corresponds to a button in this
  prover framed box. Clicking on this button starts the prover on the
  selected goal(s).

\item[Split] This splits the current goal into subgoals if it is a
  conjunction of two or more goals.

\item[Inline] If the goal is headed by a defined predicate symbol,
  expands it with this definition.

\item[Edit] Start an editor on the selected task.

  For automatic provers, this allows to see the file sent to the
  prover.

  For interactive provers, this also allows to add or modify the
  corresponding proof script. The modifications are saved, and can be
  retrieved later even if the goal was modified.

\item[Replay] Replay all obsolete proofs

  If ``only unproved goals'' is selected, only formerly successful
  proofs are rerun. If ``all goals'', then all obsolete proofs are
  rerun, successful or not.

\item[Remove] Removes a proof attempt or a transformation.

\item[Clean] Removes any unsuccessful proof attempt for which there is
  another successful proof attempt for the same goal

\item[Interrupt] Cancels all the proof attempts currently scheduled
  but not yet started.

\end{description}

\subsection{Menus}

\begin{description}
\item[Menu \textsf{File}]~
\begin{description}
\item[Add File] adds a file in the GUI.
%\item[Detect provers] runs provers auto-detection
\item[Preferences] opens a window for modifying preferred
  configuration parameters, see details below.
\item[Reload] reloads the input files from disk, and update the session state accordingly.
\item[Save session] saves current session state on disk. The policy to decide when to save the session is configurable, as described in the preferences below.
\item[Quit] exits the GUI.
\end{description}

\item[Menu \textsf{View}]~
\begin{description}
\item[Expand All] expands all the rows of the tree view.
\item[Collapse proved goals] closes all the rows of the tree view
  which are proved.
% \item[Hide proved goals] completely hides the proved rows of the tree
%   view [EXPERIMENTAL]
\end{description}

\item[Menu \textsf{Tools}]
A copy of the tools already available in the left toolbar, plus:
\begin{description}
\item[Mark as obsolete] marks all the proof as obsolete. This allows to
  replay every proofs.
\end{description}

\item[Menu \textsf{Help}]
A very short online help, and some information about this software.
\end{description}

\subsection{Preferences}

The preferences window allows you customize
\begin{itemize}
\item the default editor to use when the \textsf{Edit} button is
  pressed\footnote{This might be overridden for a specific prover. The only way
  to do that for the moment is to manually edit the configuration file.}
\item the time limit given to provers, in seconds
\item the maximal number of simultaneous provers allowed to run in parallel.
\item the policy for saving session:
  \begin{itemize}
  \item always save on exit (default): the current state of the proof session is saving on exit
  \item never save on exit: the current state of the session is never save automatically, you must use menu \textsf{File/Save session} to save when wanted
  \item ask whether to save: on exit, a popup window ask whether you
    want to save or not.
  \end{itemize}
\end{itemize}

\subsection{Structure of the database file}

The session state is stored in an XML file named
\texttt{\textsl{<dir>}/why3session.xml}, where \texttt{\textsl{<dir>}}
is the directory of the project.
The XML file follows the DTD given in \texttt{share/why3session.dtd} and reproduced below.
\verbatiminput{../share/why3session.dtd}

\section{The \texttt{why3ml} tool}

The \texttt{why3ml} tool is a layer on  top of the \why library for
generating verification conditions from \whyml programs.
The command-line of \texttt{why3ml} is identical to that of
\texttt{why3}, but also accepts files with extension \texttt{.mlw} as
input files containing \whyml modules (see Chapter~\ref{chap:whyml}
and Section~\ref{sec:syntax:whyml}). Modules are turned into
theories containing verification conditions as goals, and then
\texttt{why3ml} behaves exactly as \texttt{why3} for the remaining of
the process.
Note that files with extension \texttt{.mlw} can also be loaded in
\texttt{why3ide}.

For those who want to experiment with \whyml, many examples are provided in
\texttt{examples/programs}.

\section{The \texttt{why3bench} tool}

The \texttt{why3bench} tool adds a scheduler on top of the \why
library. \texttt{why3bench} is designed to compare various components
of automatic proofs: automatic provers, transformations, definitions
of a theory. For that goal it tries to prove predefined goals using
each component to compare. \texttt{why3bench} allows to output the
comparison in various formats:
\begin{itemize}
\item csv: the simpler and more informative format, the results are
  represented in an array, the rows corresponds to the
  compared components, the columns correspond to the result
  (Valid, Unknown, Timeout, Failure, Invalid) and the CPU time taken in seconds.
\item average: summarizes the number of the five different answers
  for each component. It also gives the average time taken.
\item timeline: for each component it gives the number of valid goals
  along the time (10 slices between 0 and the longest time a component
  takes to prove a goal)
\end{itemize}

The compared components can be defined in an \emph{rc-file},
\texttt{examples/programs/\ prgbench.rc} is such an example. More
generally a bench configuration file:
\begin{verbatim}
[probs "myprobs"]
   file = "examples/monbut.why" #relatives to the rc file
   file = "examples/monprogram.mlw"
   theory = "monprogram.T"
   goal = "monbut.T.G"

   transform = "split_goal" #applied in this order
   transform = "..."
   transform = "..."

[tools "mytools"]
   prover = cvc3
   prover = altergo
   #or only one
   driver = "..."
   command = "..."

   loadpath = "..." #added to the one in why3.conf
   loadpath = "..."

   timelimit = 30
   memlimit = 300

   use = "toto.T" #use the theory toto (allow to add metas)

   transform = "simplify_array" #only 1 to 1 transformation

[bench "mybench"]
   tools = "mytools"
   tools = ...
   probs = "myprobs"
   probs = ...
   timeline = "prgbench.time"
   average = "prgbench.avg"
   csv = "prgbench.csv"
\end{verbatim}

Such a file can define three families \texttt{tools}, \texttt{probs},
\texttt{bench}. A \texttt{tools} section defines a set of components to
compare. A \texttt{probs} section defines a set of goals on which to compare some
components. A \texttt{bench} section defines which components to
compare using which goals. It refers by name to some sections
\texttt{tools} and \texttt{probs} defined in the same file. The order
of the definitions is irrelevant. Notice that one can use
\texttt{loadpath} in a \texttt{tools} section to compare different
axiomatizations.

One can run all the bench given in one bench configuration file with
\texttt{why3bench}:
\begin{verbatim}
why3bench -B path_to_my_bench.rc
\end{verbatim}

\section{The \texttt{why3replayer} tool}
\label{sec:why3replayer}

The purpose of the \texttt{why3replayer} tool is to execute the proofs
stored in a \why session file, as the one produced by the IDE. Its
main goal is to play non-regression tests, \eg you can find in
\texttt{examples/regtests.sh} a script that runs regression tests on
all the examples.

The tool is invoked in a terminal or a script using
\begin{flushleft}\ttfamily
  why3replayer \textsl{[options] <project directory>}
\end{flushleft}
The session file \texttt{why3session.xml} stored in the given
directory is loaded and all the proofs it contains are rerun. Then,
all the differences between the information stored in the session file and
the new run are shown.

Nothing is shown when there is no change in the results, whether the
considered goal is proved or not. When all the proof
runs are done, a summary of what is proved or not is displayed using a
tree-shape pretty print, similar to the IDE tree view after doing
``Collapse proved goals''. In other words, when a goal, a theory, or a
file is fully proved, the subtree is not shown.

\paragraph{Obsolete proofs}

When some proofs stored in the session file are obsolete, the replay is
run anyway, as with the replay button in the IDE. Then, if all the
replayed proofs went OK, the session file is updated. Otherwise you have
to use the IDE to update it.

\paragraph{Exit code and options}

\begin{itemize}
\item The exit code is 0 if no difference was detected, 1 if there
  was. Other exit codes mean some failure in running the replay.
\item Option \verb|-s| suppresses the output of the final tree view.
\item Option \texttt{-I \textsl{<path>}} adds \texttt{\textsl{<path>}} to the loadpath.
\item Option \verb|-force| writes a new session file even if
  some proofs did not replay correctly.
\item Option \texttt{-smoke-detector \{none|top|deep\}} tries to detect
  if the context is self-contradicting.
\item Option \texttt{-latex \textsl{<dir>}} produces a summary of
  the replay under the form of a tabular environment in LaTeX, one
  tabular for each theory, one per file, in directory \texttt{\textsl{<dir>}}.
\item Option \texttt{-latex2 \textsl{<dir>}} produces an alternate version of
  LaTeX output, with a different layout of the tables.
% \item option \texttt{-html \textsl{<file.html>}}: produce a summary of
%   the replay in HTML syntax.
\end{itemize}

\paragraph{Smoke Detector}
The smoke detector tries to detect if the context is self-contradicting
 and, thus, that anything can be proved in this context. The smoke
 detector can't be run on outdated session and does not modify the session.
 It has three possible configurations:
 \begin{itemize}
\item \texttt{none}: Do not run the smoke detector.
\item \texttt{top}: The negation of each proved goal is sent with the
same timeout to the prover that proved the original goal.
\begin{verbatim}
Goal G : forall x:int. q x -> (p1 x \/ p2 x)
\end{verbatim}
becomes
\begin{verbatim}
Goal G : ~ (forall x:int. q x -> (p1 x \/ p2 x))
\end{verbatim}
\item \texttt{deep}: This is the same technique as \texttt{top} but the
   negation is pushed under the universal quantification (without
   changing them) and under the implication. The previous example becomes
\begin{verbatim}
Goal G : forall x:int. q x /\ ~ (p1 x \/ p2 x)
\end{verbatim}
 \end{itemize}

\noindent
The name of the goals that triggered the smoke detector are printed:
\begin{verbatim}
   goal 'G', prover 'Alt-Ergo 0.93.1': Smoke detected!!!
\end{verbatim}
Moreover \texttt{Smoke detected} (exit code 1) is printed at the end if the smoke
detector has been triggered, or \texttt{No smoke detected} (exit code 0)
otherwise.


\paragraph{Customizing LaTeX output}

The generated LaTeX files contain some macros that must be defined
externally.  Various definitions can be given to them to customize the
output. 
\begin{itemize}
\item \verb|\provername|: macro with one parameter, a prover name
\item \verb|\valid|: macro with one parameter, used where the corresponding prover answers that the goal is valid. The parameter is the time in seconds.
\item \verb|\noresult|: macro without parameter, used where no result
  exists for the corresponding prover
\item \verb|\timeout|: macro without parameter, used where the corresponding prover reached the time limit
\item \verb|\explanation|: macro with one parameter, the goal name or its explanation
\end{itemize}

\begin{figure}[t]
  \begin{center}
    \begin{tabular}{| l |c |c |c |c |c |}
\hline \multicolumn{2}{|c|}{Proof obligations } & \provername{Alt-Ergo 0.93} & \provername{Coq 8.2pl1} & \provername{Simplify 1.5.4} \\ 
\hline 
\explanation{G1} & \explanation{ }& \noresult& \noresult& \valid{0.01} \\ 
\hline 
\explanation{G2} & \explanation{ }& \noresult& \noresult& \unknown \\ 
\cline{2-5} 
\explanation{ }& \explanation{ }\explanation{G2.1} & \unknown & \unknown & \unknown \\ 
\cline{2-5} 
\explanation{ }& \explanation{ }\explanation{G2.2} & \valid{0.02} & \noresult& \valid{0.01} \\ 
\hline 
\explanation{G3} & \explanation{ }& \valid{0.02} & \noresult& \unknown \\ 
\hline \end{tabular}

  \end{center}
  \verbatiminput{./replayer_macros.tex}
  \caption{Sample macros for the LaTeX option of why3replayer}
\label{fig:replayer}
\end{figure}

Figure~\ref{fig:replayer} proposes some suggestions for these macros, together
with the table obtained from the hello proof example of Section~\ref{chap:starting}.

\section{The \texttt{why3session} tool}
\label{sec:why3session}

The program \texttt{why3session} allows to manipulate why3 session on
the command line. It thus allows to batch modifications to several
sessions at once. This tool provides several subcommands:
\texttt{info}, \texttt{rm}, \texttt{copy}, \texttt{mod}.

Currently this tool reports or modifies only proof attempts.

\texttt{why3session info} reports informations:
\begin{itemize}
\item Option \verb|--provers| prints the provers that appear
  inside the session, one by line.
\item Option \verb|--edited-files| prints all the files that
  appear in the session as edited proofs.
\item Option \verb|--tree| prints the structure of the session as
  an ASCII tree. The files, theories, goals are marked with a question
  mark \verb|?|, if they are not verified. A proof is usually said to be verified
  if the proof result is \verb|valid| and the proof is not
  obsolete. However here specially
  we separate these two properties. On the one hand if the proof suffers from an internal
  failure we mark it with an exclamation mark \verb|!|, otherwise if
  it is not valid we mark it with a question mark \verb|?|, finally
  if it is valid we add nothing. On the other hand if the proof is obsolete we mark it
  with an \verb|O| and if it is archived we mark it with an
  \verb|A|.
\item Option \verb|--print0| separates the results of the options
  \verb|provers| and \verb|--edited-files| by the character number
  0 instead of end of line \verb|\n|. That allows you to
  safely use (even if the filename contains space or carriage return)
  the result with other commands. For example you can count the number
  of proof line in all the coq edited files in a session with:
\begin{verbatim}
why3session info --edited-files vstte12_bfs --print0 | xargs -0 coqwc
\end{verbatim}
  or you can add all the edited files in your favorite repository
  with:
\begin{verbatim}
why3session info --edited-files --print0 vstte12_bfs.mlw | \
    xargs -0 git add
\end{verbatim}

\end{itemize}

The subcommands \texttt{mod}, \texttt{copy}, and \texttt{rm} share the
same set of options for selecting the proof attempts to work on:
\begin{itemize}
\item Option \verb|--filter-prover| selects only the proof attempt with
  the given prover. This option can be specified multiple times to
  allow to select all the proofs that corresponds to one of the given
  provers. If this option is not specified, the proof are simply not
  filtered by there corresponding prover.
\item Option \verb|--filter-verified yes| restricts the selection to
  the proofs that are valid and not obsoletes. On contrary the option
  \verb|--filter-verified no| select the ones that are not verified.
  \verb|--filter-verified all|, the default, does not select on this property.
\item Option \verb|--filter-verified-goal yes| restricts the selection
  to the proofs of verified goals (that does not mean that the proof is
  verified). Same for the other cases \verb|no| and \verb|all|.
\item Option \verb|--filter-archived yes| restricts the selection
  to the proofs that are archived. Same for the other cases \verb|no|
  and \verb|all| except the default is \verb|no|.
\end{itemize}

\noindent
The subcommand \texttt{mod} modifies properties of proof
attempts:
\begin{itemize}
\item Option \verb|--set-obsolete| marks the selected proofs as
  obsolete.
\item Option \verb|--set-archived| marks the selected proofs as archived.
\item Option \verb|--unset-archived| removes the archived mark from the selected proofs.
\end{itemize}

The subcommand \texttt{copy} copies the proof attempt of a given goal to another
prover. The new prover is specified by the option
\verb|--to-prover|, for example \texttt{-{}-to-prover Alt-Ergo,0.94}.
A conflict arises if a proof with this prover already exists.
You can choose between four behaviors of \texttt{why3session}:
\begin{itemize}
\item replace the proof (\verb|-f|, \verb|--force|);
\item do not replace thr proof (\verb|-n|, \verb|--never|);
\item replace the proof unless it is verified (valid and not
  obsolete) (\verb|-c|, \verb|--conservative|); this is the default;
\item ask the user each time the case arises (\verb|-i|, \verb|--interactive|).
\end{itemize}


If you just want to update one session with updated provers you can
use \verb|--convert-unknown| instead of the option \verb|--to-prover|.
\begin{verbatim}
why3session copy  --convert-unknown
\end{verbatim}
For each proof attempt associated to an unknown prover (a prover not in
\verb|.why3.conf|) and not archived, it will try to find a known prover
with the same name. If it finds one, the proof attempt is copied to this
prover and the old proof is set to archived. The corresponding edited
files, if any, are copied and regenerated for the new prover An archived
proof is not replayed by why3replayer.

The subcommand \texttt{rm} removes the selected proof
attempts. The options \verb|-i, --interactive|, \verb|-f, --force| and
\verb|-c, --conservative| can also be used to respectively ask before
each suppression, suppress all the selected proof (default) and remove
only the proof that are not verified. The macro option \verb|--clean|
corresponds to \verb|--filter-verified-goal --conservative| and
removes the proof attempts that are not verified but which correspond
to verified goals.

\section{The \texttt{why3.conf} configuration file}
\label{sec:whyconffile}


\begin{figure}[t]
\begin{verbatim}
[main ]
loadpath = "/usr/local/share/why3/theories"
magic = 2
memlimit = 0
running_provers_max = 2
timelimit = 10

[ide ]
default_editor = "emacs"
task_height = 384
tree_width = 438
verbose = 0
window_height = 779
window_width = 638

[prover coq]
command = "coqc %f"
driver = "/usr/local/share/why3/drivers/coq.drv"
editor = "coqide"
name = "Coq"
version = "8.2pl2"

[prover alt-ergo]
command = "why3-cpulimit %t %m alt-ergo %f"
driver = "/usr/local/share/why3/drivers/alt_ergo.drv"
editor = ""
name = "Alt-Ergo"
version = "0.91"
\end{verbatim}
  \caption{Sample why3.conf file}
\label{fig:why3conf}
\end{figure}



One can use a custom configuration file. \texttt{why3config}
and other \texttt{why3} tools use priorities for which
user's configuration file to consider:
\begin{itemize}
\item the file specified by the \texttt{-C} or \texttt{-{}-config} options,
\item the file specified by the environment variable
  \texttt{WHY3CONFIG} if set.
\item the file \texttt{\$HOME/.why3.conf}
  (\texttt{\$USERPROFILE/.why3.conf} under Windows) or, in the case of
  local installation, \texttt{why3.conf} in Why3 sources top directory.
\end{itemize}
If none of these files exists, a built-in default configuration is used.

The configuration file is a human-readable text file, which consists
of association pairs arranged in sections. Figure~\ref{fig:why3conf}
shows an example of configuration file.

A section begins with a header inside square brackets and ends at the
beginning of the next section. The header of a
section can be only one identifier, \texttt{main} and \texttt{ide} in
the example, or it can be composed by a family name and one family
argument, \texttt{prover} is one family name, \texttt{coq} and
\texttt{alt-ergo} are the family argument.

Inside a section, one key can be associated with an integer (\eg{} -555),
a boolean (true, false) or a string (\eg{} "emacs"). One key can appear
only once except if its a multi-value key. The order of apparition of
the keys inside a section matter only for the multi-value key.

\section{Drivers of External Provers}
\label{sec:drivers}

The drivers of external provers are readable files, in directory
\texttt{drivers}. Experimented users can modify them to change the way
the external provers are called, in particular which transformations
are applied to goals.

[TO BE COMPLETED LATER]

\section{Transformations}
\label{sec:transformations}

Here is a quick documentation of provided transformations. We give
first the non-splitting ones, \eg{} those which produce one goal as
result, and others which produces any number of goals.

Notice that the set of available transformations in your own
installation is given by
\begin{verbatim}
why3 --list-transforms
\end{verbatim}

\subsection{Non-splitting transformations}

\begin{description}

\item[eliminate\_algebraic] Replaces algebraic data types by first-order
definitions~\cite{paskevich09rr}

\item[eliminate\_builtin] Suppress definitions of symbols which are
  declared as builtin in the driver, i.e. with a ``syntax'' rule.
\item[eliminate\_definition\_func]
  Replaces all function definitions with axioms.
\item[eliminate\_definition\_pred]
  Replaces all predicate definitions with axioms.
\item[eliminate\_definition]
  Apply both transformations above.
\item[eliminate\_mutual\_recursion]
  Replaces mutually recursive definitions with axioms.
\item[eliminate\_recursion]
  Replaces all recursive definitions with axioms.

\item[eliminate\_if\_term] replaces terms of the form \texttt{if
    formula then t2 else t3} by lifting them at the level of formulas.
  This may introduce \texttt{if then else } in formulas.

\item[eliminate\_if\_fmla] replaces formulas of the form \texttt{if f1 then f2
  else f3} by an equivalent formula using implications and other
  connectives.

\item[eliminate\_if]
  Apply both transformations above.

\item[eliminate\_inductive] replaces inductive predicates by
  (incomplete) axiomatic definitions, i.e. construction axioms and
  an inversion axiom.

\item[eliminate\_let\_fmla]
  Eliminates \texttt{let} by substitution, at the predicate level.

\item[eliminate\_let\_term]
  Eliminates \texttt{let} by substitution, at the term level.

\item[eliminate\_let]
  Apply both transformations above.

% \item[encoding\_decorate\_mono]

% \item[encoding\_enumeration]

\item[encoding\_smt]
  Encode polymorphic types into monomorphic type~\cite{conchon08smt}.

\item[encoding\_tptp]
  Encode theories into unsorted logic. %~\cite{cruanes10}.

% \item[filter\_trigger] *)

% \item[filter\_trigger\_builtin] *)

% \item[filter\_trigger\_no\_predicate] *)

% \item[hypothesis\_selection] *)
%   Filter hypothesis of goals~\cite{couchot07ftp,cruanes10}. *)

\item[inline\_all]
  expands all non-recursive definitions.

\item[inline\_goal] Expands all outermost symbols of the goal that
  have a non-recursive definition.

\item[inline\_trivial]
  removes definitions of the form

\begin{verbatim}
function  f x_1 .. x_n = (g e_1 .. e_k)
predicate p x_1 .. x_n = (q e_1 .. e_k)
\end{verbatim}
when each $e_i$ is either a ground term or one of the $x_j$, and
each $x_1$ .. $x_n$ occur at most once in the $e_i$

\item[introduce\_premises] moves antecedents of implications and
  universal quantifications of the goal into the premises of the task.

% \item[remove\_triggers] *)
%   removes the triggers in all quantifications. *)

\item[simplify\_array] Automatically rewrites the task using the lemma
  \verb|Select_eq| of theory \verb|array.Array|.

\item[simplify\_formula] reduces trivial equalities $t=t$ to true and
  then simplifies propositional structure: removes true, false, ``f
  and f'' to ``f'', etc.

\item[simplify\_recursive\_definition] reduces mutually recursive
  definitions if they are not really mutually recursive, e.g.:
\begin{verbatim}
function f : ... = .... g ...

with g : .. = e
\end{verbatim}
becomes
\begin{verbatim}
function g : .. = e
function f : ... = .... g ...
\end{verbatim}
if f does not occur in e

\item[simplify\_trivial\_quantification]
  simplifies quantifications of the form
\begin{verbatim}
  forall x, x=t -> P(x)
\end{verbatim}
or
\begin{verbatim}
  forall x, t=x -> P(x)
\end{verbatim}
  when x does not occur in t
  into
\begin{verbatim}
P(t)
\end{verbatim}
  More generally, it applies this simplification whenever x=t appear
  in a negative position.

\item[simplify\_trivial\_quantification\_in\_goal]
  same as above but applies only in the goal.

\item[split\_premise]
  splits conjunctive premises.

\end{description}

\subsection{Splitting transformations}

\begin{description}

\item[full\_split\_all]
  composition of \texttt{split\_premise} and \texttt{full\_split\_goal}.

\item[full\_split\_goal] puts the goal in a conjunctive form,
  returns the corresponding set of subgoals. The number of subgoals
  generated may be exponential in the size of the initial goal.

\item[simplify\_formula\_and\_task] same as \texttt{simplify\_formula}
  but also removes the goal if it is equivalent to true.

\item[split\_all]
  composition of \texttt{split\_premise} and \texttt{split\_goal}.

\item[split\_goal] if the goal is a conjunction of goals, returns the
  corresponding set of subgoals. The number of subgoals generated is linear in
  the size of the initial goal.

\item[split\_intro]
  when a goal is an implication, moves the antecedents into the premises.

\end{description}



%%% Local Variables:
%%% mode: latex
%%% TeX-PDF-mode: t
%%% TeX-master: "manual"
%%% End:


\chapter{Language Reference}
\label{chap:syntaxref}
%HEVEA\cutname{syntaxref.html}

In this chapter, we describe the syntax and semantics of \whyml.

%This chapter is not yet fully updated to the new syntax of
%\why 1.00, so it not distributed for the moment.
%
%\endinput

\section{Lexical Conventions}
\label{sec:lexer}

%Lexical conventions of \whyml are similar to those of OCaml.
%
Blank characters are space, horizontal tab, carriage return,
and line feed. Blanks separate lexemes but are otherwise ignored.
%
Comments are enclosed by \texttt{(*} and \texttt{*)} and can be nested.
Note that \texttt{(*)} does not start a comment.

Strings are enclosed in double quotes (\verb!"!). Double quotes can be
escaped inside strings using the backslash character (\verb!\!).
The other special sequences are \verb!\n! for line feed and \verb!\t!
for horizontal tab.
In the following, strings are referred to with the non-terminal
\nonterm{string}{}\spacefalse.

%\subsection{Constants}
The syntax for numerical constants is given by the following rules:
%\begin{figure}[ht]
\begin{center}\input{./generated/constant_bnf.tex}\end{center}
%\caption{Syntax for numerical constants.}
%\label{fig:bnf:constant}
%\end{figure}
%in Figure~\ref{fig:bnf:constant}.
Integer and real constants have arbitrary precision.
Integer constants can be given in base 10, 16, 8 or 2.
Real constants can be given in base 10 or 16.
Notice that the exponent in hexadecimal real constants is written in base 10.

%\subsection{Identifiers}
Identifiers are composed of letters, digits, underscores,
and primes. %, as shown in Figure~\ref{fig:bnf:ident}.
The syntax distinguishes identifiers that start with a lowercase letter
or an underscore (\nonterm{lident}{}\spacefalse), identifiers that
start with an uppercase letter (\nonterm{uident}{}\spacefalse),
and identifiers that start with a prime
(\nonterm{qident}{}\spacefalse, used exclusively for type variables):
%\begin{figure}[ht]
\begin{center}\input{./generated/ident_bnf.tex}\end{center}
%\caption{Syntax for identifiers.}
%\label{fig:bnf:ident}
%\end{figure}
Identifiers that contain a prime followed by a letter,
such as \texttt{int32'max}, are reserved for symbols
introduced by \why and cannot be used for user-defined symbols.

In order to refer to symbols introduced in different namespaces
(\textit{scopes}), we can put a dot-separated ``qualifier prefix''
in front of an identifier (e.g.~\texttt{Map.S.get}).
This allows us to use the symbol \texttt{get}
from the scope \texttt{Map.S} without importing
it in the current namespace:
%\begin{figure}[ht]
\begin{center}\input{./generated/qualid_bnf.tex}\end{center}
%\caption{Syntax for qualified identifiers.}
%\label{fig:bnf:qualid}
%\end{figure}
All parenthesised expressions in \whyml (types,
patterns, logical terms, program expressions)
admit a qualifier before the opening parenthesis,
e.g.~\texttt{Map.S.(get m i)}. This imports
the indicated scope into the current namespace during
the parsing of the expression under the qualifier.
For the sake of convenience, the parentheses can be omitted
when the expression itself is enclosed in parentheses,
square brackets or curly braces.

%\subsection{Operators}
Prefix and infix operators are built from characters organized in four
precedence groups (\textsl{op-char-1} to \textsl{op-char-4}), with
optional primes at the end:
%as shown in Figure~\ref{fig:bnf:operator}.
%\begin{figure}[ht]
\begin{center}\input{./generated/operator_bnf.tex}\end{center}
%\caption{Syntax for operators.}
%\label{fig:bnf:operator}
%\end{figure}
Infix operators from a high-numbered group bind stronger
than the infix operators from a low-numbered group.
For example, infix operator \texttt{.*.} from group 3
would have a higher precedence than infix operator
\texttt{->-} from group 1.
Prefix operators always bind stronger than infix operators.
The so-called ``tight operators'' are prefix operators that have even
higher precedence than the juxtaposition (application) operator,
allowing us to write expressions like \texttt{inv !x}
without parentheses.
%An operator from \textsl{infix-op-4} or \textsl{prefix-op}
%cannot start with \texttt{!} or \texttt{?}: such operators
%are always recognized as tight operators.
%Infix operators from groups 2-4 are left-associative.
%Infix operators from group 1 are non-associative and
%may instead be chained, as explained in Section~\ref{sec:terms}.

%An operator inside parenthesis can act as a lowercase identifier.
%\begin{figure}[ht]
%\begin{center}\input{./generated/ident_op_bnf.tex}\end{center}
%\caption{Syntax for qualified identifiers.}
%\label{fig:bnf:qualid}
%\end{figure}

Finally, any identifier, term, formula, or expression
in a \whyml source can be tagged either with a string
\textit{attribute} or a location:
\begin{center}\input{./generated/attribute_bnf.tex}\end{center}
An attribute cannot contain newlines or closing square brackets;
leading and trailing spaces are ignored.
A location consists of a file name in double quotes,
a line number, and starting and ending character positions.

\section{Type expressions}
\label{sec:types}

\whyml features an ML-style type system with polymorphic types,
variants (sum types), and records that can have mutable fields.
The syntax for type expressions is the following:
\begin{center}\input{./generated/type_bnf.tex}\end{center}
Built-in types are \texttt{int} (arbitrary precision integers),
\texttt{real} (real numbers), \texttt{bool}, the arrow type
(also called the \textit{mapping type}),
and the tuple types.
The empty tuple type is also called the \textit{unit type}
and can be written as \texttt{unit}.

Note that the syntax for type expressions notably differs from
the usual ML syntax. In particular, the type of polymorphic lists
is written \texttt{list 'a}, and not \texttt{'a list}.

\textit{Snapshot types} are specific to \whyml, they denote
the types of ghost values produced by pure logical functions in
\whyml programs. A snapshot of an immutable type is the type
itself: thus, \texttt{\{int\}} is the same as \texttt{int} and
\texttt{\{list 'a\}} is the same as \texttt{list 'a}.
A snapshot of a mutable type, however, represents a snapshot
value which cannot be modified anymore. Thus, a snapshot array
\texttt{a} of type \texttt{\{array int\}} can be read from
(\texttt{a[42]} is accepted) but not written into
(\texttt{a[42] <- 0} is rejected). Generally speaking,
a program function that expects an argument of a mutable type
will accept an argument of the corresponding snapshot type
as long as it is not modified by the function.

\section{Logical expressions: terms and formulas}
\label{sec:terms}

\begin{figure}[p!]
\begin{center}\input{./generated/term1_bnf.tex}\end{center}
\caption{\whyml terms (part I).}
\label{fig:bnf:term1}
\end{figure}

\begin{figure}[ht]
\begin{center}\input{./generated/term2_bnf.tex}\end{center}
\caption{\whyml terms (part II).}
\label{fig:bnf:term2}
\end{figure}

A significant part of a typical \whyml source file is occupied
by non-executable logical content intended for specification
and proof: function contracts, assertions, definitions of
logical functions and predicates, axioms, lemmas, etc.

Logical expressions are called \textit{terms}. Boolean
terms are called \textit{formulas}. Internally, \why distinguishes
the proper formulas (produced by predicate symbols, propositional
connectives and quantifiers) and the terms of type \texttt{bool}
(produced by Boolean variables and logical functions that return
\texttt{bool}). However, this distinction is not enforced on the
syntactical level, and \why will perform the necessary conversions
behind the scenes.

The syntax of \whyml terms is given in
Figures~\ref{fig:bnf:term1}-\ref{fig:bnf:term3}.
The constructions are listed in the order of
decreasing precedence.
For example, as was mentioned above,
tight operators have the highest precedence of all operators,
so that \texttt{-p.x} denotes the negation of the
record field \texttt{p.x}, whereas \texttt{!p.x}
denotes the field \texttt{x} of a record stored
in the reference \texttt{p}.

An operator in parentheses acts as an identifier
referring to that operator, for example, in a definition.
To distinguish between prefix and infix operators, an
underscore symbol is appended at the end: for example,
\texttt{(-)} refers to the binary subtraction and
\texttt{(-\_)} to the unary negation.
Tight operators cannot be used as infix operators,
and thus do not require disambiguation.

In addition to prefix and infix operators, \whyml
supports several mixfix bracket operators to
manipulate various collection types: dictionaries,
arrays, sequences, etc. Bracket operators do not have
any predefined meaning and may be used to denote access
and update operations for various user-defined collection types.
We can introduce multiple bracket operations in the same scope
by disambiguating them with primes after the closing
bracket: for example, \texttt{a[i]} may denote array access
and \texttt{s[i]'} sequence access.
Notice that the in-place update operator \texttt{a[i] <- v}
cannot be used inside logical terms: all effectful operations
are restricted to program expressions. To represent the result
of a collection update, we should use a pure logical update
operator \texttt{a[i <- v]} instead.
%Overloading of bracket operations, which would allow to use
%the same bracket operator for different collections,
%is currently not supported in \whyml.

\whyml supports ``associated'' names for operators, obtained
by adding a suffix after the parenthesised operator name.
For example, an axiom that represents the specification of the
infix operator \texttt{(+)} may be called \texttt{(+)'spec}
or \texttt{(+)\_spec}. As with normal identifiers, names
with a letter after a prime, such as \texttt{(+)'spec},
can only be introduced by \why, and not by the user in a \whyml
source.

The \texttt{at} and \texttt{old} operators are used inside
postconditions and assertions to refer to the value of
a mutable program variable at some past moment of execution
(see the next section for details).
These operators have higher precedence than the infix
operators from group 1 (\textsl{infix-op-1}): \texttt{old i > j}
is parsed as \texttt{(old i) > j} and not as \texttt{old (i > j)}.

Infix operators from groups 2-4 are left-associative.
Infix operators from group 1 are non-associative and
can be chained. For example, the term \texttt{0 <= i < j < length a}
is parsed as the conjunction of three inequalities \texttt{0 <= i},
\texttt{i < j}, and \texttt{j < length a}.

As with normal identifiers,
we can put a qualifier over a parenthesised operator,
e.g.~\texttt{Map.S.([]) m i}. Also, as noted above,
a qualifier can be put over a parenthesised term,
%e.g.~\texttt{Map.S.(m[i])},
and the parentheses
can be omitted if the term is a record or a record update.

The propositional connectives in \whyml formulas are listed in
Figure~\ref{fig:bnf:term2}. The non-standard connectives ---
asymmetric conjunction (\texttt{\&\&}), asymmetric disjunction
(\texttt{||}), proof indication (\texttt{by}), and consequence
indication (\texttt{so}) --- are used to control the goal-splitting
transformations of \why and provide integrated proofs for
\whyml assertions, postconditions, lemmas, etc.
The semantics of these connectives
follows the rules below:
\begin{itemize}\setlength{\itemsep}{0ex}
\item A proof task for \texttt{A \&\& B} is split into
separate tasks for \texttt{A} and \texttt{A -> B}.
If \texttt{A \&\& B} occurs as a premise, it behaves
as a normal conjunction.
\item A case analysis over \texttt{A || B} is split into
disjoint cases \texttt{A} and \texttt{not A {/\char92} B}.
If \texttt{A || B} occurs as a goal, it behaves
as a normal disjunction.
\item An occurrence of \texttt{A by B} generates a side condition
\texttt{B -> A} (the proof justifies the affirmation).
When \texttt{A by B} occurs as a premise,
it is reduced to \texttt{A} (the proof is discarded).
When \texttt{A by B} occurs as a goal,
it is reduced to \texttt{B} (the proof is verified).
\item An occurrence of \texttt{A so B} generates a side condition
\texttt{A -> B} (the premise justifies the conclusion).
When \texttt{A so B} occurs as a premise,
it is reduced to the conjunction \mbox{\texttt{A {/\char92} B}}
(we use both the premise and the conclusion).
When \texttt{A so B} occurs as a goal,
it is reduced to \texttt{A} (the premise is verified).
\end{itemize}
For example, full splitting of the goal
\texttt{(A by (exists x. B so C)) \&\& D}
produces four subgoals:
\texttt{exists x. B} (the premise is verified),
\texttt{forall x. B -> C} (the premise justifies the conclusion),
\texttt{(exists x. B {/\char92} C) -> A} (the proof justifies the affirmation),
and finally, \texttt{A -> D} (the proof of \texttt{A} is discarded
and \texttt{A} is used to prove \texttt{D}).

%Figure~\ref{fig:byso} contains more examples of usage of
%\texttt{\&\&}, \texttt{||}, \texttt{by}, and \texttt{so}.
%\begin{figure}[ht]
%\begin{center}
%\begin{tabular}{c|c}
%\multicolumn{1}{c|}{Initial goal} &
%\multicolumn{1}{c}{Goals after full splitting} \\
%\hline
%\texttt{A -> (B {/\char92} C)} & \texttt{A -> B}, \:\: \texttt{A -> C} \\
%\texttt{(A {\char92/} B) -> C} & \texttt{A -> C}, \:\: \texttt{B -> C} \\[1ex]
%\texttt{A -> (B {\&\&} C)} & \texttt{A -> B}, \:\: \texttt{A -> (B -> C)} \\
%\texttt{(A || B) -> C} & \texttt{A -> C}, \:\: \texttt{(not A {/\char92} B) -> C} \\[1ex]
%\texttt{A -> (B by C)} & \texttt{A -> C}, \:\: \texttt{A -> (C -> B)} \\
%\texttt{(A so B) -> C} & \texttt{A -> B}, \:\: \texttt{(A {/\char92} B) -> C} \\[1ex]
%\texttt{A by (B by C)} & \texttt{C}, \:\:
%  \texttt{C -> B}, \:\: \texttt{B -> A} \\
%\texttt{A by (B so C)} & \texttt{B}, \:\:
%  \texttt{B -> C}, \:\: \texttt{(B {/\char92} C) -> A} \\
%\end{tabular}
%\end{center}
%\caption{Non-standard propositional connectives.}
%\label{fig:byso}
%\end{figure}

The behaviour of the splitting transformations is further
controlled by attributes \texttt{[@stop\_split]} and
\texttt{[@case\_split]}. Consult Section~\ref{tech:trans:split}
for details.

Among the propositional connectives,
\texttt{not} has the highest precedence,
\texttt{\&\&} has the same precedence as \texttt{/\char92}
(weaker than negation),
\texttt{||} has the same precedence as \texttt{\char92/}
(weaker than conjunction),
\texttt{by}, \texttt{so}, \texttt{->}, and \texttt{<->}
all have the same precedence (weaker than disjunction).
All binary connectives except equivalence are right-associative.
Equivalence is non-associative and is chained instead:
\texttt{A <-> B <-> C} is transformed into a conjunction
of \texttt{A <-> B} and \texttt{B <-> C}.
To reduce ambiguity, \whyml forbids to place
a non-parenthesised implication at the right-hand side
of an equivalence: \texttt{A <-> B -> C} is rejected.

\begin{figure}[ht]
\begin{center}\input{./generated/term3_bnf.tex}\end{center}
\caption{\whyml terms (part III).}
\label{fig:bnf:term3}
\end{figure}

In Figure~\ref{fig:bnf:term3}, we find the more advanced
term constructions: conditionals, let-bindings, pattern
matching, and local function definitions,
either via the \texttt{let-in} construction or the
\texttt{fun} keyword. The pure logical functions
defined in this way are called \emph{mappings};
they are first-class values of ``arrow'' type
\texttt{$\tau_1$ -> $\tau_2$}.

The patterns are similar to those of OCaml, though the \texttt{when}
clauses and numerical constants are not supported. Unlike in OCaml,
\texttt{as} binds stronger than the comma: in the pattern
\texttt{($p_1$,$p_2$ as x)}, variable \texttt{x} is bound to
the value matched by pattern $p_2$. Also notice the closing
\texttt{end} after the \texttt{match-with} term.
A \texttt{let-in} construction with a non-trivial pattern is
translated as a \texttt{match-with} term with a single branch.

Inside logical terms, pattern matching must be exhaustive:
\whyml rejects a term like \texttt{let Some x = o in $\dots$},
where \texttt{o} is a variable of an option type.
In program expressions, non-exhaustive pattern matching
is accepted and a proof obligation is generated to show
that the values not covered cannot occur in execution.

The syntax of parameters %(non-terminal
%\nonterm{param}{} in Figure~\ref{fig:bnf:term3})
in user-defined operations---%
first-class mappings,
top-level logical functions and predicates,
and program functions---%
is rather flexible in \whyml.
Like in OCaml, the user can specify the name of a parameter
without its type and let the type be inferred from the
definition. Unlike in OCaml, the user can also specify
the type of the parameter without giving its name.
This is convenient when the symbol declaration does not
provide the actual definition or specification of
the symbol, and thus only the type signature is of
relevance.
For example, one can declare an abstract binary function
that adds an element to a set simply by writing
\texttt{function add 'a (set 'a) : set 'a}.
A standalone non-qualified lowercase identifier without
attributes is treated as a type name when the definition
is not provided, and as a parameter name otherwise.

Ghost patterns, ghost variables after \texttt{as},
and ghost parameters in function definitions are only used
in program code, and not allowed in logical terms.

\section{Program expressions}
\label{sec:exprs}

The syntax of program expressions is given in
Figures~\ref{fig:bnf:expr1}-\ref{fig:bnf:expr2}.
As before, the constructions are listed in the order of decreasing
precedence. The rules for tight, prefix, infix, and bracket operators
are the same as for logical terms. In particular, the infix operators
from group~1 can be chained. Notice that binary operators \texttt{\&\&}
and \texttt{||} denote here the usual lazy conjunction and disjunction,
respectively.

\begin{figure}[ht]
\begin{center}\input{./generated/expr1_bnf.tex}\end{center}
\caption{\whyml program expressions (part I).}
\label{fig:bnf:expr1}
\end{figure}

%Two new operators make appearance in Figure~\ref{fig:bnf:expr1}.
Keyword \texttt{ghost} marks the expression as
ghost code added for verification purposes. Ghost code is
removed from the final code intended for execution, and thus
cannot affect the computation of the program results nor the
content of the observable memory.

Assignment updates in place
a mutable record field or an element of a collection.
The former can be done simultaneously
on a tuple of values: \texttt{x.f, y.g <- a, b}. The latter
form, \texttt{a[i] <- v}, amounts to a call of the ternary
bracket operator \texttt{([]<-)} and cannot be used in a
multiple assignment.

\newpage
\begin{figure}[ht]
\begin{center}\input{./generated/expr2_bnf.tex}\end{center}
\caption{\whyml program expressions (part II).}
\label{fig:bnf:expr2}
\end{figure}


%%%%%%%%%%%%%%%%%%%%%%%%%%%%%%%%%%%%%%%%%%%%%%%%%%%%%%%%%%%%%%%%%%%%%%%%%%%%%%

\section{The \why Language}

\subsection{Terms}

The syntax for terms is given in Figure~\ref{fig:bnf:term1}.
The various constructs have the following priorities and
associativities, from lowest to greatest priority:
\begin{center}
  \begin{tabular}{|l|l|}
    \hline
    construct & associativity \\
    \hline\hline
    \texttt{if then else} / \texttt{let in} & -- \\
    label & -- \\
    cast  & -- \\
    infix-op level 1 & left \\
    infix-op level 2 & left \\
    infix-op level 3 & left \\
    infix-op level 4 & left \\
    prefix-op     & --   \\
    function application & left \\
    brackets / ternary brackets & -- \\
    bang-op       & --   \\
    \hline
  \end{tabular}
\end{center}

Note the curryfied syntax for function application, though partial
application is not allowed (rejected at typing).

\subsection{Formulas}

The syntax for formulas is given Figure~\ref{fig:bnf:formula}.
The various constructs have the following priorities and
associativities, from lowest to greatest priority:
\begin{center}
  \begin{tabular}{|l|l|}
    \hline
    construct & associativity \\
    \hline\hline
    \texttt{if then else} / \texttt{let in} & -- \\
    label & -- \\
    \texttt{->} / \texttt{<->} & right \\
    \texttt{by} / \texttt{so} & right \\
    \verb!\/! / \verb!||! & right \\
    \verb|/\| / \verb!&&! & right \\
    \texttt{not}  & -- \\
    infix level 1 & left \\
    infix level 2 & left \\
    infix level 3 & left \\
    infix level 4 & left \\
    prefix        & --   \\
    \hline
  \end{tabular}
\end{center}
Note that infix symbols of level 1 include equality (\texttt{=}) and
disequality (\texttt{<>}).

\begin{figure}
  \begin{center}\framebox{\input{./generated/formula_bnf.tex}}\end{center}
  \caption{Syntax for formulas.}
\label{fig:bnf:formula}
\end{figure}

Notice that there are two symbols for the conjunction: \verb|/\|
and \verb|&&|, and similarly for disjunction. They are logically
equivalent, but may be treated slightly differently by some
transformations. For instance, \texttt{split} transforms the goal
\verb|A /\ B| into subgoals \verb|A| and \verb|B|, whereas it transforms
\verb|A && B| into subgoals \verb|A| and \verb|A -> B|. Similarly, it
transforms \verb!not (A || B)! into subgoals \verb|not A| and
\verb|not ((not A) /\ B)|.
The \texttt{by}/\texttt{so} connectives are proof indications. They are
logically equivalent to their first argument, but may affect the result
of some transformations. For instance, the \texttt{split\_goal}
transformations interpret those connectives as introduction of logical cuts
 (see~\ref{tech:trans:split} for details).

\subsection{Theories}

The syntax for theories is given on Figure~\ref{fig:bnf:theorya} and~\ref{fig:bnf:theoryb}.

\begin{figure}
  \begin{center}\framebox{\input{./generated/theory_bnf.tex}}\end{center}
  \caption{Syntax for theories (part 1).}
\label{fig:bnf:theorya}
\end{figure}

\begin{figure}
  \begin{center}\framebox{\input{./generated/theory2_bnf.tex}}\end{center}
  \caption{Syntax for theories (part 2).}
\label{fig:bnf:theoryb}
\end{figure}

\subsubsection{Algebraic types}

TO BE COMPLETED

\subsubsection{Record types}

TO BE COMPLETED

\subsubsection{Range types}
\label{sec:rangetypes}

A declaration of the form \texttt{type r = < range \textit{a b} >}
defines a type that projects into the integer range
\texttt{[\textit{a,b}]}. Note that in order to make such a declaration
the theory \texttt{int.Int} must be imported.

Why3 let you cast an integer literal in a range type
(e.g. \texttt{(42:r)}) and will check at typing that the literal is in
range. Defining such a range type $r$ automatically introduces the
following:
\begin{whycode}
  function  r'int r : int
  constant  r'maxInt : int
  constant  r'minInt : int
\end{whycode}
The function \texttt{r'int} projects a term of type \texttt{r} to its
integer value. The two constants represent the high bound and low
bound of the range respectively.

Unless specified otherwise with the meta \texttt{"keep:literal"} on
\texttt{r}, the transformation \emph{eliminate\_literal} introduces an
axiom
\begin{whycode}
axiom r'axiom : forall i:r. r'minInt <= r'int i <= r'maxInt
\end{whycode}
and replaces all casts of the form \texttt{(42:r)} with a constant and
an axiom as in:
\begin{whycode}
constant rliteral7 : r
axiom rliteral7_axiom : r'int rliteral7 = 42
\end{whycode}

This type is used in the standard library in the theories
\texttt{bv.BV8}, \texttt{bv.BV16}, \texttt{bv.BV32}, \texttt{bv.BV64}.


\subsubsection{Floating-point Types}

A declaration of the form \texttt{type f = < float \textit{eb sb} >}
defines a type of floating-point numbers as specified by the IEEE-754
standard~\cite{ieee754-2008}. Here the literal \texttt{\textit{eb}}
represents the number of bits in the exponent and the literal
\texttt{\textit{sb}} the number of bits in the significand (including
the hidden bit). Note that in order to make such a declaration the
theory \texttt{real.Real} must be imported.

Why3 let you cast a real literal in a float type
(e.g. \texttt{(0.5:f)}) and will check at typing that the literal is
representable in the format. Note that Why3 do not implicitly round a
real literal when casting to a float type, it refuses the cast if the
literal is not representable.

Defining such a type \texttt{f} automatically introduces the following:
\begin{whycode}
  predicate f'isFinite f
  function  f'real f : real
  constant  f'eb : int
  constant  f'sb : int
\end{whycode}
As specified by the IEEE standard, float formats includes infinite
values and also a special NaN value (Not-a-Number) to represent
results of undefined operations such as $0/0$.  The predicate
\texttt{f'isFinite} indicates whether its argument is neither infinite
nor NaN. The function \texttt{f'real} projects a finite term of type
\texttt{f} to its real value, its result is not specified for non finite
terms.

Unless specified otherwise with the meta \texttt{"keep:literal"} on
\texttt{f}, the transformation \emph{eliminate\_literal} will
introduce an axiom
\begin{whycode}
axiom f'axiom :
  forall x:f. f'isFinite x -> -. max_real <=. f'real x <=. max_real
\end{whycode}
where \texttt{max\_real} is the value of the biggest finite float in
the specified format. The transformation also replaces all casts of
the form \texttt{(0.5:f)} with a constant and an axiom as in:
\begin{whycode}
constant fliteral42 : f
axiom fliteral42_axiom : f'real fliteral42 = 0.5 /\ f'isFinite fliteral42
\end{whycode}

This type is used in the standard library in the theories
\texttt{ieee\_float.Float32} and \texttt{ieee\_float.Float64}.

\subsection{Files}

A \why input file is a (possibly empty) list of theories.
\begin{center}\framebox{\input{./generated/why_file_bnf.tex}}\end{center}


%%%%%%%%%%%%%%%%%%%%%%%%%%%%%%%%%%%%%%%%%%%%%%%%%%%%%%%%%%%%%%%%%%%%%%%%%%%%%%
\clearpage
\section{The \whyml Language}\label{sec:syntax:whyml}

\subsection{Specification}

The syntax for specification clauses in programs
is given in Figure~\ref{fig:bnf:spec}.
\begin{figure}
  \begin{center}\framebox{\input{./generated/spec_bnf.tex}}\end{center}
  \caption{Specification clauses in programs.}
\label{fig:bnf:spec}
\end{figure}
Within specifications, terms are extended with new constructs
\verb|old| and \verb|at|:
\begin{center}\framebox{\input{./generated/term_old_at_bnf.tex}}\end{center}
Within a postcondition, $\verb|old|~t$ refers to the value of term $t$
in the prestate. Within the scope of a code mark $L$,
the term $\verb|at|~t~\verb|'|L$ refers to the value of term $t$ at the program
point corresponding to $L$.

\subsection{Expressions}

The syntax for program expressions is given in
Figure~\ref{fig:bnf:expra} and~Figure~\ref{fig:bnf:exprb}.
\begin{figure}
  \begin{center}\framebox{\input{./generated/expr_bnf.tex}}\end{center}
  \caption{Syntax for program expressions (part 1).}
\label{fig:bnf:expra}
\end{figure}

\begin{figure}
  \begin{center}\framebox{\input{./generated/expr2_bnf.tex}}\end{center}
  \caption{Syntax for program expressions (part 2).}
\label{fig:bnf:exprb}
\end{figure}

In applications, arguments are evaluated from right to left.
This includes applications of infix operators, with the only exception of
lazy operators \verb|&&| and \verb+||+ that evaluate from left to
right, lazily.


% In the following we describe the informal semantics of each
% constructs, and provide the corresponding rule for computing the
% weakest precondition.


% \subsubsection{Constant Expressions, Unary and Binary Operators}


% Integer and real constants are as in the logic language, as weel as the unary and binary operators.


% \subsubsection{Array accesses and updates, fields access and updates}

% \todo{}

% \subsubsection{Let binding, sequences}

% \todo{}

% \subsubsection{Function definition}

% \todo{fun, let rec}

% \subsubsection{Function call}

% \todo{}

% \subsubsection{Exception throwing and catching}

% \todo{raise, try with end}

% \subsubsection{Conditional expression, pattern matching}

% \todo{if then else. Discuss standard WP versus fast WP}

% \subsubsection{Iteration Expressions}

% There are three kind of expressions for iterating: bounded, unbounded and infinite.

% \begin{itemize}
% \item A bounded iteration has the form
% \begin{flushleft}\ttfamily
%   for $i$=$a$ to $b$ do invariant \{ $I$ \} $e$ done
% \end{flushleft}
% Expressions $a$ and $b$ are evaluated first and only once, then expression $e$ si evaluated successively for $i$ from $a$ to $b$ included. Nothing is executed if $a > b$. The invariant $I$ must hold at each iteration including before entering the loop and when leaving it. The rule for computing WP is as follows:
% \begin{eqnarray*}
%   WP(\texttt{for} \ldots, Q) &=& I(a) \land \\
% && \forall \vec{w} (\forall i, a \leq i \leq b \land I(i) \rightarrow WP(e,I(i+1))) \land (I(b+1) \rightarrow Q)
% \end{eqnarray*}
% where $\vec{w}$ is the set of references modified by $e$.

% A downward bounded iteration is also available, under the form
% \begin{flushleft}\ttfamily
%   for $i$=$a$ downto $b$ do invariant \{ $I$ \} $e$ done
% \end{flushleft}

% \item An unbounded iteration has the form
% \begin{flushleft}\ttfamily
%   while $c$ do invariant \{ $I$ \} $e$ done
% \end{flushleft}
% it iterates the loop body $e$ until the condition $c$ becomes false.
% \begin{eqnarray*}
%   WP(\texttt{while} \ldots, Q) &=& I \land \\
% && \forall \vec{w} (c \land I \rightarrow WP(e,I)) \land (\neg c \land I \rightarrow Q)
% \end{eqnarray*}
% where $\vec{w}$ is the set of references modified by $e$.

% Additionally, such a loop can be given a variant $v$, a quantity that must decreases ar each iteration, so as to prove its termination.


% \item An infinite iteration has the form
% \begin{flushleft}\ttfamily
%   loop invariant \{ $I$ \} $e$ end
% \end{flushleft}
% it iterates the loop forever, hence the only way to exit such a loop is to raise an exception.
% \begin{eqnarray*}
%   WP(\texttt{loop} \ldots, Q \mid Exc \Rightarrow R) &=& I \land \\
% && \forall \vec{w} (I \rightarrow WP(e,I)) \land (I \rightarrow WP(e,Exc \Rightarrow R))
% \end{eqnarray*}
% \end{itemize}

% \subsubsection{Assertions, Code Contracts}

% There are several form of expressions for inserting annotations in the code.
% The first form of those are the \emph{assertions} which have the form
% \begin{flushleft}\ttfamily
%   \textsl{keyword} \{ \textsl{proposition} \}
% \end{flushleft}
% where \textsl{keyword} is either \texttt{assert}, \texttt{assume} or
% \texttt{check}. They all state that the given proposition holds at the given program point. The differences are:
% \begin{itemize}
% \item \texttt{assert} requires to prove that the proposition holds, and then make it available in the context of the remaining of the code
% \item \texttt{check} requires to prove that the proposition holds, but
%   does not make it visible in the remaining
% \item \texttt{assume} assumes that the proposition holds and make it
%   visible in the context of the remaining code, without requiring to
%   prove it. It acts like an axiom, but within a program.
% \end{itemize}
% The corresponding rules for computing WP are as follows:
% \begin{eqnarray*}
%   WP(\texttt{assert} \{ P \}, Q) &=& P \mathop{\&\&} Q = P \land (P \rightarrow Q)\\
%   WP(\texttt{check} \{ P \}, Q) &=& P \land Q \\
%   WP(\texttt{assume} \{ P \}, Q) &=& P \rightarrow Q
% \end{eqnarray*}

% The other forms of code contracts allow to abstract a piece of code by specifications.
% \begin{itemize}
% \item $\texttt{any}~\{ P \}~\tau~\epsilon~\{ Q \}$ is a
%   non-deterministic expression that requires the precondition $P$ to
%   hold, then makes some side effects $\epsilon$, and returns any value
%   of type $\tau$ such that $Q$ holds. This construct acts as an axiom
%   in the sense that it does not check whether there exists any program
%   that can effectively establish the post-condition (similarly as the
%   introduction of a \texttt{val} at the global level).
% \item $\texttt{abstract}~e~\{ Q \}$ makes sure that the evaluation of
%   expression $e$ establishes the post-condition $Q$, and then abstract
%   away the program state by the post-condition $Q$ (similarly to the
%   \texttt{any} construct).
% \end{itemize}
% The corresponding rules for computing WP are as follows:
% \[
% \begin{array}{l}
%   WP(\texttt{any}~\{ P \}~\tau~\epsilon~\{ Q \mid exn_i \Rightarrow R_i \} ,
%   Q'  exn_i \Rightarrow R'_i) = \\
%   \qquad\qquad P \mathop{\&\&} \forall result, \epsilon.
%   (Q \rightarrow Q') \land (R_i \rightarrow R'_i) \\
%   WP(\texttt{abstract}~e~\{ Q \mid exn_i \Rightarrow R_i \} ,
%   Q' \mid exn_i \Rightarrow R'_i) = \\
%   \qquad\qquad WP(e,Q \mid exn_i \Rightarrow R_i) \land
%   \forall result, \epsilon, Q \rightarrow Q' \land R_i \rightarrow R'_i
% \end{array}
% \]

% \subsubsection{Labels, Operators \texttt{old} and \texttt{at}}

% \todo{Labels, old, at}

\subsection{Modules}

The syntax for modules is given in Figure~\ref{fig:bnf:module}.
\begin{figure}
  \begin{center}\framebox{\input{./generated/module_bnf.tex}}\end{center}
  \caption{Syntax for modules.}
\label{fig:bnf:module}
\end{figure}
Any declaration which is accepted in a theory is also accepted in a
module. Additionally, modules can introduce record types with mutable
fields and declarations which are specific to programs (global
variables, functions, exceptions).

\subsection{Files}

A \whyml input file is a (possibly empty) list of theories and modules.
\begin{center}\framebox{\input{./generated/whyml_file_bnf.tex}}\end{center}
A theory defined in a \whyml file can only be used within that
file. If a theory is supposed to be reused from other files, be they
\why or \whyml files, it should be defined in a \why file.


\section{The \why Standard Library}
\label{sec:library}\index{standard library}\index{library}

The \why standard library provides general-purpose
modules, to be used in logic and/or programs.
It can be browsed on-line at \url{http://why3.lri.fr/stdlib/}.
Each file contains one or several modules.
To \texttt{use} or \texttt{clone} a module \texttt{M} from file
\texttt{file}, use the syntax \texttt{file.M}, since \texttt{file} is
available in \why's default load path. For instance, the module of
integers and the module of references are imported as follows:
\begin{whycode}
  use import int.Int
  use import ref.Ref
\end{whycode}
A sub-directory \texttt{mach/} provides various modules to model
machine arithmetic.
For instance, the module of 63-bit integers and the module of arrays
indexed by 63-bit integers are imported as follows:
\begin{whycode}
  use import mach.int.Int63
  use import mach.array.Array63
\end{whycode}
In particular, the types and operations from these modules are mapped
to native OCaml's types and operations when \why code is extracted to
OCaml (see Section~\ref{sec:extract}).

%%% Local Variables:
%%% mode: latex
%%% TeX-PDF-mode: t
%%% TeX-master: "manual"
%%% End:



\chapter{Executing \whyml Programs}
\label{chap:exec}\index{whyml@\whyml}

This chapter shows how \whyml code can be executed, either by being
interpreted or compiled to some existing programming language.

\begin{latexonly}
Let us consider the program in Figure~\ref{fig:MaxAndSum}
on page~\pageref{fig:MaxAndSum} that computes the maximum and the sum
of an array of integers.
\end{latexonly}
\begin{htmlonly}
Let us consider the program of Section~\ref{sec:MaxAndSum} that computes
the maximum and the sum of an array of integers.
\end{htmlonly}
Let us assume it is contained in a file \texttt{maxsum.mlw}.

\section{Interpreting \whyml Code}
\label{sec:execute}
\index{execute@\texttt{execute}}\index{interpretation!of \whyml}
\index{testing \whyml code}

To test function \texttt{max\_sum}, we can introduce a \whyml test function
in module \texttt{MaxAndSum}
\begin{whycode}
  let test () =
    let n = 10 in
    let a = make n 0 in
    a[0] <- 9; a[1] <- 5; a[2] <- 0; a[3] <- 2;  a[4] <- 7;
    a[5] <- 3; a[6] <- 2; a[7] <- 1; a[8] <- 10; a[9] <- 6;
    max_sum a n
\end{whycode}
and then we use the \texttt{execute} command to interpret this function,
as follows:
\begin{verbatim}
> why3 execute maxsum.mlw MaxAndSum.test
Execution of MaxAndSum.test ():
     type: (int, int)
   result: (45, 10)
  globals:
\end{verbatim}
We get the expected output, namely the pair \texttt{(45, 10)}.

\section{Compiling \whyml to OCaml}
\label{sec:extract}
\index{OCaml}\index{extraction}
\index{extract@\texttt{extract}}

An alternative to interpretation is to compile \whyml to OCaml.
We do so using the \texttt{extract} command, as follows:
\begin{verbatim}
> why3 extract -D ocaml64 maxsum.mlw -o max_sum.ml
\end{verbatim}
The \texttt{extract} command requires the name of a driver, which indicates
how theories/modules from the \why standard library are translated to
OCaml. Here we assume a 64-bit architecture and thus we pass
\texttt{ocaml64}. We also specify an output file using option
\verb+-o+, namely \texttt{max\_sum.ml}.
After this command, the file \texttt{max\_sum.ml} contains an OCaml
code for function \texttt{max\_sum}.
To compile it, we create a file \texttt{main.ml}
containing a call to \texttt{max\_sum}, \emph{e.g.},
\begin{whycode}
let a = Array.map Z.of_int [| 9; 5; 0; 2; 7; 3; 2; 1; 10; 6 |]
let m, s = Max_sum.max_sum a (Z.of_int 10)
let () = Format.printf "sum=%s, max=%s@." (Z.to_string s) (Z.to_string m)
\end{whycode}
It is convenient to use \texttt{ocamlbuild} to compile and link both
files \texttt{max\_sum.ml} and \texttt{main.ml}:
\begin{verbatim}
> ocamlbuild -pkg zarith main.native
\end{verbatim}
Since Why3's type
\texttt{int} is translated to OCaml arbitrary precision integers using
the \texttt{ZArith} library, we have to pass option \texttt{-pkg
  zarith} to \texttt{ocamlbuild}. In order to get extracted code that
uses OCaml's native integers instead, one has to use Why3's types for
63-bit integers from libraries \texttt{mach.int.Int63} and
\texttt{mach.array.Array63}.


%%% Local Variables:
%%% mode: latex
%%% TeX-PDF-mode: t
%%% TeX-master: "manual"
%%% End:


% maintaining library.tex up to date is hopeless
% \chapter{Standard Library}
\label{chap:library}

\begin{description}

\item[algebra]

\item[int]

division: Euclidean versus Computer


\item[array]

\item[bool]

\item[comparison]

\item[floating\_point]

\cite{ayad10ijcar}

\item[graph]

\item[list]

\item[option]

\item[real]

\item[relations]

\item[set]

\item[sum]

\end{description}




%%% Local Variables:
%%% mode: latex
%%% TeX-PDF-mode: t
%%% TeX-master: "manual"
%%% End:


\chapter{Interactive Proof Assistants}
%HEVEA\cutname{itp.html}

% ... We then provide specific information about some ITPs.

\section{Using an Interactive Proof Assistant to Discharge Goals}

Instead of calling an automated theorem prover to discharge a goal,
\why offers the possibility to call an interactive theorem prover
instead. In that case, the interaction is decomposed into two distinct
phases:
\begin{itemize}
\item Edition of a proof script for the goal, typically inside a proof editor
  provided by the external interactive theorem prover;
\item Replay of an existing proof script.
\end{itemize}
An example of such an interaction is given in the tutorial
section~\ref{sec:gui}.

Some proof assistants offer more than one possible editor, \eg a
choice between the use of a dedicated editor and the use of the Emacs
editor and the ProofGeneral mode. Selection of the preferred mode can
be made in \texttt{why3ide} preferences, under the ``Editors'' tab.

\section{Theory Realizations}
\label{sec:realizations}

Given a \why theory, one can use a proof assistant to make a
\emph{realization} of this theory, that is to provide definitions for
some of its uninterpreted symbols and proofs for some of its
axioms. This way, one can show the consistency of an axiomatized
theory and/or make a connection to an existing library (of the proof
assistant) to ease some proofs.
%Currently, realizations are supported for the proof assistants Coq and PVS.

\subsection{Generating a realization}

Generating the skeleton for a theory is done by passing to the
\texttt{realize} command a driver suitable for realizations, the names of
the theories to realize, and a target directory.
\index{realize@\texttt{realize}}

\begin{verbatim}
why3 realize -D path/to/drivers/prover-realize.drv
             -T env_path.theory_name -o path/to/target/dir/
\end{verbatim}
\index{driver@\verb+--driver+}
\index{theory@\verb+--theory+}

The theory is looked into the files from the environment, \eg the standard
library. If the theory is stored in a different location, option \texttt{-L}
should be used.

The name of the generated file is inferred from the theory name. If the
target directory already contains a file with the same name, \why
extracts all the parts that it assumes to be user-edited and merges them in
the generated file.

Note that \why does not track dependencies between realizations and
theories, so a realization will become outdated if the corresponding
theory is modified.
It is up to the user to handle such dependencies, for instance using a
\texttt{Makefile}.

\subsection{Using realizations inside proofs}

If a theory has been realized, the \why printer for the corresponding prover
will no longer output declarations for that theory but instead simply put
a directive to load the realization. In order to tell the printer
that a given theory is realized, one has to add a meta declaration in the
corresponding theory section of the driver.
\index{driver file}

\begin{verbatim}
theory env_path.theory_name
  meta "realized_theory" "env_path.theory_name", "optional_naming"
end
\end{verbatim}

The first parameter is the theory name for \why. The second
parameter, if not empty, provides a name to be used inside generated
scripts to point to the realization, in case the default name is not
suitable for the interactive prover.
\index{realized_theory@\verb+realized_theory+}

\subsection{Shipping libraries of realizations}

While modifying an existing driver file might be sufficient for local
use, it does not scale well when the realizations are to be shipped to
other users. Instead, one should create two additional files: a
configuration file that indicates how to modify paths, provers, and
editors, and a driver file that contains only the needed
\verb+meta "realized_theory"+ declarations. The configuration file should be as
follows.
\index{configuration file}

\begin{verbatim}
[main]
loadpath="path/to/theories"

[prover_modifiers]
name="Coq"
option="-R path/to/vo/files Logical_directory"
driver="path/to/file/with/meta.drv"

[editor_modifiers coqide]
option="-R path/to/vo/files Logical_directory"

[editor_modifiers proofgeneral-coq]
option="--eval \"(setq coq-load-path (cons '(\\\"path/to/vo/files\\\" \
  \\\"Logical_directory\\\") coq-load-path))\""
\end{verbatim}

This configuration file can be passed to \why thanks to the
\verb+--extra-config+ option.
\index{extra-config@\verb+--extra-config+}
\index{prover_modifiers@\verb+prover_modifiers+}
\index{editor_modifiers@\verb+editor_modifiers+}
\index{option@\verb+option+}
\index{driver@\verb+driver+}



\subsection{Coq}
\label{sec:coq}

This section describes the content of the Coq files generated by \why for
both proof obligations and theory realizations. When reading a Coq
script, \why is guided by the presence of empty lines to split the
script, so the user should refrain from removing empty lines around
generated parts or adding empty lines inside them.

\begin{enumerate}
\item	The header of the file contains all the library inclusions
	required by the driver file. Any user-made changes to this part
	will be lost when the file is regenerated by \why. This part ends
	at the first empty line.
\item	Abstract logic symbols are assumed with the vernacular directive
	\verb+Paramater+. Axioms are assumed with the \verb+Axiom+
	directive. When regenerating a script, \why assumes that all such
	symbols have been generated by a previous run. As a consequence,
	the user should not introduce new symbols with these two
	directives, as they would be lost.
\item	Definitions of functions and inductive types in theories are
	printed in a block that starts with \verb+(* Why3 assumption *)+.
	This comment should not be removed; otherwise \why will assume
	that the definition is user-made.
\item	Finally, proof obligations and symbols to be realized are
	introduced by \verb+(* Why3 goal *)+. The user is supposed to
	fill the script after the statement. \why assumes that the
	user-made part extends up to \verb+Qed+, \verb+Admitted+,
	\verb+Save+, or \verb+Defined+, whichever comes first. In the
	case of definitions, the original statement can be replaced by
	a \verb+Notation+ directive, in order to ease the usage of
	already defined symbols. \why also recognizes \verb+Variable+
	and \verb+Hypothesis+ and preserves them; they should be used in
	conjunction with Coq's \verb+Section+ mechanism to realize
	theories that still need some abstract symbols and axioms.
\end{enumerate}

Currently, the parser for Coq scripts is rather naive and does not know
much about comments. For instance, \why can easily be confused by
some terminating directive like \verb+Qed+ that would be present in a
comment.


\section{Isabelle/HOL}
\label{sec:isabelle}

\index{Isabelle proof assistant}

\subsection{Installation}

You need version Isabelle2013-2.

\todo{put ... in ... etc/components}

\subsection{Usage}

\todo{Explain Isabelle/jedit server interaction}

When using Isabelle, files generated from Why3 theories are stored in
a dedicated XML format. Those files should not be edited. Instead,
realizations must be designed in some \texttt{.thy} as follows. 

\subsection{Realizations}

The realization file corresponding to some Why3 file \url{f.why}
should have the following form. 
\begin{verbatim}
theory Why3_f
imports Why3_Setup
begin

section {* realization of theory T *}

why3_open "f/T.xml"

why3_vc <some lemma> 
<proof>

why3_vc <some other lemma> by proof

[...]

why3_end
\end{verbatim}


See directory \url{lib/isabelle} for examples.


%%% Local Variables:
%%% mode: latex
%%% TeX-PDF-mode: t
%%% TeX-master: "manual"
%%% End:


\subsection{PVS}

When a PVS file is regenerated, the old version is split into chunks,
according to blank lines. Chunks corresponding to \why declarations
are identified with a comment starting with \verb+% Why3+, \eg
\begin{verbatim}
  % Why3 f
  f(x: int) : int
\end{verbatim}
Other chunks are considered to be user PVS declarations.
Thus a comment such as \verb+% Why3 f+ must not be removed;
otherwise, there will be two
declarations for \texttt{f} in the next version of the file (one being
regenerated and another one considered to be a user-edited chunk).

The user is allowed to perform the following actions on a PVS
realization:
\begin{itemize}
\item give a definition to an uninterpreted symbol (type, function, or
  predicate symbol), by adding an equal sign (\texttt{=}) and a
  right-hand side to the definition. When the declaration is
  regenerated, the left-hand side is updated and the right-hand side
  is reprinted as is. In particular, the names of a function or
  predicate arguments should not be modified. In addition, the
  \texttt{MACRO} keyword may be inserted and it will be kept in
  further generations.

\item turn an axiom into a lemma, that is to replace the PVS keyword
  \texttt{AXIOM} with either \texttt{LEMMA} or \texttt{THEOREM}.

\item insert anything between generated declarations, such as a lemma,
  an extra definition for the purpose of a proof, etc.
\end{itemize}
\why makes some effort to merge the new declarations with the old ones
and with the user chunks. If it happens that some chunks could not be
merged, they are appended at the end of the file, in comments.



%%% Local Variables:
%%% mode: latex
%%% TeX-PDF-mode: t
%%% TeX-master: "manual"
%%% End:



% \chapter{Complete API documentation} *)
% \label{chap:apidoc} *)

% \input{./apidoc.tex} *)

\chapter{Technical Informations}
%HEVEA\cutname{technical.html}

\section{Structure of Session Files}

The proof session state is stored in an XML file named
\texttt{\textsl{<dir>}/why3session.xml}, where \texttt{\textsl{<dir>}}
is the directory of the project.
The XML file follows the DTD given in \texttt{share/why3session.dtd} and reproduced below.
\lstinputlisting{../share/why3session.dtd}


\section{Prover Detection}
\label{sec:proverdetecttiondata}

The data configuration for the automatic detection of
installed provers is stored in the file
\texttt{provers-detection-data.conf} typically located in directory
\verb|/usr/local/share/why3| after installation. The content of this
file is reproduced below.
%BEGIN LATEX
{\footnotesize
%END LATEX
\lstinputlisting{../share/provers-detection-data.conf}
%BEGIN LATEX
}
%END LATEX

\section{The \texttt{why3.conf} Configuration File}
\label{sec:whyconffile}
\index{why3.conf@\texttt{why3.conf}}\index{configuration file}

One can use a custom configuration file. The \why
tools look for it in the following order:
\begin{enumerate}
\item the file specified by the \texttt{-C} or \texttt{-{}-config} options,
\item the file specified by the environment variable
  \texttt{WHY3CONFIG} if set,
\item the file \texttt{\$HOME/.why3.conf}
  (\texttt{\$USERPROFILE/.why3.conf} under Windows) or, in the case of
  local installation, \texttt{why3.conf} in the top directory of \why sources.
\end{enumerate}
If none of these files exist, a built-in default configuration is used.

The configuration file is a human-readable text file, which consists
of association pairs arranged in sections.
%BEGIN LATEX
Figure~\ref{fig:why3conf} shows an example of configuration file.
%END LATEX
%HEVEA Below is an example of configuration file.

%BEGIN LATEX
\begin{figure}[p]
{\footnotesize
%END LATEX
\begin{verbatim}
[main]
loadpath = "/usr/local/share/why3/theories"
loadpath = "/usr/local/share/why3/modules"
magic = 14
memlimit = 0
plugin = "/usr/local/lib/why3/plugins/tptp"
plugin = "/usr/local/lib/why3/plugins/genequlin"
plugin = "/usr/local/lib/why3/plugins/hypothesis_selection"
running_provers_max = 4
timelimit = 2

[ide]
default_editor = "editor %f"
error_color = "orange"
goal_color = "gold"
iconset = "fatcow"
intro_premises = true
premise_color = "chartreuse"
print_attributes = false
print_locs = false
print_time_limit = false
saving_policy = 2
task_height = 404
tree_width = 512
verbose = 0
window_height = 1173
window_width = 1024

[prover]
command = "'why3-cpulimit' 0 %m -s coqtop -batch -I %l/coq-tactic -R %l/coq Why3 -l %f"
driver = "/usr/local/share/why3/drivers/coq.drv"
editor = "coqide"
in_place = false
interactive = true
name = "Coq"
shortcut = "coq"
version = "8.3pl4"

[prover]
command = "'why3-cpulimit' %t %m -s alt-ergo %f"
driver = "/usr/local/share/why3/drivers/alt_ergo_0.93.drv"
editor = ""
in_place = false
interactive = false
name = "Alt-Ergo"
shortcut = "altergo"
shortcut = "alt-ergo"
version = "0.93.1"

[editor coqide]
command = "coqide -I %l/coq-tactic -R %l/coq Why3 %f"
name = "CoqIDE"
\end{verbatim}
%BEGIN LATEX
}
\caption{Sample \texttt{why3.conf} file}
\label{fig:why3conf}
\end{figure}
%END LATEX

A section begins with a header inside square brackets and ends at the
beginning of the next section. The header of a
section can be only one identifier, \texttt{main} and \texttt{ide} in
the example, or it can be composed by a family name and one family
argument, \texttt{prover} is one family name, \texttt{coq} and
\texttt{alt-ergo} are the family argument.

Sections contain associations \texttt{key=value}. A value is either
an integer (\eg \texttt{-555}), a boolean (\texttt{true}, \texttt{false}),
or a string (\eg \texttt{"emacs"}). Some specific keys can be attributed
multiple values and are
thus allowed to occur several times inside a given section. In that
case, the relative order of these associations matter.

\section{Drivers for External Provers}
\label{sec:drivers}

Drivers for external provers are readable files from directory
\texttt{drivers}. Experimented users can modify them to change the way
the external provers are called, in particular which transformations
are applied to goals.

[TO BE COMPLETED LATER]

\section{Transformations}
\label{sec:transformations}

This section documents the available transformations. We first
describe the most important ones, and then we provide a quick
documentation of the others, first the non-splitting ones, \eg those
which produce exactly one goal as result, and the others which produce any
number of goals.

Notice that the set of available transformations in your own
installation is given by
\begin{verbatim}
why3 --list-transforms
\end{verbatim}
\index{list-transforms@\verb+--list-transforms+}

\subsection{Inlining definitions}

Those transformations generally amount to replace some applications of
function or predicate symbols with its definition.

\begin{description}

\item[inline\_trivial]
  expands and removes definitions of the form
\begin{whycode}
function  f x_1 ... x_n = (g e_1 ... e_k)
predicate p x_1 ... x_n = (q e_1 ... e_k)
\end{whycode}
when each $e_i$ is either a ground term or one of the $x_j$, and
each $x_1 \dots x_n$ occurs at most once in all the $e_i$.
\index{inline-trivial@\verb+inline_trivial+}

\item[inline\_goal] expands all outermost symbols of the goal that
  have a non-recursive definition.
\index{inline-goal@\verb+inline_goal+}

\item[inline\_all]
  expands all non-recursive definitions.
\index{inline-all@\verb+inline_all+}

\end{description}


\subsection{Induction Transformations}

\begin{description}
\item[induction\_ty\_lex]
  \index{induction-ty-lex@\verb+induction_ty_lex+}
  performs structural, lexicographic induction on
  goals involving universally quantified variables of algebraic data
  types, such as lists, trees, etc. For instance, it transforms the
  following goal
\begin{whycode}
goal G: forall l: list 'a. length l >= 0
\end{whycode}
  into this one:
\begin{whycode}
goal G :
  forall l:list 'a.
     match l with
     | Nil -> length l >= 0
     | Cons a l1 -> length l1 >= 0 -> length l >= 0
     end
\end{whycode}
  When induction can be applied to several variables, the transformation
  picks one heuristically. A label \verb|"induction"| can be used to
  force induction over one particular variable, \eg with
\begin{whycode}
goal G: forall l1 "induction" l2 l3: list 'a.
        l1 ++ (l2 ++ l3) = (l1 ++ l2) ++ l3
\end{whycode}
induction will be applied on \verb|l1|. If this label is attached to
several variables, a lexicographic induction is performed on these
variables, from left to right.

%\item[] Induction on inductive predicates.

%[TO BE COMPLETED]

% TODO: implement also induction on int !

\end{description}

\subsection{Simplification by Computation}

These transformations simplify the goal by applying several kinds of
simplification, described below. The transformations differ only by
the kind of rules they apply:
\begin{description}
\item[compute\_in\_goal] aggressively applies all known
  computation/simplification rules.
  \index{compute-in-goal@\verb+compute_in_goal+}

\item[compute\_specified] performs rewriting using only built-in
  operators and user-provided rules.
  \index{compute-specified@\verb+compute_specified+}
\end{description}

The kinds of simplification are as follows.
\begin{itemize}
\item Computations with built-in symbols, \eg operations on integers,
  when applied to explicit constants, are evaluated. This includes
  evaluation of equality when a decision can be made (on integer
  constants, on constructors of algebraic data types), Boolean
  evaluation, simplification of pattern-matching/conditional expression,
  extraction of record fields, and beta-reduction.
  At best, these computations reduce the goal to
  \verb|true| and the transformations thus does not produce any sub-goal.
  For example, a goal
  like \verb|6*7=42| is solved by those transformations.
\item Unfolding of definitions, as done by \verb|inline_goal|. Transformation
  \verb|compute_in_goal| unfolds all definitions, including recursive ones.
  For \verb|compute_specified|, the user can enable unfolding of a specific
  logic symbol by attaching the meta \verb|rewrite_def| to the symbol.
\begin{whycode}
function sqr (x:int) : int = x * x
meta "rewrite_def" function sqr
\end{whycode}
\item Rewriting using axioms or lemmas declared as rewrite rules. When
  an axiom (or a lemma) has one of the forms
\begin{whycode}
axiom a: forall ... t1 = t2
\end{whycode}
  or
\begin{whycode}
axiom a: forall ... f1 <-> f2
\end{whycode}
  then the user can declare
\begin{whycode}
meta "rewrite" prop a
\end{whycode}
  to turn this axiom into a rewrite rule. Rewriting is always done
  from left to right. Beware that there is no check for termination
  nor for confluence of the set of rewrite rules declared.
\end{itemize}
Instead of using a meta, it is possible to declare an axiom as a
rewrite rule by adding the label \verb|"rewrite"| on the axiom name or
on the axiom itself, e.g.:
\begin{whycode}
axiom a "rewrite": forall ... t1 = t2
lemma b: "rewrite" forall ... f1 <-> f2
\end{whycode}
The second form allows some form of local rewriting, e.g.
\begin{whycode}
lemma l: forall x y. ("rewrite" x = y) -> f x = f y
\end{whycode}
can be proved by \verb|introduce_premises| followed by \verb|"compute_specified"|.

\paragraph{Bound on the number of reductions}
The computations performed by these transformations can take an
arbitrarily large number of steps, or even not terminate. For this
reason, the number of steps is bounded by a maximal value, which is
set by default to 1000. This value can be increased by another meta,
\eg
\begin{whycode}
meta "compute_max_steps" 1_000_000
\end{whycode}
When this upper limit is reached, a warning is issued, and the
partly-reduced goal is returned as the result of the transformation.


\subsection{Other Non-Splitting Transformations}

\begin{description}

\item[eliminate\_algebraic] replaces algebraic data types by first-order
definitions~\cite{paskevich09rr}.
\index{eliminate-algebraic@\verb+eliminate_algebraic+}

\item[eliminate\_builtin] removes definitions of symbols that are
  declared as builtin in the driver, \ie with a ``syntax'' rule.
\index{eliminate-builtin@\verb+eliminate_builtin+}

\item[eliminate\_definition\_func]
  replaces all function definitions with axioms.
\index{eliminate-definition-func@\verb+eliminate_definition_func+}

\item[eliminate\_definition\_pred]
  replaces all predicate definitions with axioms.
\index{eliminate-definition-pred@\verb+eliminate_definition_pred+}

\item[eliminate\_definition]
  applies both transformations above.
\index{eliminate-definition@\verb+eliminate_definition+}

\item[eliminate\_mutual\_recursion]
  replaces mutually recursive definitions with axioms.
\index{eliminate-mutual-recursion@\verb+eliminate_mutual_recursion+}

\item[eliminate\_recursion]
  replaces all recursive definitions with axioms.
\index{eliminate-recursion@\verb+eliminate_recursion+}

\item[eliminate\_if\_term] replaces terms of the form \texttt{if
    formula then t2 else t3} by lifting them at the level of formulas.
  This may introduce \texttt{if then else} in formulas.
\index{eliminate-if-term@\verb+eliminate_if_term+}

\item[eliminate\_if\_fmla] replaces formulas of the form \texttt{if f1 then f2
  else f3} by an equivalent formula using implications and other
  connectives.
\index{eliminate-if-fmla@\verb+eliminate_if_fmla+}

\item[eliminate\_if]
  applies both transformations above.
\index{eliminate-if@\verb+eliminate_if+}

\item[eliminate\_inductive] replaces inductive predicates by
  (incomplete) axiomatic definitions, \ie construction axioms and
  an inversion axiom.
\index{eliminate-inductive@\verb+eliminate_inductive+}

\item[eliminate\_let\_fmla]
  eliminates \texttt{let} by substitution, at the predicate level.
\index{eliminate-let-fmla@\verb+eliminate_let_fmla+}

\item[eliminate\_let\_term]
  eliminates \texttt{let} by substitution, at the term level.
\index{eliminate-let-term@\verb+eliminate_let_term+}

\item[eliminate\_let]
  applies both transformations above.
\index{eliminate-let@\verb+eliminate_let+}

% \item[encoding\_decorate\_mono]

% \item[encoding\_enumeration]

\item[encoding\_smt]
  encodes polymorphic types into monomorphic types~\cite{conchon08smt}.
\index{encoding-smt@\verb+encoding_smt+}

\item[encoding\_tptp]
  encodes theories into unsorted logic. %~\cite{cruanes10}.
\index{encoding-tptp@\verb+encoding_tptp+}

% \item[filter\_trigger] *)

% \item[filter\_trigger\_builtin] *)

% \item[filter\_trigger\_no\_predicate] *)

% \item[hypothesis\_selection] *)
%   Filter hypothesis of goals~\cite{couchot07ftp,cruanes10}. *)

\item[introduce\_premises] moves antecedents of implications and
  universal quantifications of the goal into the premises of the task.
\index{introduce-premises@\verb+introduce_premises+}

% \item[remove\_triggers] *)
%   removes the triggers in all quantifications. *)

\item[simplify\_array] automatically rewrites the task using the lemma
  \verb|Select_eq| of theory \verb|map.Map|.
\index{simplify-array@\verb+simplify_array+}

\item[simplify\_formula] reduces trivial equalities $t=t$ to true and
  then simplifies propositional structure: removes true, false, simplifies
  $f \land f$ to $f$, etc.
\index{simplify-formula@\verb+simplify_formula+}

\item[simplify\_recursive\_definition] reduces mutually recursive
  definitions if they are not really mutually recursive, \eg
\begin{whycode}
function f : ... = ... g ...
with g : ... = e
\end{whycode}
becomes
\begin{whycode}
function g : ... = e
function f : ... = ... g ...
\end{whycode}
if $f$ does not occur in $e$.
\index{simplify-recursive-definition@\verb+simplify_recursive_definition+}

\item[simplify\_trivial\_quantification]
  simplifies quantifications of the form
\begin{whycode}
forall x, x = t -> P(x)
\end{whycode}
into
\begin{whycode}
P(t)
\end{whycode}
  when $x$ does not occur in $t$.
  More generally, this simplification is applied whenever $x=t$ or
  $t=x$ appears in negative position.
\index{simplify-trivial-quantification@\verb+simplify_trivial_quantification+}

\item[simplify\_trivial\_quantification\_in\_goal]
  is the same as above but it applies only in the goal.
\index{simplify-trivial-quantification-in-goal@\verb+simplify_trivial_quantification_in_goal+}

\item[split\_premise] replaces axioms in conjunctive form
  by an equivalent collection of axioms.
  In absence of case analysis labels (see \texttt{split\_goal} for details),
  the number of axiom generated per initial axiom is
  linear in the size of that initial axiom.
\index{split-premise@\verb+split_premise+}

\item[split\_premise\_full] is similar to \texttt{split\_premise}, but it
  also converts the axioms to conjunctive normal form. The number of
  axioms generated per initial axiom may be exponential in the size of
  the initial axiom.
\index{split-premise-full@\verb+split_premise_full+}

\end{description}

\subsection{Other Splitting Transformations}
\label{tech:trans:split}

\begin{description}

\item[simplify\_formula\_and\_task] is the same as \texttt{simplify\_formula}
  but it also removes the goal if it is equivalent to true.
\index{simplify-formula-and-task@\verb+simplify_formula_and_task+}

\item[split\_goal] changes conjunctive goals into the
  corresponding set of subgoals. In absence of case analysis labels,
  the number of subgoals generated is linear in the size of the initial goal.

  \paragraph{Behavior on asymmetric connectives and
    \texttt{by}/\texttt{so}}

  The transformation treats specially asymmetric and
  \texttt{by}/\texttt{so} connectives. Asymmetric conjunction
  \verb|A && B| in goal position is handled as syntactic sugar for
  \verb|A /\ (A -> B)|.  The conclusion of the first subgoal can then
  be used to prove the second one.

  Asymmetric disjunction \verb+A || B+ in hypothesis position is handled as
  syntactic sugar for \verb|A \/ ((not A) /\ B)|.
  In particular, a case analysis on such hypothesis would give the negation of
  the first hypothesis in the second case.

  The \texttt{by} connective is treated as a proof indication. In
  hypothesis position, \verb|A by B| is treated as if it were
  syntactic sugar for its regular interpretation \verb|A|. In goal
  position, it is treated as if \verb|B| was an intermediate step for
  proving \verb|A|. \verb|A by B| is then replaced by \verb|B| and the
  transformation also generates a side-condition subgoal \verb|B -> A|
  representing the logical cut.

  Although splitting stops at disjunctive points like symmetric
  disjunction and left-hand sides of implications, the occurrences of
  the \texttt{by} connective are not restricted. For instance:
  \begin{itemize}
  \item Splitting
\begin{whycode}
goal G : (A by B) && C
\end{whycode}
generates the subgoals
\begin{whycode}
goal G1 : B
goal G2 : A -> C
goal G3 : B -> A (* side-condition *)
\end{whycode}
\item Splitting
\begin{whycode}
goal G : (A by B) \/ (C by D)
\end{whycode}
generates
\begin{whycode}
goal G1 : B \/ D
goal G2 : B -> A (* side-condition *)
goal G3 : D -> C (* side-condition *)
\end{whycode}
\item Splitting
\begin{whycode}
goal G : (A by B) || (C by D)
\end{whycode}
generates
\begin{whycode}
goal G1 : B || D
goal G2 : B -> A        (* side-condition *)
goal G3 : B || (D -> C) (* side-condition *)
\end{whycode}
Note that due to the asymmetric disjunction, the disjunction is kept in the
second side-condition subgoal.
\item Splitting
\begin{whycode}
goal G : exists x. P x by x = 42
\end{whycode}
generates
\begin{whycode}
goal G1 : exists x. x = 42
goal G2 : forall x. x = 42 -> P x (* side-condition *)
\end{whycode}
Note that in the side-condition subgoal, the context is universally closed.
\end{itemize}

The \texttt{so} connective plays a similar role in hypothesis position, as it serves as a consequence indication. In goal position, \verb|A so B| is treated as if it were syntactic sugar for its regular interpretation \verb|A|. In hypothesis position, it is treated as if both \verb|A| and \verb|B| were true because \verb|B| is a consequence of \verb|A|. \verb|A so B| is replaced by \verb|A /\ B| and the transformation also generates a side-condition subgoal \verb|A -> B| corresponding to the consequence relation between formula.

As with the \texttt{by} connective, occurrences of \texttt{so} are
unrestricted. For instance:
\begin{itemize}
\item Splitting
\begin{whycode}
goal G : (((A so B) \/ C) -> D) && E
\end{whycode}
generates
\begin{whycode}
goal G1 : ((A /\ B) \/ C) -> D
goal G2 : (A \/ C -> D) -> E
goal G3 : A -> B               (* side-condition *)
\end{whycode}
\item Splitting
\begin{whycode}
goal G : A by exists x. P x so Q x so R x by T x
(* reads: A by (exists x. P x so (Q x so (R x by T x))) *)
\end{whycode}
generates
\begin{whycode}
goal G1 : exists x. P x
goal G2 : forall x. P x -> Q x               (* side-condition *)
goal G3 : forall x. P x -> Q x -> T x        (* side-condition *)
goal G4 : forall x. P x -> Q x -> T x -> R x (* side-condition *)
goal G5 : (exists x. P x /\ Q x /\ R x) -> A (* side-condition *)
\end{whycode}
In natural language, this corresponds to the following proof scheme
for \verb|A|: There exists a \verb|x| for which \verb|P| holds. Then,
for that witness \verb|Q| and \verb|R| also holds. The last one holds
because \verb|T| holds as well. And from those three conditions on
\verb|x|, we can deduce \verb|A|.
\end{itemize}

\paragraph{Labels controlling the transformation}

The transformations in the split family can be controlled by using
labels on formulas.

The label \verb|"stop_split"| can be used to block the splitting of a
formula.  The label is removed after blocking, so applying the
transformation a second time will split the formula. This is can be
used to decompose the splitting process in several steps. Also, if a
formula with this label is found in non-goal position, its
\texttt{by}/\texttt{so} proof indication will be erased by the
transformation. In a sense, formulas tagged by \verb|"stop_split"| are
handled as if they were local lemmas.

The label \verb|"case_split"| can be used to force case analysis on hypotheses.
For instance, applying \texttt{split\_goal} on
\begin{whycode}
goal G : ("case_split" A \/ B) -> C
\end{whycode}
generates the subgoals
\begin{whycode}
goal G1 : A -> C
goal G2 : B -> C
\end{whycode}
Without the label, the transformation does nothing because undesired case analysis
may easily lead to an exponential blow-up.

Note that the precise behavior of splitting transformations in presence of
the \verb|"case_split"| label is not yet specified
and is likely to change in future versions.

\index{split-goal@\verb+split_goal+}

\item[split\_all]
  performs both \texttt{split\_premise} and \texttt{split\_goal}.
\index{split-all@\verb+split_all+}

\item[split\_intro]
  performs both \texttt{split\_goal} and \texttt{introduce\_premises}.
\index{split-intro@\verb+split_intro+}

\item[split\_goal\_full]
  has a behavior similar
  to \texttt{split\_goal}, but also converts the goal to conjunctive normal form.
  The number of subgoals generated may be exponential in the size of the initial goal.
\index{split-goal-full@\verb+split_goal_full+}

\item[split\_all\_full]
  performs both \texttt{split\_premise} and \texttt{split\_goal\_full}.
\index{split-all-full@\verb+split_all_full+}


\end{description}


\section{Proof Strategies}
\label{sec:strategies}

As seen in Section~\ref{sec:ideref}, the IDE provides a few buttons
that trigger the run of simple proof strategies on the selected goals.
Proof strategies can be defined using a basic assembly-style language,
and put into the Why3 configuration file. The commands of this basic
language are:
\begin{itemize}
\item \texttt{c $p$ $t$ $m$} calls the prover $p$ with a time limit
  $t$ and memory limit $m$. On success, the strategy ends, it
  continues to next line otherwise
\item \texttt{t $n$ $lab$} applies the transformation $n$. On success,
  the strategy continues to label $lab$, and is applied to each
  generated sub-goals.  It continues to next line otherwise.
\item \texttt{g $lab$} inconditionally jumps to label $lab$
\item \texttt{$lab$:} declares the label $lab$. The default label
  \texttt{exit} allows to stop the program.
\end{itemize}

To examplify this basic programming language, we give below the
default strategies that are attached to the default buttons of the
IDE, assuming that the provers Alt-Ergo 1.30, CVC4 1.5 and Z3 4.5.0
were detected by the \verb|why3 config --detect| command
\begin{description}
\item[Split] is bound to the 1-line strategy
\begin{verbatim}
t split_goal_wp exit
\end{verbatim}

\item[Auto level 0] is bound to
\begin{verbatim}
c Z3,4.5.0, 1 1000
c Alt-Ergo,1.30, 1 1000
c CVC4,1.5, 1 1000
\end{verbatim}
  The three provers are tried for a time limit of 1 second and memory
  limit of 1~Gb, each in turn. This is a perfect strategy for a first
  attempt to discharge a new goal.

\item[Auto level 1] is bound to
\begin{verbatim}
start:
c Z3,4.5.0, 1 1000
c Alt-Ergo,1.30, 1 1000
c CVC4,1.5, 1 1000
t split_goal_wp start
c Z3,4.5.0, 10 4000
c Alt-Ergo,1.30, 10 4000
c CVC4,1.5, 10 4000
\end{verbatim}
  The three provers are first tried for a time limit of 1 second and
  memory limit of 1~Gb, each in turn. If none of them succeed, a
  split is attempted. If the split works then the same strategy is
  retried on each sub-goals. If the split does not succeed, the provers
  are tried again with a larger limits.

\item[Auto level 2] is bound to
\begin{verbatim}
start:
c Z3,4.5.0, 1 1000
c Eprover,2.0, 1 1000
c Spass,3.7, 1 1000
c Alt-Ergo,1.30, 1 1000
c CVC4,1.5, 1 1000
t split_goal_wp start
c Z3,4.5.0, 5 2000
c Eprover,2.0, 5 2000
c Spass,3.7, 5 2000
c Alt-Ergo,1.30, 5 2000
c CVC4,1.5, 5 2000
t introduce_premises afterintro
afterintro:
t inline_goal afterinline
g trylongertime
afterinline:
t split_goal_wp start
trylongertime:
c Z3,4.5.0, 30 4000
c Eprover,2.0, 30 4000
c Spass,3.7, 30 4000
c Alt-Ergo,1.30, 30 4000
c CVC4,1.5, 30 4000
\end{verbatim}
  Notice that now 5 provers are used.  The provers are first tried for
  a time limit of 1 second and memory limit of 1~Gb, each in turn. If
  none of them succeed, a split is attempted. If the split works then
  the same strategy is retried on each sub-goals. If the split does
  not succeed, the prover are tried again with limits of 5 s and 2
  Gb. If all fail, we attempt the transformation of introduction of
  premises in the context, followed by an inlining of the definitions
  in the goals. We then attempt a split again, if the split succeeds,
  we restart from the beginning, if it fails then provers are tried
  again with 30s and 4 Gb.

\end{description}

%%% Local Variables:
%%% mode: latex
%%% TeX-PDF-mode: t
%%% TeX-master: "manual"
%%% End:


\part{Appendix}

\appendix

\chapter{Release Notes}
%HEVEA\cutname{changes.html}

\section{Release Notes for version 0.90}

TO DISCUSS:

comment mettre a jour l'example bag ? Parametrer par une egalit'e sur
les elements ?


Attention, ne pas introduire 1 variable par champ complique le boulot des prouveurs

``inline'' ne doit pas inliner Map.set

egalite sur les type algebriques ? engendrees automatiquement ?

\subsection{Syntax Changes}

\begin{center}
  \begin{tabular}{|c|c|}
\hline
    0.87 & 0.90 \\
\hline
\texttt{'L:} & \texttt{label L in} \\
\texttt{at x 'L} & \texttt{x at L} \\
\texttt{assert \{ ... (old x) ... \}} & \texttt{ assert \{... (x at Init) .. \}} \\
\verb|\|\texttt{ x. e} & \texttt{fun x -> e} \\
\texttt{use HighOrd} & nothing, not needed anymore \\
\texttt{HighOrd.pred ty} & \texttt{ty -> bool} \\
\texttt{type t model ...} & \texttt{type t = abstract ... } \\
\texttt{abstract e ensures \{ Q \}} & \texttt{begin ensures \{ Q \} e end} \\
\hline
  \end{tabular}
\end{center}

\subsection{Model types, abstract types}

explain \texttt{private} and ghost fields, \texttt{abstract mutable},
\texttt{private mutable}

\subsection{Polymorphic Equality}

No polymorphic equality in programs. Consequence : no List.mem in
programs, need for List.mem, List.filter, parameterized with a
predicate.

No default equality on algebraic datatypes

\section{Release Notes for version 0.80: syntax changes w.r.t. 0.73}

The syntax of \whyml programs changed in release 0.80.
The table in Figure~\ref{fig:syntax080} summarizes the changes.

\begin{figure}[thbp]
  \centering
\begin{tabular}{|p{0.45\textwidth}|p{0.45\textwidth}|}
\hline
\textbf{version 0.73} & \textbf{version 0.80} \\
\hline
\ttfamily
type t = \{| field~:~int |\}
&
\ttfamily
type t = \{ field~:~int \}
\\
\hline
\ttfamily
\{| field = 5 |\}
&
\ttfamily
\{ field = 5 \}
\\
\hline
\ttfamily
use import module M
&
\ttfamily
use import M
\\
\hline
\ttfamily
let rec f (x:int) (y:int)~:~t \newline
\null~~~~variant \{ t \} with rel = \newline
\null~~~~\{ P \} \newline
\null~~~~e \newline
\null~~~~\{ Q \} \newline
\null~~~~| Exc1 -> \{ R1 \} \newline
\null~~~~| Exc2 n -> \{ R2 \}
&
\ttfamily
let rec f (x:int) (y:int)~:~t \newline
\null~~~~variant \{ t with rel \} \newline
\null~~~~requires \{ P \} \newline
\null~~~~ensures \{ Q \} \newline
\null~~~~raises \{ Exc1 -> R1 \newline
\null~~~~~~~~~~~| Exc2 n -> R2 \} \newline
\null~~~~= e
\\
\hline
\ttfamily
val f (x:int) (y:int)~:\newline
\null~~~~\{ P \} \newline
\null~~~~t \newline
\null~~~~writes a b \newline
\null~~~~\{ Q \} \newline
\null~~~~| Exc1 -> \{ R1 \} \newline
\null~~~~| Exc2 n -> \{ R2 \}
&
\ttfamily
val f (x:int) (y:int)~:~t \newline
\null~~~~requires \{ P \} \newline
\null~~~~writes \{ a, b \} \newline
\null~~~~ensures \{ Q \} \newline
\null~~~~raises \{ Exc1 -> R1 \newline
\null~~~~~~~~~~~| Exc2 n -> R2 \}
\\
\hline
\ttfamily
val f~:~x:int -> y:int ->\newline
\null~~~~\{ P \} \newline
\null~~~~t \newline
\null~~~~writes a b \newline
\null~~~~\{ Q \} \newline
\null~~~~| Exc1 -> \{ R1 \} \newline
\null~~~~| Exc2 n -> \{ R2 \}
&
\ttfamily
val f (x y:int)~:~t \newline
\null~~~~requires \{ P \} \newline
\null~~~~writes \{ a, b \} \newline
\null~~~~ensures \{ Q \} \newline
\null~~~~raises \{ Exc1 -> R1 \newline
\null~~~~~~~~~~~| Exc2 n -> R2 \}
\\
\hline
\ttfamily
abstract e \{ Q \}
&
\ttfamily
abstract e ensures \{ Q \}
\\
\hline
\end{tabular}
\caption{Syntax changes from version 0.73 to version 0.80}
\label{fig:syntax080}
\end{figure}

\section{Summary of Changes w.r.t. Why 2}

The main new features with respect to Why 2.xx
are the following.
\begin{enumerate}
\item Completely redesigned input syntax for logic declarations
  \begin{itemize}
  \item new syntax for terms and formulas
  \item enumerated and algebraic data types, pattern matching
  \item recursive definitions of logic functions and predicates, with
    termination checking
  \item inductive definitions of predicates
  \item declarations are structured in components called ``theories'',
    which can be reused and instantiated
  \end{itemize}

\item More generic handling of goals and lemmas to prove
  \begin{itemize}
  \item concept of proof task
  \item generic concept of task transformation
  \item generic approach for communicating with external provers
  \end{itemize}

\item Source code organized as a library with a documented API, to
  allow access to \why features programmatically.

\item GUI with new features with respect to the former GWhy
  \begin{itemize}
  \item session save and restore
  \item prover calls in parallel
  \item splitting, and more generally applying task transformations,
    on demand
  \item ability to edit proofs for interactive provers (Coq only for
    the moment) on any subtask
  \end{itemize}

\item Extensible architecture via plugins
  \begin{itemize}
  \item users can define new transformations
  \item users can add connections to additional provers
  \end{itemize}
\end{enumerate}

% \begin{itemize}
% \item New syntax for terms and formulas
% \item Algebraic data types, pattern matching
% \item Recursive definitions
% \item Inductive predicates
% \item Declaration encapsulated in theories. Using and cloning theories.
% \item Concept of proof task transformation
% \item Generic communication with provers
% \item OCaml library with documented API
% \item New GUI with session save and restore
% % \item New syntax for programs, new VC generator, intentionaly left *)
% %   undocumented, since the syntax is likely to evolve significantly in *)
% %   the future. Examples are available in \texttt{examples/programs}. *)
% \end{itemize}

\bibliographystyle{abbrvurl}
\bibliography{manual}
%\bibliography{abbrevs,demons,demons2,demons3,team,crossrefs}


% \cleardoublepage
% % \newcommand{\why}{\textsc{Why}}

\newglossaryentry{ident}{
  name={identifier},
  description={
    (type \texttt{Ident.ident}) --- a common type to represent
    a name of an object. Every \gls{tysymbol}, \gls{lsymbol},
    \gls{prsymbol}, \gls{tvsymbol}, \gls{vsymbol}, and
    \gls{theory} has a unique identifier that can be used
    to distinguish it from any other object of the same type.
\glspar
    Every identifier has an associated, possibly non-unique,
    string. To avoid collisions in \gls{prettyprinting} and
    to make the output conform to a given format, various
    \glspl{printer} may \glslink{sanitization}{sanitize}
    identifiers. Identifiers may also bear a \gls{location}
    of their origin and a list of \glspl{label}.
\glspar
    An identifier is \gls{unigen} from a \gls{preid} and
    is suitable for \gls{weakmemo}.
\nopostdesc}
}

\newglossaryentry{label}{
  name={label},
  description={
\nopostdesc}
}

\newglossaryentry{location}{
  name={location},
  description={
\nopostdesc}
}

\newglossaryentry{decl}{
  name={declaration},
  description={(type \texttt{Decl.decl})
\nopostdesc}
}

\newglossaryentry{algtype}{
  name={algebraic type},
  description={
\nopostdesc}
}

\newglossaryentry{constructor}{
  name={constructor},
  description={--- a \gls{lsymbol} introduced in
  an \gls{algtype} \gls{decl}. Can be used in \glspl{pattern}
  and as a usual function symbol.
\nopostdesc}
}

\newglossaryentry{lsymbol}{
  name={logical symbol},
  description={(type \texttt{Term.lsymbol}) --- a type representing
  function and predicate symbols. A logical symbol bears a unique
  \gls{ident} and a type signature, describing what types a symbol
  admits in its arguments.
\glspar
  A logical symbol is \gls{unigen} from a \gls{preid} and
  is suitable for \gls{weakmemo}.
\nopostdesc}
}

\newglossaryentry{preid}{
  name={pre-identifier},
  description={(type \texttt{Ident.preid}) --- a preliminary
  non-unique object used to produce unique \glspl{ident}.
\nopostdesc}
}

\newglossaryentry{prettyprinting}{
  name={pretty-printing},
  description={
\nopostdesc}
}

\newglossaryentry{printer}{
  name={printer},
  description={
\nopostdesc}
}

\newglossaryentry{proposition}{
  name={printer},
  description={
\nopostdesc}
}

\newglossaryentry{prsymbol}{
  name={proposition symbol},
  description={(type \texttt{Decl.prsymbol}) --- a type representing
  \gls{proposition} names. A proposition symbol bears a unique \gls{ident},
  is \gls{unigen} from a \gls{preid}, and is suitable for \gls{weakmemo}.
\nopostdesc}
}

\newglossaryentry{sanitization}{
  name={sanitization},
  description={
\nopostdesc}
}

\newglossaryentry{theory}{
  name={theory},
  description={
\nopostdesc}
}

\newglossaryentry{tvsymbol}{
  name={type variable},
  description={
    (type \texttt{Ty.tvsymbol}) --- a type representing symbols of
    type variables. A type variable bears a unique \gls{ident}
    and is \gls{unigen} from a \gls{preid}.
\nopostdesc}
}

\newglossaryentry{hashcons}{
  name={hash-consed},
  description={
    Objects of a given type are hash-consed whenever (a) every
    two semantically equal objects are also physically equal;
    and (b) contrary to \gls{unigen} objects, one can construct
    an hash-consed object equal to an existing one.
\glspar
    Hash-consed objects provide efficient comparison and hash
    operations and are suitable for \gls{weakmemo}.
\nopostdesc}
}

\newglossaryentry{pattern}{
  name={pattern},
  description={(type \texttt{Term.pattern}) --- objects used
  in pattern-matching expressions. A pattern is built of
  \glspl{constructor} and \glspl{vsymbol} with the help of
  smart constructors guaranteeing that every pattern is well-typed.
\glspar
  Patterns are \gls{hashcons}.
\nopostdesc}
}

\newglossaryentry{type}{
  name={type},
  description={(type \texttt{Ty.ty}) --- a type representing, well,
  types in {\why}. Types are built with the help of smart constructors
  guaranteeing that every type is well-formed.
\glspar
  Types are \gls{hashcons} and suitable for \gls{weakmemo}.
\nopostdesc}
}

\newglossaryentry{term}{
  name={term},
  description={(type \texttt{Term.term}) --- a type of logical terms.
  Every term has a \gls{type}. Terms are built with the help of
  smart constructors guaranteeing that every term is well-typed.
  A term can bear a list of \glspl{label}.
\glspar
  In addition to usual first-order terms, {\why} terms can contain
  if-then-else expressions, let-expressions, and \gls{pattern} matching.
\glspar
  Terms are \gls{hashcons}.
\nopostdesc}
}

\newglossaryentry{formula}{
  name={formula},
  description={(type \texttt{Term.fmla}) --- a type of logical formulas.
  Formulas are built with the help of smart constructors guaranteeing
  that every formula is well-typed. A formula can bear a list of
  \glspl{label}.
\glspar
  In addition to usual first-order formulas, {\why} formulas can contain
  if-then-else expressions, let-expressions, and \gls{pattern} matching.
\glspar
  Formulas are \gls{hashcons}.
\nopostdesc}
}

\newglossaryentry{tysymbol}{
  name={type symbol},
  description={(type \texttt{Ty.tysymbol}) --- a type representing
  \gls{type} constructors. A type symbol bears a unique \gls{ident}
  and an arity, is \gls{unigen} from a \gls{preid}, and
  is suitable for \gls{weakmemo}.
\nopostdesc}
}

\newglossaryentry{unigen}{
  name={uniquely generated},
  description={
    Objects of a given type are uniquely generated whenever every
    newly constructed object is physically and semantically distinct
    from any other value of the same type. For example, \glspl{ident}
    are uniquely generated, while \glspl{preid} or \glspl{term} are not.
\glspar
    Uniquely generated objects usually provide efficient comparison
    and hash operations. Thus, they can be used in \gls{hashcons}
    objects and are suitable for \gls{weakmemo}.
\nopostdesc}
}

\newglossaryentry{vsymbol}{
  name={variable},
  description={(type \texttt{Ty.vsymbol}) --- a type representing
  variable symbols. A variable symbol bears a unique \gls{ident}
  and has an associated \gls{type}.
\glspar
  A variable is \gls{unigen} from a \gls{preid}.
\nopostdesc}
}

\newglossaryentry{weakmemo}{
  name={forgetful memoization},
  description={(module \texttt{Hashweak}) --- memoization technique
  that does not create an additional reference to a key, allowing it
  (and its associated value) to be garbage-collected even while
  the memoization table is still accessible. Forgetful memoization
  is used, in particular, to implement \gls{task} \glspl{trans}:
  the transformation results are memoized, but as soon as the
  original task is garbage-collected, these results can be
  dropped, too.
\glspar
  \Glspl{tysymbol}, \glspl{lsymbol}, \glspl{prsymbol},
  \glspl{type}, \glspl{decl}, \glspl{task}, and \glspl{env}
  can be used as keys in forgetful memoization.
\nopostdesc}
}

\newglossaryentry{trans}{
  name={transformation},
  description={(type \texttt{Trans.trans})
\nopostdesc}
}

\newglossaryentry{task}{
  name={task},
  description={(type \texttt{Task.task})
\nopostdesc}
}

\newglossaryentry{env}{
  name={environment},
  description={(type \texttt{Env.env})
\nopostdesc}
}


\printglossary

\gls{type} \gls{term} \gls{formula} \gls{ident} \gls{weakmemo}
\gls{vsymbol} \gls{unigen} \gls{tysymbol} \gls{pattern}


\cleardoublepage
\listoffigures
\cleardoublepage
\printindex

\end{document}

%%% Local Variables:
%%% mode: latex
%%% TeX-PDF-mode: t
%%% TeX-master: t
%%% End:
