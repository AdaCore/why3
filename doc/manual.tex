\documentclass[a4paper,12pt]{report}

\usepackage{fullpage}
\usepackage{url}
\usepackage{makeidx}
\usepackage{verbatim}
\usepackage{graphicx}
\input{./version.tex}

%\newcommand{\todo}[1]{{\Huge\bfseries TODO: #1}}

\newcommand{\why}{\textsf{Why3}\xspace}
\newcommand{\whyml}{\textsf{WhyML}\xspace}
\newcommand{\eg}{\emph{e.g.}\xspace}
\newcommand{\ie}{\emph{i.e.}\xspace}
\newcommand{\opam}{\textsf{OPAM}\xspace}

\newcommand{\emptyitem}{%
%BEGIN LATEX
~
%END LATEX
}

% BNF grammar
\newcommand{\keyword}[1]{\texttt{#1}}
\newcommand{\indextt}[1]{\index{#1@\protect\keyword{#1}}}
\newcommand{\indexttbs}[1]{\index{#1@\protect\keywordbs{#1}}}
\newif\ifspace
\newif\ifnewentry
\newcommand{\addspace}{\ifspace \: \spacefalse \fi}
\newcommand{\term}[2]{\addspace\mbox{\lstinline|#1|%
\ifthenelse{\equal{#2}{}}{}{\indexttbase{#2}{#1}}}\spacetrue}
\newcommand{\nonterm}[2]{%
  \ifthenelse{\equal{#2}{}}%
             {\addspace\mbox{\textsl{#1}\ifnewentry\index{#1@\textsl{#1}}\fi}\spacetrue}%
             {\addspace\mbox{\textsl{#1}\footnote{#2}\ifnewentry\index{grammar entries!\textsl{#1}}\fi}\spacetrue}}
\newcommand{\repetstar}{$^*$\spacetrue}
\newcommand{\repetplus}{$^+$\spacetrue}
\newcommand{\repetone}{$^?$\spacetrue}
\newcommand{\lparen}{\addspace(}
\newcommand{\rparen}{)}
\newcommand{\orelse}{\addspace$\mid$\spacetrue}
\newcommand{\sep}{ \\[2mm] \spacefalse\newentrytrue}
\newcommand{\newl}{ \\ & & \spacefalse}
\newcommand{\alt}{ \\ & $\mid$ & \spacefalse}
\newcommand{\is}{ & $::=$ & \spacefalse\newentryfalse}
\newenvironment{syntax}{\begin{tabular}{@{}rrll@{}}\spacefalse\newentrytrue}{\end{tabular}}
\newcommand{\synt}[1]{$\spacefalse#1$}
\newcommand{\emptystring}{$\epsilon$}
\newcommand{\below}{See\; below}

%%% listings for Why3 %%%%%%%%%%%%%%%%%%%%%%%%%%%%%%%%%%%%%%%%%%%%%%%%


% cannot use \usepackage here because it breaks hevea (no colored keywords anymore :-()

\RequirePackage{listings}
\RequirePackage{amssymb}

\lstdefinelanguage{why3}
{
basicstyle=\ttfamily,%
morekeywords=[1]{abstract,absurd,any,assert,assume,axiom,by,%
check,clone,coinductive,constant,diverges,do,done,downto,%
else,end,ensures,exception,exists,export,for,forall,fun,%
function,ghost,goal,if,import,in,inductive,invariant,lemma,%
let,loop,match,meta,model,module,mutable,namespace,not,old,%
predicate,private,raise,raises,reads,rec,requires,result,%
returns,so,then,theory,to,try,type,use,val,variant,while,%
with,writes},%
string=[b]",%
%keywordstyle=[1]{\color{red}},%
morekeywords=[2]{true,false},%
%keywordstyle=[2]{\color{blue}},%
otherkeywords={},%
commentstyle=\itshape,%
columns=[l]fullflexible,%
sensitive=true,%
morecomment=[s]{(*}{*)},%
escapeinside={*?}{?*},%
keepspaces=true,%
literate=%
% {'a}{$\alpha$}{1}%
% {'b}{$\beta$}{1}%
% {<}{$<$}{1}%
% {>}{$>$}{1}%
% {<=}{$\le$}{1}%
% {>=}{$\ge$}{1}%
% {<>}{$\ne$}{1}%
% {/\\}{$\land$}{1}%
% {\\/}{ $\lor$ }{3}%
% {\ or(}{ $\lor$(}{3}%
% {not\ }{$\lnot$ }{1}%
% {not(}{$\lnot$(}{1}%
% {+->}{\texttt{+->}}{2}%
% {+->}{$\mapsto$}{2}%
% {-->}{\texttt{-\relax->}}{2}%
% {-->}{$\longrightarrow$}{2}%
% {->}{$\rightarrow$}{2}%
% {<-}{$\leftarrow$}{2}%
% {<->}{$\leftrightarrow$}{2}%
%
%
}

\lstnewenvironment{why3}{\lstset{language=why3}}{}

\newcommand{\whyf}[1]{\lstinline[language=why3]{#1}}


% \RequirePackage{listings}
% \RequirePackage{amssymb}

\lstset{
  basicstyle={\ttfamily},
  framesep=2pt,
  frame=single,
  keywordstyle={\color{blue}},
  stringstyle=\itshape,
  commentstyle=\itshape,
  columns=[l]fullflexible,
  showstringspaces=false,
}

\lstnewenvironment{whycode}{\lstset{language=why3}}{}
\lstnewenvironment{ocamlcode}{\lstset{language={[Objective]Caml}}}{}

%%% Local Variables:
%%% mode: latex
%%% TeX-master: "manual"
%%% TeX-PDF-mode: t
%%% End:


\makeindex

\begin{document}

%%% coverpage
\whytitlepage{The WHY verification tool}{Tutorial and Reference
  Manual}{\whyversion}{Team-Project ProVal}


\tableofcontents

%%%%%%%%%%%%%%%%%%%%%%%%%%%%%%%%%%%%%%%%%%%%%%%%%%%%%%%%%%%%%%%%%%%%%%
\chapter*{Foreword}
\addcontentsline{toc}{chapter}{Foreword}



\subsection*{License}

The \why\ tool is \copyright\ 2002--2010 team-project ProVal
(\url{proval.lri.fr}).
It is open source and freely available under the terms of the Gnu
Library General Public License Version 2.1 (see the file
\texttt{LICENSE} included in the distribution).

\subsection*{Availability}

The \why\ tool is available from \url{http://why.lri.fr/}, in source
and binary formats, together with this documentation and many
examples.


\subsection*{Contact}

There is a mailing list for \why;
see \url{http://lists.lri.fr/mailman/listinfo/why}.

% TODO BTS


%%%%%%%%%%%%%%%%%%%%%%%%%%%%%%%%%%%%%%%%%%%%%%%%%%%%%%%%%%%%%%%%%%%%%%
\chapter{Syntax of input files}
\label{syntax}

\section{Lexical conventions}

Comments are opened with \texttt{(*}, closed
with \texttt{*)} and can be nested.

Identifiers are made of letters, digits,
the underscore character \texttt{\_} and the single quote \texttt{'},
starting with a letter. In many cases, identifiers starting with
lowercase or uppercase characters are distinguished.

\begin{center}\framebox{\begin{syntax}
  \nt{lident}\indexnt{lident}
      \is \te{a}..\te{z} \nt{ident\_char}\repstar \\
  \nt{uident}\indexnt{uident}
      \is \te{A}..\te{Z} \nt{ident\_char}\repstar \\
  \nt{ident\_char} 
      \is \te{A}..\te{Z} \orelse \te{a}..\te{z} \orelse
      \te{0}..\te{9} \orelse \te{\_} \orelse \te{'} \\[0.1em]
  \nt{ident} 
      \is \nt{lident} \orelse\ \nt{uident}
\end{syntax}}\end{center}

Keywords are the following:
\begin{center}
{\tt\begin{tabular}{l@{\qquad}l@{\qquad}l@{\qquad}l@{\qquad}l}
        absurd &
	and &
        array &
	as &
	assert \\
	axiom &
	begin &
        bool &
	do &
	done \\
        else &
	end &
	exception &
	exists &
	external \\
        false &
	for &
	forall &
	fun &
	function \\
	goal &
	if &
	in &
	int &
	invariant \\
%	label &
	let &
	logic &
	not &
	of \\
	or &
	parameter &
	predicate &
	prop &
	raise \\
	raises &
	reads &
	real &
	rec &
	ref \\
	returns &
	then &
	true &
	try &
	type \\
	unit &
	variant &
	void &
	while &
	with \\
        writes &
\end{tabular}}
\end{center}

\section{Grammar}
%% file := decl*
%% decl := | "theory" uident decl* "end"
%%         | "theory" uident "." uqualid ("with" instance+ )?
%%         | "use" ("open" | "include")?  uqualid ("as" ident)
%%         | ("axiom"|"lemma"|"goal") ident ":" fmla
%%         | "type" typarams lident ("=" (ty-alg | ty))?
%%           ("and" typarams lident ("=" (ty-alg | ty))?)* 
%% Il faut verifier qu'il y a toujours un constructeur entre deux
%% définitions récursives. 
%%         | "logic" def ("and" def)*
%%         | "inductive" ...

%% def := lident_ou_infix ("(" formal+, ")")? (":" ty)? ("=" lexpr)?
%% formal := ((lident|_)":")? ty         

%% instance := | lqualid "=" lqualid
%% ty-alg := | uident ("(" ty,+ ")")? ("|" uident ...)*

\subsection{Qualified names}

\begin{center}\framebox{\begin{syntax}
  \nt{lqualid} \is 
\end{syntax}}\end{center}

\subsection{Terms}


\begin{figure}[htbp]
\begin{center}
\hrulefill\\
\begin{tabular}{lrl}
  \nt{term}\indexnt{term}
    \is   \nt{constant} 
    \newl \nt{term} \nt{arith\_op} \nt{term} \\
      & $|$ & \te{-} \nt{term} \\
      & $|$ & \nt{lab\_identifier} \\
      & $|$ & \nt{identifier} \te{(} \nt{term}\repplussep{\te{,}} \te{)} \\
      & $|$ & \nt{lab\_identifier} \te{[} \nt{term} \te{]} \\
      & $|$ & \te{if} \nt{term} \te{then} \nt{term} \te{else} \nt{term} \\
      & $|$ & \te{let} \nt{identifier} \te{=} \nt{term} \te{in} \nt{term} \\
      & $|$ & \te{(} \nt{term} \te{)} \\
  \\[0.1em]

  \nt{constant}\indexnt{constant}
    & $::=$ & \nt{integer-constant} \\
      & $|$ & \nt{real-constant} \\
\end{tabular}\\
\hrulefill
\caption{Syntax of terms}
\label{fig:terms}
\end{center}            
\end{figure}

\begin{figure}[htbp]
\begin{center}
\hrulefill\\
\begin{tabular}{lrl}
  \nt{predicate}\indexnt{predicate}
    & $::=$ & \te{true} \\
      & $|$ & \te{false} \\
      & $|$ & \nt{identifier} \\
      & $|$ & \nt{identifier} \te{(} \nt{term}\repplussep{\te{,}} \te{)} \\
      & $|$ & \nt{term} \nt{relation} \nt{term} 
              $[$ \nt{relation} \nt{term} $]$ \\
      & $|$ & \nt{predicate} \te{->} \nt{predicate} \\
      & $|$ & \nt{predicate} \te{<->} \nt{predicate} \\
      & $|$ & \nt{predicate} \te{or} \nt{predicate} \\
      & $|$ & \nt{predicate} \te{and} \nt{predicate} \\
      & $|$ & \te{not} \nt{predicate} \\
      & $|$ & \te{if} \nt{term} \te{then} \nt{predicate} 
              \te{else} \nt{predicate} \\
      & $|$ & \te{let} \nt{identifier} \te{=} \nt{term} 
              \te{in} \nt{predicate} \\
      & $|$ & \te{forall} \nt{identifier}\repplussep{\te{,}}
              \te{:} \nt{primitive\_type} $[$ \nt{triggers} $]$
              \te{.} \nt{predicate} \\
      & $|$ & \te{exists} \nt{identifier}\repplussep{\te{,}}
              \te{:} \nt{primitive\_type}
              \te{.} \nt{predicate} \\
      & $|$ & \te{(} \nt{predicate} \te{)} \\
      & $|$ & (\nt{identifier} $|$ \nt{string}) \te{:} \nt{predicate} \\
  \\[0.1em]
 
  \nt{triggers}
    & $::=$ & \te{[} \nt{trigger}\repplussep{\te{|}} \te{]} \\
  \nt{trigger}
    & $::=$ & \nt{term}\repplussep{\te{,}} \\
  \\[0.1em]

  \nt{primitive\_type}\indexnt{primitive\_type}
    & $::=$ & \te{int} $|$ \te{bool} $|$ \te{real} $|$ 
              \te{unit} $|$ \nt{identifier} $|$ \te{'} \nt{identifier} \\
    & $|$ & \nt{primitive\_type} \nt{identifier} $|$ \te{(}
    \nt{primitive\_type}\repstarsep{\te{,}} \te{)} \nt{identifier} \\
  \\[0.1em]

  \nt{relation}\indexnt{relation}
    & $::=$ & \te{=} $|$ \te{<>} $|$ 
              \te{<} $|$ \te{<=} $|$ \te{>} $|$ \te{>=}
\end{tabular}\\
\hrulefill
\caption{Syntax of predicates}
\label{fig:predicates}
\end{center}            
\end{figure}

\begin{figure}[htbp]
\begin{center}
\hrulefill\\
\begin{tabular}{lrl}
  \nt{l\_declaration}
    & $::=$ & $[$ \te{external} $]$ \te{type} $[$ \nt{type\_parameters} $]$
              \nt{identifier} \\\indextt{type} \indextt{external}
      & $|$ & $[$ \te{external} $]$ \te{logic} \nt{identifier}\repplussep{\te{,}}
              \te{:} \nt{logic\_type} \\\indextt{logic} \indextt{external}
      & $|$ & \te{function} \nt{identifier}
              \te{(} \nt{logic\_binder}\repstarsep{\te{,}}
              \te{)} \te{:} \nt{primitive\_type} \\
          & & \te{=} \nt{term} \\ \indextt{function}
      & $|$ & \te{predicate} \nt{identifier}
              \te{(} \nt{logic\_binder}\repstarsep{\te{,}}
              \te{)} \te{=} \nt{predicate} \\ \indextt{predicate}
      & $|$ & \te{inductive} \nt{identifier}
              \te{:} \nt{logic\_type} \te{=}\\ 
          & &(\te{|} \nt{identifier} \te{:} \nt{predicate})\repstar{}
              \te{\}} \\ \indextt{predicate}
      & $|$ & \te{axiom} \nt{identifier} \te{:} 
              \nt{predicate} \\\indextt{axiom}
      & $|$ & \te{goal} \nt{identifier} \te{:} 
              \nt{predicate} \\\indextt{goal}
   \\[0.1em]

  \nt{logic\_type}
    & $::=$ & \nt{logic\_arg\_type}\repstarsep{\te{,}} \te{->} \te{prop} 
              \\ \indextt{prop}
      & $|$ & \nt{logic\_arg\_type}\repstarsep{\te{,}} \te{->} 
              \nt{primitive\_type} \\ \indextt{logic}
   \\[0.1em]

  \nt{logic\_arg\_type}
    & $::=$ & \nt{primitive\_type} $|$ \nt{primitive\_type} \te{array} \\
   \\[0.1em]

   \nt{logic\_binder}
    & $::=$ & \nt{identifier} \te{:} \nt{primitive\_type} \\
   \\[0.1em]

   \nt{type\_parameters}
    & $::=$ & \te{'}\nt{identifier} $|$ 
              \te{(} (\te{'}\nt{identifier})\repplussep{\te{,}} \te{)}
\end{tabular}\\
\hrulefill
\caption{Syntax of logic declarations}
\label{fig:ldecl}
\end{center}           
\end{figure}

\subsubsection{Programs}

Syntax of types is given Figure~\ref{fig:types}.
Syntax of annotated programs is given Figure~\ref{fig:caml}.
Syntax of input files is given Figure~\ref{fig:input}.

\begin{figure}[htbp]
\begin{center}
\hrulefill\\
\begin{tabular}{lrl}
  \nt{simple\_value\_type}\indexnt{simple\_value\_type}
    & $::=$ & \nt{primitive\_type} \\
      & $|$ & \nt{primitive\_type} \te{ref} \\
      & $|$ & \nt{primitive\_type} \te{array} \\
      & $|$ & \te{(} \nt{value\_type} \te{)} \\
  \\[0.1em]

  \nt{value\_type}\indexnt{value\_type}
    & $::=$ & \nt{simple\_value\_type} \\
      & $|$ & \nt{simple\_value\_type} \te{->} \nt{computation\_type} \\
      & $|$ & \nt{identifier} \te{:} \nt{simple\_value\_type} 
              \te{->} \nt{computation\_type} \\
  \\[0.1em]

  \nt{computation\_type}\indexnt{computation\_type}
    & $::=$ & \te{\{} $[$ \nt{precondition} $]$ \te{\}} \\
      &     & $[$ \te{returns} \nt{identifier} \te{:} $]$ \nt{value\_type}
              \nt{effects} \\
      &     & \te{\{} $[$ \nt{postcondition} $]$ \te{\}} \\
      & $|$ & \nt{value\_type} \\
  \\[0.1em]

  \nt{effects}
    & $::=$ & $[$ \te{reads} \nt{identifier}\repstarsep{\te{,}} $]$
              $[$ \te{writes}  \nt{identifier}\repstarsep{\te{,}}  $]$ 
              $[$ \te{raises}  \nt{identifier}\repstarsep{\te{,}}  $]$ \\
  \\[0.1em]

  \nt{precondition}\indexnt{precondition}
    & $::=$ & \nt{assertion} \\
  \\[0.1em]

  \nt{postcondition}\indexnt{postcondition}
    & $::=$ & \nt{assertion} \nt{exn\_condition}\repstar \\
  \\[0.1em]

  \nt{exn\_condition} 
    & $::=$ & \te{|} \nt{identifier} \te{=>} \nt{assertion} \\
  \\[0.1em]

  \nt{assertion} 
    & $::=$ & \nt{predicate} $[$ \te{as} \nt{identifier} $]$ \\
\end{tabular}\\
\hrulefill
\caption{Syntax of types}
\label{fig:types}
\end{center}            
\end{figure}

\begin{figure}[htbp]
\begin{center}
\hrulefill\\
\begin{tabular}{lrl}
  \nt{prog}\indexnt{prog}
    & $::=$ & \nt{constant} \\
      & $|$ & \nt{identifier} \\
      & $|$ & \te{!} \nt{identifier} \\
      & $|$ & \nt{identifier} \te{:=} \nt{prog} \\
      & $|$ & \nt{identifier} \te{[} \nt{prog} \te{]} \\
      & $|$ & \nt{identifier} \te{[} \nt{prog} \te{]} \te{:=} \nt{prog} \\
      & $|$ & \nt{prog} \nt{infix} \nt{prog} \\
      & $|$ & \nt{prefix} \nt{prog} \\
      & $|$ & \te{let} \nt{identifier} \te{=} \nt{prog} 
              \te{in} \nt{prog} \\
      & $|$ & \te{let} \nt{identifier} \te{=} \te{ref} 
              \nt{prog} \te{in} \nt{prog} \\
      & $|$ & \te{if} \nt{prog} \te{then} \nt{prog}
              $[$ \te{else} \nt{prog} $]$ \\
      & $|$ & \te{while} \nt{prog} \te{do}
              [ \nt{loop\_annot} ] \nt{prog} \te{done} \\
      & $|$ & \nt{prog} \te{;} \nt{prog} \\
      & $|$ & \nt{identifier} \te{:} \nt{prog} \\
      & $|$ & \te{assert} (\te{\{} \nt{assertion} \te{\}})\repplus\
              \te{;} \nt{prog} \\
      & $|$ & \nt{prog} \te{\{}\ \nt{postcondition} \te{\}} \\
      & $|$ & \nt{prog} \te{\{\{}\ \nt{postcondition} \te{\}\}} \\
      & $|$ & \te{fun} \nt{binders} \te{->} \nt{prog} \\
      & $|$ & \te{let} \nt{identifier} \nt{binders} \te{=} \nt{prog} 
              \te{in} \nt{prog} \\
      & $|$ & \te{let} \te{rec} \nt{recfun} $[$ \te{in} \nt{prog} $]$ \\
      & $|$ & \nt{prog} \nt{prog} \\
      & $|$ & \te{raise} \nt{identifier} $[$ \te{:} \nt{value\_type} $]$ \\
      & $|$ & \te{raise} \te{(} \nt{identifier} \nt{prog} \te{)}
              $[$ \te{:} \nt{value\_type} $]$ \\
      & $|$ & \te{try} \nt{prog} \te{with} 
              \nt{handler}\repplussep{\te{|}} \te{end} \\
      & $|$ & \te{absurd} $[$ \te{:} \nt{value\_type} $]$ \\ \indextt{absurd}
      & $|$ & \te{(} \nt{prog} \te{)} \\
      & $|$ & \te{begin} \nt{prog} \te{end} \\
  \\[0.1em]

  \nt{infix}
    & $::=$ & \te{+} $|$ \te{-} $|$ \te{*} $|$ \te{/} $|$ \te{\%} $|$ 
              \te{=} $|$ \te{<>} $|$ 
              \te{<} $|$ \te{<=} $|$ \te{>} $|$ \te{>=} $|$
              \te{||} $|$ \te{\&\&} \\
  \nt{prefix}
    & $::=$ & \te{-} $|$ \te{not} \\
  \\[0.1em]

  \nt{binders}\indexnt{binders}
    & $::=$ & \te{(} \nt{identifier}\repplussep{\te{,}} \te{:}
              \nt{value\_type} \te{)}\repplus \\
  \\[0.1em]

  \nt{recfun}
    & $::=$ & \nt{identifier} \nt{binders} \te{:}
              value\_type \\
      &     & \te{\{} \te{variant} \nt{wf\_arg} \te{\}}
              \te{=} \nt{prog} \\
  \\[0.1em]

  \nt{loop\_annot}
    & $::=$ & \te{\{} [ \te{invariant} \nt{assertion} ]
              [ \te{variant} \nt{wf\_arg} ] \te{\}} \\
  \\[0.1em]

  \nt{wf\_arg} 
    & $::=$ & \nt{term} $[$ \te{for} \nt{identifier} $]$ \\

  \\[0.1em]

  \nt{handler}\indexnt{handler}
    & $::=$ & \nt{identifier} \te{->} \nt{prog} \\
      & $|$ & \nt{identifier} \nt{identifier} \te{->} \nt{prog} \\
  
\end{tabular}\\
\hrulefill\caption{Syntax of annotated programs}
\label{fig:caml}
\end{center}
\end{figure}


\begin{figure}[htbp]
\begin{center}
\hrulefill\\
\begin{tabular}{lrl}
  \nt{file}
    & $::=$ & \nt{declaration}\repstar\ \\
  \\[0.1em]

  \nt{declaration}
    & $::=$ & \te{let} \nt{identifier} $[$ \nt{binders} $]$ \te{=} \nt{prog} \\
      & $|$ & \te{let} \te{rec} \nt{recfun} \\
      & $|$ & $[$ \te{external} $]$ 
              \te{parameter} \nt{identifier}\repplussep{\te{,}}
              \te{:} \nt{value\_type} \\ \indextt{parameter}\indextt{external}
      & $|$ & \te{exception} \nt{identifier} 
              $[$ \te{of} \nt{primitive\_type} $]$ \\ \indextt{exception}
      & $|$ & \nt{l\_declaration}
\end{tabular}\\
\hrulefill
\caption{Syntax of input files}
\label{fig:input}
\end{center}           
\end{figure}

\section{Semantics}\label{semantics}\index{Semantics}

\subsection{Logic}\label{semantics:logic}

The abstract syntax for types ($\tau$), terms ($t$) and predicates
($P$) is given by the following grammars:
\begin{displaymath}
  \begin{array}{rrl}
    \tau & ~::=~ & \alpha ~|~ (\tau,\dots,\tau)~s \\
    t & ~::=~ & x ~|~ f(t,\dots,t) \\
    P & ~::=~ & p(t,\dots,t) \\ 
      &    |~~ & \top ~|~ \bot ~|~ P \land P ~|~ P \lor P
                 ~|~ \lnot P ~|~ P\Rightarrow P \\
      &    |~~ & \forall x:\tau.\,P ~|~ \exists x:\tau.\,P
  \end{array}
\end{displaymath}
A theory $\Sigma$ is a finite list of declarations $\delta$ where
\begin{displaymath}
  \begin{array}{rrl}
  \delta & ~::=~ & 
      \texttt{type}~\vec{\alpha}~s ~|~ x:\tau ~|~
      \mathtt{logic}~f:\forall\vec{\alpha}.\,\tau,\dots,\tau\rightarrow\tau \\
         &    |~~ &   
      \mathtt{logic}~p:\forall\vec{\alpha}.\,\tau,\dots,\tau\rightarrow\mathtt{prop} ~|~ 
      \texttt{axiom}~\forall\vec{\alpha}.\,P ~|~
      \texttt{goal}~\forall\vec{\alpha}.\,P \\
  \end{array} 
\end{displaymath}

\subsubsection{Typing}

\newcommand{\Subst}[2]{\ensuremath{\mathsf{Subst}(#1,#2)}}

Well-formed types ($\Sigma\vdash\wf{\tau}$):
% types wf
\begin{displaymath}
  \frac{}{\Sigma\vdash\wf{\alpha}}(\mathsf{Ty}_1)
  \qquad
  \frac{\mathtt{type}~(\alpha_1,\dots,\alpha_n)~s\in\Sigma
        \quad
        \forall i,\,\Sigma\vdash\wf{\tau_i}}
       {\Sigma\vdash\wf{(\tau_1,\dots,\tau_n)~s}}(\mathsf{Ty}_2)
\end{displaymath}
Well-typed terms ($\Sigma\vdash t:\tau$):
\begin{displaymath}
  \frac{x:\tau\in\Sigma}{\Sigma\vdash x:\tau}(\mathsf{T}_1)
  \qquad
  \frac{
    \begin{array}{c}
      \texttt{logic}~f:\forall\vec{\alpha}.\,\tau_1,\dots,\tau_n\rightarrow\tau\in\Sigma \\[0.2em]
      \Subst{\sigma}{\Sigma} \quad
       \forall i,\, \Sigma\vdash t_i:\sigma(\tau_i) \\
    \end{array}}
       {\Sigma\vdash f(t_1,\dots,t_n):\sigma(\tau)}(\mathsf{T}_2)
\end{displaymath}
\begin{description}
\item[~~~]
  where $\sigma$ is a mapping from type variables to types, naturally
  extended to types with $\sigma((\tau_1,\dots,\tau_n)~s) =
  (\sigma(\tau_1),\dots,\sigma(\tau_n))~s$. 
  We write $\Subst{\sigma}{\Sigma}$ whenever
  $\Sigma\vdash\wf{\sigma(\alpha)}$ holds for any type variable $\alpha$.
\end{description}
Well-typed predicates ($\Sigma\vdash\wf{P}$):
\begin{displaymath}
 \frac{
    \begin{array}{c}
      \texttt{logic}~p:\forall\vec{\alpha}.\,\tau_1,\dots,\tau_n\rightarrow\mathtt{prop}\in\Sigma \quad
        \Subst{\sigma}{\Sigma} \quad
        \forall i,\,\Sigma\vdash t_i:\sigma(\tau_i) \\
    \end{array}}
       {\Sigma\vdash\wf{p(t_1,\dots,t_n)}}(\mathsf{P}_1)
\end{displaymath}
\begin{displaymath}
  \frac{}
       {\Sigma\vdash\wf{\top}}(\mathsf{P}_2)
  \qquad
  \frac{}
       {\Sigma\vdash\wf{\bot}}(\mathsf{P}_3)
  \qquad
  \frac{\Sigma\vdash\wf{P_1} \quad \Sigma\vdash\wf{P_2}}
       {\Sigma\vdash\wf{P_1\land P_2}}(\mathsf{P}_4)
\end{displaymath}
\begin{displaymath}
  \frac{\Sigma\vdash\wf{P_1} \quad \Sigma\vdash\wf{P_2}}
       {\Sigma\vdash\wf{P_1\lor P_2}}(\mathsf{P}_5)
  \qquad
  \frac{\Sigma\vdash\wf{P}}
       {\Sigma\vdash\wf{\lnot P}}(\mathsf{P}_6)
\end{displaymath}
\begin{displaymath}
  \frac{\Sigma\vdash\wf{P_1} \quad \Sigma\vdash\wf{P_2}}
       {\Sigma\vdash\wf{P_1\Rightarrow P_2}}(\mathsf{P}_7)
\end{displaymath}
\begin{displaymath}
  \frac{\Sigma,x:\tau\vdash\wf{P}}
       {\Sigma\vdash\wf{\forall x:\tau.\,P}}(\mathsf{P}_8)
  \qquad
  \frac{\Sigma,x:\tau\vdash\wf{P}}
       {\Sigma\vdash\wf{\exists x:\tau.\,P}}(\mathsf{P}_8)
\end{displaymath}
Well-formed theories ($\vdash\wf{\Sigma}$):
\begin{displaymath}
  \frac{}{\vdash\wf{\emptyset}}(\textsf{Th}_1)
  \qquad
  \frac{\texttt{type}~s\not\in\Sigma}
       {\vdash\wf{\Sigma,\texttt{type}~(\alpha_1,\dots,\alpha_n)~s}}(\textsf{Th}_2)
  \qquad
  \frac{x\not\in\Sigma \quad \Sigma\vdash\wf{\tau}}
       {\vdash\wf{\Sigma,x:\tau}}(\textsf{Th}_3)
\end{displaymath}
\begin{displaymath}
  \frac{\texttt{logic}~f\not\in\Sigma \quad
        \forall i,\, \Sigma\vdash\wf{\tau_i}}
       {\vdash\wf{\Sigma,\mathtt{logic}~f:\forall\vec{\alpha}.\,\tau_1,\dots,\tau_n\rightarrow\tau_{n+1}}}(\textsf{Th}_4)
\end{displaymath}
\begin{displaymath}
  \frac{\texttt{logic}~p\not\in\Sigma \quad
        \forall i,\, \Sigma\vdash\wf{\tau_i}}
       {\vdash\wf{\Sigma,\mathtt{logic}~p:\forall\vec{\alpha}.\,\tau_1,\dots,\tau_n\rightarrow\mathtt{prop}}}(\textsf{Th}_5)
\end{displaymath}
\begin{displaymath}
  \frac{\Sigma\vdash\wf{P}}
       {\vdash\wf{\Sigma,\texttt{axiom}~\forall\vec{\alpha}.\,P}}(\textsf{Th}_6)  \qquad
  \frac{\Sigma\vdash\wf{P}}
       {\vdash\wf{\Sigma,\texttt{goal}~\forall\vec{\alpha}.\,P}}(\textsf{Th}_6)  
\end{displaymath}
Well-formed definitions of functions and predicates:
\begin{displaymath}
  \frac{\Sigma\vdash\wf{\tau_i} \quad
        \Sigma,x_1:\tau_1,\dots,x_n:\tau_n\vdash t:\tau}
       {\Sigma\vdash\wf{\mathtt{function}~f(x_1:\tau_1,\dots,x_n:\tau_n) : \tau = t}}(\textsf{Th}_7)
\end{displaymath}
\begin{displaymath}
  \frac{\Sigma\vdash\wf{\tau_i} \quad
        \Sigma,x_1:\tau_1,\dots,x_n:\tau_n\vdash\wf{P}}
       {\Sigma\vdash\wf{\mathtt{predicate}~p(x_1:\tau_1,\dots,x_n:\tau_n) = P}}(\textsf{Th}_8)
\end{displaymath}
Then, as far as typing and validity in concerned, the function $f$
(resp. the predicate $p$) is added to $\Sigma$ as 
$\mathtt{logic}~f:\tau_1,\dots,\tau_n\rightarrow\tau$
(resp. $\mathtt{logic}~p:\tau_1,\dots,\tau_n\rightarrow\mathtt{prop}$).

\subsubsection{Validity}

Validity is defined as a set of natural deduction rules
($\Sigma\models P$). For the sake of clarity, 
we write $\Sigma,P$ for $\Sigma,\texttt{axiom}~P$ in the following.
A substitution $\sigma$ over from type variables to types in extended
to terms and predicates in the obvious way. We write $P[t/x]$ for the
substitution of all the occurrences of a free variable $x$ in $P$ by a
term $t$. 
\begin{displaymath}
  \frac{\texttt{axiom}~\forall\vec{\alpha}.\,P\in\Sigma \quad
        \Subst{\sigma}{\Sigma}}
       {\Sigma\models\sigma(P)}(\mathsf{Ax})
  \qquad
  \frac{\Sigma\models Q \quad 
        \Sigma,Q\models P}
       {\Sigma\models P}(\mathsf{Cut})
\end{displaymath}
\begin{displaymath}
  \frac{}{\Sigma\models\top}(\mathsf{True})
  \qquad
  \frac{\Sigma\models\bot \quad \Sigma\vdash\wf{P}}
       {\Sigma\models P}(\mathsf{False})
  \qquad
  \frac{\Sigma\vdash\wf{P}}{\Sigma\models P\lor\lnot P}(\mathsf{EM})
\end{displaymath}
% and
\begin{displaymath}
  \frac{\Sigma\models P \quad \Sigma\models Q}
       {\Sigma\models P\land Q}(\mathsf{And}_1)
  \qquad
  \frac{\Sigma\models P\land Q}
       {\Sigma\models P}(\mathsf{And}_2)
  \qquad
  \frac{\Sigma\models P\land Q}
       {\Sigma\models Q}(\mathsf{And}_3)
\end{displaymath}
% or
\begin{displaymath}
  \frac{\Sigma\models P \quad \Sigma\vdash\wf{Q}}
       {\Sigma\models P\lor Q}(\mathsf{Or}_1)
  \qquad
  \frac{\Sigma\models Q \quad \Sigma\vdash\wf{P}}
       {\Sigma\models P\lor Q}(\mathsf{Or}_2)
\end{displaymath}
\begin{displaymath}
  \frac{\Sigma\models P\lor Q \quad
        \Sigma,P\models R \quad \Sigma,Q\models R}
       {\Sigma\models R}(\mathsf{Or}_3)
\end{displaymath}
% not
\begin{displaymath}
  \frac{\Sigma\vdash\wf{P} \quad \Sigma,P\models\bot}
       {\Sigma\models\lnot P}(\mathsf{Not}_1)
  \qquad
  \frac{\Sigma\models P \quad \Sigma\models\lnot P}
       {\Sigma\models\bot}(\mathsf{Not}_2)
\end{displaymath}
% implies
\begin{displaymath}
  \frac{\Sigma\vdash\wf{P} \quad \Sigma,P\models Q}
       {\Sigma\models P\Rightarrow Q}(\mathsf{Imp}_1)
  \qquad
  \frac{\Sigma\models P\Rightarrow Q \quad \Sigma\models P}
       {\Sigma\models Q}(\mathsf{Imp}_2)
\end{displaymath}
% forall
\begin{displaymath}
  \frac{x\not\in\Sigma \quad
        \Sigma\vdash\wf{\tau} \quad 
        \Sigma,x:\tau\models P}
       {\Sigma\models\forall x:\tau.\,P}(\mathsf{Forall}_1)
\end{displaymath}
\begin{displaymath}
  \frac{\Sigma\models\forall x:\tau.\,P \quad
        \Sigma\vdash t:\tau}
       {\Sigma\models P[t/x]}(\mathsf{Forall}_2)
\end{displaymath}
% exists
\begin{displaymath}
  \frac{%\Sigma\vdash\wf{\tau} \quad INUTILE CAR CONSEQUENCE
        \Sigma\vdash t:\tau \quad
        x\not\in\Sigma \quad
        \Sigma\models P[t/x]}
       {\Sigma\models\exists x:\tau.\,P}(\mathsf{Exists}_1)
\end{displaymath}
\begin{displaymath}
  \frac{\Sigma\models\exists x:\tau.\,P \quad
        x\not\in\Sigma \quad
        \Sigma,x:\tau,P\models Q}
       {\Sigma\models Q}(\mathsf{Exists}_2)
\end{displaymath}
Deduction rules for equality: 
\begin{displaymath}
  \frac{\Sigma\vdash t:\tau}
       {\Sigma\models t=t}(\mathsf{Eq}_1)
  \qquad
  \frac{\Sigma\models x=y \quad
        \Sigma,z:\tau\vdash\wf{P} \quad 
        \Sigma\models P[x/z]}
       {\Sigma\models P[y/z]}(\mathsf{Eq}_2)
\end{displaymath}
Arithmetic:
An arithmetic proposition $P$ is a proposition built from 
variables of type \textit{int}, integers constants,
the functions symbols \textit{add\_int} and \textit{sub\_int}, the predicates
\textit{lt\_int, le\_int, gt\_int, ge\_int} and the equality.
If $x_1,\dots,x_n$ are the free variables of $P$,
we write $x_1,\dots,x_n\models_A P$ whenever $P$ is valid in
Presburger arithmetic (which is decidable).
Then we have the following deduction rule for arithmetic:
\begin{displaymath}
  \frac{x_1,\dots,x_n\models_A P}
       {\Sigma\models\forall x_1:\mathit{int}.\,\dots\forall
         x_n:\mathit{int}.\, P}(\mathsf{Arith})
\end{displaymath}


\subsection{Programs}\label{semantics:programs}

\newcommand{\pget}[1]{\ensuremath{\texttt{!}#1}}
\newcommand{\pset}[2]{\ensuremath{#1~\texttt{:=}~#2}}


\newcommand{\pblock}[1]{\ensuremath{\kw{begin}~#1~\kw{end}}}
\newcommand{\pseq}[2]{\ensuremath{#1\texttt{;}~#2}}
\newcommand{\ppre}[2]{\ensuremath{\{#1\}~#2}}
\newcommand{\passert}[2]{\ensuremath{\kw{assert}~\{#1\};~#2}}
\newcommand{\passume}[2]{\ensuremath{\{\kw{assume}~#1\}~#2}}
\newcommand{\ppost}[2]{\ensuremath{#1~\{#2\}}}
\newcommand{\popost}[2]{\ensuremath{#1~\{\{#2\}\}}}
\newcommand{\prefine}[4]{\ensuremath{\langle\{#1\}~#2~#3~\{#4\}\rangle}}
\newcommand{\ploop}[2]{\ensuremath{\kw{loop}~#1~\{\kw{variant}~#2\}}}
\newcommand{\ploopI}[3]{\ensuremath{\kw{loop}~#1~\{\kw{invariant}~#2~\kw{variant}~#3\}}}
\newcommand{\pwhile}[3]{\ensuremath{\kw{while}~#1~\kw{do}~#2~\{\kw{variant}~#3\}}}
\newcommand{\pwhileI}[4]{\ensuremath{\kw{while}~#1~\kw{do}~#2~\{\kw{invariant}~#3~\kw{variant}~#4\}}}
\newcommand{\pite}[3]{\ensuremath{\kw{if}~#1~\kw{then}~#2~\kw{else}~#3}}
\newcommand{\pfun}[4]{\ensuremath{\kw{fun}~(#1:#2)\rightarrow\{#3\}~#4}}
\newcommand{\tapp}[2]{\ensuremath{#1\texttt{(}#2\texttt{)}}}
\newcommand{\papp}[2]{\ensuremath{#1~#2}}
\newcommand{\precfun}[5]{\ensuremath{\kw{rec}~#1:#2~\{\kw{variant}~#3\}=\{#4\}~#5}}
\newcommand{\closure}[3]{\ensuremath{\kw{rec}~#1~#2=#3}}
\newcommand{\pletin}[3]{\ensuremath{\kw{let}~#1~\texttt{=}~#2~\kw{in}~#3}}
\newcommand{\pletref}[3]{\ensuremath{\kw{let}~#1~\texttt{=}~\kw{ref}~#2~\kw{in}~#3}}
\newcommand{\praisex}[3]{\ensuremath{\kw{raise}~({#1}~{#2}):#3}}
\renewcommand{\exn}[2]{\ensuremath{\kw{exception}~#1~\kw{of}~#2}}
\newcommand{\Exit}{\ensuremath{\mathit{Exit}}}
\newcommand{\ptrywith}[4]{\ensuremath{\kw{try}~#1~\kw{with}~#2~#3\rightarrow#4~\kw{end}}}
\newcommand{\pcoerce}[2]{\ensuremath{(#1:#2)}}

\newcommand{\tint}{\kw{int}}
\renewcommand{\tarray}[1]{\ensuremath{#1~\kw{array}}}
\newcommand{\tfun}[3]{\ensuremath{#1:#2\rightarrow#3}}

\renewcommand{\prepost}[3]{\ensuremath{\{#1\}\,#2\,\{#3\}}}
\newcommand{\prepostE}[4]{\ensuremath{\{#1\}\,#2\,\{#3 ~|~ E \Rightarrow #4\}}}
\renewcommand{\result}{\ensuremath{\mathit{result}}}

\newcommand{\ptrue}{\ensuremath{\top}}
\newcommand{\pfalse}{\ensuremath{\kw{false}}}
\newcommand{\pnot}[1]{\ensuremath{\kw{not}~#1}}
\newcommand{\pand}[2]{\ensuremath{#1~\kw{and}~#2}}
\newcommand{\por}[2]{\ensuremath{#1~\kw{or}~#2}}
\newcommand{\pimplies}[2]{\ensuremath{#1~\kw{->}~#2}}
\newcommand{\pif}[3]{\ensuremath{\kw{if}~#1~\kw{then}~#2~\kw{else}~#3}}
\newcommand{\pforall}[2]{\ensuremath{\kw{forall}~#1\texttt{.}~#2}}
\newcommand{\pexists}[2]{\ensuremath{\kw{exists}~#1\texttt{.}~#2}}
\newcommand{\eqdef}{\equiv}

Program types ($\theta$) and specifications ($\kappa$) 
are classified as follows:
\begin{displaymath}
  \begin{array}{rrl}
    \theta & ~::=~ & \tau ~|~ \tref{\tau} ~|~ (x:\theta) \rightarrow\kappa \\
    \kappa &~::=~& \prepost{p}{\theta~\epsilon}{q} \\
    q &~::=~& p;E\Rightarrow p;\dots;E\Rightarrow p \\
    \epsilon &~::=~&
    \kw{reads}~x,\dots,x~\kw{writes}~x,\dots,x ~ \kw{raises}~E,\dots,E 
  \end{array}
\end{displaymath}
A value of type $\theta$ is either an immutable variable of a
pure type ($\tau$), a reference containing a value of a pure type
($\tref{\tau}$) or a function of type
$(x:\theta)\rightarrow\prepost{p}{\theta'~\epsilon}{q}$ 
mapping the formal parameter $x$ to the specification of its body, that
is a precondition $p$, the type $\theta'$ for the returned value,
an effect $\epsilon$ and a postcondition $q$. An effect
is made of tree lists of variables: the references possibly accessed
(\texttt{reads}), the references possibly modified (\texttt{writes})
and the exceptions possibly raised (\texttt{raises}). A postcondition
$q$ is made of several parts: one 
for the normal termination and one for each possibly raised exception
($E$ stands for an exception name).

When a function specification $\prepost{p}{\theta~\epsilon}{q}$ has no
precondition and no postcondition (both being \ptrue) and no effect
($\epsilon$ is made of three empty lists) it can be shortened to
$\theta$. In particular,
$(x_1:\theta_1)\rightarrow\cdots\rightarrow(x_n:\theta_n)\rightarrow\kappa$
denotes the type of a function with $n$ arguments that has no effect
as long as it not applied to $n$ arguments. Note that functions can be
partially applied.

The syntax for program expressions is given in Figure~\ref{fig:prog:expr}.
In particular, programs contain \emph{pure terms} ($t$) made of
constants, variables, dereferences (written $\pget{x}$) and application
of function symbols from the logic to pure terms. 
$\ploopI{e}{p}{t}$ is an infinite loop of body $e$, invariant $p$ and
which termination is ensured by the variant $t$. 

\begin{figure}[htbp]
  %\centering
  \begin{displaymath}
    \begin{array}{rrl}
      t & ~::=~ & x ~|~ \pget{x} ~|~ \tapp{f}{t,\dots,t} \\
      e & ~::=~ & t \\
        &  |~ & \pletin{x}{e}{e} \\
        &  |~ & \pletref{x}{e}{e} \\
        &  |~ & \pite{e}{e}{e} \\
        &  |~ & \ploopI{e}{p}{t} \\
        &  |~ & \plabel{L}{e} \\
        &  |~ & \praisex{E}{e}{\theta} \\
        &  |~ & \ptrywith{e}{E}{x}{e} \\
        &  |~ & \passert{p}{e} \\
        &  |~ & \ppost{e}{q} \\
        &  |~ & \popost{e}{q} \\
        &  |~ & \pfun{x}{\theta}{p}{e} \\
        &  |~ & \precfun{x~(x:\theta)\dots(x:\theta)}{\theta}{t}{p}{e} \\
        &  |~ & \papp{e}{e}
    \end{array}
  \end{displaymath}
  \caption{Abstract syntax of program expressions}
  \label{fig:prog:expr}
\end{figure}

\paragraph{Syntactic sugar.}
The traditional sequence construct is only syntactic sugar for a
\texttt{let-in} binder where the variable does not occur in $e_2$: 
\begin{displaymath}
\pseq{e_1}{e_2} ~\eqdef~ \pletin{\_}{e_1}{e_2}
\end{displaymath}
The assignment construct is syntactic sugar for the application of the
function \texttt{ref\_set} from \why's prelude (see~\ref{prelude}):
\begin{displaymath}
  \pset{x}{e} ~\eqdef~ \mathtt{ref\_set}~x~e
\end{displaymath}
We also simplify the \texttt{raise} construct whenever both the exception
contents and the whole \texttt{raise} expression have type \texttt{unit}:
\begin{displaymath}
  \kw{raise}~E ~\eqdef~ \praisex{E}{\void}{\unit}
\end{displaymath}
The traditional \texttt{while} loop is also syntactic sugar for a
combination of an infinite loop and the use of an exception $\Exit$ to
exit the loop:
\begin{displaymath}
  \begin{array}{ll}
    \pwhileI{e_1}{e_2}{p}{t} ~\equiv \\
    \quad \kw{try} \\
    \quad \quad \kw{loop} ~ \pite{e_1}{e_2}{\kw{raise}~\Exit} \\
    \quad \quad \{\kw{invariant} ~ p ~ \kw{variant} ~ t\} \\
    \quad \kw{with} ~ \Exit~\_ ~ \texttt{->} ~ \void ~ \kw{end}
  \end{array}
\end{displaymath}

\subsubsection{Typing}

The typing of program expressions is straightforward, with
polymorphism but no type inference.

% TODO: give typing rules

\subsubsection{Operational semantics}

The language is strict (all arguments always evaluated before the
function call), with a left-to-right evaluation order.

% TODO: give operation semantics rules

\subsubsection{Weakest preconditions}\label{wp}
\index{Weakest preconditions}

\renewcommand{\wpre}[2]{\ensuremath{\mathit{wp}(#1,#2)}}
\renewcommand{\wprx}[3]{\ensuremath{\mathit{wp}(#1,#2;#3)}}
\newcommand{\unref}[1]{#1}
\newcommand{\old}[1]{\ensuremath{\mathtt{old}(#1)}}
\newcommand{\at}[2]{\ensuremath{\mathtt{at}(#1,#2)}}

Programs correctness is defined using a calculus of weakest
preconditions. We note $\wprx{e}{q}{r}$ the weakest precondition for a
program expression $e$ and a postcondition $q;r$ where $q$ is the
property to hold when terminating normally and $r = E_1\Rightarrow
q_1; \dots; E_n\Rightarrow q_n$ is the set of properties to hold for
each possibly uncaught exception.
Expressing the correctness of a program $e$ is simply a matter of
computing $\wpre{e}{\textsf{True}}$.

The rules for the basic constructs are the following:
\begin{displaymath}
  \begin{array}{rl}
    \wprx{t}{q}{r} & = 
       q[\result \leftarrow \unref{t}] \\
%    \wprx{\pset{x}{e}}{q}{r} & = 
%       \wprx{e}{q[\result\leftarrow\void; x\leftarrow\result]}{r} \\
    \wprx{\pletin{x}{e_1}{e_2}}{q}{r} & =
       \wprx{e_1}{\wprx{e_2}{q}{r}[x\leftarrow\result]}{r} \\
    \wprx{\pletref{x}{e_1}{e_2}}{q}{r} & =
       \wprx{e_1}{\wprx{e_2}{q}{r}[x\leftarrow\result]}{r} \\
    \wprx{\pite{e_1}{e_2}{e_3}}{q}{r} & =
       \wprx{e_1}{\pite{\result}{\wprx{e_2}{q}{r}}{\wprx{e_3}{q}{r}}}{r} \\
    \wprx{\plabel{L}{e}}{q}{r} & =
       \wprx{e}{q}{r}[\at{x}{L}\leftarrow x] \\
  \end{array}
\end{displaymath}
On the case of an exception free sequence, we retrieve the usual rule:
\begin{displaymath}
  \wpre{\pseq{e_1}{e_2}}{q} = \wpre{e_1}{\wpre{e_2}{q}}
\end{displaymath}
The cases of exceptions and annotations are also straightforward:
\begin{displaymath}
  \begin{array}{rl}
    \wprx{\praisex{E}{e}{\theta}}{q}{r} & = 
       \wprx{e}{r(E)}{r} \\
    \wprx{\ptrywith{e_1}{E}{x}{e_2}}{q}{r} & = 
       \wprx{e_1}{q}{E\Rightarrow\wprx{e_2}{q}{r}[x\leftarrow\result];r} \\
       % FAUX \wprx{e_1}{q}{\wprx{e_2}{q}{r}[v\leftarrow\result]} \\
    \wprx{\passert{p}{e}}{q}{r} & =
       p \land \wprx{e}{q}{r} \\ 
    \wprx{\ppost{e}{q';r'}}{q}{r} & =
       \wprx{e}{q' \land q}{r'\land r} \\ 
    \wprx{\popost{e}{q';r'}}{q}{r} & =
       \wprx{e}{q'}{r'} \land
       \forall\omega.\forall\result.q'\Rightarrow q\land r'\Rightarrow r \\ 
  \end{array}
\end{displaymath}
The last implication $r'\Rightarrow r$ is actually an abuse for the 
conjunction of all implications for each postcondition part.
The case of an infinite loop is more subtle:
\begin{displaymath}
  \begin{array}{rl}
    \wprx{\ploopI{e}{p}{t}}{q}{r} & = 
       p ~\land~ \forall\omega. ~ 
       p \Rightarrow \wprx{\plabel{L}{e}}{p\land t<\at{t}{L}}{r} \\
  \end{array}
\end{displaymath}
where $\omega$ stands for the set of references possibly modified by
the loop body (the \texttt{writes} part of $e$'s effect).
Here the weakest precondition expresses that the invariant must hold
initially and that for each turn in the loop (represented by
$\omega$), either
$p$ is preserved by $e$ and $e$ decreases the value of $t$ (to ensure
termination), or $e$ raises an exception and thus must establish $r$
directly. 

By combining this rule and the rule for the conditional, we can retrieve
the rule for the usual while loop:
\begin{displaymath}
  \begin{array}{ll}
    & \wprx{\pwhileI{e_1}{e_2}{p}{t}}{q}{r} \\
   =& p ~\land~ \forall\omega. ~ p \Rightarrow \\
    &    \mathit{wp}(\plabel{L}{\pite{e_1}{e_2}{\kw{raise}~E}},
         p\land t<\at{t}{L},E\Rightarrow q;r) \\
   =&p ~\land~ \forall\omega. ~ p \Rightarrow \\
    & \mathit{wp}(e_1,\pite{\result}{\wpre{e_2}{p\land t<\at{t}{L}}}{q},r)%
      [\at{x}{L}\leftarrow x]
  \end{array}
\end{displaymath}

Finally, we give the rules for functions and function calls.
Since a function cannot be mentioned within the postcondition, the
weakest preconditions for function constructs \texttt{fun} and
\texttt{rec} are only expressing the correctness of the function body:
\begin{displaymath}
  \begin{array}{rl}
    \wprx{\pfun{x}{\theta}{p}{e}}{q}{r} 
    & = q ~\land~ \forall x.\forall\rho. p \Rightarrow\wpre{e}{\textsf{True}}
  \end{array}
\end{displaymath}
\begin{displaymath}
  \begin{array}{l}
    \wprx{\precfun{f~(x_1:\theta_1)\dots(x_n:\theta_n)}{\theta}{t}{p}{e}}{q}{r}
    \\ = q ~\land~ \forall x_1.\dots\forall x_n.\forall\rho. 
    p \Rightarrow\wpre{\plabel{L}{e}}{\textsf{True}}
  \end{array}
\end{displaymath}
where $\rho$ stands for the set of references possibly accessed by
the loop body (the \texttt{reads} part of $e$'s effect).
In the case of a recursive function,
$\wpre{\plabel{L}{e}}{\textsf{True}}$ must be computed within an
environment where $f$ is assumed to have type
$(x_1:\theta_1)\rightarrow\cdots\rightarrow(x_n:\theta_n)\rightarrow\prepost{p\land
  t<\at{t}{L}}{\theta~\epsilon}{q}$
i.e. where the decreasing of the variant $t$ has been added to the
precondition of $f$.

The case of a function call $\papp{e1}{e2}$ can be simplified to the
case of an application $\papp{x_1}{x_2}$ of one variable to another,
using the following transformation if needed:
\begin{displaymath}
    \papp{e_1}{e_2} \equiv 
      \pletin{x_1}{e_1}{\pletin{x_2}{e_2}{\papp{x_1}{x_2}}}
\end{displaymath}
Then assuming that $x_1$ has type
$(x:\theta)\rightarrow\prepost{p'}{\theta'~\epsilon}{q'}$, we define
\begin{displaymath}
  \wpre{x_1~x_2}{q} = p'[x\leftarrow x_2] ~\land~
  \forall\omega. \forall\result. 
    (q'[x\leftarrow x_2] \Rightarrow q)[\old{t}\leftarrow t]
\end{displaymath}
that is (1) the precondition of the function must hold and (2) its
postcondition must imply the expected property $q$ whatever the
values of the modified references and of the result are.
Note that $q$ and $q'$ may contain exceptional parts and thus the
implication is again an abuse for the conjunction of all implications for
each postcondition part.

In the case of an assignment $\pset{x}{e}$ the rules above combined
with the specification of \texttt{ref\_set} give
(assuming that $\result$ does not occur in $q$):
\begin{displaymath}
  \wprx{\pset{x}{e}}{q}{r} = \wprx{e}{x=\result \Rightarrow q}{r} 
\end{displaymath}
And in the case of the assigment of a side-effect free expression $t$,
it simplifies to
\begin{displaymath}
  \wpre{\pset{x}{t}}{q} = (x=t \Rightarrow q)
\end{displaymath}
which is equivalent to the usual rule $\wpre{\pset{x}{t}}{q} =
q[x\leftarrow t]$.

\section{Prelude}
\label{prelude}\index{Prelude}

Apart from the type names \texttt{unit}, \texttt{bool}, \texttt{int},
\texttt{real}, and the type constructor \texttt{ref}, nothing is
built-in in the \why\ tool. But two prelude 
files are read before considering the files given on the command
line. The prelude files are
\begin{itemize}
\item \url{/usr/local/lib/why/prelude.why}, and
\item \url{/usr/local/lib/why/arrays.why}
\end{itemize}
(unless your library path is set differently).
The interpretation of this prelude is prover-dependent. 
It is possible not to load the prelude using the command-line option
\texttt{--no-prelude}, or to load only the first prelude file using
the command-line option \texttt{--no-arrays}.
The remaining of this section describes the contents of the prelude files.

\subsection{\texttt{prelude.why}}

{\small
\verbatiminput{../lib/why/prelude.why}
}

\subsection{\texttt{arrays.why}}

{\small
\verbatiminput{../lib/why/arrays.why}
}

\section{Provers specificities}

\subsection{Coq}
\label{lib:coq}\index{Coq library}

Several \coq\ modules are delivered with \why. If installation is
properly done, they should be found in subdirectory \texttt{user-contrib}
of the \coq\ standard library (usually \texttt{/usr/lib/coq}
unless some other path was specified when installing \coq). 

% \coq\ files generated by \why\ usually start with \texttt{Require Why}
% (unless specified otherwise using command line option
% \texttt{--coq-preamble}). For this module to be found by \coq, the
% \coq\ load path must be set accordingly, either using the \coq\ command
% line option \texttt{-I} or the \coq\ command \texttt{Add LoadPath} (to
% be put in a \texttt{.coqrc} file for instance).

\paragraph{Types.} 
Types are mapped as follows:
\begin{center}
  \begin{tabular}{|l|l|}
    \hline
    \why & \coq \\
    \hline
    \texttt{unit} & \texttt{unit} \\
    \hline
    \texttt{bool} & \texttt{bool} \\
    \hline
    \texttt{int} & \texttt{Z} \\
    \hline
    \texttt{real} & \texttt{R} \\
    \hline
    $\tau$ \texttt{array} & \texttt{(array $\tau$)},
                            from module \texttt{WhyArrays} \\
    \hline
  \end{tabular}
\end{center}

\paragraph{Arrays.} 
Arrays are introduced in module \texttt{WhyArrays}. 
Some tactics are provided to help the user simplifying array expressions
in proof obligations:
\begin{description}
\item[\texttt{WhyArrays}] : repeatedly simplifies
  \texttt{access}/\texttt{update}using combinations and
  \texttt{array\_length} expressions
\item[\texttt{AccessSame}] : rewrites
  $\texttt{access}~(\mathtt{update}~t~i~v)~i$  into $v$,
  simplifies with \texttt{WhyArrays} and attempts \texttt{omega} on
  every subgoal
\item[\texttt{AccessOther}] : rewrites 
    $\texttt{access}~(\mathtt{update}~t~i~v)~j$  into $\mathtt{access}~t~j$,
  simplifies with \texttt{WhyArrays} and attempts \texttt{omega} on
  every subgoal
\item[\texttt{ArraySubst} $t$] : similar to \texttt{subst $t$}, with
  additional simplifications
\end{description}

\subsection{PVS}
\label{lib:pvs}\index{PVS library}

\paragraph{Types.} 
Types are mapped as follows:
\begin{center}
  \begin{tabular}{|l|l|}
    \hline
    \why & \coq \\
    \hline
    \texttt{unit} & \texttt{unit}, from theory \texttt{why} \\
    \hline
    \texttt{bool} & \texttt{bool} \\
    \hline
    \texttt{int} & \texttt{int} \\
    \hline
    \texttt{real} & \texttt{real} \\
    \hline
    $\tau$ \texttt{array} & \texttt{warray[$\tau$]},
                            from theory \texttt{why} \\
    \hline
  \end{tabular}
\end{center}

\paragraph{Arrays.} An array whose elements are of type \texttt{T} is a
\pvs\ value of type \texttt{warray[T]}. This type is defined in the
theory \texttt{why\_arrays}, as
a pair of an integer, the array length, and a function mapping
indices to elements :
\begin{verbatim}
     warray: TYPE = [ n:int, [ below(n) -> T ] ]
\end{verbatim}
Two other theories, \texttt{why\_int\_array\_pred} and
\texttt{why\_array\_pred}, introduce the various predicates over
arrays presented in Section~\ref{prelude}.

\appendix
\chapter{DTD of Why project files}

\verbatiminput{../lib/why.dtd}


\nocite{*}
\bibliographystyle{plain}
\bibliography{./biblio}



\newpage
\addcontentsline{toc}{chapter}{Index}
\printindex

\end{document}

%%% Local Variables: 
%%% mode: latex
%%% TeX-master: t
%%% End: 
