\section{Isabelle/HOL}
\label{sec:isabelle}

\index{Isabelle proof assistant}

When using Isabelle from \why, files generated from \why theories and
goals are stored in a dedicated XML format. Those files should not be
edited. Instead, the proofs must be completed in a file with the same
name and extension \texttt{.thy}. This is the file that is opened when
using ``Edit'' action in \texttt{why3 ide}.

\subsection{Installation}

You need version Isabelle2015 or Isabelle2016. Former versions are not
supported. We assume below that your version is 2016, please replace
2016 by 2015 otherwise.

Isabelle must be installed before compiling \why. After compilation
and installation of \why, you must manually add the path
\begin{verbatim}
<Why3 lib dir>/isabelle
\end{verbatim}
into either the user file
\begin{verbatim}
.isabelle/Isabelle2016/etc/components
\end{verbatim}
or the system-wide file
\begin{verbatim}
<Isabelle install dir>/etc/components
\end{verbatim}

\subsection{Usage}

The most convenient way to call Isabelle for discharging a \why goal
is to start the Isabelle/jedit interface in server mode. In this mode,
one must start the server once, before launching \texttt{why3 ide},
using
\begin{verbatim}
isabelle why3_jedit
\end{verbatim}
Then, inside a \texttt{why3 ide} session, any use of ``Edit'' will
transfer the file to the already opened instance of jEdit. When the
proof is completed, the user must send back the edited proof to
\texttt{why3 ide} by closing the opened buffer, typically by hitting
\texttt{Ctrl-w}.

\subsection{Realizations}

Realizations must be designed in some \texttt{.thy} as follows.
The realization file corresponding to some \why file \texttt{f.why}
should have the following form.
\begin{verbatim}
theory Why3_f
imports Why3_Setup
begin

section {* realization of theory T *}

why3_open "f/T.xml"

why3_vc <some lemma>
<proof>

why3_vc <some other lemma> by proof

[...]

why3_end
\end{verbatim}

See directory \texttt{lib/isabelle} for examples.


%%% Local Variables:
%%% mode: latex
%%% TeX-PDF-mode: t
%%% TeX-master: "manual"
%%% End:
