\chapter{Reference manuals for the \why\ tools}
\label{chap:manpages}

\section{Compilation, Installation}
\label{sec:install}

Compilation of \why\ must start with a configuration phase which is run as
\begin{verbatim}
./configure
\end{verbatim}
This analyzes you current configuration and check if requirements hold.
Compilation requires:
\begin{itemize}
\item The Objective Caml compiler, version 3.10 or higher. It is
  available as a binary package for most Unix distributions. For
  debian-based Linux distributions, you can install the packages
\begin{verbatim}
ocaml ocaml-native-compilers
\end{verbatim}
It is also installable from sources, downloadable from the Web site
\url{http://caml.inria.fr/ocaml/}
\end{itemize}

For the IDE, additional Ocaml libraries are needed:
\begin{itemize}
\item The Lablgtk2 library for Ocaml bindings of the gtk2 graphical library.
 For debian-based Linux distributions, you can install the packages
\begin{verbatim}
liblablgtk2-ocaml-dev liblablgtksourceview2-ocaml-dev
\end{verbatim}
It is also installable from sources, available from the site \url{http://wwwfun.kurims.kyoto-u.ac.jp/soft/olabl/lablgtk.html}

\item The Ocaml bindings of the sqlite3 library
For debian-based Linux distributions, you can install the package
\begin{verbatim}
libsqlite3-ocaml-dev
\end{verbatim}
It is also installable from sources, available from the site
\url{http://ocaml.info/home/ocaml_sources.html#ocaml-sqlite3}
\end{itemize}

\subsection{Local use, without installation}

It is not mandatory to install \why\ to use it. Local use is obtained via
\begin{verbatim}
./configure --enable-local
make
\end{verbatim}
The \why\ executables are then available in subdirectory \texttt{bin/}.

\subsection{Installation of the \why\ library}
\label{sec:installlib}

By default, the \why\ library is not installed. It can be installed using
\begin{verbatim}
make byte opt
make install_lib
\end{verbatim}

\section{Installation of external provers}

\why\ can use a wide range of external theorem provers. These need to
be installed separately, and then \why\ needs to be configured to use
them. There is no need to install these provers before compiling and
installing Why. 

For installation of external provers, please look at the Why provers
tips page \url{http://why.lri.fr/provers.en.html}.

For configuring \why\ to use the provers, follow intructions given in
Section~\ref{sec:why3config}.

\section{The \texttt{why3config} command-line tool}
\label{sec:why3config}.

\why\ must be configured to access external provers. Typically, this is done
by running either the command line tool
\begin{verbatim}
why3config
\end{verbatim}
or using the menu
\begin{verbatim}
File/Detect provers
\end{verbatim}
of the IDE. This must be done again each time a new prover is installed.

The set of all provers which are attempted to detect is described in
the readable configuration file \texttt{provers-detection-data.conf}
of the \why\ data directory (\eg{}
\texttt{/usr/local/share/why3}). Advanced users may try to modify this
file to add support for detection of other provers. (In that case,
please consider submitting a new prover configuration on the bug
tracking system).

The result of the prover detection is stored in the user's
configuration file (\eg{} \texttt{~/.why.conf}). Again, this file is
human readable, and advanced users may modify it in order to
experiment different ways of calling provers, \eg{} different versions
of the same prover, or with different options.

The provers which are typically attemped for detection are
\begin{itemize}
\item Alt-Ergo~\cite{conchon08smt,ergo}: \url{http://alt-ergo.lri.fr}
\item CVC3~\cite{BarTin-CAV-07}: \url{http://cs.nyu.edu/acsys/cvc3/}
\item Coq~\cite{CoqArt}: \url{http://coq.inria.fr}
\item Eprover~\cite{schulz04ijcar}: \url{http://www4.informatik.tu-muenchen.de/~schulz/WORK/eprover.html}
\item Gappa~\cite{melquiond08rnc}: \url{http://gappa.gforge.inria.fr/}
\item Simplify~\cite{simplify05}: \url{http://secure.ucd.ie/products/opensource/Simplify/}
\item Spass~: \url{http://www.spass-prover.org/}
\item veriT~: \url{http://www.verit-solver.org/}
\item Yices~\cite{DM06}: \url{http://yices.csl.sri.com/}
\item Z3~\cite{z3}: \url{http://research.microsoft.com/en-us/um/redmond/projects/z3/}
\end{itemize}

\texttt{why3config} also detects the plugins installed in the \why\
plugins directory (\eg{} \texttt{/usr/local/lib/why3/plugins}). A
plugin must register itself as a parser, a transformation or a
printer, as explained in the corresponding section.

If the user's configuration file is already present,
\texttt{why3config} will only reset unset variables to default value.
The option \texttt{--autodetect-provers} will detect again the available
provers and will replace them in the file configuration.
\texttt{--autodetect-plugins} will do the same for plugins.

\section{The \texttt{why3} command-line tool}
\label{sec:why3ref}

\why\ is primarily used to call provers on goals contains by file in
why3 format \texttt{.why} extension. However plugins can register
parser which can extend the known format. \texttt{why3ml} apply the
following steps :
\begin{enumerate}
\item Parse the command line and report error if needed
\item Read the configuration file using the priority defined in
  section\ref{sec:whyconffile}
\item Load the plugins mentionned in the configuration. It will not
  stop if a plugin fail to load.
\item Parse and type the given files using the correct parser in order
  to obtain a set of why theory for each files. It uses
  the filename extension or the \texttt{--format} options to choose
  among the available parsers. \texttt{--list-format} gives the list
  of them.
\item Extract the selected goals inside each selected theories into
  tasks. The goals and theories are selected using the options
  \texttt{-G/--goal} and \texttt{-T/--theory}. One
  \texttt{-T/--theory} applies to the last file appearing on the
  commandline, one \texttt{-G/--goal} applies to the last theory
  appearing on the commandline. If none theories are selected in one
  file, they are all selected. If none goals are selected inside one
  selected theory, they are all selected.
\item Apply the transformation requested
  with \texttt{-a/--apply-transform} in the order they appear on the
  command line. \texttt{--list-transforms} list the known
  transformations, plugins can add more of them.
\item Apply the driver selected with the \texttt{-D/--driver} option,
  or the driver of the prover selected with \texttt{-P/--prover}
  option. \texttt{--list-provers} lists the known provers, ie the one
  which appear in the configuration file.
\item Print the result of the standard output if \texttt{-D/--driver}
  is used or call the prover and print the result if
  \texttt{-P/--prover} is used.
\end{enumerate}

% \texttt{why3} call the prover sequentially, use \texttt{why3bench} if *)
% you want to call the provers concurrently.  *)

% \why\ can also be *)
% used to provide other informations : *)
% \begin{itemize} *)
% \item \texttt{print-namespace} print the namespace of the selected *)
%   theories *)
% \item TO BE COMPLETED *)
% \end{itemize} *)

\section{The \texttt{why3ide} GUI}
\label{sec:ideref}

The basic usage of the GUI is described by the tutorial of
Section~\ref{sec:gui}. We describe here the command-line options and
the actions of the various menus and buttons of the interface.

\subsection{Command-line options}

\begin{description}
\item[-I] $d$: adds $d$ in the load path, to search for theories.
\end{description}

\subsection{Left toolbar actions}

\begin{description}
\item[Context] The context in which the other tools below will
  apply. If ``only unproved goals'' is selected, no action will ever
  be applied to an already proved goal.  If ``all goals'', then
  actions are performed even if the goal is already proved. The second
  choice allows to compare provers on the same goal.

\item[Provers] To each detected prover corresponds to a button in this
  prover framed box. Clicking on this button starts the prover on the
  selected goal(s).

\item[Split] This splits the current goal into subgoals if it is a
  conjunction of two or more goals.

\item[Inline] If the goal is headed by a defined predicate symbol,
  expands it with this definition. [NOT YET AVAILABLE]

\item[Edit] Start an editor on the selected task.

  For automatic provers, this allows to see the file sent to the
  prover.

  For interactive provers, this also allows to add or modify the
  corresponding proof script. The modifications are saved, and can be
  retrieved later even if the goal was modified.

\item[Replay] replay all obsolete proofs [NOT YET AVAILABLE]

\item[Remove] Remove a proof attempt or a transformation. 
  [Removing transformation NOT YET AVAILABLE]

\end{description}

\subsection{Menus}

\begin{description}
\item[File/Detect provers] runs provers auto-detection
\item[File/Preferences] opens a window for modifying preferred configuration
\end{description}

\subsection{Preferences}

The preferences window allows you customize
\begin{itemize}
\item the default editor to use when the \textsf{Edit} button is
  pressed. This might be overidden for a specific prover (the only way
  to do that for the moment is to manually edit the config file)
\item the time limit given to provers, in seconds
\item the maximal number of simultaneous provers allowed to run in parallel. 
\end{itemize}

\subsection{Structure of the database file}

[TO BE COMPLETED LATER]

\section{The \texttt{why3ml} tool}

The \texttt{why3ml} is an additional layer on \why\ library for
generating verification conditions from WhyML programs. This tool and
the syntax of WhyML programs is intentionally left undocumented since
it might evolve significantly in the near future.

For those who want to experiment with it, examples are provided in
\texttt{examples/programs}. The files \texttt{*.mlw} can be loaded in
the GUI.

[TO BE COMPLETED LATER]

\section{The \texttt{why3bench} tool}

[TO BE COMPLETED LATER]

\section{The \texttt{why.conf} configuration file}
\label{sec:whyconffile}
One can defined more than one configuration file. \texttt{why3config}
and all the others \texttt{why3}'s tools use priorities for which
user's configuration file to consider:
\begin{itemize}
\item the file specified by the \texttt{-C} or \texttt{--config} options,
\item the file specified by the environment variable
  \texttt{\$WHY\_CONFIG} if set.
\item the file \texttt{why.conf} or \texttt{.why.conf} in the current
  directory.
\item the file \texttt{\$HOME/.why.conf} or \texttt{\$USERPROFILE/.why.conf}
\end{itemize}
If none of this file exists a built-in default configuration is used
which is saved in a default configuration filename, which is usually
\texttt{\$HOME/.why.conf}.

The configuration file is a human-readable text file, which is
composed by association pairs arranged by sections. Here an example of
a configuration file.

\begin{verbatim}
[main ]
datadir = "/usr/local/share/why3"
libdir = "/usr/local/lib/why3"
loadpath = "/usr/local/share/why3/theories"
memlimit = 0
running_provers_max = 2
timelimit = 10

[ide ]
default_editor = "emacs"
task_height = 384
tree_width = 438
verbose = 0
window_height = 779
window_width = 638

[prover coq]
command = "coqc %f"
driver = "/usr/local/share/why3/drivers/coq.drv"
editor = "coqide"
name = "Coq"
version = "8.2pl2"

[prover alt-ergo]
command = "why3-cpulimit %t %m alt-ergo %f"
driver = "/usr/local/share/why3/drivers/alt_ergo.drv"
editor = ""
name = "Alt-Ergo"
version = "0.91"
\end{verbatim}

A section begin with an header inside square brackets and end at the
next square brackets. Sections can't be nested. The header of a
section can be only one identifier, \texttt{main} and \texttt{ide} in
the example, or it can be composed by a family name and one family
argument, \texttt{prover} is one family name, \texttt{coq} and
\texttt{alt-ergo} are the family argument.

Inside a section, one key can be associated to an integer (.eg -555),
a boolean (true, false) or a string (\eg{} "emacs"). One key can appear
only once except if its a multi-value key. The order of apparition of
the keys inside a section matter only for the multi-value key.

\section{Drivers of external provers}

The drivers of external provers are readable files, in directory
\texttt{drivers}. Experimented users can modify them to change the way
the external provers are called, in particular which transformations
are applied to goals.

[TO BE COMPLETED LATER]

\section{Transformations}
\label{sec:transformations}

Here is a quick documentation of provided transformations. We give
first the non-splitting ones, \eg{} those which produce one goal as
result, and others which produces any number of goals.

\subsection{Non-splitting transformations}

\begin{description}

\item[eliminate\_algebraic] Replaces algebraic data types by first-order
definitions~\cite{paskevich09rr}

\item[eliminate\_builtin] Suppress definitions of symbols which are
  declared as builtin in the driver, i.e. with a ``syntax'' rule.
\item[eliminate\_definition\_func]
  Replaces all function definitions with axioms.
\item[eliminate\_definition\_pred]
  Replaces all predicate definitions with axioms.
\item[eliminate\_definition]
  Apply both transformations above.
\item[eliminate\_mutual\_recursion]
  Replaces mutually recursive definitions with axioms.
\item[eliminate\_recursion]
  Replaces all recursive definitions with axioms.

\item[eliminate\_if\_term] replaces terms of the form \texttt{if
    formula then t2 else t3} by lifting them at the level of formulas.
  This may introduce \texttt{if then else } in formulas.

\item[eliminate\_if\_fmla] replaces formulas of the form \texttt{if f1 then f2
  else f3} by an equivalent formula using implications and other
  connectives. 

\item[eliminate\_if]
  Apply both transformations above.

\item[eliminate\_inductive] replaces inductive predicates by
  (incomplete) axiomatic definitions, i.e. construction axioms and
  an inversion axiom.

\item[eliminate\_let\_fmla]
  Eliminates \texttt{let} by substitution, at the predicate level.

\item[eliminate\_let\_term]
  Eliminates \texttt{let} by substitution, at the term level.

\item[eliminate\_let]
  Apply both transformations above.

% \item[encoding\_decorate\_mono]

% \item[encoding\_enumeration]

\item[encoding\_smt]
  Encode polymorphic types into monomorphic type~\cite{conchon08smt}.

\item[encoding\_tptp]
  Encode theories into unsorted logic. %~\cite{cruanes10}.

% \item[filter\_trigger] *)

% \item[filter\_trigger\_builtin] *)

% \item[filter\_trigger\_no\_predicate] *)

% \item[hypothesis\_selection] *)
%   Filter hypothesis of goals~\cite{couchot07ftp,cruanes10}. *)

\item[inline\_all]
  expands all non-recursive definitions.

\item[inline\_goal] Expands all outermost symbols of the goal that
  have a non-recursive definition.

\item[inline\_trivial]
  removes definitions of the form

\begin{verbatim}
logic f x_1 .. x_n = (g e_1 .. e_k)
\end{verbatim}
when each $e_i$ is either a ground term or one of the $x_j$, and
each $x_1$ .. $x_n$ occur at most once in the $e_i$

\item[introduce\_premises] moves antecedents of implications and
  universal quantifications of the goal into the premises of the task.

% \item[remove\_triggers] *)
%   removes the triggers in all quantifications. *)

\item[simplify\_array] Automatically rewrites the task using the lemma
  \verb|Select_eq| of theory \verb|array.Array|.

\item[simplify\_formula] reduces trivial equalities $t=t$ to true and
  then simplifies propositional structure: removes true, false, ``f
  and f'' to ``f'', etc.

\item[simplify\_recursive\_definition] reduces mutually recursive
  definitions if they are not really mutually recursive, e.g.:
\begin{verbatim}
logic f : ... = .... g ...

with g : .. = e
\end{verbatim}
becomes
\begin{verbatim}
logic g : .. = e
logic f : ... = .... g ...
\end{verbatim}
if f does not occur in e

\item[simplify\_trivial\_quantification]
  simplifies quantifications of the form
\begin{verbatim}
  forall x, x=t -> P(x)
\end{verbatim}
or
\begin{verbatim}
  forall x, t=x -> P(x)
\end{verbatim}
  when x does not occur in t
  into 
\begin{verbatim}
P(t)
\end{verbatim}
  More generally, it applies this simplification whenever x=t appear
  in a negative position.
  
\item[simplify\_trivial\_quantification\_in\_goal]
  same as above but applies only in the goal.

\item[split\_premise]
  splits conjunctive premises.

\end{description}

\subsection{Splitting transformations}

\begin{description}

\item[full\_split\_all]
  composition of \texttt{split\_premise} and \texttt{full\_split\_goal}.

\item[full\_split\_goal] puts the goal in a conjunctive form,
  returns the corresponding set of subgoals. The number of subgoals
  generated may be exponential in the size of the initial goal.

\item[simplify\_formula\_and\_task] same as \texttt{simplify\_formula}
  but also removes the goal if it is equivalent to true.

\item[split\_all]
  composition of \texttt{split\_premise} and \texttt{split\_goal}.

\item[split\_goal] if the goal is a conjunction of goals, returns the
  corresponding set of subgoals. The number of subgoals generated is linear in
  the size of the initial goal.

\item[split\_intro]
  when a goal is an implication, moves the antecedents into the premises.

\end{description}



%%% Local Variables:
%%% mode: latex
%%% TeX-PDF-mode: t
%%% TeX-master: "manual"
%%% End:
