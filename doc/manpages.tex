\chapter{Reference manuals for the Why3 tools}
\label{chap:manpages}

\section{Compilation, Installation}
\label{sec:install}

Compilation of Why3 must start with a configuration phase which is run as
\begin{verbatim}
./configure
\end{verbatim}
This analyzes you current configuration and check if requirements hold.
Compilation requires:
\begin{itemize}
\item The Objective Caml compiler, version 3.10 or higher. It is
  available as a binary package for most Unix distributions. For
  debian-based Linux distributions, you can install the packages
\begin{verbatim}
ocaml ocaml-native-compilers
\end{verbatim}
It is also installable from sources, downloadable from the Web site
\url{http://caml.inria.fr/ocaml/}
\end{itemize}

For the IDE, additional Ocaml libraries are needed:
\begin{itemize}
\item The Lablgtk2 library for Ocaml bindings of the gtk2 graphical library.
 For debian-based Linux distributions, you can install the packages
\begin{verbatim}
liblablgtk2-ocaml-dev liblablgtksourceview2-ocaml-dev
\end{verbatim}
It is also installable from sources, available from the site \url{http://wwwfun.kurims.kyoto-u.ac.jp/soft/olabl/lablgtk.html}

\item The Ocaml bindings of the sqlite3 library
For debian-based Linux distributions, you can install the package
\begin{verbatim}
libsqlite3-ocaml-dev
\end{verbatim}
It is also installable from sources, available from the site
\url{http://ocaml.info/home/ocaml_sources.html#ocaml-sqlite3}
\end{itemize}

\subsection{Local use, without installation}

It is not mandatory to install Why3 to use it. Local use is obtained via
\begin{verbatim}
./configure --enable-local
make
\end{verbatim}
The Why3 executables are then available in subdirectory \texttt{bin/}.

\section{Installation of external provers}

Why3 can use a wide range of external theorem provers. These need to
be installed separately, and then Why3 needs to be configured to use
them. There is no need to install these provers before compiling and
installing Why. 

For installation of external provers, please look at the Why provers
tips page \url{http://why.lri.fr/provers.en.html}.

For configuring Why3 to use the provers, follow intructions given in
Section~\ref{sec:why3config}.

\section{The \texttt{why3config} command-line tool}
\label{sec:why3config}.

Why3 must be configured to access external provers. Typically, this is done
by running either the command line tool
\begin{verbatim}
why3config
\end{verbatim}
or using the menu
\begin{verbatim}
File/Detect provers
\end{verbatim}
of the IDE. This must be done again each time a new prover is installed.

The set of all provers which are attempted to detect is described in
the readable configuration file \texttt{provers-detection-data.conf}
of the Why3 data directory (\eg{}
\texttt{/usr/local/share/why3}). Advanced users may try to modify this
file to add support for detection of other provers. (In that case,
please consider submitting a new prover configuration on the bug
tracking system).

The result of the prover detection is stored in the user's
configuration file (\eg{} \texttt{~/.why.conf}). Again, this file is
human readable, and advanced users may modify it in order to
experiment different ways of calling provers, \eg{} different versions
of the same prover, or with different options.

The provers which are typically attemped for detection are
\begin{itemize}
\item Alt-Ergo~\cite{conchon08smt,ergo}: \url{}
\item CVC3~\cite{BarTin-CAV-07}: \url{}
\item Coq~\cite{CoqArt}: \url{}
\item Eprover~: \url{}
\item Gappa~\cite{melquiond08rnc}: \url{}
\item Simplify~\cite{simplify05}: \url{}
\item Spass~: \url{}
\item veriT~: \url{}
\item Yices~\cite{DM06}: \url{}
\item Z3~\cite{z3}: \url{}
\end{itemize}

\section{The \texttt{why3} command-line tool}
\label{sec:why3ref}

\section{The \texttt{why3ml} tool}

\section{The \texttt{why3ide} tool}
\label{sec:ideref}

\subsection{Command-line options}

\begin{description}
\item[-I] $d$: adds $d$ in the load path, to search for theories.
\end{description}

\subsection{Left toolbar}

\begin{description}
\item[Provers] To each detected prover corresponds to a button in this
  prover framed box. Clicking on this button starts the prover on the
  selected goal(s).

\item[Edit] Start an editor on the selected task.

  For automatic provers, this allows to see the file sent to the
  prover.

  For interactive provers, this also allows to add or modify the
  corresponding proof script. The modifications are saved, and can be
  retrieved later even if the goal was modified.

\item[Split] This splits the current goal into subgoals if it is a
  conjunction of two or more goals.

\end{description}

\subsection{Menus}

\begin{description}
\item[File/Detect provers] 

\end{description}

\subsection{Preferences}

\subsection{Structure of the database file}

[TODO]

\section{The \texttt{why.conf} configuration file}

\section{Drivers of external provers}

\section{Transformations}

\subsection{Non-splitting transformations}

\begin{description}
\item[eliminate\_algebraic] Replaces algebraic data types by first-order
definitions~\cite{paskevich09rr}
\item[eliminate\_builtin] Suppress definitions of symbols which are
  declared as builtin in the driver, i.e. with a ``syntax'' rule.
  
\item[eliminate\_definition]
\item[eliminate\_definition\_func]
\item[eliminate\_definition\_pred]
\item[eliminate\_if\_fmla] replaces formulas of the form if f1 then f2
  else f3 by an equivalent formula using implications and other
  connectives. (TODO: detail)
\item[eliminate\_if\_term] replaces terms of the form if formula then
  t2 else t3 by lift it at the level of the formula (TODO: detail)
\item[eliminate\_if]
  apply both two above transformations
\item[eliminate\_inductive] replaces inductive predicates by
  (incomplete) axiomatic definitions, i.e construction axioms and an
  inversion axiom (TODO: detail)
\item[eliminate\_let\_fmla]
\item[eliminate\_let\_term]
\item[eliminate\_let]
  apply both two above transformations
\item[eliminate\_mutual\_recursion]
\item[eliminate\_recursion]
\item[encoding\_decorate\_mono]
\item[encoding\_enumeration]
\item[encoding\_simple2]
\item[encoding\_smt]
Should we cite \cite{conchon08smt} here?
\item[encoding\_tptp]
\item[filter\_trigger]
\item[filter\_trigger\_builtin]
\item[filter\_trigger\_no\_predicate]
\item[hypothesis\_selection]
\item[inline\_all]
\item[inline\_trivial]
  removes definitions of the form
\begin{verbatim}
logic f x_1 .. x_n = (g e_1 .. e_k)
\end{verbatim}
when each $e_i$ is either a constant or one of the $x_j$, and
each $x_1$ .. $x_n$ occur at most once in the $e_i$ 

\item[remove\_triggers]
\item[simplify\_array]
\item[simplify\_formula] reduces trivial equalities $t=t$ to True and
  then simplifies propositional structure: removes True, False, ``f
  and f'' to ``f'', etc.
\item[simplify\_recursive\_definition]
  reduces mutually recursive definitions if they are not really mutually recursive, e.g.:
\begin{verbatim}
logic f : ... = .... g ...

with g : .. = e
\end{verbatim}
becomes
\begin{verbatim}
logic g : .. = e
logic f : ... = .... g ...
\end{verbatim}
if f does not occur in e

\item[simplify\_trivial\_quantification]
  simplifies quantifications of the form
\begin{verbatim}
  forall x, x=t -> P(x)
\end{verbatim}
or
\begin{verbatim}
  forall x, t=x -> P(x)
\end{verbatim}
  when x does not occur in t
  into 
\begin{verbatim}
P(t)
\end{verbatim}
  More generally, it applies this simplification whenever x=t appear
  in a negative position.
  
\item[simplify\_trivial\_quantification\_in\_goal]
\item[split\_premise]
\end{description}

\subsection{Splitting transformations}

\begin{description}
\item[right\_split]
\item[simplify\_formula\_and\_task]
\item[split\_all]
\item[split\_goal]
\item[split\_goal\_pos\_all]
\item[split\_goal\_pos\_axiom]
\item[split\_goal\_pos\_goal]
\item[split\_goal\_pos\_neg\_all]
\item[split\_goal\_pos\_neg\_axiom]
\item[split\_goal\_pos\_neg\_goal]
\end{description}



%%% Local Variables:
%%% mode: latex
%%% TeX-PDF-mode: t
%%% TeX-master: "manual"
%%% End:
