
\chapter{Executing \whyml Programs}
\label{chap:exec}\index{whyml@\whyml}

This chapter shows how \whyml code can be executed, either by being
interpreted or compiled to some existing programming language.

\begin{latexonly}
Let us consider the program in Figure~\ref{fig:MaxAndSum}
on page~\pageref{fig:MaxAndSum} that computes the maximum and the sum
of an array of integers.
\end{latexonly}
\begin{htmlonly}
Let us consider the program of Section~\ref{sec:MaxAndSum} that computes
the maximum and the sum of an array of integers.
\end{htmlonly}
Let us assume it is contained in a file \texttt{maxsum.mlw}.

\section{Interpreting \whyml Code}
\label{sec:execute}
\index{execute@\texttt{execute}}\index{interpretation!of \whyml}
\index{testing \whyml code}

To test function \texttt{max\_sum}, we can introduce a \whyml test function
in module \texttt{MaxAndSum}
\begin{whycode}
  let test () =
    let n = 10 in
    let a = make n 0 in
    a[0] <- 9; a[1] <- 5; a[2] <- 0; a[3] <- 2;  a[4] <- 7;
    a[5] <- 3; a[6] <- 2; a[7] <- 1; a[8] <- 10; a[9] <- 6;
    max_sum a n
\end{whycode}
and then we use the \texttt{execute} command to interpret this function,
as follows:
\begin{verbatim}
> why3 execute maxsum.mlw MaxAndSum.test
Execution of MaxAndSum.test ():
     type: (int, int)
   result: (45, 10)
  globals:
\end{verbatim}
We get the expected output, namely the pair \texttt{(45, 10)}.

\section{Compiling \whyml to OCaml}
\label{sec:extract}
\index{OCaml}\index{extraction}
\index{extract@\texttt{extract}}

An alternative to interpretation is to compile \whyml to OCaml.
We do so using the \texttt{extract} command, as follows:
\begin{verbatim}
> why3 extract -D ocaml64 maxsum.mlw -o max_sum.ml
\end{verbatim}
The \texttt{extract} command requires the name of a driver, which indicates
how theories/modules from the \why standard library are translated to
OCaml. Here we assume a 64-bit architecture and thus we pass
\texttt{ocaml64}. We also specify an output file using option
\verb+-o+, namely \texttt{max\_sum.ml}.
After this command, the file \texttt{max\_sum.ml} contains an OCaml
code for function \texttt{max\_sum}.
To compile it, we create a file \texttt{main.ml}
containing a call to \texttt{max\_sum}, \emph{e.g.},
\begin{whycode}
let a = Array.map Z.of_int [| 9; 5; 0; 2; 7; 3; 2; 1; 10; 6 |]
let m, s = Max_sum.max_sum a (Z.of_int 10)
let () = Format.printf "sum=%s, max=%s@." (Z.to_string s) (Z.to_string m)
\end{whycode}
It is convenient to use \texttt{ocamlbuild} to compile and link both
files \texttt{max\_sum.ml} and \texttt{main.ml}:
\begin{verbatim}
> ocamlbuild -pkg zarith main.native
\end{verbatim}
Since \why's type
\texttt{int} is translated to OCaml arbitrary precision integers using
the \texttt{ZArith} library, we have to pass option \texttt{-pkg
  zarith} to \texttt{ocamlbuild}. In order to get extracted code that
uses OCaml's native integers instead, one has to use \why's types for
63-bit integers from libraries \texttt{mach.int.Int63} and
\texttt{mach.array.Array63}.

\paragraph{Extraction starting point} The \texttt{extract} command accepts three
different targets for extraction: a \whyml file, a module, or a symbol
(function, type, exception). To extract all the symbols from every module of a
file named \texttt{f.mlw}, one should write
\begin{verbatim}
> why3 extract -D <driver> f.mlw
\end{verbatim}
To extract only the symbols from module \texttt{M} of file \texttt{f.mlw}, one
should write
\begin{verbatim}
> why3 extract -D <driver> -L <dir> f.M
\end{verbatim}
To extract only the symbol \texttt{s} (a function, a type, or an exception) from
module \texttt{M} of file \texttt{f.mlw}, one should write
\begin{verbatim}
> why3 extract -D <driver> -L <dir> f.M.s
\end{verbatim}
Note the use of~\texttt{-L <dir>}, for both extraction of a module and a symbol,
in order to state the location of file \texttt{f.mlw}.

\paragraph{Options}
\begin{description}
\item[\texttt{-{}-flat}] performs a flat extraction, \emph{i.e.}, everything is
  extracted into a single file. This is the default behavior. The \texttt{-o}
  option should be given the name of a file or, if omitted, the result of
  extraction is printed to the standard output.
\item[\texttt{-{}-modular}] each module is extracted in its own, separated
  file. The \texttt{-o} option cannot be omitted, and it should be given the
  name of an existing directory. This directory will be populated with the
  resulting OCaml files.
\item[\texttt{-{}-recursive}] recursively extracts all the dependencies of the
  chosen entry point. This option is valid for both \texttt{modular} and
  \texttt{flat} options.
\end{description}

\paragraph{Examples}
We illustrate different ways of using the \texttt{extract} command though some
examples. Consider the program in Figure~\ref{fig:AQueue} on
page~\pageref{fig:AQueue}. If we are only interested in extracting function
\texttt{aenqueue}, we can proceed as follows:
\begin{verbatim}
> why3 extract -D ocaml64 -L . aqueue.AmortizedQueue.enqueue -o aqueue.ml
\end{verbatim}
Here we assume that file \texttt{aqueue.mlw} contains this program, and that
we invoke \texttt{extract} from the directory where this file is stored. File
\texttt{aqueue.ml} now contains the following OCaml code:
\begin{whycode}
let enqueue (x: 'a) (q: 'a queue) : 'a queue =
  create (q.front) (q.lenf) (x :: (q.rear))
    (Z.add (q.lenr) (Z.of_string "1"))
\end{whycode}
Choosing a function symbol as the entry point of extraction allows us to focus
only on specific parts of the program. However, the generated code cannot be
type-checked by the OCaml compiler, as it depends on function \texttt{create}
and on type \texttt{'a queue}, whose definitions are not given. In order to
obtain a \emph{complete} OCaml implementation, we can perform a recursive
extraction:
\begin{verbatim}
> why3 extraction --recursive -D ocaml64 -L . \
    aqueue.AmortizedQueue.enqueue -o aqueue.ml
\end{verbatim}
This updates the contents of file \texttt{aqueue.ml} as follows:
%\begin{figure}
\begin{whycode}
type 'a queue = {
  front: 'a list;
  lenf: Z.t;
  rear: 'a list;
  lenr: Z.t;
  }

let create (f: 'a list) (lf: Z.t) (r: 'a list) (lr: Z.t) : 'a queue =
  if Z.geq lf lr
  then
    { front = f; lenf = lf; rear = r; lenr = lr }
  else
    let f1 = List.append f (List.rev r) in
    { front = f1; lenf = Z.add lf lr; rear = []; lenr = (Z.of_string "0") }

let enqueue (x: 'a) (q: 'a queue) : 'a queue =
  create (q.front) (q.lenf) (x :: (q.rear))
    (Z.add (q.lenr) (Z.of_string "1"))
\end{whycode}
This new version of the code is now accepted by the OCaml compiler.

Let us now consider the
% \label{fig:extract-queens}
% \caption{Recursive extraction of \texttt{queens} function.}
% \end{figure}

%%% Local Variables:
%%% mode: latex
%%% TeX-PDF-mode: t
%%% TeX-master: "manual"
%%% End:
