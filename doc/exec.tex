
\chapter{Executing \whyml Programs}
\label{chap:exec}\index{whyml@\whyml}

This chapter show how \whyml code can be executed, either by being
interpreted or compiled to some existing programming language.

Let us consider the program in Fig.~\ref{fig:MaxAndSum}
on page~\pageref{fig:MaxAndSum} that computes the maximum and the sum
of an array of integers.
Let us assume it is contained in a file \texttt{maxsum.mlw}.

\section{Interpreting \whyml Code}
\index{exec@\verb+--exec+}\index{interpretation!of \whyml}
\index{testing \whyml code}

To test function \texttt{max\_sum}, we can introduce a \whyml test function
in module \texttt{MaxAndSum}
\begin{whycode}
  let test () =
    let n = 10 in
    let a = make n 0 in
    a[0] <- 9; a[1] <- 5; a[2] <- 0; a[3] <- 2;  a[4] <- 7;
    a[5] <- 3; a[6] <- 2; a[7] <- 1; a[8] <- 10; a[9] <- 6;
    max_sum a n
\end{whycode}
and then we use option \verb+--exec+ to interpret this function,
as follows:
\begin{verbatim}
  > why3  maxsum.mlw --exec MaxAndSum.test
  Execution of MaxAndSum.test ():
       type: (int, int)
     result: Tuple2(45, 10)
    globals:
\end{verbatim}
We get the excepted output, namely the pair \texttt{(45, 10)}.

\section{Compiling \whyml to OCaml}
\index{OCaml}\index{extraction}
\index{E@\verb+-E+|see{\texttt{-{}-extract}}}
\index{extract@\verb+--extract+}

An alternative to interpretation is to compile \whyml to OCaml.
We do so using option \verb+--extract+, as follows:
\begin{verbatim}
  > mkdir dir
  > why3 --extract ocaml64 maxsum.mlw -o dir
\end{verbatim}
Option \verb+--extract+ requires the name of a driver, which indicates
how theories/modules from the \why standard library are translated to
OCaml. Here we assume a 64-bit architecture and thus we pass
\texttt{ocaml64}. On a 32-bit architecture, we would use
\texttt{ocaml32} instead. Extraction also requires a target directory
to be specified using option \verb+-o+. Here we pass a freshly created
directory \texttt{dir}.

Directory \texttt{dir} is now populated with a bunch of OCaml files,
among which we find a file \texttt{maxsum\_\_MaxAndSum.ml} containing
the OCaml code for functions \texttt{max\_sum} and \texttt{test}.

To compile it, we create a file \texttt{main.ml}
containing a call to \texttt{test}, that is
\begin{whycode}
  let () = test ()
\end{whycode}
and we pass both files \texttt{maxsum\_\_MaxAndSum.ml} and
\texttt{main.ml} to the OCaml compiler:
\begin{verbatim}
  > cd dir
  > ocamlopt zarith.cmxa why3extract.cmxa maxsum__MaxAndSum.ml main.ml
\end{verbatim}
OCaml code extracted from \why must be linked with the library
\texttt{why3extract.cmxa} that is shipped with \why. It is typically
stored in subdirectory \texttt{why3} of the OCaml standard library.
Depending on the way \why was installed, it depends either on library
\texttt{nums.cmxa} or \texttt{zarith.cmxa} for big integers. Above we
assumed the latter. it is likely that additional options \texttt{-I}
must be passed to the OCaml compiler for libraries
\texttt{zarith.cmxa} and \texttt{why3extract.cmxa} to be found.


%%% Local Variables:
%%% mode: latex
%%% TeX-PDF-mode: t
%%% TeX-master: "manual"
%%% End:
