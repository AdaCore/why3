\chapter{Why3 Language}
\label{chap:syntax}

This chapter describes the input syntax, and informally gives its semantics,
illustrated by examples.

A \why\ text contains a list of \emph{theories}. 
A theory is a list of \emph{declarations}. Declarations introduce new
types, functions and predicates, state axioms, lemmas and goals. 
These declarations can be directly written in the theory or taken from
existing theories. The base logic of \why\ is a first-order
polymorphic logic.

The Figure~\ref{fig:tutorial1} contains an example of \why\ input
text, containing three theories. The first theory, \texttt{List}, 
declares a new algebraic type for polymorphic lists, \texttt{list 'a}.
As in ML, \texttt{'a} stands for a type variable.
The type \texttt{list 'a} has two constructors, \texttt{Nil} and
\texttt{Cons}. Both constructors can be used as usual function
symbols, respectively of type \texttt{list 'a} and \texttt{'a
  $\times$ list 'a $\rightarrow$ list 'a}.
We deliberately make this theory that short, for reasons which will be
discussed later.

\begin{figure}
  \centering
\begin{verbatim}
theory List

  type list 'a = Nil | Cons 'a (list 'a)

end

theory Length

  use import List
  use import int.Int

  logic length (l : list 'a) : int = 
    match l with
    | Nil      -> 0
    | Cons _ r -> 1 + length r
    end

  lemma Length_nonnegative : forall l:list 'a. length(l) >= 0

end

theory Sorted

  use import List
  use import int.Int
 
  inductive sorted (l : list int) =
    | Sorted_Nil : 
        sorted Nil
    | Sorted_One : 
        forall x:int. sorted (Cons x Nil)
    | Sorted_Two : 
        forall x y : int, l : list int. 
        x <= y -> sorted (Cons y l) -> sorted (Cons x (Cons y l))
 
end
\end{verbatim}
\caption{Example of Why3 text.}
\label{fig:tutorial1}
\end{figure}

The next theory, \texttt{Length}, introduces the notion of list
length. The \texttt{use import List} command indicates that this new
theory may refer to symbols from theory \texttt{List}. These symbols
are accessible in a qualified form, such as \texttt{List.list} or
\texttt{List.Cons}. The \texttt{import} qualifier additionally allows
use to use them without qualification.

Similarly, the next command \texttt{use import int.Int} adds to our
context the theory \texttt{int.Int} from the standard library. The
prefix \texttt{int} indicates the file in the standard library
containing theory \texttt{Int}. Theories referred to without prefix
either appear earlier in the current file, \eg\ \texttt{List}, or are
predefined. 

The next declaration defines a recursive function, \emph{length},
which computes the length of a list. The \texttt{logic} keyword is
used to introduce or define both function and predicate symbols. 
\why\ checks every recursive, or mutually recursive, definition for
termination. Basically, we require a lexicographic and structural
descent for every recursive call for some reordering of arguments. 
Note that matching must be exhaustive and that every \texttt{match}
expression must be terminated by the \texttt{end} keyword.

The last declaration in theory \texttt{Length} is a lemma stating that
the length of a list is non-negative. 

% \section{Terms and Formulas} *)

% \section{Declarations, Theories} *)

% \section{Using and Cloning Theories} *)

\section*{Another Example}
\index{Einstein's logic problem}

We now consider another, slightly more complex example: to use \why\
to solve a little puzzle known as ``Einstein's logic
problem''\footnote{This was contributed by St\'ephane Lescuyer.}.
The problem is stated as follows. 
\begin{itemize}
\item TODO
\end{itemize}
The question is: what is the nationality of the fish's owner?

We start by introducing a general-purpose theory defining the notion
of \emph{bijection}, as two abstract types together with two functions from
one to the other and two axioms stating that these functions are
inverse of each other.
\begin{verbatim}
theory Bijection
  type t
  type u

  logic of t : u
  logic to u : t

  axiom To_of : forall x : t. to (of x) = x
  axiom Of_to : forall y : u. of (to y) = y
end
\end{verbatim}

We now start a new theory, \texttt{Einstein}, which will contain all
the individuals of the problem.
\begin{verbatim}
theory Einstein "Einstein's problem"
\end{verbatim}
First we introduce enumeration types for houses, colors, persons,
drinks, cigars and pets.
\begin{verbatim}
  type house  = H1 | H2 | H3 | H4 | H5
  type color  = Blue | Green | Red | White | Yellow
  type person = Dane | Englishman | German | Norwegian | Swede
  type drink  = Beer | Coffee | Milk | Tea | Water
  type cigar  = Blend | BlueMaster | Dunhill | PallMall | Prince
  type pet    = Birds | Cats | Dogs | Fish | Horse
\end{verbatim}
We now express that each house is associated bijectively to a color,
by cloning the \texttt{Bijection} theory appropriately.
\begin{verbatim}
  clone Bijection as Color with type t = house, type u = color
\end{verbatim}
It introduces two functions, namely \texttt{Color.of} and
\texttt{Color.to}, from houses to colors and colors to houses,
respectively, and the two axioms relating them.
Similarly, we express that each house is associated bijectively to a
person
\begin{verbatim}
  clone Bijection as Owner with type t = house, type u = person
\end{verbatim}
and that drinks, cigars and pets are all associated bijectively to persons:
\begin{verbatim}
  clone Bijection as Drink with type t = person, type u = drink
  clone Bijection as Cigar with type t = person, type u = cigar
  clone Bijection as Pet   with type t = person, type u = pet
\end{verbatim}
Next we need a way to state that a person lives next to another. We
first define a predicate \texttt{leftof} over two houses.
\begin{verbatim}
  logic leftof (h1 h2 : house) =
    match h1, h2 with
    | H1, H2 
    | H2, H3 
    | H3, H4 
    | H4, H5 -> true
    | _      -> false
    end
\end{verbatim}
Note how we advantageously used pattern-matching, with a or-pattern
for the four positive cases and a universal pattern for the remaining
21 cases. It is then immediate to define a \texttt{neighbour}
predicate over two houses, which completes theory \texttt{Einstein}.
\begin{verbatim}
  logic rightof (h1 h2 : house) =
    leftof h2 h1
  logic neighbour (h1 h2 : house) =
    leftof h1 h2 or rightof h1 h2
end
\end{verbatim}

The next theory contains the 15 hypotheses. It starts by importing
theory \texttt{Einstein}.
\begin{verbatim}
theory EinsteinHints "Hints"
  use import Einstein
\end{verbatim}
Then each hypothesis is stated in terms of \texttt{to} and \texttt{of}
functions. For instance, the hypothesis ``The Englishman lives in a
red house'' is declared as the following axiom.
\begin{verbatim}
  axiom Hint1: Color.of (Owner.to Englishman) = Red
\end{verbatim}
And so on for all other hypotheses, up to
``The man who smokes Blends has a neighbour who drinks water'', which completes
this theory.
\begin{verbatim}
  ...
  axiom Hint15:
    neighbour (Owner.to (Cigar.to Blend)) (Owner.to (Drink.to Water))
end
\end{verbatim}

TODO


%%% Local Variables:
%%% mode: latex
%%% TeX-PDF-mode: t
%%% TeX-master: "manual"
%%% End:
