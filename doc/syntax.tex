\chapter{Why3 Language}
\label{chap:syntax}

This chapter describes the input syntax, and informally gives its semantics,
illustrated by examples.

A \why\ text contains a list of \emph{theories}. 
A theory is a list of \emph{declarations}. Declarations introduce new
types, functions and predicates, state axioms, lemmas and goals. 
These declarations can be directly written in the theory or taken from
existing theories. The base logic of \why\ is a first-order
polymorphic logic.

The Figure~\ref{fig:tutorial1} contains an example of \why\ input
text, containing three theories. The first theory, \texttt{List}, 
declares a new algebraic type for polymorphic lists, \texttt{list 'a}.
As in ML, \texttt{'a} stands for a type variable.
The type \texttt{list 'a} has two constructors, \texttt{Nil} and
\texttt{Cons}. Both constructors can be used as usual function
symbols, respectively of type \texttt{list 'a} and \texttt{'a
  $\times$ list 'a $\rightarrow$ list 'a}.
We deliberately make this theory that short, for reasons which will be
discussed later.

\begin{figure}
  \centering
\begin{verbatim}
theory List

  type list 'a = Nil | Cons 'a (list 'a)

end

theory Length

  use import List
  use import int.Int

  logic length (l : list 'a) : int = 
    match l with
    | Nil      -> 0
    | Cons _ r -> 1 + length r
    end

  lemma Length_nonnegative : forall l:list 'a. length(l) >= 0

end

theory Sorted

  use import List
  use import int.Int
 
  inductive sorted (l : list int) =
    | Sorted_Nil : 
        sorted Nil
    | Sorted_One : 
        forall x:int. sorted (Cons x Nil)
    | Sorted_Two : 
        forall x y : int, l : list int. 
        x <= y -> sorted (Cons y l) -> sorted (Cons x (Cons y l))
 
end
\end{verbatim}
\caption{Example of Why3 text.}
\label{fig:tutorial1}
\end{figure}

The next theory, \texttt{Length}, introduces the notion of list
length. The \texttt{use import List} command indicates that this new
theory may refer to symbols from theory \texttt{List}. These symbols
are accessible in a qualified form, such as \texttt{List.list} or
\texttt{List.Cons}. The \texttt{import} qualifier additionally allows
use to use them without qualification.

Similarly, the next command \texttt{use import int.Int} adds to our
context the theory \texttt{int.Int} from the standard library. The
prefix \texttt{int} indicates the file in the standard library
containing theory \texttt{Int}. Theories referred to without prefix
either appear earlier in the current file, \eg\ \texttt{List}, or are
predefined. 

The next declaration defines a recursive function, \emph{length},
which computes the length of a list. The \texttt{logic} keyword is
used to introduce or define both function and predicate symbols. 
\why\ checks every recursive, or mutually recursive, definition for
termination. Basically, we require a lexicographic and structural
descent for every recursive call for some reordering of arguments. 
Note that matching must be exhaustive and that every \texttt{match}
expression must be terminated by the \texttt{end} keyword.

The last declaration in theory \texttt{Length} is a lemma stating that
the length of a list is non-negative. 

% \section{Terms and Formulas} *)

% \section{Declarations, Theories} *)

% \section{Using and Cloning Theories} *)


%%% Local Variables:
%%% mode: latex
%%% TeX-PDF-mode: t
%%% TeX-master: "manual"
%%% End:
