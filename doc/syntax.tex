\chapter{The \why Language}
\label{chap:syntax}
%HEVEA\cutname{syntax.html}

This chapter describes the input syntax, and informally gives its semantics,
illustrated by examples.

A \why text contains a list of \emph{theories}.
A theory is a list of \emph{declarations}. Declarations introduce new
types, functions and predicates, state axioms, lemmas and goals.
These declarations can be directly written in the theory or taken from
existing theories. The base logic of \why is first-order
logic with polymorphic types.

\section{Example 1: Lists}

%BEGIN LATEX
Figure~\ref{fig:tutorial1} contains an example of \why input
text, containing three theories.
%END LATEX
%HEVEA The code below is an example of \why input text, containing three theories.

%BEGIN LATEX
\begin{figure}
\centering
%END LATEX
\begin{whycode}
theory List
  type list 'a = Nil | Cons 'a (list 'a)
end

theory Length
  use import List
  use import int.Int

  function length (l : list 'a) : int =
    match l with
    | Nil      -> 0
    | Cons _ r -> 1 + length r
    end

  lemma Length_nonnegative : forall l:list 'a. length l >= 0
end

theory Sorted
  use import List
  use import int.Int

  inductive sorted (list int) =
    | Sorted_Nil :
        sorted Nil
    | Sorted_One :
        forall x:int. sorted (Cons x Nil)
    | Sorted_Two :
        forall x y : int, l : list int.
        x <= y -> sorted (Cons y l) -> sorted (Cons x (Cons y l))
end
\end{whycode}
%BEGIN LATEX
\vspace*{-1em}%\hrulefill
\caption{Example of \why text}
\label{fig:tutorial1}
\end{figure}
%END LATEX

The first theory, \texttt{List},
declares a new algebraic type for polymorphic lists, \texttt{list 'a}.
As in ML, \texttt{'a} stands for a type variable.
The type \texttt{list 'a} has two constructors, \texttt{Nil} and
\texttt{Cons}. Both constructors can be used as usual function
symbols, respectively of type \texttt{list 'a} and \texttt{'a
  $\times$ list 'a $\rightarrow$ list 'a}.
We deliberately make this theory that short, for reasons which will be
discussed later.

The next theory, \texttt{Length}, introduces the notion of list
length. The \texttt{use import List} command indicates that this new
theory may refer to symbols from theory \texttt{List}. These symbols
are accessible in a qualified form, such as \texttt{List.list} or
\texttt{List.Cons}. The \texttt{import} qualifier additionally allows
us to use them without qualification.

Similarly, the next command \texttt{use import int.Int} adds to our
context the theory \texttt{int.Int} from the standard library. The
prefix \texttt{int} indicates the file in the standard library
containing theory \texttt{Int}. Theories referred to without prefix
either appear earlier in the current file, \eg\ \texttt{List}, or are
predefined.

The next declaration defines a recursive function, \texttt{length},
which computes the length of a list. The \texttt{function} and
\texttt{predicate} keywords are used to introduce function and
predicate symbols, respectively.
\why checks every recursive, or mutually recursive, definition for
termination. Basically, we require a lexicographic and structural
descent for every recursive call for some reordering of arguments.
Notice that matching must be exhaustive and that every \texttt{match}
expression must be terminated by the \texttt{end} keyword.

Despite using higher-order ``curried'' syntax, \why does not permit
partial application: function and predicate arities must be respected.

The last declaration in theory \texttt{Length} is a lemma stating that
the length of a list is non-negative.

The third theory, \texttt{Sorted}, demonstrates the definition of
an inductive predicate. Every such definition is a list of clauses:
universally closed implications where the consequent is an instance
of the defined predicate. Moreover, the defined predicate may only
occur in positive positions in the antecedent. For example, a clause:
\begin{whycode}
  | Sorted_Bad :
      forall x y : int, l : list int.
      (sorted (Cons y l) -> y > x) -> sorted (Cons x (Cons y l))
\end{whycode}
would not be allowed. This positivity condition assures the logical
soundness of an inductive definition.

Note that the type signature of \lstinline{sorted} predicate does not
include the name of a parameter (see \texttt{l} in the definition
of \texttt{length}): it is unused and therefore optional.

\section{Example 1 (continued): Lists and Abstract Orderings}

In the previous section we have seen how a theory can reuse the
declarations of another theory, coming either from the same input
text or from the library. Another way to referring to a theory is
by ``cloning''. A \texttt{clone} declaration constructs a local
copy of the cloned theory, possibly instantiating some of its
abstract (\ie declared but not defined) symbols.

%BEGIN LATEX
Consider the continued example in Figure~\ref{fig:tutorial2}.
%END LATEX
%HEVEA Consider the continued example below.
We write an abstract theory of partial orders, declaring an
abstract type \texttt{t} and an abstract binary predicate
\texttt{<=}. Notice that an infix operation must be enclosed
in parentheses when used outside a term. We also specify
three axioms of a partial order.

%BEGIN LATEX
\begin{figure}
\centering
%END LATEX
\begin{whycode}
theory Order
  type t
  predicate (<=) t t

  axiom Le_refl : forall x : t. x <= x
  axiom Le_asym : forall x y : t. x <= y -> y <= x -> x = y
  axiom Le_trans: forall x y z : t. x <= y -> y <= z -> x <= z
end

theory SortedGen
  use import List
  clone import Order as O

  inductive sorted (l : list t) =
    | Sorted_Nil :
        sorted Nil
    | Sorted_One :
        forall x:t. sorted (Cons x Nil)
    | Sorted_Two :
        forall x y : t, l : list t.
        x <= y -> sorted (Cons y l) -> sorted (Cons x (Cons y l))
end

theory SortedIntList
  use import int.Int
  clone SortedGen with type O.t = int, predicate O.(<=) = (<=)
end
\end{whycode}
%BEGIN LATEX
\vspace*{-1em}%\hrulefill
\caption{Example of \why text (continued)}
\label{fig:tutorial2}
\end{figure}
%END LATEX

There is little value in \texttt{use}'ing such a theory: this
would constrain us to stay with the type \texttt{t}. However,
we can construct an instance of theory \texttt{Order} for
any suitable type and predicate. Moreover, we can build some
further abstract theories using order, and then instantiate
those theories.

Consider theory \texttt{SortedGen}. In the beginning, we
\texttt{use} the earlier theory \texttt{List}. Then we
make a simple \texttt{clone} theory \texttt{Order}.
This is pretty much equivalent to copy-pasting every
declaration from \texttt{Order} to \texttt{SortedGen};
the only difference is that \why traces the history
of cloning and transformations and drivers often make
use of it (see Section~\ref{sec:drivers}).

Notice an important difference between \texttt{use}
and \texttt{clone}. If we \texttt{use} a theory, say
\texttt{List}, twice (directly or indirectly: \eg by
making \texttt{use} of both \texttt{Length} and
\texttt{Sorted}), there is no duplication: there is
still only one type of lists and a unique pair
of constructors. On the contrary, when we \texttt{clone}
a theory, we create a local copy of every cloned
declaration, and the newly created symbols, despite
having the same names, are different from their originals.

Returning to the example, we finish theory \texttt{SortedGen}
with a familiar definition of predicate \texttt{sorted};
this time we use the abstract order on the values of type
\texttt{t}.

Now, we can instantiate theory \texttt{SortedGen} to any
ordered type, without having to retype the definition of
\texttt{sorted}. For example, theory \texttt{SortedIntList}
makes \texttt{clone} of \texttt{SortedGen} (\ie copies its
declarations) substituting type \texttt{int} for type
\texttt{O.t} of \texttt{SortedGen} and the default order
on integers for predicate \texttt{O.(<=)}. \why will
control that the result of cloning is well-typed.

Several remarks ought to be made here. First of all, why should
we clone theory \texttt{Order} in \texttt{SortedGen} if we make
no instantiation? Couldn't we write \texttt{use import Order as O}
instead? The answer is no, we could not. When cloning a theory,
we only can instantiate the symbols declared locally in this theory,
not the symbols imported with \texttt{use}. Therefore, we create
a local copy of \texttt{Order} in \texttt{SortedGen} to be able
to instantiate \texttt{t} and \texttt{(<=)} later.

Secondly, when we instantiate an abstract symbol, its declaration
is not copied from the theory being cloned. Thus, we will not create
a second declaration of type \texttt{int} in \texttt{SortedIntList}.

The mechanism of cloning bears some resemblance to modules and functors
of ML-like languages. Unlike those languages, \why makes no distinction
between modules and module signatures, modules and functors. Any \why
theory can be \texttt{use}'d directly or instantiated in any of its
abstract symbols.

The command-line tool \texttt{why3} (described in
Section~\ref{sec:batch}), allows us to see the effect of cloning.
If the input file containing our example is called \texttt{lists.why},
we can launch the following command:
\begin{verbatim}
> why3 lists.why -T SortedIntList
\end{verbatim}
to see the resulting theory \texttt{SortedIntList}:
\begin{whycode}
theory SortedIntList
  (* use BuiltIn *)
  (* use Int *)
  (* use List *)

  axiom Le_refl : forall x:int. x <= x
  axiom Le_asym : forall x:int, y:int. x <= y -> y <= x -> x = y
  axiom Le_trans : forall x:int, y:int, z:int. x <= y -> y <= z
    -> x <= z

  (* clone Order with type t = int, predicate (<=) = (<=),
    prop Le_trans1 = Le_trans, prop Le_asym1 = Le_asym,
    prop Le_refl1 = Le_refl *)

  inductive sorted (list int) =
    | Sorted_Nil : sorted (Nil:list int)
    | Sorted_One : forall x:int. sorted (Cons x (Nil:list int))
    | Sorted_Two : forall x:int, y:int, l:list int. x <= y ->
        sorted (Cons y l) -> sorted (Cons x (Cons y l))

  (* clone SortedGen with type t1 = int, predicate sorted1 = sorted,
    predicate (<=) = (<=), prop Sorted_Two1 = Sorted_Two,
    prop Sorted_One1 = Sorted_One, prop Sorted_Nil1 = Sorted_Nil,
    prop Le_trans2 = Le_trans, prop Le_asym2 = Le_asym,
    prop Le_refl2 = Le_refl *)
end
\end{whycode}

In conclusion, let us briefly explain the concept of namespaces
in \why. Both \texttt{use} and \texttt{clone} instructions can
be used in three forms (the examples below are given for \texttt{use},
the semantics for \texttt{clone} is the same):
\begin{itemize}
\item \texttt{use List as L} --- every symbol $s$ of theory \texttt{List}
is accessible under the name \texttt{L.$s$}. The \texttt{as L} part is
optional, if it is omitted, the name of the symbol is \texttt{List.$s$}.

\item \texttt{use import List as L} --- every symbol $s$ from
\texttt{List} is accessible under the name \texttt{L.$s$}. It is also
accessible simply as \texttt{$s$}, but only up to the end of the current
namespace, \eg the current theory. If the current theory, that is the
one making \texttt{use}, is later used under the name \texttt{T},
the name of the symbol would be \texttt{T.L.$s$}. (This is why we
could refer directly to the symbols of \texttt{Order} in theory
\texttt{SortedGen}, but had to qualify them with \texttt{O.}
in \texttt{SortedIntList}.)
As in the previous case, \texttt{as L} part is optional.

\item \texttt{use export List} --- every symbol $s$ from \texttt{List}
is accessible simply as \texttt{$s$}. If the current theory
is later used under the name \texttt{T}, the name of the symbol
would be \texttt{T.$s$}.
\end{itemize}

\why allows to open new namespaces explicitly in the text. In particular,
the instruction ``\texttt{clone import Order as O}'' can be equivalently
written as:
\begin{whycode}
namespace import O
  clone export Order
end
\end{whycode}
However, since \why favors short theories over long and complex ones,
this feature is rarely used.


%%% Local Variables:
%%% mode: latex
%%% TeX-PDF-mode: t
%%% TeX-master: "manual"
%%% End:
