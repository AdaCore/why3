\chapter{Standard Library}
\label{chap:library}

This chapter provides a short description of the contents of \why
standard library. It contains both standard logic theories, described
in Section~\ref{sec:stdlib}, and standard ML modules, described in
Section~\ref{sec:mllibrary}.

Only a rough introduction of the theories and modules is given
here. For detailed information, one should refer to the on-line
documentation automatically generated from the actual sources,
available at \url{http://why3.lri.fr/stdlib/}.


\section{Theories}
\label{sec:stdlib}

We present the most important theories here, see the URL above for the
others.

Notice there is an alternative way to explore the contents of a
library, using the \verb|why3| command with option 
\verb|-T| and a qualified theory name, for example:
\index{theory@\verb+--theory+}

\begin{verbatim}
> why3 -T bool.Ite
theory Ite
  (* use BuiltIn *)

  (* use Bool *)

  function ite (b:bool) (x:'a) (y:'a) : 'a =
    match b with
    | True -> x
    | False -> y
    end
end
\end{verbatim}

In the following, for each library, we describe the main theories
defined in it.

\subsection{Library \texttt{bool}}

\begin{description}

\item[Bool] provides the Boolean data type \verb|bool| with
  constructors \verb|True| and \verb|False|; and operations \verb|andb|, \verb|orb|, \verb|xorb|, \verb|notb|.
\indextt{bool}
\indextt{True}
\indextt{False}

\item[Ite] provides the polymorphic if-then-else operator written as \verb|ite|.

\end{description}

\subsection{Library \texttt{int}}

\begin{description}

\item[Int] provides the basic operations \verb|+|, \verb|-|, and
  \verb|*|, and the comparison operators \verb|<|, \verb|>|, \verb|>=|, and
  \verb|<=|.

\item[Abs] provides the absolute value written as \verb|abs|.

\item[EuclideanDivision] defines division and modulo, where division rounds
  down, written as \verb|div| and \verb|mod|.

\item[ComputerDivision] defines division and modulo, where division rounds to
  zero, also written as \verb|div| and \verb|mod|.

\item[MinMax] provides \verb|min| and \verb|max| operators.

\end{description}

See the on-line web documentation for the other theories defined in the
\texttt{int} library.

\subsection{Library \texttt{real}}

\begin{description}

\item[Real] provides basic operations \verb|+|, \verb|-|, \verb|*| and \verb|/|;
  comparison operators.

\item[RealInfix] provides basic operations with alternative syntax \verb|+.|,
  \verb|-.|, \verb|*.|, \verb|/.|, \verb|<.|, \verb|>.|, \verb|<=.|, and \verb|>=.|, to
  allow simultaneous use of integer and real operators.

\item[Abs] provides absolute value written as \verb|abs|.

\item[MinMax] provides \verb|min| and \verb|max| operators.

\item[FromInt] provides the operator \verb|from_int| to convert an integer to a real.

\item[Truncate] provides conversion operators from real to integers:
  \verb|truncate| rounds to 0, \verb|floor| rounds down, and
  \verb|ceil| rounds up.

\item[Square] provides operators \verb|sqr| and \verb|sqrt| for square and square root.

% \item[ExpLog] functions \verb|exp|, \verb|log|, \verb|log2|, and \verb|log10|.

% \item[PowerReal] function \verb|pow| with two real parameters.

% \item[PowerInt] function \verb|pow| with integer only exponents.

% \item[Trigonometry] functions \verb|cos|, \verb|sin|, \verb|tan|, and
%   \verb|atan|. Constant \verb|pi|.

% \item[Hyperbolic] functions \verb|cosh|, \verb|sinh|, \verb|tanh|,
%   \verb|acosh|, \verb|asinh|, \verb|atanh|.

% \item[Polar] functions \verb|hypot| and \verb|atan2|.

\end{description}

See the on-line web documentation for the other theories defined in the
\texttt{real} library, such exponential, logarithm, power,
trigonometric functions.

\subsection{Library \texttt{floating\_point}}

This library provides a theory of IEEE-754 floating-point numbers. It
is inspired by~\cite{ayad10ijcar}.

\subsection{Library \texttt{map}}

This library provides the data type of purely applicative maps. It is
polymorphic both in the index type and the contents. There are also a
few theories and operators specialized to maps indexed by integers,
such as the sorted predicate, permutation, etc.

\subsection{Library \texttt{option}}

This library provides the classical ML option type with constructors
\verb|None| and \verb|Some|.
\indextt{None}
\indextt{Some}

\subsection{Library \texttt{list}}

This library provides the classical ML type of polymorphic lists, with
constructors \verb|Nil| and \verb|Cons|. Most of the classical list
operators are provided in separate theories.

\section{Modules}
\label{sec:mllibrary}

The standard ML modules provided allow to write imperative
programs. The two main modules are the one providing ML references,
and the one providing arrays.

\subsection{Library \texttt{ref}}


\begin{description}
\item[Ref] provides references \emph{i.e.} mutable variables:
  type \verb|ref 'a| and functions \verb|ref| for creation,
  \verb|(!)| for access, and \verb|(:=)| for mutation.
\item[Refint] provides additional functions \texttt{incr},
  \texttt{decr} and a few others, over integer references.
\end{description}
\indextt{ref}
\index{"!@\texttt{"!}}
\indextt{:=}

\subsection{Library \texttt{array}}

\begin{description}
\item[Array]polymorphic arrays (type \texttt{array 'a}, infix
  syntax $a[i]$ for access and $a[i] \leftarrow e$ for update,
  functions \texttt{length}, \texttt{make}, \texttt{append},
  \texttt{sub}, \texttt{copy}, \texttt{fill}, and \texttt{blit})
\item[ArraySorted] an array of integers is sorted
  (\verb|array_sorted_sub| and \verb|array_sorted|)
\item[ArrayEq] two arrays are identical
  (\verb|array_eq_sub| and \verb|array_eq|)
\item[ArrayPermut] two arrays are permutation of each other
  (\verb|permut_sub| and \verb|permut|)
\end{description}

\subsection{Standard Data Types}

A few other classical data types are provided, such as queues (library
\texttt{queue}), stacks (library \texttt{stack}), hash tables (library
\texttt{hashtbl}), characters and strings (library \texttt{string}),
etc.


%%% Local Variables:
%%% mode: latex
%%% TeX-PDF-mode: t
%%% TeX-master: "manual"
%%% End:
