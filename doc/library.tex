\chapter{Standard Library: Why3 Theories}
\label{chap:library}

We provide here a short description of logic symbols defined in the
standard library. Only the most general-purpose ones are
described. For more details, one should directly read the
corresponding file, or alternatively, use the \verb|why3| with option
\verb|-T| and a qualified theory name, for example:
\begin{verbatim}
> why3 -T bool.Ite
theory Ite
  (* use BuiltIn *)

  (* use Bool *)

  function ite (b:bool) (x:'a) (y:'a) : 'a =
    match b with
    | True -> x
    | False -> y
    end
end
\end{verbatim}

In the following, for each library, we describe the (main) symbols
defined in it.

\section{Library \texttt{bool}}

\begin{description}

\item[Bool] boolean data type \verb|bool| with constructors \verb|True| and
  \verb|False|; operations \verb|andb|, \verb|orb|, \verb|xorb|, \verb|notb|.

\item[Ite] polymorphic if-then-else operator written as \verb|ite|.

\end{description}

\section{Library \texttt{int}}

\begin{description}

\item[Int] basic operations \verb|+|, \verb|-| and \verb|*|; comparison
  operators \verb|<|, \verb|>|, \verb|>=| and \verb|<=|.

\item[Abs] absolute value written as \verb|abs|.

\item[EuclideanDivision] division and modulo, where division rounds
  down, written as \verb|div| and \verb|mod|.

\item[ComputerDivision] division and modulo, where division rounds to
  zero, also written as \verb|div| and \verb|mod|.

\item[MinMax] \verb|min| and \verb|max| operators.

\end{description}

\section{Library \texttt{real}}

\begin{description}

\item[Real] basic operations \verb|+|, \verb|-|, \verb|*| and \verb|/|;
  comparison operators.

\item[RealInfix] basic operations with alternative syntax \verb|+.|,
  \verb|-.|, \verb|*.|, \verb|/.|, \verb|<.|, \verb|>.|, \verb|<=.| and \verb|>=.|, to
  allow simultaneous use of integer and real operators.

\item[Abs] absolute value written as \verb|abs|.

\item[MinMax] \verb|min| and \verb|max| operators.

\item[FromInt] operator \verb|from_int| to convert an integer to a real.

\item[Truncate] conversion operators from real to integers:
  \verb|truncate| rounds to 0, \verb|floor| rounds down and
  \verb|ceil| rounds up.

\item[Square] operators \verb|sqr| and \verb|sqrt| for square and square root.

\item[ExpLog] functions \verb|exp|, \verb|log|, \verb|log2|, and \verb|log10|.

\item[Power] function \verb|pow| with two real parameters.

\item[Trigonometry] functions \verb|cos|, \verb|sin|, \verb|tan|, and
  \verb|atan|. Constant \verb|pi|.

\item[Hyperbolic] functions \verb|cosh|, \verb|sinh|, \verb|tanh|,
  \verb|acosh|, \verb|asinh|, \verb|atanh|.

\item[Polar] functions \verb|hypot| and \verb|atan2|.

\end{description}

\section{Library \texttt{floating\_point}}

This library provides a theory of IEEE-754 floating-point numbers. It
is inspired by~\cite{ayad10ijcar}.

\begin{description}
\item[Rounding] type \verb|mode| with 5 constants
  \verb|NearestTiesToEven|, \verb|ToZero|, \verb|Up|, \verb|Down| and
  \verb|NearTiesToAway|.
\item[SpecialValues] handling of infinities and NaN.
\item[GenFloat] generic floats parameterized by the maximal
  representable number. Functions \verb|round|, \verb|value|,
  \verb|exact|, \verb|model|, predicate \verb|no_overflow|.
\item[Single] instance of GenFloat for 32-bits single precision numbers.
\item[Double] instance of GenFloat for 64-bits double precision numbers.
\end{description}


\section{Library \texttt{array}}

\begin{description}

\item[Array] polymorphic arrays, a.k.a maps. Type \verb|t|
  parameterized by both the type of indices and the type of
  data. Functions \verb|get| and \verb|set| to access and update
  arrays. Function \verb|create_const| to produce an array initialized
  by a given constant.

\item[ArrayLength] arrays indexed by integers and holding their
  length. Function \verb|length|.

\item[ArrayRich] additional functions on arrays indexed by
  integers. Functions \verb|sub| and \verb|app| to extract a sub-array
  and append arrays.

\end{description}

\section{Library \texttt{option}}

\begin{description}
\item[Option] data type \verb|option 'a| with constructors \verb|None| and
  \verb|Some|.
\end{description}


\section{Library \texttt{list}}

\begin{description}
\item[List] data type \verb|list 'a| with constructors \verb|Nil| and
  \verb|Cons|.
\item[Length] function \verb|length|
\item[Mem] function \verb|mem| for testing for list membership.
\item[Nth] function \verb|nth| for extract the $n$-th element.
\item[HdTl] functions \verb|hd| and \verb|tl|.
\item[Append] function \verb|append|, concatenation of lists.
\item[Reverse] function \verb|reverse| for list reversal.
\item[Sorted] predicate \verb|sorted| for lists of integers.
\item[NumOcc] number of occurrences in a list.
\item[Permut] list permutations.
\item[Induction] structural induction on lists.
\item[Map] list map operator.
\end{description}


\chapter{Standard Library: Why3ML Modules}
\label{chap:mllibrary}

\section{Library \texttt{ref}}

\begin{description}
\item[Ref] references \emph{i.e.} mutable variables:
  type \verb|ref 'a| and functions \verb|ref| for creation,
  \verb|(!)| for access, and \verb|(:=)| for mutation
\item[Refint] references with additional functions \texttt{incr} and
  \texttt{decr} over integer references
\end{description}

\section{Library \texttt{array}}

\begin{description}
\item[Array]polymorphic arrays (type \texttt{array 'a}, infix
  syntax $a[i]$ for access and $a[i] \leftarrow e$ for update,
  functions \texttt{length}, \texttt{make}, \texttt{append},
  \texttt{sub}, \texttt{copy}, \texttt{fill}, and \texttt{blit})
\item[ArraySorted] an array of integers is sorted
  (\verb|array_sorted_sub| and \verb|array_sorted|)
\item[ArrayEq] two arrays are identical
  (\verb|array_eq_sub| and \verb|array_eq|)
\item[ArrayPermut] two arrays are permutation of each other
  (\verb|permut_sub| and \verb|permut|)
\end{description}

\section{Library \texttt{queue}}

\begin{description}
\item[Queue] polymorphic mutable queues (type \texttt{t 'a} and
  functions \texttt{create}, \texttt{push}, \texttt{pop},
  \texttt{top}, \texttt{clear}, \texttt{copy}, \texttt{is\_empty},
  \texttt{length})
\end{description}

\section{Library \texttt{stack}}

\begin{description}
\item[Stack] polymorphic mutable stacks (type \texttt{t 'a} and
  functions \texttt{create}, \texttt{push}, \texttt{pop},
  \texttt{top}, \texttt{clear}, \texttt{copy}, \texttt{is\_empty},
  \texttt{length})
\end{description}

\section{Library \texttt{hashtbl}}

\begin{description}
\item[Hashtbl] hash tables with monomorphic keys (type \texttt{key})
  and polymorphic values (type \texttt{t 'a} of hash tables, syntax
  $h[k]$ for access, functions \texttt{create}, \texttt{clear},
  \texttt{add}, \texttt{mem}, \texttt{find}, \texttt{find\_all},
  \texttt{copy}, \texttt{remove}, and \texttt{replace})
\end{description}

\section{Library \texttt{string}}

\begin{description}
\item[Char]
\item[String]
\end{description}

%%% Local Variables:
%%% mode: latex
%%% TeX-PDF-mode: t
%%% TeX-master: "manual"
%%% End:
