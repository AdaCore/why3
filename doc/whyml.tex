\chapter{The Why3ML Programming Language}
\label{chap:whyml}

This chapter describes the \whyml\ programming language.
A \whyml\ input text contains a list of theories (see
chapter~\ref{chap:syntax}) and/or modules.
Modules extend theories with \emph{programs}.
Programs can use all types, symbols, and constructs from the logic.
They also provide extra features:
\begin{itemize}
\item In a record type declaration, some fields can be declared
  \texttt{mutable}.
\item There are programming constructs with no counterpart in the logic:
  \begin{itemize}
  \item mutable field assignment;
  \item sequence;
  \item loops;
  \item exceptions;
  \item local and anonymous functions;
  \item annotations: pre- and postconditions, assertions, loop invariants.
  \end{itemize}
\item A program function can be non-terminating or can be proved
  to be terminating using a variant (a term together with a well-founded
  order relation).
\end{itemize}
% TODO: model types

% files .mlw
% command line
% tutorial with examples: same fringe, max/sum

%%% Local Variables:
%%% mode: latex
%%% TeX-PDF-mode: t
%%% TeX-master: "manual"
%%% End:
