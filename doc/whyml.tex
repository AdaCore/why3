\chapter{The Why3ML Programming Language}
\label{chap:whyml}

This chapter describes the \whyml\ programming language.
A \whyml\ input text contains a list of theories (see
chapter~\ref{chap:syntax}) and/or modules.
Modules extend theories with \emph{programs}.
Programs can use all types, symbols, and constructs from the logic.
They also provide extra features:
\begin{itemize}
\item
  In a record type declaration, some fields can be declared
  \texttt{mutable}.
\item
  There are programming constructs with no counterpart in the logic:
  \begin{itemize}
  \item mutable field assignment;
  \item sequence;
  \item loops;
  \item exceptions;
  \item local and anonymous functions;
  \item annotations: pre- and postconditions, assertions, loop invariants.
  \end{itemize}
\item
  A program function can be non-terminating or can be proved
  to be terminating using a variant (a term together with a well-founded
  order relation).
\item
  An abstract program type $t$ can be introduced with a logical
  \emph{model} $\tau$: inside programs, $t$ is abstract, and inside
  annotations, $t$ is an alias for $\tau$.
\end{itemize}
%
Programs are contained in files with suffix \verb|.mlw|.
They are handled by the tool \texttt{why3ml}, which has a command line
similar to \texttt{why3}. For instance
\begin{verbatim}
  % why3ml myfile.mlw
\end{verbatim}
will display the verification conditions extracted from modules in
file \texttt{myfile.mlw}, as a set of corresponding theories, and
\begin{verbatim}
  % why3ml -P alt-ergo myfile.mlw
\end{verbatim}
will run the SMT solver Alt-Ergo on these verification conditions.
Program files are also handled by the GUI tool \texttt{why3ide}.
See Chapter~\ref{chap:manpages} for more details regarding command lines.

\medskip
As an introduction to \whyml, we use the five problems from the VSTTE
2010 verification competition~\cite{vstte10comp}.

\subsection{Problem 1: Sum and Maximum}

The first problem is stated as follows:
\begin{quote}
  Given an $N$-element array of natural numbers,
  write a program to compute the sum and the maximum of the
  elements in the array.
\end{quote}
We  assume $N \ge 0$ and $a[i] \ge 0$ for $0 \le i < N$, as precondition,
and we have to prove the following postcondition:
\begin{displaymath}
  sum \le N \times max.
\end{displaymath}


\subsection{Problem 2: Inverting an Injection}

\subsection{Problem 3: Searching a Linked List}

\subsection{Problem 4: N-Queens}

\subsection{Problem 5: Amortized Queue}

% other examples: same fringe ?

%%% Local Variables:
%%% mode: latex
%%% TeX-PDF-mode: t
%%% TeX-master: "manual"
%%% End:
