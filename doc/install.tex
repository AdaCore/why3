
\chapter{Compilation, Installation}
\label{sec:install}


In short, installation proceeds as follows.
\begin{flushleft}\ttfamily
  ./configure\\
  make\\
  make install \mbox{\rmfamily (as super-user)}
\end{flushleft}

\section{Installation Instructions from Source Distribution}

After unpacking the distribution, go to the newly created directory
\texttt{why3-\whyversion}. Compilation must start with a
configuration phase which is run as 
\begin{verbatim}
./configure
\end{verbatim}
This analyzes your current configuration and checks if requirements hold.
Compilation requires:
\begin{itemize}
\item The Objective Caml compiler, version 3.11.2 or higher. It is
  available as a binary package for most Unix distributions. For
  Debian-based Linux distributions, you can install the packages
\begin{verbatim}
ocaml ocaml-native-compilers
\end{verbatim}
It is also installable from sources, downloadable from the site
\url{http://caml.inria.fr/ocaml/}
\end{itemize}

\noindent
For some of the \why tools, additional OCaml libraries are needed:
\begin{itemize}
\item For the graphical interface, the Lablgtk2 library is needed.
  It provides OCaml
  bindings of the gtk2 graphical library. For Debian-based Linux
  distributions, you can install the packages
\begin{verbatim}
liblablgtk2-ocaml-dev liblablgtksourceview2-ocaml-dev
\end{verbatim}
It is also installable from sources, available from the site
\url{http://wwwfun.kurims.kyoto-u.ac.jp/soft/olabl/lablgtk.html}

\item For \texttt{why3 bench}, the OCaml bindings of the sqlite3 library
are needed.
For Debian-based Linux distributions, you can install the package
\begin{verbatim}
libsqlite3-ocaml-dev
\end{verbatim}
It is also installable from sources, available from the site
\url{http://ocaml.info/home/ocaml_sources.html#ocaml-sqlite3}
\end{itemize}


If you want to use the specific Coq features, \ie the Coq tactic
(Section~\ref{sec:coqtactic}) and Coq realizations
(Section~\ref{sec:realizations}), then Coq has to be installed before
\why. Look at the summary printed at the end of the configuration
script to check if Coq has been detected properly. Similarly, for
using PVS (Section~\ref{sec:pvs}) or Isabelle (Section~\ref{sec:isabelle}) to discharge proofs, PVS and Isabelle must be
installed before \why. You should check that those proof assistants
are correctly detected by the configure script.

When configuration is finished, you can compile \why.
\begin{verbatim}
make
\end{verbatim}
Installation is performed (as super-user if needed) using
\begin{verbatim}
make install
\end{verbatim}
Installation can be tested as follows:
\begin{enumerate}
\item install some external provers (see~Section~\ref{provers} below)
\item run \verb|why3 config --detect|
\item run some examples from the distribution, \eg you should
obtain the following:
\begin{verbatim}
$ cd examples
$ why3 replay logic/scottish-private-club
Opening session... done
Progress: 4/4
 1/1
Everything OK.
$ why3 replay programs/same_fringe
Opening session... done
Progress: 12/12
 3/3
Everything OK.
\end{verbatim}
\end{enumerate}

\section{Local Use, Without Installation}

It is not mandatory to install \why into system directories.
\why can be configured and compiled for local use as follows:
\begin{verbatim}
./configure --enable-local
make
\end{verbatim}
The \why executables are then available in the subdirectory
\texttt{bin/}. This directory can be added in your \texttt{PATH}.

\section{Installation of the \why API}
\label{sec:installlib}\index{API}

By default, the \why API is not installed. It can be installed using
\begin{flushleft}\ttfamily
make byte opt \\
make install-lib \mbox{\rmfamily (as super-user)}
\end{flushleft}

\section{Installation of External Provers}
\label{provers}

\why can use a wide range of external theorem provers. These need to
be installed separately, and then \why needs to be configured to use
them. There is no need to install automatic provers, \eg SMT solvers,
before compiling and installing \why.

For installation of external provers, please refer to the specific
section about provers on the Web page \url{http://why3.lri.fr/}.

For configuring \why to use the provers, follow instructions given in
Section~\ref{sec:why3config}.

\section{Multiple Versions of the Same Prover}

\why is able to use several versions of the same
prover, \eg it can use both CVC3 2.2 and CVC3 2.4.1 at the same time.
The automatic detection of provers looks for typical names for their
executable command, \eg \texttt{cvc3} for CVC3. However, if you
install several version of the same prover it is likely that you would
use specialized executable names, such as \texttt{cvc3-2.2} or
\texttt{cvc3-2.4.1}. To allow the \why detection process to recognize
these, you can use the option \verb|--add-prover| with the
\texttt{config} command, \eg
\index{add-prover@\verb+--add-prover+}
\begin{verbatim}
why3 config --detect --add-prover cvc3-2.4 /usr/local/bin/cvc3-2.4.1
\end{verbatim}
the first argument (here \verb|cvc3-2.4|) must be one of the class of
provers known in the file \verb|provers-detection-data.conf| typically
located in \verb|/usr/local/share/why3| after installation. See
Appendix~\ref{sec:proverdetecttiondata} for details.


\section{Session Update after Prover Upgrade}
\label{sec:uninstalledprovers}

If you happen to upgrade a prover, \eg installing CVC3 2.4.1 in place
of CVC3 2.2, then the proof sessions formerly recorded will still
refer to the old version of the prover. If you open one such a session
with the GUI, and replay the proofs, you will be asked to choose
between 3 options:
\begin{itemize}
\item Keep the former proofs as they are. They will be marked as
  ``archived''.
\item Upgrade the former proofs to an installed prover (typically a
  upgraded version). The corresponding proof attempts will become
  attached to this new prover, and marked as obsolete,
  to make their replay mandatory.
\item Copy the former proofs to an installed prover. This is a
  combination of the actions above: each proof attempt is duplicated,
  one with the former prover marked as archived, and one for the new
  prover marked as obsolete.
\end{itemize}

Notice that if the prover under consideration is an interactive one, then
the copy option will duplicate also the edited proof scripts, whereas
the upgrade-without-archive option will just reuse the former proof scripts.

Your choice between the three options above will be recorded, one for
each prover, in the \why configuration file. Within the GUI, you can
discard these choices via the \textsf{Preferences} dialog.

Outside the GUI, the prover upgrades are handled as follows. The
\texttt{replay} command will just ignore proof attempts marked as
archived\index{archived}.
Conversely, a non-archived proof attempt with a non-existent
prover will be reported as a replay failure. The
\texttt{session} command performs move or copy operations on
proof attempts in a fine-grained way, using filters, as detailed in
Section~\ref{sec:why3session}.


% pour l'instant on ne documente pas
% {que devient l'option -to-known-prover de why3session ?
%   (d'ailleurs documenté en tant que --convert-unknown ??) Pourrait-on
%   permettre à why3session d'appliquer les choix d'association
%   vieux-prover/nouveau-prouveur stockés par l'IDE ?}


%%% Local Variables:
%%% mode: latex
%%% TeX-PDF-mode: t
%%% TeX-master: "manual"
%%% End:
